\thm{Q.226 $\bigstar 2$ (2017慶大 総合政策)}
B,Hの半径を求めよ。
\enthm
B,C,Dの半径を$x$\\
E,Fの半径を$y$とする。\\
Eの直径$=$Bの直径+Cの直径より $y=2x$\\
Hの半径はEの直径+Aの直径なので$2y+2=4x+2$\\
Hの中心をOとし, B,Cの交点をKとすると\\
OKの長さは三平方の定理から $\sqrt{x^2+(2x)^2}~\sqrt{5}x$\\
EとHの円弧との接点をLとすると, KLはEの半径$2x$に等しい。\\
よって, OL$=(2+\sqrt{5})x$\\
これはHの半径にも等しいので OL$=4x+2$\\
この2式から, Bの半径は$x=\dfrac{2}{\sqrt{5}-2}=\dfrac{2}{3}(2+\sqrt{5})$\\
Hの半径は $(2+\sqrt{5})x=\dfrac{2}{3}(9+4\sqrt{5})$ 
