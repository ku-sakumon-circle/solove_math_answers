\thm{Q.63 $\bigstar 6$ (京大OP)}
$p$を無理数, $q$を実数とする。数列$\{a_n\}$が
\[a_{n+1}=pa_n+q  (n=1,2,3,\cdots)\]
を満たす。すべての自然数$n$に対して$a_n$が有理数であるとき, $a_n$は$n$によらない定数であることを証明せよ。
\enthm
\begin{proof}
与えられた式から, $n$をひとつ固定したとき
\[ a_{n+2}-a_{n+1}=p(a_{n+1}-a_n) \]
が成立する。$a_{n+1}\neq a_n$ であるとすると, 
\[p=\dfrac{a_{n+2}-a_{n+1}}{a_{n+1}-a_{n}}\]
であり, 条件より $a_n, a_{n+1}, a_{n+2}$は有理数なのでこの右辺は有理数となる。$p$は無理数であるからこれは矛盾する。従って $a_{n+1}=a_n$でなければならず, これがすべての$n=1,2,\cdots$ に対して成立するから
\[a_1=a_2=a_3=\cdots =a_n=a_{n+1}=\cdots \]
となるので,各$n$に対して $a_n=a_1$ という$n$によらない定数になる。
\end{proof}
