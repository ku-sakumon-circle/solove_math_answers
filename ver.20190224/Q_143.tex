\thm{Q.143 (漸化式)}
(17) $Q_1=4$\,\, $Q_{n+1}^2 =Q_n^2-2$
\enthm
\begin{enumerate}
\item[(17)] $Q_n=q_n+\dfrac{1}{q_n}$ となる実数$q_n$が存在することを示す。$n=1$のときは $q_1=2+\sqrt{3}$でよい。$Q_{n+1}=Q_n^2-2= q_n^2+\dfrac{1}{q_n^2}$ より, $q_{n+1}=q_n^2$ とすればよいので 帰納的に $Q_n=q_n+\dfrac{1}{q_n}$ と置くことができ, 構成の仕方から $q_n=q_1^{2^{n-1}}=(2+\sqrt{3})^{2^{n-1}}$ とすればよい。よって
\[Q_n= (2+\sqrt{3})^{2^{n-1}}+ (2-\sqrt{3})^{2^{n-1}}\]
\end{enumerate}
