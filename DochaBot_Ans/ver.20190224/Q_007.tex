\thm{Q.7 \hosi 4 (東工大2017)}
次の条件 (i), (ii)をともに満たす正の整数$N$を全て求めよ。\\
(i) $N$の正の約数は12個\\
(ii) $N$の正の約数を小さい順にならべたとき, 7番目の数は12である。
\enthm
$N$の約数を小さい順に$d_1, d_2,\cdots , d_{12}$ とする。このとき,
\[1=d_1<d_2<\cdots <d_{12}=N\]
で, $6\leq d_6$であり,
\[N=\dfrac{N}{d_1}>\dfrac{N}{d_2}>\cdots > \dfrac{N}{d_{12}}=1\]
である。また, $d$を$N$の約数とすると $\dfrac{N}{d}$も$N$の約数なので, $\dfrac{N}{d_k}$は$k$番目に大きい約数になるから $\dfrac{N}{d_k}=d_{13-k}$である。$k=6$として 
\[ N=d_6d_7=12d_6<12^2\]
を得る。$d_6\geq 6$ を使って
\[N=12d_6\geq 12\times 6\]
である。$N$は12を約数に持つので $N=12$m とおくと, 
\[12\times 6\leq 12m<12^2\]
より $m=6,7,8,9,10,11$が必要である。このうち, $m=10$では$N$が約数を12個以上持つので不適であり, $m=6$では $d_7=9$なので不適。それ以外の $N$ はすべて条件を満たす。よって {\boldmath{$N=84,96,108, 132$}}
