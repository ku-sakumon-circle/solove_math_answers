\thm{Q.210 $\bigstar 7$ (2017 京大理系)}
 $w$ を0でない複素数, $x,y$ を $w+\dfrac{1}{w}=x+yi$ を満たす実数とする.\footnote{これはジェーコフスキー変換という名がついており,航空学などに応用がされている。}\\
(1) 実数 $R$ は $R>1$ を満たす定数とする. $w$ が絶対値 $R$ の複素数全体を動く
\indent とき, $xy$ 平面上の点 $(x,y)$ の軌跡を求めよ.\\
(2) 実数 $\alpha$ は $0<\alpha<\dfrac{\pi}{2}$ を満たす定数とする. $w$ が偏角 $\alpha$ の複素数全体を動
\indent くとき, $xy$ 平面上の点 $(x,y)$ の軌跡を求めよ.\\
\enthm
