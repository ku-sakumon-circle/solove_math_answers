\thm{Q.5 }
(1) $x>0$のとき, $\left(x+\dfrac{1}{x}\right)\left(x+\dfrac{4}{x}\right)$ の最小値を求めよ。(1997千葉工大)\\
(2) $x>0$のとき, $x^2+\dfrac{1}{x^3}$の最小値を求めよ。\\
(3) 実数$a,b$が $a+b=17$を満たすとき, $2^a+4^b$の最小値を求めよ。(JMO予選2005)
\end{itembox}
\begin{enumerate}
\item[(1)] $\left(x+\dfrac{1}{x}\right)\left(x+\dfrac{4}{x}\right)=x^2+\dfrac{4}{x^2}+5$ で, $x^2,\dfrac{4}{x^2}>0$より相加相乗平均不等式が使えて $x^2+\dfrac{4}{x^2}\geq 2\sqrt{x^2\cdot \dfrac{4}{x^2}}=4$ となる。(等号は$x=\sqrt{2}$のときに起こる)よって最小値は $4+5=9$
\item[(2)] $x^2>0, x^3>0$なので$x^2+\dfrac{1}{x^3}=\dfrac{x^2}{3}+\dfrac{x^2}{3}+\dfrac{x^2}{3}+\dfrac{1}{2x^3}+\dfrac{1}{2x^3}\geq 5\sqrt[5]{\left(\dfrac{x^2}{3}\right)^3\left(\dfrac{1}{2x^3}\right)^2}=\sqrt[5]{\dfrac{1}{108}}$  (等号成立は$x=\sqrt[5]{\dfrac{3}{2}}$のときに起こる)より最小値は$\dfrac{1}{\sqrt[5]{108}}$
\item[(3)] $2^{a-1}+2^{a-1}+4^b\geq 3\sqrt[3]{2^{a-1}\cdot 2^{a-1}\cdot 4^b}=\sqrt[3]{2^{2a-2+2b}}=\sqrt[3]{2^{32}}$ (等号成立は$a=\dfrac{35}{3}, b=\dfrac{16}{3}$)より最小値は$3072\sqrt[3]{4}$
\end{enumerate}
