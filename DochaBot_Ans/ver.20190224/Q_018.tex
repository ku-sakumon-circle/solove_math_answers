\thm{Q.18 (並び替え数列)}
数列$1,2,\cdots, n$ のある並び替えを $a_1.a_2,\cdots , a_n$ とする。\\
(1) $\disp\sum_{k=1}^n (a_k-k)^2 + \disp\sum_{k=1}^n (a_k-n+k-1)^2$ を $n$で表せ。\\
(2) $\disp\sum_{k=1}^n(a_k-k)^2$ の最大値を求めよ。\\
(3) $\disp\sum_{k=1}^nka_k$ が最小となるとき, $a_k=n+1-k$ であることを証明せよ。
\enthm
\begin{enumerate}
\item[(1)] 括弧を展開して一つのシグマの中に入れると 
\begin{eqnarray*}
&&\disp\sum_{k=1}^n \{2a_k^2-2(n+1)a_k + k^2 + (n+1-k)^2 \}\\
&&=2\disp\sum_{k=1}^na_k^2 -2(n+1)\disp\sum_{k=1}^na_k + 2\disp\sum_{k=1}^n k^2\\
&&=4\disp\sum_{k=1}^nk^2 - 2(n+1) \disp\sum_{k=1}^nk\\
&&=\dfrac{n(n+1)(8n+4)}{6} - \dfrac{n(n+1)(6n+6)}{6}\\
&&=\dfrac{n(n+1)(n-1)}{3}\\
&&\dfrac{n^3-n}{3}
\end{eqnarray*}
\item[(2)] (1) より
\[\disp\sum_{k=1}^n(a_k-k)^2 = \dfrac{n^3-n}{3}-\disp\sum_{k=1}^n (a_k-n+k-1)^2\]
であって, 右辺を最大化したい。 2乗の和なので$\disp\sum_{k=1}^n (a_k-n+k-1)^2\geq 0$が成り立ち, $a_k=n-k+1$ とすることで, $a_1,a_2,\cdots, a_n$は$1,2,\cdots ,n$の並び替えでありながら右辺が最大化されると分かる。よって, 最大値はそのような$a_k$のときに実現される $\dfrac{n^3-n}{3}$ である。
\item[(3)] (2) より $\disp\sum_{k=1}^n (a_k^2+k^2 -2ka_k)=2\disp\sum_{k=1}^n k^2 -2\disp\sum_{k=1}^n ka_k$ が最大化されるのが $a_k=n+1-k$ のときであるが, これは $\disp\sum_{k=1}^n ka_k$ を最小化させているときとも見れるので明らか。
\end{enumerate}
