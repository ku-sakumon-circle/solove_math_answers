\documentclass[twocolumn]{jbook}
\usepackage{bm,braket,ascmac}
\usepackage{amsmath,amssymb,amsthm,mathrsfs}
\usepackage{amsmath,amssymb,amsfonts,fancyhdr,ulem}
\usepackage[all]{xy}
\usepackage[%
top    = 15truemm,%
bottom = 15truemm,%
left   = 6truemm,%
right  = 6truemm]{geometry}
\usepackage{amsthm, bm}
\usepackage{tikz}
\usetikzlibrary{shadows}
\pagestyle{fancy}
\lhead{\rightmark}
% Theorem 環境
\theoremstyle{definition}
\newtheorem*{prob}{\textbf{問}}

% Enumerate 環境
\def\theenumi{\arabic{enumi}}
\def\labelenumi{(\theenumi)}

% (Re)New commands
\newcommand*{\vvv}{\overrightarrow}
\newcommand{\thm}{\begin{itembox}[l]}
\newcommand{\jap}{\mbox}

\newcommand{\enthm}{\end{itembox}\\}
\newcommand{\disp}{\displaystyle}
\newcommand{\dou}{\Leftrightarrow}
\newcommand{\del}{\partial}
\newcommand{\ben}{\begin{enumerate}}
\newcommand{\een}{\end{enumerate}}
\newcommand{\beqn}{\begin{eqnarray*}}
\newcommand{\eeqn}{\end{eqnarray*}}
\newcommand{\bcas}{\begin{cases}}
\newcommand{\ecas}{\end{cases}}
\newcommand{\nara}{\Rightarrow}
\newcommand{\mb}{\mathbb}
\newcommand{\maf}{\mathfrak}
\newcommand{\mr}{\mathrm}
\newcommand{\hana}{\mathcal} %花文字
\newcommand{\chat}{\hat{\mathbb{C}}} %リーマン球面
\renewcommand{\leq}{\leqq}
\renewcommand{\geq}{\geqq}
\newcommand{\nea}{\nearrow}
\newcommand{\exer}{ 演習\arabic{chapter}.\arabic{section}.}
\newcommand{\sea}{\searrow}
\newcommand{\bracket}[1]{%
\[
 \left(
 \begin{tabular}{p{0.9\hsize}}
  #1
 \end{tabular}
 \right)
\]}
\begin{document}



{\Large Q.216}\\
\\
(1)\\
$f(x)=x-1- \log{x} $とおいて, $x>0$で$f(x)\geq 0$であることを示せば良い。\\
$f'(x) = 1- \dfrac{1}{x}$より, $f(x)$は$0<x<1$で減少し, $x=1$で極小値を取り, $x>1$で増加する。従って, $f(x)>f(1) = 0$だから示された。\qed\\
\\
(2) $p=\log{101}$とおく。$p=2\log{10}+\log{1.01}$なので, (1) の不等式より
\[p\leq 2\log{10}+ 0.01 < 2\times (2.30259) + 0.01 = 4.61518\]
(1)の不等式の$x$を$\frac{1}{x}>0$に置き換えて, $f(\frac{1}{x})\geq 0$ という不等式が成立することを利用すれば, 
\[f(\frac{1}{x})= \dfrac{1}{x}- 1 + \log{x} \geq 0\]
が$x>0$で成立する。整理すると
\[1-\dfrac{1}{x}\leq \log{x}\]
となるので, これに$x=1.01$を代入すれば
\[\log{1.01} \geq 1- \dfrac{1}{1.01} = \dfrac{1}{101} > 0.0099\]
という$\log{1.01}$の下からの評価を得る。これによって, 
\[p \geq 2\log{10} + \log{1.01}  > 2\times (2.30258) + 0.0099 =4.61506 \]
となる。従って,
\[4.61506 < p < 4.61518\]
だから, $p= 4.615\cdots $である。






\end{document}