\documentclass[twocolumn]{jbook}
\usepackage[dvipdfmx]{graphicx}
\usepackage{bm,braket,ascmac,extarrows}
\usepackage{amsmath,amssymb,amsthm,mathrsfs,makeidx}
\usepackage{amsfonts,graphics,fancyhdr,ulem,stmaryrd}
\usepackage[all]{xy}
\usepackage[%
top    = 15truemm,%
bottom = 15truemm,%
left   = 6truemm,%
right  = 6truemm]{geometry}
\usepackage{amsthm, bm}

\usepackage{tikz}
\usetikzlibrary{shadows}
\usetikzlibrary{patterns}

\pagestyle{fancy}
\lhead{\rightmark}
% Theorem 環境
\theoremstyle{definition}


%メモでつかう%
\newtheorem{theo}{定理}
\newtheorem{lem}[theo]{補題}
\newtheorem{prop}[theo]{命題}
\newtheorem{prob}[theo]{問題}
\newtheorem{example}[theo]{例}
 
%Enumerate 環境
\def\theenumi{\arabic{enumi}}
\def\labelenumi{(\theenumi)}

% (Re)New commands

\newcommand{\maru}{\textcircled}
\newcommand{\symdiff}{\bigtriangleup}
\newcommand{\ovl}{\overline}
\newcommand{\lan}{\langle}
\newcommand{\ran}{\rangle}
\newcommand*{\vvv}{\overrightarrow}
\newcommand{\vin}{\rotatebox{90}{$\in$}}
\newcommand{\thm}{\begin{itembox}[l]}
\newcommand{\jap}{\mbox}
\newcommand{\enthm}{\end{itembox}\\}
\newcommand{\disp}{\displaystyle}
\newcommand{\dou}{\Leftrightarrow}
\newcommand{\del}{\partial}
\newcommand{\ben}{\begin{enumerate}}
\newcommand{\een}{\end{enumerate}}
\newcommand{\beqn}{\begin{eqnarray*}}
\newcommand{\eeqn}{\end{eqnarray*}}
\newcommand{\bcas}{\begin{cases}}
\newcommand{\ecas}{\end{cases}}
\newcommand{\nara}{\Rightarrow}
\newcommand{\mb}{\mathbb}
\newcommand{\mr}{\mathrm}
\newcommand{\maf}{\mathfrak}
\newcommand{\masc}{\mathscr}
\newcommand{\hana}{\mathcal} %花文字(花文字じゃない)
\newcommand{\chat}{\widehat{\mathbb{C}}} %リーマン球面
\newcommand{\khat}{\widehat{K}} %%Lemma4.5.6
\renewcommand{\leq}{\leqq}
\renewcommand{\geq}{\geqq}
\renewcommand\thefootnote{*\arabic{footnote}}
\newcommand{\nea}{\nearrow}
\newcommand{\exer}{{\bf 演習}\arabic{chapter}.\arabic{section}.}
\newcommand{\sea}{\searrow}
\newcommand{\witi}{\widetilde}
\newcommand{\bracket}[1]{%
\[
 \left(
 \begin{tabular}{p{0.9\hsize}}
  #1
 \end{tabular}
 \right)
\]}
\makeatletter
\def\mapstofill@{%
   \arrowfill@{\mapstochar\relbar}\relbar\rightarrow}
\newcommand*\xmapsto[2][]{%
   \ext@arrow 0395\mapstofill@{#1}{#2}}
\newcommand{\xequal}[2][]{\ext@arrow 0055{\equalfill@}{#1}{#2}}
\def\equalfill@{\arrowfill@\Relbar\Relbar\Relbar}
\makeatother

\begin{document}








本問は, 次の命題がカラクリとなっている。
\begin{lem} \\
$c(n) = \dfrac{d(n) + \phi(n)}{\sigma(n)}$とおく。このとき, 次が成り立つ。
\ben
\item[(1)] $n$が素数ならば, $c(n) = 1$
\item[(2)] $n\geq 2$が素数でないならば, $0<c(n) < 1$
\een
\end{lem} \\
({\it Proof of Lemma})\\
\\
{\LARGE (1)}\\
$n$を素数とする。このとき,\\
$n$の正の約数は$1,n$の{\bf $2$個} で, その和は$n+1$であり, $n$以下の$n$と互いに素な自然数の個数は$1$から$n-1$までの全ての整数であり, {\bf $n-1$個}である。したがって, $d(n) = 2, \sigma(n)=n+1, \phi(n) = n-1$なので, 
\[c(n) = \dfrac{2+ (n-1)}{n+1} = 1\]
よりよい。\qed\\
\\
\\
{\LARGE (2)}\\
\\
$n\geq 2$を素数でないとする。$0<c(n)$であることは明らか。\\
$n$は$1$と$n$以外にも約数を持ち, $1\neq n$であるから, $n+1< \sigma(n) $が分かる。\\
\\
$1$から$n$までの任意の整数$k$に関して, 次の条件(i),(ii)を考える。
\ben
\item[(i)] $k$は$n$の約数である。
\item[(ii)] $k$は$n$と互いに素である。
\een
(i)を満たす$k$の個数が$d(n)$であり, (ii)を満たす$k$の個数が$\phi(n)$である。したがって, 
\beqn
D &=& \{ k\in \mb{Z}\cap [1,n]\mid kは(i)を満たす.\}\\
\Phi &=& \{ k\in \mb{Z}\cap [1,n] \mid kは(ii)を満たす. \}
\eeqn
という集合$D,\Phi$を定めると, $d(n),\phi(n)$は$D$, $\Phi$の元の個数である。\\
\\
 さて, $k=1$とすると, $1$は$n$の約数であり, $n$と互いに素であるから, $1\in D, 1\in \Phi$であることが分かる。\\
 続いて, $k\geq 2$とする。このとき, $k\in D$を満たすとすると, $k$と$n$の最大公約数は$k\neq 1$であるから(ii)を満たさない。\\
以上から, $D\cap \Phi = \{ 1 \}$であると分かる。有限集合$X$に対してその元の個数を$|X|$で書くことにする。定義より明らかに$D\cup \Phi \subset \{ 1,2,\cdots, n\}$であるから, 
\beqn
d(n) + \phi(n) &=& |D| + |\Phi|\\
&=& |D\cup \Phi| + |D \cap \Phi|\\
&=& |D\cup \Phi| + | \{ 1 \} |\\
&=& |D\cup \Phi| + 1 \\
&\leq & |\{ 1,2,\cdots, n\}| + 1\\
&=& n+1\\
&<& \sigma(n) 
\eeqn
となる。ゆえに, $c(n) <1$が得られた。\qed\\
\\
補題より, $k\geq 2$に対して
\[
\lfloor c(k) \rfloor = \bcas
1 (kが素数)\\
0 (kが素数でない)
\ecas
\]
となるので, $\disp\sum_{k=2}^{n} \lfloor c(k) \rfloor $は$2$から$n$までの素数の個数を計上する。したがって$\pi(n)$に一致する。\qed


\end{document}













