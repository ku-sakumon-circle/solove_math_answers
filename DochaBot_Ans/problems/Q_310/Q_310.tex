\begin{thm}{310}{\hosi 8-高}{自作, 学コン2024-12-3}
十進法で1の位が0でない正の整数$n$に対し, $f(n)$は, $n$を1の位から逆の順番で読んで得られる正の整数として定める. たとえば$f(123456789) = 987654321$である. 
%\begin{enumerate}
    %\item 
    $n+f(n)$が81の倍数となるような十進法で10桁の$n$の個数を求めよ. 
  %  \item $n$が十進法で9桁のとき, $n$と$f(n)$の両方が81の倍数となるのは$n=999999999$のみであることを証明せよ. 
%\end{enumerate}
\end{thm}


\[n = \sum_{k=1}^{10} a_k 10^{k-1}\]
とおく. 以下, 合同式は法81とする. 二項定理より$10^n = (9 + 1)^n \equiv 9n + 1\pmod{81}$なので, 
\[
n\equiv \sum_{k=1}^{10}(9k-8)a_k
\]
である.


% 同様に, 
% \[
% f(n) \equiv \sum_{k=1}^{10} (91 - 9k)a_{k}
% \]
% である. もし$n,f(n) \equiv 0$なら, 足すことで
% \[
% n + f(n) \equiv 0 \equiv \sum_{k=1}^{10} 83\cdot a_{k} 
% \]
% となる. $n+f(n)\equiv 0$のとき, $83$は3の倍数でないから, $a_1 +  \dots + a_{10}\equiv 0$である. $a_1 + \dots + a_{10} > 0$だから, $a_1 + \dots + a_{10} \geq 81$となる. $a_i \leq 9$だから $a_1 + \dots + a_{10} < 81\times 2$. よって $a_1 + \dots + a_{10} = 81$でなければならず,  このようになるのは$a_1 = \dots = a_9 = 9$の場合のみである. 



% 同様に桁で考えるとき, $n\equiv \sum_{k=1}^{10} (9k - 8)a_k$ なら
同様に$f(n) \equiv \sum_{k=1}^{10} (91-9k)a_k$となるので, 
\[
n + f(n) \equiv \sum_{k=1}^{10} 91 a_k \equiv 0
\]
より$a_1 + \dots + a_{10}$は81の倍数である. この総和は90を越えないので, $a_1 + \dots + a_{10} = 81$である. 

$b_i = 9-a_i$として$b_i$を定めると, $90 - (b_1 + \dots + b_{10}) = 81$なので 
\[
b_1 + \dots + b_{10} = 9,\qquad 0\leq b_i\leq 9,\qquad b_1,b_9\neq 9, 
\]
となる. 最初の2条件を満たす組$(b_1,\dots, b_{10})$は「9個の〇と9個の$|$の並べ替えの数」に等しいので, ${}_{18}\mathrm{C}_{9}$個ある. $b_1,b_9 \neq 9$を満たす組は$(9,0,0,\dots, 0)$と$(0,0,\dots,0,9)$の二つ以外($n = 9999999990, 0999999999$に対応)である. 以上より, 求める個数は
\[
{}_{18}\mathrm{C}_{9} - 2 = \boldsymbol{48618}\text{個}
\]
