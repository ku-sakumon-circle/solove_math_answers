\begin{thm}{056}{}{}
 \begin{enumerate}
  \item $3<\pi<4$ を示せ。\hosi 1
  \item $2<e<3$ を示せ。 \hosi 3
  \item 円周率が3.05より大きいことを証明せよ。 \hosi 3 (東大)
 \end{enumerate}
\end{thm}

\syoumon{1}
単位円 (周長は$2\pi$) と、それに外接する正方形 (周長は8) を比較することで$2\pi<8$が従う。また内接する正六角形 (周長は6) を比較することで$6<2\pi$が従う。よって、$3<\pi<4$が成り立つ。\footnote{本当は周長が 内接正六角形$<$円$<$外接正方形 であることは調べなければならないが、見るからに明らかであり、高校数学ではこの方法で十分であると思われる。}

\syoumon{2}
$e$の定義は$\disp \lim_{n\to\infty} \left(1+\frac{1}{n}\right)^2$ である\footnote{$e=\disp \sum_{k=0}^\infty\frac{1}{k!}$ という事実はあるけれども、高校数学の範囲で認めてよいか怪しい部分があると思われる。定義に立ち返るのが安全である。}。以降$n$は十分大きい自然数とする。まず二項定理によって
\[ 1+\frac{\combi{n}{1}}{n}=2<\left(1+\frac{1}{n}\right)^n \]
が成り立つので$e>2$である。次に展開した際に現れる$a_k=\dfrac{\combi{n}{k}}{n^k}$を上から評価する。$n^k\ge n(n-1)(n-2)\cdots(n-k+1)$ であることに注意すれば、$n^k(n-k)!\ge n!$ なので、
\[ a_k=\frac{\combi{n}{k}}{n^k}=\frac{n!}{k!(n-k)!n^k}\le \frac{n!}{k!n!}=\frac{1}{k!} \]
が従う。なお$k>1$なら等号は成り立たない。よって、
\[ \left(1+\frac{1}{n}\right)^n<1+\frac{1}{1!}+\dots +\frac{1}{n!}=1+\sum_{k=1}^n \frac{1}{k!} \]
となる。最期に$2^{k-1}\le k!$ が$k\ge 1$ で常に成り立つから、
\[ 1+\sum_{k=1}^n\frac{1}{k!}<1+\sum_{k=1}^n \frac{1}{2^{k-1}}=1+2\left(1-\frac{1}{2^n}\right) < 3 \]
がわかる。よって$e<3$が示された。

\syoumon{3}
単位円に内接する正12角形を考える。円の中心を点$\mr{O}$、隣り合う2頂点を$\mr{A,B}$とすると、これは頂角が$30^\circ$の二等辺三角形となり、$\mr{OA}=\mr{OB}=1$。$\mr{A}$から$\mr{OB}$に垂線を下ろし、その足を$\mr{H}$とする。$\angle\mr{AOH}=30^\circ$だから、$\mr{OH}=\dfrac{\sqrt{3}}{2}$, $\mr{AH}=\dfrac{1}{2}$, $\mr{BH}=1-\dfrac{\sqrt{3}}{2}$。直角三角形$\mr{ABH}$について三平方の定理を用いて、
\[ \mr{AB}=\sqrt{\left(\frac{1}{2}\right)^2+\left(1-\frac{\sqrt{3}}{2}\right)^2}=\sqrt{2-\sqrt{3}} \]
$\mr{AB}$の12倍が正12角形の周長であり、円周はこれより長いから、
\[ 12\sqrt{2-\sqrt{3}} < 2\pi \quad\dou\quad 6\sqrt{2-\sqrt{3}} < \pi \]
が成り立つ。よって、$6\sqrt{2-\sqrt{3}}$が$3.05$よりも大きいことを示せばよい。双方とも正であるから、これらの大小は2乗した値、$36(2-\sqrt{3})$ と$3.05^2=9.3025$ との大小に一致する。$1.74^2=3.0276$ より、$\sqrt{3}<1.74$である。よって、
\[ 36(2-\sqrt{3}) > 36(2-1.74)=9.36>9.3025=3.05^2 \]
が成り立つ。よって$6\sqrt{2-\sqrt{3}}>3.05$が従う。

以上により、$\pi > 3.05$ が示された。