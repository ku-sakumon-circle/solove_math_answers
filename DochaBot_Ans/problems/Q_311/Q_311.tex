\begin{thm}{311}{\hosi 4-競}{OMCB031(D) / writer:aonagi}
$BC=7$なる三角形ABCにおいて, 辺$AB$上に点$P$を取ると, 以下が成立しました. 
\[
\angle{ABC} + \angle{ACP} = 90^{\circ},\quad CA:CP = 7:8,\quad BP = 5
\]
このとき, 三角形ABCの外接円の直径の長さは互いに素な正整数$a,b$を用いて$\sqrt{\frac{a}{b}}$と表されるので, $a+b$の値を解答してください. 
\end{thm}

解答は省略(OMCのリンクよりご確認ください)

ヒントとしては, 角度の条件に目をつけて補助点を取ります. 