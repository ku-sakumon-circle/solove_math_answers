\begin{thm}{223}{\hosi 8}{東レ模試}
 点$O$を中心とする半径$R$の円を$K$とする。与えられた正の数$a, b, c$に対して$R$を十分大きく取り、$K$の周上に4点$A, B, C, D$を$AB=a$, $BC=b$, $CD=c$かつ、$\angle{ABC}$と$\angle{ACD}$が鈍角となるように取る。$AC=x$, $AD=y$とする。
 \begin{enumerate}
  \item $\disp \lim_{R\to\infty} R^2(a+b-x)$ を求めよ。
  \item $\disp \lim_{R\to\infty} R^2(a+b+c-y)$ を求めよ。
 \end{enumerate}
\end{thm}

\syoumon{1}
ここに解答を記述。

\syoumon{2}
ここに解答を記述。