\begin{thm}{223}{\hosi 8}{東レ模試}
 点$O$を中心とする半径$R$の円を$K$とする。与えられた正の数$a, b, c$に対して$R$を十分大きく取り、$K$の周上に4点$A, B, C, D$を$AB=a$, $BC=b$, $CD=c$かつ、$\angle{ABC}$と$\angle{ACD}$が鈍角となるように取る。$AC=x$, $AD=y$とする。
 \begin{enumerate}
  \item $\disp \lim_{R\to\infty} R^2(a+b-x)$ を求めよ。
  \item $\disp \lim_{R\to\infty} R^2(a+b+c-y)$ を求めよ。
 \end{enumerate}
\end{thm}

本問で$R$に依存する量は$x,y,\angle{\mr{ABC}},\angle{\mr{ACD}}$などであることに注意する。

\syoumon{1}
$\angle{\mr{ABC}}=\theta_R$とおく。三角不等式より
\begin{eqnarray}
x<a+b\label{eq:triangle}
\end{eqnarray}
である。正弦定理および(\ref{eq:triangle})より
\begin{eqnarray}
\sin{\theta_R}=\dfrac{x}{2R}<\dfrac{a+b}{2R} \to 0 (R\to\infty)\label{eq:sinthm}
\end{eqnarray}
なので, $\theta_R>\dfrac{\pi}{2}$より
\begin{eqnarray}
\theta_R\to\pi (R\to\infty)\label{eq:limtheta}
\end{eqnarray}
余弦定理より
\[x^2=a^2+b^2-2ab\cos{\theta_R}\to (a+b)^2 (R\to\infty)\]
だから, $a+b>0, x>0$ より 
\begin{eqnarray}
x\to a+b (R\to\infty)\label{eq:limx}
\end{eqnarray}
ふたたび余弦定理, (\ref{eq:sinthm})により
\begin{align*}
 R^2(a+b-x)&=R^2\cdot\dfrac{(a+b)^2-x^2}{a+b+x}\\
 &=R^2\cdot\dfrac{2ab(1+\cos{\theta_R})}{a+b+x}\\
 &= R^2\cdot\dfrac{2ab}{(a+b+x)(1-\cos{\theta_R})}\sin^2{\theta_R}\\
 &= R^2\dfrac{2ab}{(a+b+x)(1-\cos{\theta_R})}\left(\dfrac{x}{2R}\right)^2\\
 &= \dfrac{2abx^2}{4(a+b+x)(1-\cos{\theta_R})}
\end{align*}
(\ref{eq:limtheta}), (\ref{eq:limx})より
\[\disp\lim_{R\to\infty}R^2(a+b-x)=\dfrac{2ab(a+b)^2}{4(2a+2b)(1-(-1))}=\dfrac{ab(a+b)}{8}\]

\syoumon{2}
まず
\[R^2(a+b+c-y)=R^2(a+b-x)+R^2(x+c-y)\]
であるから $R^2(x+c-y)$の極限を見れば求まる。$\angle{\mr{ACD}}=\phi_R$とおく。三角不等式, (\ref{eq:triangle})により
\begin{eqnarray}
y<x+c<a+b+c\label{eq:triangle2}
\end{eqnarray}
正弦定理, (\ref{eq:triangle2})より
\begin{eqnarray}
\sin{\phi_R}=\dfrac{y}{2R}<\dfrac{a+b+c}{2R}\to 0 (R\to \infty)\label{eq:sinthm2}
\end{eqnarray}
である。$\phi_R>\dfrac{\pi}{2}$だから
\begin{eqnarray}
\phi_R\to\pi (R\to\infty)\label{eq:limphi}
\end{eqnarray}
余弦定理,  (\ref{eq:limx}), (\ref{eq:limphi})より
\[y^2=(x+c)^2-2xc(1+\cos{\phi_R})\to (a+b+c)^2 (R\to\infty)\]
だから, $a+b+c>0, y>0$ より 
\begin{eqnarray}
y\to a+b+c (R\to\infty)\label{eq:limy}
\end{eqnarray}
ふたたび余弦定理, (\ref{eq:sinthm2})により
\begin{align*}
 R^2(x+c-y)&=R^2\cdot\dfrac{(x+c)^2-y^2}{x+c+y}\\
 &=R^2\cdot\dfrac{2xc(1+\cos{\phi_R})}{x+c+y}\\
 &=R^2\cdot\dfrac{2xc}{(x+c+y)(1-\cos{\phi_R})}\sin^2{\phi_R}\\
 &=R^2\dfrac{2xc}{(x+c+y)(1-\cos{\phi_R})}\left(\dfrac{y}{2R}\right)^2\\
 &=\dfrac{2xcy^2}{4(x+c+y)(1-\cos{\phi_R})}
\end{align*}
(\ref{eq:limx}),(\ref{eq:sinthm2}),(\ref{eq:limphi}),(\ref{eq:limy})より
\begin{align*}
 \lim_{R\to\infty}R^2(x+c-y)&=\dfrac{2(a+b)c(a+b+c)^2}{4(2a+2b+2c)(1-(-1))}\\
 &=\dfrac{c(a+b)(a+b+c)}{8}
\end{align*}
よって, 本問の(1)より
\begin{align*}
 \lim_{R\to\infty}R^2(a+b-x)&+\disp\lim_{R\to\infty}R^2(x+c-y)\\
 &=\disp\lim_{R\to\infty}R^2(a+b+c-y)\\
 &=\dfrac{ab(a+b)}{8}+\dfrac{c(a+b)(a+b+c)}{8}\\
 &=\dfrac{(ab+bc+ca)(a+b+c)}{8}
\end{align*}
