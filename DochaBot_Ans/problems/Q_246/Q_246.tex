\begin{thm}{246}{\hosi 5}{自作、学コン2020-10-5}
 四面体$\mr{OABC}$は、$\mr{OA}=1$, $\mr{OB}=\mr{OC}=2$ を満たし、面$\mr{ABC}$は正三角形であるとする。
 \begin{enumerate}
  \item 正三角形$\mr{ABC}$の一辺の長さ$x$の取り得る値の範囲を求めよ。
  \item 四面体$\mr{OABC}$の体積の最大値を求めよ。
 \end{enumerate}
\end{thm}

\syoumon{1}
簡単のため$x=2d$とおく。まず, 面$\mr{OAB}$の三辺は$1,2,2d$であり, この三辺について三角不等式が成り立つので 
\begin{align*}
 \left\{
 \begin{aligned}
  1&<2+2d \\
  2&<1+2d \\
  2d&<1+2
 \end{aligned}
 \right. \quad\dou\quad \frac{1}{2}<d<\frac{3}{2} \quad\dou\quad 1<x<3 \quad \cdots\marunum{1}
\end{align*}
が必要。また, この範囲であれば面$\mr{OAC}, \mr{OBC}$に関しても三角不等式が成り立つ。

次に, $\mr{BC}$の中点を$\mr{M}$とすると, $\mr{BM}=d, \mr{OM}=\sqrt{4-d^{2}}, \mr{AM}=\sqrt{3}d$である。点$\mr{A}$, 点$\mr{O}$はともに, $\mr{M}$を通る$\mr{BC}$に垂直な平面$p$上にある。一辺が$x=2d$のときにこの四面体が成り立つとするなら, $p$上で三角形$\mr{OAB}$が成り立つので, 三角不等式により
\begin{align*}
 \frac{1}{2}<d<\frac{3}{2} \,\text{かつ}\,\left\{
 \begin{aligned}
  \sqrt{3}d&<1+\sqrt{4-d^2} \\
  1&<\sqrt{3}d+\sqrt{4-d^2} \\
  \sqrt{4-d^2}&<\sqrt{3}d+1
 \end{aligned}
 \right.
\end{align*}
を満たすことが必要。逆にこれを満たせば, 先に1辺$2d$の正三角形$\mr{ABC}$を与えて, $p$上の点$\mr{O}$であって, $\mr{OM}=\sqrt{4-d^{2}}$かつ$\mr{AM}=\sqrt{3}d$, したがって $\mr{OA}=1, \mr{OB}=\mr{OC}=2$を満たすようなものを取ることができる。

そこでこの不等式を解く。1式目と2式目は$(1-\sqrt{3}d)^{2} < 4-d^{2}$を考えることに同値。これを整理すると
\begin{align*}
 & 4d^2-2\sqrt{3}d-3=x^2-\sqrt{3}x-3 <0 \\
 \dou\quad &\frac{\sqrt{3}-\sqrt{15}}{2}<x<\frac{\sqrt{3}+\sqrt{15}}{2} \quad\cdots\marunum{2}
\end{align*}
となる。3式目は, $\dfrac{1}{2}< d < \dfrac{3}{2}$のもとで両辺が正であるから, 二乗しても同値であり
\begin{align*}
 & 4-d^2<(\sqrt{3}d+1)^2 \quad\dou\quad x^2+\sqrt{3}x-x>0 \\
 \dou\quad & x<\frac{-\sqrt{3}-\sqrt{15}}{2} \,\text{or}\, \frac{-\sqrt{3}+\sqrt{15}}{2} < x \quad\cdots\marunum{3}
\end{align*}
である。以上\marunum{1},\marunum{2},\marunum{3}を満たせば良いので, 求める範囲は
\[ \dfrac{\sqrt{15}-\sqrt{3}}{2} < x  < \dfrac{\sqrt{15}+\sqrt{3}}{2} \]

\syoumon{2}
$\angle{\mr{OAM}}=\theta$とおく。余弦定理により, 
\[\cos{\theta} = \dfrac{1^{2}+(\sqrt{3}d)^{2} - (\sqrt{4-d^{2}})^{2}}{2\sqrt{3}d} = \dfrac{4d^{2} - 3}{2\sqrt{3}d}\]
なので, 
\[ \sin^{2}{\theta} = 1- \left(\dfrac{4d^{2} - 3}{2\sqrt{3}d}\right)^{2} =\dfrac{1}{12d^{2}}\cdot (-16d^{4} + 36d^{2} - 9) \]
であり, $0<\theta<\pi$なので$\sin{\theta}>0$であるから, 
\[\sin{\theta} = \dfrac{1}{2\sqrt{3}d}\sqrt{-16d^{4} + 36d^{2} - 9}\]
となる。$\mr{OA}=1$であるから, $\triangle{}\mr{OAM}$の$\mr{AM}$を底辺としたときの高さが$\sin{\theta}$である。$p$は$\mr{O}$を通り, 平面$\mr{ABC}$に垂直であるから, この高さは四面体の$\mr{ABC}$を底面とみたときの高さでもある。面$\mr{ABC}$の面積は$\sqrt{3}d^{2}$であるから, $\mr{AB}=x$のときの四面体の体積を$V(x)$とおくと
\begin{align*}
 V(x)&=(\sqrt{3}d^2)\times \frac{1}{2\sqrt{3}d}\sqrt{-16d^4+36d^2-9}\times\frac{1}{3} \\
 &= \frac{1}{6}\sqrt{d^2(-16d^4+36d^2-9)}=\frac{1}{12}\sqrt{x^2(-x^4+9x^2-9)}
\end{align*}
$x^{2} = X$とおいて, $V(x) = \dfrac{1}{12}\sqrt{-X^{3} + 9X^{2} - 9X} $となる。

$W(X)=-X^{3} + 9X^{2} - 9X $とおくと, 
\[W'(X) =-3(X^{2} - 6X +3) = -3\left( X-(3-\sqrt{6}) \right)\left(X- (3+\sqrt{6})\right)\]
となる。問(1)により, $X$の取る範囲は$\dfrac{9-3\sqrt{5}}{2} < X < \dfrac{9+3\sqrt{5}}{2}$であって,数の大小に注意すると, 次の増減表を得る:
\begin{center}
$
\begin{array}{|c|*5{c|}}\hline
  X & \dfrac{9-3\sqrt{5}}{2} & \cdots  & 3+\sqrt{6} & \cdots  & \dfrac{9+3\sqrt{5}}{2} \\ \hline
  W'(x)&       &  + &   0    & - &      \\ \hline
  W(x) &    & \nea &  \mbox{極大}   &\sea &   \\ \hline
\end{array}
$
\end{center}
よって$V(x)=\dfrac{1}{12}\sqrt{W(X)}$も$X=3+\sqrt{6}$で最大となるので, 
\begin{align*}
 &V(x)\text{の最大値} \\
 =& \frac{1}{12}\sqrt{W(3+\sqrt{6})} \\
 =& \dfrac{1}{12}\sqrt{ -(3+\sqrt{6})\left\{ -(3+\sqrt{6})^{2} + 9(3+\sqrt{6}) - 9 \right\} }\\
 =& \dfrac{1}{12}\sqrt{-(3+\sqrt{6})\left\{3+3\sqrt{6} \right\}}\\
 =& \frac{\sqrt{3}}{12}\sqrt{9+4\sqrt{6}}
\end{align*}
