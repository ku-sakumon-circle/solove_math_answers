\begin{thm}{263}{\hosi ?}{}
10進法の有限列を任意に与える. このとき, $2^n$の10進表記がその列から始まるような自然数$n$が存在することを示せ. 
\end{thm}

10進法の任意の有限列$a_1a_2\dots a_r$ ($1\leq a_i \leq 9$, $a_1\neq 0$)を考える. $2^n$の10進表記がこの列から始まるための必要十分条件は, $n$に対してある自然数$m(n)$が存在して
\[10^{m(n)}\times \ovl{a_1a_2\dots a_r} \leq 2^n < 10^{m(n)}\times (\ovl{a_1\dots a_r} + 1)\]
となることである. ただし上線はこの列を数字とみなす記号である. この式は
\[\ovl{a_1\dots a_r} \leq 2^n\cdot 10^{-m(n)} < (\ovl{a_1\dots a_r} + 1)\]
と同値で, $\log_{10}(\cdot )$を取り
\[\log_{10}{(\ovl{a_1\dots a_r})}  \leq n\log_{10}{2}-m(n)  < \log_{10}{(\ovl{a_1\dots a_r} + 1)}\]
である. $a= \log_{10}{(\ovl{a_1\dots a_r})}$, $b= \log_{10}{(\ovl{a_1\dots a_r} + 1)}$として, 示すべきことは, ある自然数$m,n$が存在して
\[a\leq n\log_{10}{2} - m < b\]
を満たすということである. ここで $b-a > 0$であるが, $0 < \delta < b-a$を満たすような正数$\delta$を一つ取る. このとき, $\log_{2}{10}$が無理数であるから, Kroneckerの稠密定理によって, ある自然数$k$が存在して$k\log_{10}{2}$の小数部分は$\delta$未満である. つまり, $l = [k\log_{10}{2}]$としたときに$0<k\log_{10}{2} - l< \delta$を満たす. $\epsilon = k\log_{10}{2} - l$とおく. このとき$\epsilon$の整数倍全体の集合$S:=\set{N\epsilon \mid N\in \mb{Z}}$は, 数直線上で幅$\epsilon$を空けながら並ぶので, 幅$\delta (> \epsilon)$ である開区間$(a,b)$上には必ず$S$の点が少なくとも一つ含まれる. それを$N_0\epsilon$とすれば, $N_0\epsilon \in (a,b)$である. すなわち, 
\[a < N_0k \log_{10}{2} - N_0l < b\]
である. よって$n=N_0k$, $m=N_0l$を構成すればよい. \qed 
