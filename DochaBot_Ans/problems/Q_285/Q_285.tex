\begin{thm}{285}{\hosi 4}{V.A. Lebesgue's Proof}
$x,y$を整数とし, 方程式
$$
y^2 = x^3 + 7\qquad\qquad (\star)
$$
を考える. 
\begin{enumerate}
\item $x$は偶数ではないことを示せ.
\item $x^2 - 2x + 4$を割り切る素数$p$であって, $4$で割って$3$余るものが存在することを示せ. 
\item $(\star)$ に整数解$(x,y)$は存在しないことを示せ. 
\end{enumerate}
\end{thm}

\syoumon{1}
$x$が偶数だと$y^2\equiv 3\pmod{4}$なので不合理($3\bmod{4}$は非平方剰余)

\syoumon{2}
$y^2 + 1 = (x^3 + 8) = (x^2 - 2x + 4)(x+2)$ で$x$が奇数であることより
$$
x^2 - 2x + 4 \equiv 1 - 2 + 0 \equiv 3\pmod{4}
$$
である. もし$x^2 - 2x + 4$を割り切る素数$p$がすべて$p\equiv 1\pmod{4}$であれば, その積として書けるはずの$x^2 - 2x + 4$自身が$4$で割って$1$余るはずなので矛盾. よって存在する. 

\syoumon{3} 
(2)のような$p$をとると $y^2 \equiv -1\pmod{p}$であるが, これは平方剰余第一補充則に反する. 
