\begin{thm}{274}{\hosi 6}{}
$n$は正の整数とする. $X^2-2$が$7^n$で割り切れるような整数$X$が存在することを証明せよ. 
\end{thm}

数学的帰納法により証明する. $n=1$のとき, $X=3$とすればよい. 次に, ある$k\geq 1$で$X_k^2 - 2$が$7^k$で割り切れるような整数$X_k$が存在すると仮定し, $X^2-2$が$7^{k+1}$で割り切れるような整数$X$が存在することを証明する. \par 
いま$N = 7^k q + r$と表される整数$N$に対して$N^2-2$を計算すると
\[
(7^k q+r)^2 - 2= 7^{2k} q^2 + (2\cdot 7^{k} qr + r^2 -2)
\]
となる. これより, $\bmod{7^{k+1}}$において
\begin{align}\label{274:eq-1}
N^2 - 2 \equiv 0 &\iff  2\cdot 7^k qr + r^2 - 2 \equiv 0 \\ \nonumber  
			&\iff 8\cdot 7^{k}qr + 4(r^2 - 2) \equiv 0 \\ \nonumber 
			&\iff 7^{k}qr + 4(r^2 - 2) \equiv 0 
\end{align} 
であるので, 最後の合同式をみたすような$q,r$が構成できればよい. \par
(最後の合同式を満足するためには$r^2-2$が$7^{k}$の倍数であることが必要であるので) \textbf{$r = X_k$と置いてみるのがよいと予測される.} このとき$r^2 - 2 = X_k^2 - 2 = 7^{k}s$ ($s$は整数)と表せる. これを代入すると
\[
7^{k} qr + 4(r^2 - 2) = 7^k(qX_k + 4s)
\]
であるので, $qX_k + 4s$を$7$の倍数にするような$q$を取るとよい. いま$X_{k}^2 - 2$が$7$の倍数であることから, $X_k$と$7$は互いに素である. よって, $X_kY_k \equiv 1\pmod{7}$となるような整数$Y_k$が必ず取れる\footnote{$X_k$と$7$が互いに素なら, $aX_k + 7b = 1$となる整数$a,b$があるので, ここに現れた$a$を$Y_k$とおくとよい. }. このとき
\[q = -4sY_k = -4\cdot \dfrac{X_k^2-2}{7^k}Y_k\]
と置くことで
\[
qX_k + 4s \equiv  -4s(X_kY_k) + 4s \equiv -4s + 4s \equiv 0\pmod{7}.
\]
以上で$q,r$は構成された. 結局,  
\[
N = 7^kq + r,\qquad q = -4\cdot \dfrac{X_k^2-2}{7^k}Y_k,\qquad r = X_k
\]
とおけば, 同値性(\ref{274:eq-1})によって$N^2 - 2\equiv 0\pmod{7^{k+1}}$を得るので$n=k+1$の場合は$X$としてこの$N$を取ればよい. \par 
以上より, 数学的帰納法によってすべての正の整数$n$に対して主張は正しい. 




\begin{supple}
本問はHenselの補題と呼ばれる有名な定理の特別なケースであり, 類題はたくさん構成できる. 「7進整数環」と呼ばれる$\mb{Z}$を拡張した世界$\mb{Z}_{7}$というものがあり, 本問は別の言い方をすれば「方程式$X^2 = 2$は$\mb{Z}_{7}$において解を持つ」ということである. 詳しくは雪江明彦氏の整数論1やNeukirchの代数的整数論を参照されたい. 
\end{supple}






