\begin{thm}{282}{\hosi 7◎}{自作, 学コン2022-10-6}
原点Oの$xy$平面上の点$A_n$を以下の操作で定める. 
\begin{itemize}
\item $A_0(1,0)$
\item $A_1(\cos{\frac{2\pi}{N}}, \sin{\frac{2\pi}{N}} )$
\item $n\geq 2$に対し, $A_n$は直線$OA_{n-2}$ に関して$A_{n-1}$と対称な点
\end{itemize}
\begin{enumerate}
\item $A_n$の$x$座標$x_n$を$n$で表せ. 
\item 数列$\set{x_n}$が収束するような$N$の条件を求めよ. 
\end{enumerate}
\end{thm}


\syoumon{1} 
座標平面を複素平面に対応させて$\mr{A}_n$に対応する点を$z_n$とおく. $z_0=1$, $z_1 = \cos{\dfrac{2\pi}{N}} + i\sin{\dfrac{2\pi}{N}}$である.  $n\geq 2$において, $z_{n}z_{n-2}^{-1}$は$z_{n-1}z_{n-2}^{-1}$の共役複素数である. $z_0,z_1$はいずれも単位円$|z|=1$上にあり, $z_{n-2},z_{n-1}$から$z_n$を得る操作において$|z_{n-1}|=|z_n|$であることに注意すると, $|z_n|=1$が従う. そこで$z_n = \cos{\dfrac{2\pi}{N}a_n} + i\sin{\dfrac{2\pi}{N}a_n}$と表せるような実数$a_n$を取る. $z_n = \ovl{z_{n-1}z_{n-2}^{-1}}z_{n-2} = z_{n-2}^{2}z_{n-1}^{-1}$だから($|z|=1\implies \ovl{z} = z^{-1}$に注意), ド・モアブルの定理より
\[\dfrac{2\pi}{N}a_{n} \equiv \dfrac{2\pi}{N}(2a_{n-2} - a_{n-1})\pmod{2\pi}\]
が従う. すなわち
\[\dfrac{2\pi}{N}a_n = \dfrac{2\pi}{N}(-a_{n-1} + 2a_{n-2}) + 2\pi k\quad (kは整数)\]
が成り立つので, $a_{n} \equiv -a_{n-1} + 2a_{n-2} + Nk$が成り立つ.
よって$a_n \equiv -a_{n-1} + 2a_{n-2} \pmod{N}$である. 漸化式$b_{n} = -b_{n-1} + 2b_{n-2}$, $b_0=0, b_1=1$を解くと$b_n = \dfrac{1-(-2)^n}{3}$であるから, $a_n = \dfrac{1}{3}(1-(-2)^n)$と置いてよい. したがって, 
\[\bolm{x_n = \cos{\parena{\dfrac{2\pi}{N}\cdot \dfrac{1-(-2)^n}{3}}}}\]
%A+B(-2)^n 
%A+B = 0, A-2B = 1 → A=1/3, B=-1/3


\syoumon{2}
$x_n$の取り得る値は高々, 有限集合$\set{\cos{\dfrac{2\pi}{N}k}\mid k=0,1,\dots, N-1}$の中にしかないことに注意する. 
よって, この数列が収束するなら, ある整数$0\leq k < N$が存在して, $n\geq R$ ($R>0$は十分大きい整数)では常に$x_n = \cos{\dfrac{2\pi}{N}k} = \cos{\dfrac{2\pi}{N}(N-k)}$である. よって, そのような$n$からは $z_n$の偏角は$\dfrac{2\pi}{N}k$または$\dfrac{2\pi}{N}(N-k)$である. つまり, $\alpha = \cos{\dfrac{2\pi}{N}k} + i\sin{\dfrac{2\pi}{N}}$, $\beta = \alpha^{-1} = \ovl{\alpha}$とするとき$z_n \in \set{\alpha,\beta}$である($n\geq R$). \par 
ところが, 実際には次が成り立つ. 
 
\begin{claim} 
$x_n$が収束するとき, 十分大きい$n$から$z_n$は一定である. 
\end{claim} 
\begin{proof} $\alpha = \beta$の場合は示すことは何もない. $\alpha\neq \beta$の場合を考える. この場合$\sin{\dfrac{2k\pi}{N}} \neq 0$である. よって($0\leq k<N$より) $k \neq \dfrac{N}{2}$である. \par 
いったん$z_{n} = z_{n+1}$になると, 漸化式から帰納的に$z_{n+2} = z_{n+1}$にもなるので$z_n$は一定になる. よって, 一定にならない場合, 十分大きい$n$から常に$z_n\neq z_{n+1}$, $z_{n+1} \neq z_{n+2}$, $z_n = z_{n+2}$などが成り立つことになる. そのような$n$からは$z_n^{-1} = z_{n+1}$なので, $z_{n+2} = z_n^2 z_{n+1}^{-1} = z_n^3$となる. $z_n = z_{n+2}$であったから, $z_n = z_n^3$となる. よって$1 = z_n^2$だから$\mr{arg}\parena{z_n^2} \equiv \pm \dfrac{4k}{N}\pi \equiv 0 \pmod{2\pi}$である. したがって$\dfrac{2k}{N}$は整数である. $0\leq \dfrac{2k}{N} < 2$だから$\dfrac{2k}{N}=1$. これは$k\neq \dfrac{N}{2}$に反する. よって示せた. 
\end{proof}

また, $z_n$が一定であるなら明らかに$x_n$は収束している. 以上より, 
\begin{align*}
    &\set{x_n}_n \text{が収束する} \\
    &\iff \text{十分大きい$n$からは常に$z_{n+1} = z_n$} \\ 
    &\iff \text{十分大きい$n$からは常に$\dfrac{1-(-2)^n}{3} \equiv \dfrac{1-(-2)^{n+1}}{3}\pmod{N}$}
\end{align*}
となる. 合同式の条件については
\begin{align*}
\dfrac{1-(-2)^n}{3} &\equiv \dfrac{1-(-2)^{n+1}}{3}\pmod{N} \\
&\iff 0\equiv (-2)^n \pmod{N} \iff 2^n \equiv 0\pmod{N}
\end{align*}
であるので, 
\begin{align*}
    &\set{x_n}\text{が収束する} \\
    &\iff \text{十分大きい$n$からは常に$2^n$が$N$の倍数である} \\ 
    &\iff \text{$N$は$2$の冪乗の正の約数である} \\ 
    &\iff \bolm{N=2^{s} \quad (s=0,1,2,\dots)}
\end{align*}

\begin{supple}
この問の正答率は19%でした. (2)の収束の議論は雑にできない. 
\end{supple}