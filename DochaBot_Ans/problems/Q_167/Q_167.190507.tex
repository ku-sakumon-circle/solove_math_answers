\begin{thm}{167}{\hosi 5}{}
 円$O$の半径を111111111、円$P$の半径を11111とする。$O$の円周上に点$A$が、$P$の円周上に点$B$が、それぞれ自由に動くものとする。線分$AB$の中点$M$の存在しうる範囲の面積を求めよ。 
\end{thm}

$\vvv{OA}=\vvv{a}$, $\vvv{PB}=\vvv{b}, \vvv{OP}=\vvv{p}$とする。このとき,
\[\vvv{OM}=\dfrac{\vvv{a}}{2}+\dfrac{\vvv{OB}}{2}=\dfrac{\vvv{a}+\vvv{b}}{2}+\dfrac{\vvv{p}}{2}\]
だから, OPの中点をNとすると, 
\[\vvv{OM}-\vvv{ON}=\vvv{NM}=\dfrac{\vvv{a}}{2}+\dfrac{\vvv{b}}{2}\]
である。つまり, Nを始点としたときの$\dfrac{\vvv{a}}{2}+\dfrac{\vvv{b}}{2}$の終点が通る存在範囲がMの存在範囲であり, $|\vvv{a}|=111111111, |\vvv{b}|=11111$を満たしながら自由に動けるから,存在領域は, 半径$r$が
\[\dfrac{111111111}{2}-\dfrac{11111}{2}\leq r\leq \dfrac{111111111}{2}+\dfrac{11111}{2}\]
の範囲の円環領域(輪っか)になっている。よって, 大きい円の面積から小さい円の面積を引くことで
\begin{align*}
&\pi\left(\dfrac{111111111}{2}+\dfrac{11111}{2}\right)^2 -\pi\left(\dfrac{111111111}{2}-\dfrac{11111}{2}\right)^2\\
=&2\times \dfrac{111111111\times 11111}{2} \pi\\
=& 111111111\times 11111\times \pi=1234555554321\pi
\end{align*}