\begin{thm}{212}{\hosi 4}{大阪大 (1995)}
 どのような実数$x$に対しても、不等式
 \[ |x^3+ax^2+bx+c| \le |x^3| \]
 が成立するような実数$a, b, c$の値を求めよ。
\end{thm}

$x=0$とすることで、$|c|\le 0$を満たすから$c=0$。このとき不等式は、
\[ |x|\cdot|x^2+ax+b| \le |x|^3 \]
だから、$x\neq 0$に対して
\[ |x^2+ax+b|\le |x|^2 \quad\dou\quad \left|1+\frac{a}{x}+\frac{b}{x^2}\right| \le 1 \]
となる。$\dfrac{1}{x}=t$とおいたとき、$t$は0でない実数を走るので、任意の0でない実数$t$に対して
\[ |1+at+bt^2| \le 1 \]
が成立する。$b\neq 0$だと、$t$を十分大きくとれば、$1+at+bt^2$は発散させられるため不適。$b=0$とすると、$|1+at|\le 1$が常に成り立つためには、同様の理由から$a=0$でなければならない。よって$a=b=c=0$であるが、このとき確かに
\[ |x^3+ax^2+bx+c|=|x^3|\le |x^3| \]
となっているのでよい。よって$(a, b, c)=(0, 0, 0)$。