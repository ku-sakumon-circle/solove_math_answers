\begin{thm}{040}{\hosi 8}{学コン 2018-6-6}
 一辺の長さが$\dfrac{1}{n}$の正$2n$角形 $\mr{P}_0\mr{P}_1\dots \mr{P}_{2n-1}$ がある。辺$\mr{P}_0\mr{P}_1$上に点S、辺$\mr{P}_0\mr{P}_{2n-1}$上に点Tを、$\mr{P}_0\mr{S}=\mr{P}_0\mr{T}$となるようにとり、線分$\mr{S}\mr{P}_n$上に点Uを $\angle\mr{SUT}=\angle\mr{P}_1\mr{P}_n\mr{P}_{2n-1}$ となるようにとる。ただし、$\mr{S}=\mr{P}_0$のときは$\mr{U}=\mr{P}_0$とする。Sを$\mr{P}_0$から$\mr{P}_1$まで動かすとき、Uの軌跡の長さを$U(n)$とする。
 \begin{enumerate}
  \item $U(3)$ を求めよ。
  \item $\disp \lim_{n\to\infty} U(n)$ を求めよ。
 \end{enumerate}
\end{thm}

ここに解答を記述。