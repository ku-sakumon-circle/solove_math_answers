\begin{thm}{090}{\hosi ?}{}
 \begin{enumerate}
  \item 正の整数$n$を9で割った余りは、$n$の十進法における各桁の総和を9で割った余りに等しいことを示せ。
  \item $2^{29}$は、9桁の正の整数であり、すべての桁が異なっている。実際に計算することなく、0以上9以下の整数のうち、$2^{29}$の桁には入っていない数字を求めよ。
 \end{enumerate}
\end{thm}

\syoumon{1}
$n$の10進法表示は、
\[ n=\sum_{k=0}^N a_k 10^k \quad(\text{ただし}\, N\ge 0 \,,\,\, a_k\in\{0, 1, \dots, 9\}) \]
であるが、これの$\pmod{9}$を考えると、
\[ n\equiv \sum_{k=0}^N a_k(10-9)^k=\sum_{k=0}^N a_k \,\,\pmod{9} \]
となり、右辺は桁の数字の総和であるから題意は示された。

\syoumon{2}
入っていない数字を$a$とおく。$a\in\{0, 1, \dots, 9\}$である。このとき$2^{29}$の桁の総和は、
\[ 0+1+2+\dots+8+9-a=45-a \]
である。一方で、
\[ 2^{29}=8^9\cdot2^2\equiv (-1)^9\cdot 4\equiv 5 \pmod{9} \]
であるから、$45-a\equiv 5 \pmod{9}$でなければならない。これを解くと$a\equiv 4 \pmod{9}$となり、$a\in\{0, 1, \dots, 9\}$であるから、$a=4$である。