\begin{thm}{189}{\hosi 3}{大阪大 (2002)}
 実数を係数とする3次方程式 $x^3+ax^2+bx+c=0$ が異なる3つの実数解を持つとする。$a>0$, $b>0$ならば、少なくとも2つの解は負であることを示せ。
\end{thm}

\begin{proof}
3つの実数解を 小さい順に$\alpha, \beta ,\gamma$ とおく。
\[f(x)=x^3+ax^2+bx+c\]
とする。平均値の定理より, 
\[\dfrac{f(\beta) -f(\alpha)}{\beta -\alpha}=0=f'(t) \]
\[\dfrac{f(\gamma) -f(\beta)}{\beta -\alpha}=0=f'(s) \]
を満たす実数$t\in [\alpha, \beta]$, $s\in [\beta , \gamma]$ が存在する。\footnote{グラフより・・・とすればたいへん明らかなことだが, このように平均値の定理を用いた方が丁寧かと思う。}
$s,t$は方程式 $3x^2 + 2ax +b=0$の解であるから, 解と係数の関係より
\[s+t=-\dfrac{2a}{3}<0,  st=\dfrac{b}{3}>0\]
である。二式目より$s, t$は符号が同じであるが, $s,t>0$であると1式目に矛盾する。よって $s<0 , t<0$であり, 
\[\alpha< s< \beta < t<0\]
より$\alpha,\beta <0$なので題意は示された。%\qed
\end{proof}