\begin{thm}{179}{\hosi 8}{自作 DMO3rd 1}
 平面上に相異なる3点$A, B, C$があり、$|\vvv{AB}|$, $|\vvv{BC}|$ の値は素数である。$\vvv{AB}\cdot\vvv{AC}$, $\vvv{BA}\cdot\vvv{BC}$, $\vvv{CA}\cdot\vvv{CB}$ がこの順に等比数列をなし、$|\vvv{AC}|^3+6$, $|\vvv{AC}|^3-6$ が素数になるとき、$|\vvv{AB}|$, $|\vvv{BC}|$, $|\vvv{CA}|$ の値を求めよ。
\end{thm}

ここに解答を記述。