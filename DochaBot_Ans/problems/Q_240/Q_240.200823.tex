\begin{thm}{240}{\hosi ?}{}
 定規とコンパスによって後に示す操作(a), (b)を有限回行うことだけが許されている。座標平面内の2点$(0,0)$, $(1,0)$を始めに与え、有限回の操作の組み合わせから得られる平面上の点全体を$A$とする。このとき、$(\cos\dfrac{2}{17}\pi,0)\in A$を示せ。また、正17角形が有限回の操作(a), (b)だけで得られることを示せ。

 \begin{itemize}
  \setlength{\leftskip}{3eM}
  \item[操作(a)] 与えられた2点を結ぶ直線を描く
  \item[操作(b)] 与えられた点を中心とし、与えられた長さを半径とする円を描く
 \end{itemize}

 なお、$\cos\dfrac{2}{17}\pi$が次に示す値であることは認めてよい。
 \[ \!\!\frac{1}{16}\!\!\left(\!\sqrt{17}-1\!+\!\sqrt{34-2\sqrt{17}}+2\sqrt{17\!+\!3\sqrt{17}\!-\!\sqrt{170\!+\!38\sqrt{17}}}\,\right) \]
\end{thm}

まず、次の補題240.1を設定する。
\begin{subthm}{240.1}
 \begin{enumerate}
  \item 有理数$a$に対し、$(a,0)\in A$
  \item $(a,0), (b,0) \in A$ ならば、$(a\pm b, 0) \in A$
  \item $(a,0)\in A$ ならば $(\sqrt{|a|},0) \in A$
 \end{enumerate}
\end{subthm}
(1): 題意より$(0,0), (1,0) \in A$である。針を原点においてコンパスを使うことで$(a,0)\in A$なら$(-a,0) \in A$でもあるから、$a>0$の場合を考えればよい。自然数$m,n$によって$a=\dfrac{m}{n}$と表す。コンパスによって適当な自然数倍の長さを持った線分を構成することができるので、$(\frac{1}{n},0)\in A$を示せば$(a,0)\in A$も従う。さて原点から$(1,1)$に向かう半直線を考え、原点と$(1,1)$にコンパスを合わせて繰り返し使うことで点$(n,n)$を作図できる。$(n,n)$と$(1,0)$を結ぶ直線を$l_n$とすると、点$(1,1)$を通り$l_n$に平行な直線を定規とコンパスで作図できる。この直線と$x$軸との交点が$(\frac{1}{n},0)$である。よって$(\frac{1}{n},0)\in A$である。

(2): $a,b>0$としてよく、コンパスを使えば明らかに作図できる。

(3): $a>0$としてよい。(1),(2)より、十分大きい$n$をとれば$a-\dfrac{1}{4n^2}>0$かつ $(a\pm\frac{1}{4n^2},0) \in A$である。$a+\dfrac{1}{4n^2}$を斜辺に、$a-\dfrac{1}{4n^2}$を高さにもつ直角三角形が作図でき、その残りの辺の長さは$\dfrac{\sqrt{a}}{n}$であるから、これを$n$倍にすることで$\sqrt{a}$の長さの線分が作図できる。よって$(\sqrt{a},0) \in A$。

この補題を繰り返し用いることで、与えられた$\cos\dfrac{2}{17}\pi$について$(\cos\dfrac{2}{17}\pi,0) \in A$であることがわかる。