\begin{thm}{253}{\hosi 8\maru}{自作 学コン2020-11-5 原題}
 $n$, $n+2$, $n+4$ の最小公倍数を求めよ.
\end{thm}

自然数$N$が素数$p$で割り切れる回数を$v_{p}(N)$とおく. 求める最小公倍数を$L_n$とする. 一般に, $p_1,p_2,\dots$を素数の小さい順に並べた列だとして, $n$個の(素因数分解表示された)自然数
\[
p_{1}^{e_{1j}}\cdot p_2^{e_{2j}} \cdot \dots \quad (j=1,2,\dots, n,\quad e_{ij} \geq 0)
\]
の最小公倍数は, 
\[p_1^{\max_{j} e_{1j}}\cdot p_2^{\max_{j} e_{2j}}\cdot \dots \]
で計算できる. \par 
まず, $n,n+2,n+4$のどの二つを見ても, その差は2か4である. よってユークリッドの互除法を考えれば, たとえば$n$と$n+2$は共通の奇素因数を持たない. $n+2,n+4$と$n,n+4$についても同様のことが従う. よって, 奇素数$p\geq 3$に対して次が正しい. 
\[\max{\set{v_{p}(n),v_{p}(n+2),v_{p}(n+4)}} = v_{p}(n(n+2)(n+4)).\]
つまり, \tf{奇素因数に限って言えば,} $L_n$は$n(n+2)(n+4)$と同じ素因数分解を持っている. よって
\[\dfrac{n(n+2)(n+4)}{L_n} = 2^{v_2(n(n+2)(n+4))- \max{\set{v_2(n), v_2(n+2), v_2(n+4)}}}\]
である. よってこの右辺の指数を$e(n)$とすれば, $L_n = \dfrac{n(n+2)(n+4)}{2^{e(n)}}$として求まる. \\
\step{1}{$n$が奇数のとき} 
$n,n+2,n+4$は奇数だから$v_2(n)$などはみな0である. よって$e(n) = 0$なので$L_n=n(n+2)(n+4)$. \\
\step{2}{$n$が偶数だが4で割れないとき} \\
このとき$n+2$のみが4の倍数で, $v_2(n) = v_2(n+4) = 1$である. よって$v_2(n(n+2)(n+4)) = 2+v_2(n+2)$なので
\[e(n) = \left\{ 2+v_2(n+2)\right\} - v_2(n+2) = 2\]
なので
\[L_n = \dfrac{n(n+2)(n+4)}{4}\]
\step{3}{$n$が4の倍数のとき} もし$n$が8の倍数ではないなら, $v_2(n) = 2$, $v_2(n+2) = 1$, $v_2(n+4) \geq 3$である. その場合は
\[e(n) = (3+v_2(n+4)) - v_2(n+4) = 3 \]
である. もし$n$が8の倍数であるなら, $v_2(n)\geq 3$, $v_2(n+2) = 1$, $v_2(n+4) = 2$なのでこの場合も$e(n) = 3$. よって
\[L_n = \dfrac{n(n+2)(n+4)}{8}\]

