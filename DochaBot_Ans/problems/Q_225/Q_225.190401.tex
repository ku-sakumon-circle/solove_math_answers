\begin{thm}{225}{(1) \hosi 7, (2) \hosi 11}{年賀状問題2019}
 $a, b, c, d$は整数とする。以下の問に答えよ。
 \begin{enumerate}
  \item $ac-5bd=2019$, $ad+bc=0$, $a\ge 0$を満たす$(a, b, c, d)$の組を全て求めよ。
  \item $ac+5bd=2019$, $ad+bc=0$を満たす$(a, b, c, d)$の組を全て求めよ。
 \end{enumerate}
\end{thm}

\syoumon{1}
与式は
\[(a+b\sqrt{-5})(c+d\sqrt{-5})=2019\]
が成り立つことに同値である。さらに, 次の式とも同値である。
\[(a-b\sqrt{-5})(c-d\sqrt{-5})=2019\]
この2式を辺々かけあわせれば
\[(a^2+5b^2)(c^2+5d^2)=2019^2=3^2\times 673^2\]
となる。ここから左辺の因数は$2019^2$の約数になることが必要条件である。この約数の候補をさらに限定するために次の補題を用意する。 
\begin{subthm}{225.1}
 $m=X^2+5Y^2$ ($X,Y\in \mb{Z}$) とあらわされる整数$m$に対して, 次が成立する。\\
 \begin{enumerate}[label=\roman*,align=CenterWithParen]
  \item $m\geq 0$
  \item $m\equiv 0, 1, 2 \pmod{4}$
  \item $m\equiv 0, 1, 4 \pmod{5}$
  \item $m$が673で割れるならば、$673^2$でも割れる。
 \end{enumerate}
\end{subthm}
(i)は自明。(ii), (iii)は平方剰余を列挙すればやさしい。(iv)を証明する。\\
$X$が673で割れないと仮定する。このとき,$Y$も673で割れないので
,\[1=\left(\dfrac{X^2}{673}\right)=\left(\dfrac{-5Y^2}{673}\right)=\left(\dfrac{-5}{673}\right)\]
となる。しかし, 相互法則と第一補充則により
\[ \left(\dfrac{-5}{673}\right)=(-1)^{336}\left(\dfrac{5}{673}\right)=(-1)^{2\cdot 336}\left(\dfrac{673}{5}\right)=-1 \]
より矛盾する。($\because p\equiv 2,3$)\\
よって, $X$は673で割り切れ, $Y$も673で割り切れるので $X^2-5Y^2$は$673^2$で割り切れる。\qed\\
この補題225.1により, $a^2+5b^2=1,3^2,673^2,2019^2$にまで絞られる。

$c+d\sqrt{5}=\dfrac{2019a}{a^2+5b^2}+\dfrac{2019b}{a^2+5b^2}\sqrt{5}$より, $a,b$を右辺に代入して係数を比較すれば$(c,d)$が求まることに注意する。$a\geq 0$より $c\geq 0$であるとわかる。

$a^2+5b^2=1$のとき, $a\geq 0 $より$(a,b)=(1,0)$のみ。$(c,d)=(2019,0)$を得る。

$a^2+5b^2=9$のとき, $(a,b)=(2,\pm 1), (3,0)$で, 前者は $c=\dfrac{1346}{3}$で整数にならないので不適。後者のとき, $(c,d)=(673, 0)$

$a^2+5b^2=673^2, 2019^2$のときは $c^2+5d^2=9,1$であって, $c\geq 0$が分かっているから上の2つの場合と同様になる。つまり, $(a,b,c,d)=(673,0,3,0), (2019,0,1,0)$

以上より, 求めるすべての組は, $(a,b,c,d)=$
\[\mbox{{\boldmath  $(3,0,673,0),(673,0,3,0),(1,0,2019,0),(2019,0,1,0)$} }\]

\syoumon{2}
$\sqrt{5}$が無理数であることから, 与式は
\[(a+b\sqrt{5})(c+d\sqrt{5})=2019\]
が成り立つことに同値である。さらに, 次の式とも同値である。
\[(a-b\sqrt{5})(c-d\sqrt{5})=2019\]
この2式を辺々かけあわせれば
\[(a^2-5b^2)(c^2-5d^2)=2019^2=3^2\times 673^2\]
となる。ここから左辺の因数は$2019^2$の約数になることが必要条件である。この約数の候補をさらに限定するために次の補題を用意する。 (補題225.1の(iv)とほぼ同様である。)

\begin{subthm}{225.2.1}
 $X^2-5Y^2$が素数$p$で割り切れる ($p\equiv 2,3$ (mod $5$)) とき, $X,Y$はともに$p$で割り切れ, $X^2-5Y^2$は$p^2$で割り切れる。特に, $p=3,673$で主張が成り立つ。
\end{subthm}

{\bf 証明.}\\
$X$が$p$で割れないと仮定する。このとき,$Y$も$p$で割れず, Ledgendre記号により
\[1=\left(\dfrac{X^2}{p}\right)=\left(\dfrac{5Y^2}{p}\right)=\left(\dfrac{5}{p}\right)\]
となる。しかし, 相互法則により $\left(\dfrac{5}{p}\right)=(-1)^{2\cdot 336}\left(\dfrac{p}{5}\right)=-1$ より矛盾する。($\because p\equiv 2,3$)\\
よって, $X$は$p$で割り切れ, $Y$も$p$で割り切れるので $X^2-5Y^2$は$p^2$で割り切れる。\qed

補題225.2.1により, $a^2-5b^2=\pm 1, \pm 9, \pm 673^2, \pm 2019^2$ であることが必要になる。$a^2-5b^2=\pm m^2$ ($m=1,9,673,2019$) としたとき, 補題2.1から$a,b$はともに$m$の倍数であることが言えるため, $a=ma'$, $b=mb'$ として $a'^2-5b'^2=\pm 1$ となるので, 特に $a^2-5b^2=\pm 1$ の場合について考えることを目標にする。次の補題225.2.2を示す。

\begin{subthm}{225.2.2}
$S=\{X+Y\sqrt{5}\in\mb{Z}[\sqrt{5}] |  X^2-5Y^2=\pm 1, X+\sqrt{5}Y>0\}$とする。
 \begin{enumerate}[label=\roman*,align=CenterWithParen]
  \item $(2+\sqrt{5})^n=x_n+y_n\sqrt{5}$ とおくと, $(x_n, y_n)\in S$.
  \item $n$が奇数 $\dou$ $x_n^2-5y_n^2=-1$.
  \item 任意の整数$n$と任意の$S$の元$s$に対して, $s\times(2+\sqrt{5})^n \in S$
  \item 任意の$S$の元は, ある$n\in \mb{Z}$によって上の $x_n+y_n\sqrt{5}$の形になる。
 \end{enumerate}
\end{subthm}
{\bf 証明.}
(i) $f:\mb{Z}[\sqrt{5}]\to \mb{Z}$ を $f(x+y\sqrt{5})=x^2-5y^2$ で定義すると,$x,y,z,w\in \mb{Z}$に対して次が成立することが容易な計算によりわかる。
\[f(x+y\sqrt{5})f(z+w\sqrt{5})=f((xz+5yw)+(xw+yz)\sqrt{5})\]
これを用いて, 
\[f(x_n+y_n\sqrt{5})=x_n^2-5y_n^2=f(2+\sqrt{5})^n=(-1)^n\]
である。また, $x_n+y_n\sqrt{5}=(2+\sqrt{5})^n>0$である。以上から $(x_n, y_n)\in S$である。\\
$(-1)^n$の値を見れば(ii)は明らか。\\
\\
(iii)$s=X+Y\sqrt{5}\in S$とする。$s>0, 2+\sqrt{5}>0$より $s(2+\sqrt{5})^n>0$であり,
\[f(s(2+\sqrt{5})^n)=f(s)f(2+\sqrt{5})^n=1\cdot(-1)^n=(-1)^n\]
より明らか。\\
\\
(iv) $u=X+Y\sqrt{5}\in S$ かつ$1<u$とする。このとき,  $|u(X-\sqrt{5}Y)|=1$なので $|X-\sqrt{5}Y|<1$である。すなわち
\[-1<X-\sqrt{5}Y<1,  1<X+\sqrt{5}Y\]
なので, これらから$0<X,  0<Y$を得る。そこで, $1<u\leq 2+\sqrt{5}$ という条件を考えたとき, 不等式の条件から$u=1+\sqrt{5}, 2+\sqrt{5}$ の可能性があるが, $1+\sqrt{5}\notin S$ である。したがって, $u=2+\sqrt{5}$は$S$の1より大きい元のうち最小なものである。\\
いま, $s\in S$を任意にとるとき, $1<u$で指数関数$u^n$は単調に増加していくことから, $u^n\leq s<u^{n+1}$を満たす$n\in \mb{Z}$がただひとつ存在する。したがって
\[1\leq su^{-n}<u\]
であり, (iii)より $su^{-n}\in S$かつ, $S$の1以上の元になっているので, $u$の最小性により $1=su^{-n}$である。よって $s=u^{n}=x_n+y_n\sqrt{5}$\qed

$S$のすべての元を$-1$倍した集合を $T$とすると,
\begin{align*}
%T&=\{-X-Y\sqrt{5}\in \mb{Z}[\sqrt{5}]| (-X)^2-5(-Y)^2=\pm 1, X+Y\sqrt{5}<0\}\\
 T=\{ X+Y\sqrt{5}\in \mb{Z}[\sqrt{5}]| X^2-5Y^2=\pm 1, X+Y\sqrt{5}<0 \}
\end{align*}
だから, $S\cup T=:K$として, $X^2-5Y^2=\pm 1$の解$(X,Y)$から対応する$X+Y\sqrt{5}$全体の集合は, $K$に一致する。そして, 補題225.2.2(iv)から
\[K=\{\epsilon(2+\sqrt{5})^n | n\in \mb{Z}, \epsilon =1, -1\}\]
となる。$(2+\sqrt{5})^n=x_n+y_n\sqrt{5}$のとき, $(2-\sqrt{5})^n=x_n-y_n\sqrt{5}$なので
\[x_n=\dfrac{(2+\sqrt{5})^n+(2-\sqrt{5})^n}{2}, y_n=\dfrac{(2+\sqrt{5})^n-(2-\sqrt{5})^n}{2\sqrt{5}}\]
だから,$X^2-5Y^2=\pm 1$の解は$(\pm x_n, \pm y_n)$ (複号同順, $n\in \mb{Z}$)。

補題225.2.2(ii)より$X^2-5Y^2=1$の解は$(\pm x_{2n}, \pm y_{2n})$ (複号同順, $n\in \mb{Z}$) である。

$a^2-5b^2=m^2$ ($m=1,3,673,2019$) の場合,上のことから 複号同順で$(a,b)=(\pm mx_{2n}, \pm my_{2n})$ である。$m,n$を固定し, このとき,$(c,d)$が存在するかを見る。最初の式から
\[m(\pm x_{2n}c \pm 5y_{2n}d)=2019,  \pm x_{2n}d\pm y_{2n}c=0\]
で, $c,d$の連立方程式とみてとくと
\[(c,d)=(\pm\dfrac{2019}{m}x_{2n}, \mp\dfrac{2019}{m}y_{2n})\]
 となる。$x_{2n}=x_{-2n}, y_{2n}=-y_{-2n}$より,
\[(a,b,c,d)=\left(\pm mx_{2n}, \pm my_{2n}, \pm\dfrac{2019}{m}x_{-2n}, \pm\dfrac{2019}{m}y_{-2n}\right)\]

$a^2-5b^2=-m^2$ ($m=1,3,673,2019$) の場合も,同様に考えることができて, $x_{2n-1}=-x_{1-2n}, y_{2n-1}=y_{1-2n}$ より
\[(a,b,c,d)=\left(\pm mx_{2n-1}, \pm my_{2n-1}, \mp\dfrac{2019}{m}x_{1-2n}, \mp\dfrac{2019}{m}y_{1-2n}\right)\]
以上より, 求める組のすべては, $(a,b,c,d)=$
\[\mbox{{\boldmath $\left(\pm mx_{n}, \pm my_{n}, \pm (-1)^n\dfrac{2019}{m}x_{-n}, \pm (-1)^n\dfrac{2019}{m}y_{-n}\right)$}}\]
ただし,複号同順, $m=1,3,673,2019$, $n\in\mb{Z}$
\[x_n=\dfrac{(2+\sqrt{5})^n+(2-\sqrt{5})^n}{2}, y_n=\dfrac{(2+\sqrt{5})^n-(2-\sqrt{5})^n}{2\sqrt{5}}\]
