\begin{thm}{033}{\hosi ?}{琉球大 (2009)}
 $a,b$を正の実数とする。すべての自然数$n$に対して
\[(1^a+2^a+3^a+\cdots +n^a)^2=1^b+2^b+3^b+\cdots +n^b\]
が成立するとき, $a,b$をすべて求めよ。
\end{thm}

式を変形すると
\[n^{2a+2}\left(\dfrac{1}{n}\disp\sum_{k=1}^{n}\left(\dfrac{k}{n}\right)^a\right)^2=n^{b+1}\cdot\dfrac{1}{n}\disp\sum_{k=1}^n\left(\dfrac{k}{n}\right)^b\]
なので, 両辺を$n^{b+1}$で割って
\[ n^{2a+1-b}\left(\dfrac{1}{n}\disp\sum_{k=1}^{n}\left(\dfrac{k}{n}\right)^a\right)^2=\dfrac{1}{n}\disp\sum_{k=1}^n\left(\dfrac{k}{n}\right)^b\]
$2a+1-b=c$として, 両辺の$n\to \infty$の極限が一致するので, 区分求積法により
\[\disp\lim_{n\to \infty} \left\{n^c\left(\dfrac{1}{n}\disp\sum_{k=1}^{n}\left(\dfrac{k}{n}\right)^a\right)^2\right\}=\disp\int_0^1 x^b dx = \dfrac{1}{b+1}\]
となるため, 左辺が0でない値に収束することが必要となる。上の式の両辺を, 極限値
\[\disp\lim_{n\to \infty} \left(\dfrac{1}{n}\disp\sum_{k=1}^n\left(\dfrac{k}{n}\right)^a\right)^{2} = \dfrac{1}{(a+1)^2}\]
で割ったとき,
\[\disp\lim_{n\to \infty} n^c = \dfrac{(a+1)^2}{b+1}\neq 0, \infty\]
となるから, $c>0$では左辺が$\infty$に, $c<0$では左辺が$0$になるから不適。よって$c=0$であって,
\[1=\dfrac{(a+1)^2}{b+1}\]
となる。$c=0$はすなわち $b+1=2(a+1)$ ということになるので
\[1=\dfrac{(a+1)^2}{2(a+1)}=\dfrac{a+1}{2}\]
より $a=1$となる。このとき$b=3$で, 実際,すべての$n$に対して
\[\disp\sum_{k=1}^n k^3=\left(\dfrac{n(n+1)}{2}\right)^2=\left(\disp\sum_{k=1}^n k^1\right)^2\]
なのでよい。従って$(a,b)=(1,3)$ に限る。