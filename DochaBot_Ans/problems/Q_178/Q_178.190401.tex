\begin{thm}{178}{\hosi 6}{京大オープン}
 初項1, 公差24の等差数列を$\{a_n\}$とする。数列$\{\sqrt{a_n}\}$の項には5以上の素数がすべて現れることを示せ。
\end{thm}

\begin{proof}

$p$を5以上の素数とする。$p^2-1=(p-1)(p+1)$が24の倍数であることを示す。まず, $p$は3で割って1余るか2余るので, $p-1,p+1$のいずれかは3の倍数である。\\
次に, $p\pm 1$は偶数であって, $p$を
4で割ったあまりは1か3なので$p-1,p+1$のいずれかは4の倍数である。これにより$(p-1)(p+1)$は8の倍数であるとわかり, $p^2-1$は24の倍数である。よって $p^2-1=24k$  となる自然数$k$が存在し,  $p=\sqrt{24k+1}=\sqrt{a_k}$なので題意は示された。


\end{proof}
