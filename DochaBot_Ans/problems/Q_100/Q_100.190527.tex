\begin{thm}{100}{\hosi 8}{suiso\_728660 様}
 100人の田中が同じスタート地点から体育館を永遠に周回する。$\mathrm{田中}_n$ $(1\le n \le 100, n\in\mathbb{N})$ が体育館を1周するのにかかる時間は$n$とする。このとき、どの田中に自分から半周離れた場所の他の99人の田中が位置するような孤独な瞬間が訪れるか。
\end{thm}

体育館の1周の長さを1とし、$\text{田中}_n$の速さは$\dfrac{1}{n}$としてよい。スタートの時刻を$t=0$として、時刻$t=t_1$のときの$\text{田中}_n$が移動した距離は$\dfrac{t_1}{n}$である。

体育館の周上の位置を0以上1未満の実数で表現することにすれば、時刻$t=t_1$における$\text{田中}_n$の位置は、$\left\{\dfrac{t_1}{n}\right\}$ と表される。なお、$\{x\}$で実数$x$の小数部分を表すものとする。

さて、$\text{田中}_1$と$\text{田中}_2$が集合する時刻の条件を考えると、$\{t\}=\left\{\dfrac{t}{2}\right\}$であるが、これは$t-\dfrac{t}{2}=\dfrac{t}{2}$が整数であることに同値で、すなわち$t$は2の倍数である。また、このときこの2人の田中はともにスタート地点にいる。このことから、$\text{田中}_1$と$\text{田中}_2$の2人が集合するのはスタート地点のみである。

続いて、$\text{田中}_4$と$\text{田中}_8$が集合する時刻の条件を考えると、$\left\{\dfrac{t}{4}\right\}=\left\{\dfrac{t}{8}\right\}$であるが、これは$\dfrac{t}{4}-\dfrac{t}{8}=\dfrac{t}{8}$が整数であることに同値で、すなわち$t$は8の倍数である。また、このときこの2人の田中はともにスタート地点にいる。このことから、$\text{田中}_4$と$\text{田中}_8$の2人が集合するのはスタート地点のみである。

孤独な田中の番号を$m$とする。$\text{田中}_n$は、時刻$t=nj$ (ただし$j$は0以上の整数)のときに限りスタート地点にいるので、$\text{田中}_m$を除いた99人の田中は、
\[ \text{時刻}t=j\times \mr{lcm}(1, 2, \dots, m-1, m+1, \dots, 100) \]
においてスタート地点に集合する。このとき、$\text{田中}_1$と$\text{田中}_2$、$\text{田中}_4$と$\text{田中}_8$の2組のうち少なくとも一方はスタート地点に集合している。これらの集合はスタート地点に限られていたのであるから、この99人の集合もスタート地点に限られることがわかる。

$L_m=\mr{lcm}(1, 2, \dots, m-1, m+1, \dots, 100)$とおく。時刻$t=L_mj$のときの$\text{田中}_m$の位置は、$\left\{\frac{L_mj}{m}\right\}$なので、求める$m$は、ある0以上の整数$j$を用いて$\left\{\dfrac{L_m}{m}\right\}=\dfrac{1}{2}$となるような$m$である。さらに整理して、$\dfrac{L_mj}{m}-\dfrac{1}{2}=\dfrac{2L_mj-m}{2m}$が整数となる$j$が存在するような$m$を求めることを考える。

$m=64$の場合を考える。64を除いた1以上100以下の整数は、2で高々5回しか割り切れない。そのため$L_{64}=32l$ (ただし$l$は奇数)と書ける。よって$\dfrac{2L_{64}j-64}{128}=\dfrac{lj-1}{2}$は、ある奇数$j$を用いれば整数とできる。よって$\text{田中}_{64}$は題意を満たす。

$m\neq 64$の場合、$m$は2で高々6回しか割り切れないから、$m=2^ab$ (ただし$0\le a\le 6$でかつ$b$は奇数)と表せるが、一方で$L_m$は64の倍数であるから$L_m=64c$とすると、
\[ \frac{2L_mj-m}{2m}=\frac{2^7cj-2^ab}{2^{a+1}b}=\frac{2^{7-a}cj-b}{2b} \]
が得られる。ここで$7-a\ge 1$によって、右辺の分子は$j$にかかわらず奇数であるから、これは整数でない。よって$\text{田中}_{64}$以外は題意を満たさない。

以上より孤独な田中は、$\text{田中}_{64}$。

