\begin{thm}{098}{\hosi 10}{juniormemo 様}
 正の実数$x, y, z$について、
 \begin{align*}
  x^2+xy+y^2&=16 \\
  y^2+yz+z^2&=25 \\
  z^2+zx+x^2&=36 \\
 \end{align*}
 であるとき、$x+y+z$の値を求めよ。
\end{thm}

$x+y+z=s$とおく。$t=xy+yz+zx$として条件式の総和を取ると, ($x^2+y^2+z^2=s^2-2t$から)
\[77-2s^2=-3t\]
を得る。また, 条件式を各辺引いたりすると
\begin{align*}
 (z-x)s&=25-16=9 \\
 (x-y)s&=36-25=11 \\
 (z-y)s&=36-16=20
\end{align*}
の3式を得る。
\[(z-x)^2=36-3zx, (x-y)^2=16-3xy, (z-y)^2=25-3zy\]
なので, 先の3式をそれぞれ2乗することで
\begin{align*}
 (36-3zx)s^2&=81 \\
 (16-3xy)s^2&=121 \\
 (25-3zy)s^2&=400
\end{align*}
を得る。
\[77s^2-3s^2t=602\]
となる。ところで, $77-2s^2=-3t$であったから
\[77s^2+(77-2s^2)s^2=602\]
となり, 整理すると$s^4-77s^2+301=0$を得るので,$s^2$の二次方程式とみて
\[s^2=\dfrac{77\pm 15\sqrt{21}}{2}\]
を得る。このとき, $-3t=77-2s^2=\mp 15\sqrt{21}$ であって, $x,y,z>0\naraba t>0$なので$-3t<0$が必要であるため $s^2=\dfrac{77+15\sqrt{21}}{2}$のほうが適している。$s>0$なので, 
\[s=\sqrt{\dfrac{77+15\sqrt{21}}{2}}\]
である。\footnote{これは3辺が4,5,6の三角形の「フェルマー点」からの各頂点の距離の和という幾何的解釈もできるが, 代数的にやるほうがいい気がする。ところで, 一般に三角形の内点$I$で, $I$からの各頂点の距離の和が最小化されるのは$I$がフェルマー点であるときという事実があり, この問題はその最小値を求める問題だったのである。幾何の疑問を代数で考えた一問である。}

