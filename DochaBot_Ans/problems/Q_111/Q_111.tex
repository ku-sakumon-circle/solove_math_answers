\begin{thm}{111}{\hosi 3}{}
 互いに異なる$n (\ge 3)$個の正の数の集合$S$が次の性質を持っている。
 \begin{itemize}
  \item[性質:] $S$から相異なる元$a, b$をとれば、$a-b\in S$または$b-a\in S$のいずれか一方が成立する。
 \end{itemize}
 このとき、$S$の元を上手く並べると、等差数列を作ることができることを示せ。
\end{thm}

$S$から最小の元$m$を取る。$a\in S-\{m\}$に対して, $a=qm+r$  ($q$は0以上の整数, $0\leq r<m$)と書ける。$m$の最小性より$a>m$だから, $m-a$は正でないので$m-a\notin S$である。よって条件より
\[a-m=(q-1)m+r\in S\]
である。$q\geq 2$,つまり $a-m>m$であるなら, この議論を繰り返すことができ$a-2m\in S$. 以下同様に繰り返せば $a-(q-1)m=m+r\in S$となる。もし$r>0$であるなら, 更に同様の議論を行うことができて$(m+r)-m=r\in S$となるが, $r<m$なので$m$の最小性に矛盾する。よって$r=0$でなければならず, $a=qm$と書ける。つまり, $S$の元は必ず$m$の自然数倍で書けることになる。\\
 $S$の元は$n$個だから, $S$に最大元$dm$ ($d$は$n$より大きい整数)がある。$a=dm, b=m$として条件を適用すれば$(d-1)m,(d-2)m,\cdots, m$が$S$に属することが分かり,これらは異なる$S$の元であることと$S$が$n$個の元からなることより$d=n$である。つまり,
\[S=\{m,2m,\cdots, nm\}\]
なので, $S$の元は等差数列をなす。\qed