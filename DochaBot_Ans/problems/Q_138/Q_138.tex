\begin{thm}{138}{\hosi 7}{(1) 京大理系 (2014), (2) 東大実戦理系 (2017)}
 $\triangle$ABCが条件 $\angle{\mbox{B}}=2\angle{\mbox{A}}$, BC$=1$を満たしているとする。
 \begin{enumerate}
  \item $\triangle$ABCの面積が最大になるときの$\cos\mr{B}$の値を求めよ。
  \item $\triangle$ABCの内接円半径$r$と外接円半径$R$の比 $r/R$ の取りうる値の範囲を求めよ。
 \end{enumerate}
\end{thm}

\syoumon{1}
$\angle\mr{A}=\theta$とおくと、$\angle\mr{B}=2\theta$, $\angle\mr{C}=\pi-3\theta$となる。すべての角は0より大きく$\pi$より小さいから、$0<\theta<\dfrac{\pi}{3}$。正弦定理によって、$\mr{AB}=\dfrac{\sin3\theta}{\sin\theta}$である。よって三角形の面積は、
\begin{align*}
 \triangle\mr{ABC}&=\frac{1}{2}\mr{AB}\cdot\mr{BC}\sin\mr{B}=\frac{\sin3\theta\sin2\theta}{2\sin\theta} \\
 &=\frac{(\sin\theta\cos2\theta+\cos\theta\sin2\theta)\sin2\theta}{2\sin\theta} \\
 &=\frac{1}{2}(2\cos2\theta+1)\sqrt{1-\cos^22\theta}
\end{align*}
で得られる。$0<\theta<\dfrac{\pi}{3}$のとき、$-\dfrac{1}{2}<\cos2\theta<1$であることと、三角形の面積は常に正であることから、
\[ f(x)=\frac{1}{4}(2x+1)^2(1-x^2) \qquad \left(-\frac{1}{2}<x<1\right) \]
のような関数$f(x)$が最大となる$x$の値について考えればよい。
\begin{align*}
 f'(x)&=\frac{1}{4}\bigl[2(2x+1)(2x+1)'(1-x^2)+(2x+1)^2(-2x)\bigr] \\
 &=-\frac{1}{2}(2x+1)(4x^2+x-2)
\end{align*}
により、$f'(x)=0$となるのは、$x=-\dfrac{1}{2}, \dfrac{-1\pm\sqrt{33}}{8}$のとき。これを踏まえて増減表は、以下のようになる。
\begin{align*}
 \begin{array}{c|c|c|c|c|c}
  x & -\frac{1}{2} & \cdots & \frac{-1+\sqrt{33}}{8}& \cdots & 1 \\ \hline
  f'(x) & 0 & + & 0 & - & - \\ \hline
  f(x) & 0 & \nearrow & \text{max} & \searrow & 
 \end{array}
\end{align*}
したがって、$-\dfrac{1}{2}<x<1$において$f(x)$は、$x=\dfrac{-1+\sqrt{33}}{8}$のときに最大となる。よって、$\triangle$ABCの面積が最大となるのは、$\cos\mr{B}=\dfrac{-1+\sqrt{33}}{8}$のとき。

\syoumon{2}
$\angle{\mbox{A}}=\theta$ とすると, $\angle{\mbox{B}}=2\theta, \angle{\mbox{C}}=\pi-3\theta$ であるから, これらが0より大きく$\pi$未満になることより $0<\theta<\dfrac{\pi}{3}$の範囲で動かす。内心をIとして, 内接円とABの接点をHとしたとき, 三角形IHBに注目すると, $\angle{\mbox{IBH}}=\theta$なので
\[\mbox{HB}=\dfrac{r}{\tan{\theta}}\]
である。同様に考えると
\[\mbox{HA}=\dfrac{r}{\tan{\frac{\theta}{2}}}\]
であり, 正弦定理より
\[\mr{AB}=2R\sin{ (\pi-3\theta) } =\mbox{HA$+$HB} =r\left( \dfrac{1}{ \tan{\frac{\theta}{2} }}+\dfrac{1}{\tan{\theta}}\right)\]
を得る。$\sin{\dfrac{\theta}{2}}=s, \cos{\dfrac{\theta}{2}}=c$とおいて$\sin{(\pi-3\theta)=\sin{3\theta}}$と倍角の公式から,整理すると
\begin{eqnarray*}
\dfrac{R}{r}&=&\dfrac{1}{2}\left( \dfrac{\sin{\theta}\cos{\frac{\theta}{2}}  +\cos{\theta}\sin{ \frac{\theta}{2} }  }{\sin{\frac{\theta}{2}}\sin{\theta}\sin{3\theta}}\right)=\dfrac{1}{2}\left( \dfrac{2sc\cdot c + (2c^2-1)s}{s\cdot \sin{\theta} \sin{3\theta}}\right)\\
&=&\dfrac{4c^2-1}{2\sin{\theta}\sin{3\theta}} = \dfrac{2\cos{\theta}+1}{2\sin{\theta}\sin{3\theta}}=\dfrac{2\cos{\theta}+1}{2\sin^2{\theta}(4\cos^2{\theta}-1)}\\
&=&\dfrac{1}{2(1-\cos^2{\theta})(2\cos{\theta}-1)}
\end{eqnarray*}
となる。よって, $\cos{\theta}=x$~$\left(\dfrac{1}{2}<x<1\right)$とおいて
\[\dfrac{r}{R}=2(2x-1)(1-x^2)\]
の取りうる値の範囲を考えればよい。$x$で微分すると $-12x^2+4x+4=-4(3x^2-x-1)$となるから. $x=\dfrac{1\pm \sqrt{13}}{6}$で極値をとる。$\dfrac{1}{2}<\dfrac{1+\sqrt{13}}{6}<1$で, ここで極大値かつ最大値をとる。$\dfrac{r}{R}$に$x=\dfrac{1+\sqrt{13}}{6}$を代入し
\[2\cdot\dfrac{\sqrt{13}-2}{3}\cdot\dfrac{22-2\sqrt{13}}{36} = \dfrac{1}{54}(26\sqrt{13} -70)=\dfrac{1}{27}(13\sqrt{13}-35)\]
となる。$x\to 1-0$とすればいくらでも0に近づくので,求める範囲は
\[0<\dfrac{r}{R}\leq \dfrac{1}{27}(13\sqrt{13}-35)\]
