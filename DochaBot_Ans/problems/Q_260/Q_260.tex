\begin{thm}{260}{\hosi ?}{H23 京大院試 数理解析系 II}
すべての$x\in \mb{R}$に対して$\phi(\phi(x)) = -x$を満たす実数値連続関数$\phi$は存在しないことを示せ. 
\end{thm}

条件より$-\phi(x) = \phi(\phi(\phi(x))) = \phi(-x)$なので$\phi(0) = 0$を得る. \par 
写像$\mb{R}\to \mb{R}; x\mapsto -x$が全単射なので条件より$\phi$は全単射. よって$\phi(-1) \neq \phi(1)$である. $\phi(-1),\phi(1)$が同じ符号であることはない(全単射性と中間値の定理で分かる). ここでは $\phi(-1)<0<\phi(1)$の場合を考える($\phi(1)>0>\phi(-1)$の場合も同様である). \par
\step{1}{$[-1,1]$での増加性}
すべての$-1\leq a<b\leq 1$に対し $\phi(a)<\phi(b)$を示す. いま単射性より$\phi(a)\neq \phi(b)$なので, もし$\phi(a) < \phi(b)$でないなら, $\phi(a) > \phi(b)$である. すると, $\phi(b) < p < \phi(b)$なる任意の実数$p$に対し, 二つのグラフ$y=p$と$y=\phi(x)$は, $-1< x<a$と$a<x<b$で交わっていることが分かる(中間値の定理). よって単射性に反するので, 矛盾. よって$\phi$が$[-1,1]$で増加関数であることが分かる. \\
\step{2}{矛盾を導く}
$\phi(0) = 0$と連続性より, 十分小さい$1>\epsilon > 0$が存在して, $x\in [-\epsilon,\epsilon]$なる$x$では常に$\phi(x)\in [-1,1]$が成り立つようにできる. いま, $-\epsilon \leq x\leq \epsilon$の中で$x$を増加させると, $\phi(x)$は$[-1,1]$の中で増加する. $\phi$は$[-1,1]$で増加するのであったから, $\phi(\phi(x))$も増加するはずである. しかし関数$-x$は減少しているから, 矛盾である. \qed 