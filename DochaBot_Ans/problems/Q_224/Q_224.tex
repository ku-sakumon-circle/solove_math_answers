\begin{thm}{224}{\hosi 7\maru}{学コン 2018-6-1}
 1, 2, 3と書かれたカードが各3枚、4, 5, 6と書かれたカードが各2枚、7, 8, 9と書かれたカードが各1枚ある。これら18枚のカードから4枚を無作為に1枚ずつ取り出して、左から順に並べて4桁の整数$N$を作る。$N$が6の倍数である確率を求めよ。
\end{thm}

4枚のカードの数字の和が3で割り切れ、かつ、4枚目のカードの数字が偶数 ($\cdots$ *) の場合に、$N$は6の倍数になる。1と7、2と8、3と9は、6で割った余りがそれぞれ同じなので、(*)のような選び方を考えるときには、それぞれ同一視してよい。よって、1, 2, 3が各4枚、4, 5, 6が各2枚の状況から考える。

1の位(4枚目)を指定して場合わけを行う。1の位が2, 4, 6のとき、その数字を取り方はそれぞれ4, 2, 2通り。残りのカードの取り出し方は、それら3枚の数字の和を3で割った余りがそれぞれ1, 2, 0となるようにすればよい。以降では、`残り3枚のカードの数字の和を3で割った余り'を$R$で表す。

$R$を考えるにあたっては、4を1に、5を2に、6を3にみなせるので、残ったカードが、(i) 1, 3が6枚ずつと2が5枚、(ii) 2, 3が6枚ずつと1が5枚、(iii) 1, 2が6枚ずつと3が5枚、のそれぞれの状況から考えればよい。加えて、(i)の場合では、カードの数字に全て1を足し、4を1とみなす、(ii)の場合では、カードの数字に全て1を引き、0を3とみなす、という操作を行っても$R$の値は変わらない。したがって、1, 2が6枚ずつと3が5枚の状況\footnote{最終的に確率を考えるのであるから、計17枚のカードは全て区別できるものと考える}から、$R$が1, 2, 0のそれぞれになる場合の数を調べればよい。それぞれを$n_1$, $n_2$, $n_0$とおく。

$n_1+n_2+n_0$は、1,2が6枚ずつと3が5枚ある中から3枚取り出す場合の数に相当するから、$17\times 16\times 15$。1の位の決め方はそれぞれ4, 2, 2通りであったから、$N$が6の倍数になる取り方は$4n_1+2n_2+2n_0$通りである。$R=1$となる3枚の組み合わせは、$(1, 1, 2)$, $(2, 2, 3)$, $(1, 3, 3)$。これの並べ替えが$N$の上3桁になる。場合の数は、
\[ n_1=3(6\times 5\times 6)+3(6\times 5\times 6)+3(6\times 5\times 4)=6\times 5\times 3\times 15 \]
と求まる。

$N$が6の倍数となる取り出し方は、
\begin{align*}
 &4n_1+2n_2+2n_0=2(n_1+n_2+n_0)+2n_1 \\
 =&2(17\times 16\times 15) + 2(6\times 5\times 3\times 15) = 60\times 181
\end{align*}
であるから、求める確率は、
\[ \frac{60\times 181}{\permu{18}{4}}=\frac{181}{1224} \]