\begin{thm}{148}{\hosi 6}{ElfenLied\_MS2 様}
 各桁の数が7または2である正の整数を千早数と呼ぶ。72桁であり、$2^{72}$の倍数である千早数は存在するか。
\end{thm}

任意の自然数$n$に対して、$n$桁でかつ$2^n$で割り切れる千早数が存在する ($\cdots$ *) ことを示す。

$n=1$のときは2が存在するのでよい。

2以上のある自然数$k$について(*)の成立を仮定し、この千早数を$A$とおく。このとき、$2\cdot 10^k+A$, $7\cdot 10^k+A$は、ともに$k+1$桁の千早数である。さて、$A$は$2^k$の倍数であるから、$A=2^kB$と書ける ($B$は自然数)。

$B$が奇数であるとすると、
\[ 7\cdot 10^k+A=2^k(7\cdot 5^k+B) \]
は$2^{k+1}$で割り切れる。$B$が偶数であるとすると、
\[ 2\cdot 10^k+A=2^k(2\cdot 5^k+B) \]
は$2^{k+1}$で割り切れる。よって、いずれの場合も$2\cdot10^k+A$, $7\cdot 10^k+A$の一方が、$k+1$桁でかつ$2^{k+1}$で割り切れる千早数となる。

以上より、数学的帰納法によって(*)が示された。特に$n=72$とすることで、問題の主張を得る。