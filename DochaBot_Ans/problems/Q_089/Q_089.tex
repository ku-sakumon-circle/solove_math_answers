\begin{thm}{089}{\hosi 8}{JMO予選 1991 改題}
 自然数に対して定義される関数$f$は、
 \begin{align*}
  f(1)&=1 \\
  f(2n)&=f(n) \\
  f(2n+1)&=f(n)+1
 \end{align*}
 を満たしている。$1\le k \le 114514$ における$f(k)$の最大値、及びその時の$k$の値を全て求めよ。
\end{thm}

$f(n)$は$n$を2進数表示したときの1の桁の個数である ($\cdots$ *) ことを数学的帰納法により証明する。

与えられた条件によって、$n=1$の場合は(*)が成り立っている。

$1\le k \le n$なる任意の自然数$k$で(*)が成り立つことを仮定する。

$n+1$の2進数表示を$a_ma_{m-1}\dots a_2a_1$ とする。

$n+1$が偶数ならば$a_1=0$である。よって$\dfrac{n+1}{2}$の2進数表示は$a_ma_{m-1}\dots a_2$なので、$n+1$と$\dfrac{n+1}{2}$を2進数表示したときの1の個数は等しい。一方で与えられた条件によって$f(n+1)=f\left(\dfrac{n+1}{2}\right)$であるから、$k=n+1$が偶数のときにも(*)は成り立つ。

$n+1$が奇数ならば$a_1=1$なので、$n$の2進数表示は$a_ma_{m-1}\dots a_20$。さらに$\dfrac{n}{2}$の2進数表示は$a_ma_{m-1}\dots a_2$なので、$n+1$を2進数表示したときの1の個数は、$\dfrac{n}{2}$のそれよりも1大きい。一方で与えられた条件によって$f(n+1)=f\left(\dfrac{n}{2}\right)+1$であるから、$k=n+1$が奇数のときにも(*)は成り立つ。

以上より任意の自然数$n$において(*)が成り立つことが示された。

さて、$114514=11011111101010010_2$である。これは17桁であり、$k$は$\underbrace{11\dots 11_2}_{\text{17個}}$未満なので、$f(k)<17$である。$f(k)=16$となる$k$を求める。このような$k$は、
\[ k=\underbrace{11\dots 11}_{\text{17個}}{}_2 - 1\underbrace{0\dots 00}_{i\text{個}}{}_2 \quad (i=0, 1,\dots , 16) \]
のように表される数に限られ、10進法で$131071-2^i$である。これが114514以下となる$i$は$i=15,16$のみ\footnote{$131071-2^{14}=114687$。114514はこのギリギリを攻めた結果なのである。}で、このときの$k$は、$k=98303, 65535$