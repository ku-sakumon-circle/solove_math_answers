\begin{thm}{074}{\hosi 8\maru}{東工大 (1985)}
 $a^2-2b^2=\pm 1$ かつ $a+b\sqrt{2}>0$ を満たす整数$a, b$から得られる実数$a+b\sqrt{2}$ 全体の集合を$G$とする。1より大きい$G$の元のうち、最小のものを$u$とする。
 \begin{enumerate}
  \item $u$を求めよ。
  \item 整数$n$と$g\in G$に対して、$gu^n\in G$を示せ。
  \item $G$の任意の元は、適当な整数$n$により$u^n$と表されることを示せ。
 \end{enumerate}
\end{thm}

\syoumon{1}
$u=1+\sqrt{2}$であることを示す。$G\ni 1+\sqrt{2}>1$はよい。$1<w\le 1+\sqrt{2}$なる$w\in G$を考え、$w=a+b\sqrt{2}$とおく。$|a^2-2b^2|=w|a-b\sqrt{2}|=1$であるから、
\[ |a-b\sqrt{2}|=\frac{1}{w}<1 \quad (\because w>1) \]
より、$-1<a-b\sqrt{2}<1$となる。これと$1<w=a+b\sqrt{2}$より、
\begin{align*}
 -1+1&< (a-b\sqrt{2})+(a+b\sqrt{2}) \\
 1+(a-b\sqrt{2})&<1+(a+b\sqrt{2})
\end{align*}
がそれぞれ成り立つ。整理して$a>0$, $b>0$を得るので、$a\ge 1$,$b\ge 1$でなければならず、$1+\sqrt{2}\le w$となる。$1<w\le 1+\sqrt{2}$としたから、$w=1+\sqrt{2}$となる。このことは、求める最小の元$u$は$1+\sqrt{2}$であることを示している。$u=1+\sqrt{2}$。

\syoumon{2}
次を示す。
\[ w_1 \,,\,\, w_2 \in G \quad\Rightarrow\quad w_1w_2\in G \quad\text{(*)}\]
$w_i=a_i+b_i\sqrt{2}$ (ただし$a_i^2-2b_i^2=\pm 1$, $w_i>0$) とおく。積を計算すると、
\[ w_1w_2=(a_1a_2+2b_1b_2)+(a_2b_1+a_1b_2)\sqrt{2} \]
であるが、
\begin{align*}
 &(a_1a_2+2b_1b_2)^2-2(a_2b_1+a_1b_2)^2 \\
 =& (a_1a_2)^2+4(b_1b_2)^2-2(a_2b_1)^2-2(a_1b_2)^2 \\
 =& (a_1^2-2b_1^2)(a_2^2-2b_2^2)=\pm 1 \quad (\because (a_i^2-2b_i^2)\in\{-1, 1\})
\end{align*}
となる。明らかに$w_1w_2>0$であるから、したがって(*)が示された\footnote{つまり$G$は積について閉じている。}。

次に整数$n$に対して$u^n\in G$を示す。$n=0$のときは$u^0=1$で明らかに$G$に含まれる。$n$が正なら、(*)を用いて帰納的に$u^n \in G$がいえる。$n$が負なら、$(u^{-1})^{-n}$を考えて、$u^{-1}=-1+\sqrt{2} \in G$なので、(*)を用いてこの場合でも$u^n \in G$である。よって、全ての整数$n$について$u^n \in G$である。

\syoumon{3}
$G$の元$g$をひとつ取る。このとき、ある整数$n(g)$が存在して、
\[ u^{n(g)-1 < g \le u^{n(g)}} \]
を満たす。両辺を$u^{n(g)-1}$で割ると、
\[ 1 < g u^{-n(g)+1} \le u \]
であり、(2)の結果から$g u^{-n(g)+1}$は$G$の元であり、かつ1より大きい。(1)の結果から、$u$は1より大きい$G$の元のうち最小であるから、
\[ gu^{-n(g)+1}=u \]
でなければならない。よって$g=u^{n(g)}$を得るから、題意は示された。