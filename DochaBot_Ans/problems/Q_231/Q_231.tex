\begin{thm}{231}{\hosi 6}{IMO shortlist (2013)}
 正の整数$n$であって、つぎの条件を満たすものが無限に存在することを証明せよ。
 \begin{itemize}
  \item[条件:] $n^4+n^2+1$の最大の素因数と$(n+1)^4+(n+1)^2+1$の最大の素因数が等しい 
 \end{itemize}
\end{thm}

$f(n)=n^2+n+1$とおく。このとき$f(n^2)=n^4+n^2+1$であって、因数分解すると
\[ f(n^2)=f(n)f(n-1) \quad\cdots\text{(#)} \]
であることがわかる。そこで$f(n)$の最大の素因数を$p_n$とする。(便宜上$p_0=1$とする。) $p_{n^2}=p_{(n+1)^2}$となる$n$が無限に存在することを示せばよい。式(#)から、$p_{n^2}=\mr{max}\{p_n\,,\,\,p_{n-1}\}$が全ての正の整数$n$で成り立つことに注意すると次のことがわかる。

いかなる自然数$M$を与えても、$i>j>M$を満たす自然数$i, j$であって$p_i=p_j$を満たすものが存在する。$\cdots$ (*)

$p_{n^2}=p_{(n+1)^2}$を満たす$n$が有限個だと仮定する。そのような$n$の最大値を$N_0$とおく。$m>N_0$を満たす任意の自然数$m$について、$p_{m^2}>p_{(m+1)^2}$または$p_{m^2}<p_{(m+1)^2}$が満たされる。

素数の真の無限減少数列を作ることはできないから、ある$m_0>N_0$が存在して$p_{m_0^2}<p_{(m_0+1)^2}$となる。
\[ p_{m_0}\le \mr{max}\{p_{m_0}\,,\,\,p_{m_0-1}\} < \mr{max}\{p_{m_0+1}\,,\,\,p_{m_0}\} \]
であるから、右辺は$p_{m_0}$に等しくなってはならないので、 $\mr{\max}\{p_{m_0+1}\,,\,\,p_{m_0}\}=p_{m_0+1}$ がわかる。すると $p_{m_0}<p_{m_0+1}$ となる。

さて $p_{(m_0+1)^2}=\mr{max}\{p_{m_0+1}\,,\,\,p_{m_0}\}=p_{m_0+1}$ であったから、 $p_{m_0+1}>\mr{max}\{p_{m_0+2}\,,\,\,p_{m_0+1}\}$ は成り立たない。したがって $p_{(m_0+1)^2}<p_{(m_0+2)^2}$ となり、ここから同様な議論により $p_{m_0+1}<p_{m_0+2}$ がわかる。

帰納的に、$m_0\le n$を満たす任意の$n$に対して$p_n<p_{n+1}$でなければならないことがわかる。これは(*)で$K=m_0$とした内容に矛盾する。

よって、$p_{n^2}=p_{(n+1)^2}$を満たす$n$が無限に存在する。