\begin{thm}{219}{\hosi 6}{1対1 数A}
 凸多面体の頂点、辺、面の数をそれぞれ$v, e, f$とすると、$v-e+f=2$が成り立つ。これをオイラーの多面体定理という。正五角形と正六角形の面からなる凸多面体がある。正五角形の面どうしが辺を共有しないとき、この多面体の面の数を求めよ。\\
 なお、凸多面体は次の性質をもつ。
 \begin{itemize}
  \item 1つの頂点に集まる面の数は3以上
  \item 1つの頂点に集まる角の和は$360^\circ$未満
 \end{itemize}
\end{thm}

ここに解答を記述。