\begin{thm}{180}{\hosi 9}{DMO3rd 理5}
 正の整数$n$の正の約数の個数, 総和をそれぞれ $d(n), \sigma (n)$ とする。次の極限値を求めよ。
 \[\disp\lim_{n\to \infty}\dfrac{\log{(\sigma(1) +\sigma(2) +\cdots +\sigma (n))}}{\log{(d(1)+d(2)+\cdots +d(n))}}\]
\end{thm}

$D_n=\{d| \mbox{$d$ は $n$ の正の約数}\}$ とする。
\[\displaystyle\sum_{k=1}^n d(k) = \displaystyle\sum_{k=1}^n|D_k|\] である。ここで, $i\in D_k$ と $k$ が $i$ の倍数であることは同値である。$1,2,\cdots n$ のうち $i$ の倍数は $\left\lfloor\dfrac{n}{i}\right\rfloor$ 個あるから,$i$ が $D_1, D_2, \cdots D_n$ の要素として現れる回数は, $1,2,\cdots, n$の中にある$k$の$i$の倍数の個数なので $\left\lfloor\dfrac{n}{i}\right\rfloor$ である。この回数を $1\le i\le n$ で足し合わせた $\displaystyle\sum_{i=1}^n \left\lfloor\dfrac{n}{i}\right\rfloor$ が $\displaystyle\sum_{k=1}^n |D_k|$ に等しい。

$\displaystyle\sum_{k=1}^n \sigma (k)$ も同様, $i$ が $\left\lfloor\dfrac{n}{i}\right\rfloor$ 回登場することから $\displaystyle\sum_{k=1}^n i\left\lfloor\dfrac{n}{i}\right\rfloor$ に等しい。 \\
この2つの結果を不等式で評価し, はさみうちの原理で極限値を求めることを考える。\\
一般に $\lfloor x\rfloor\le x$ が成立し, $n$ が2以上のとき
\begin{alignat*}{6}
n=\sum_{k=1}^n 1 < &\sum_{i=1}^n \left\lfloor\dfrac{n}{i}\right\rfloor& \le &\sum_{i=1}^n \dfrac{n}{i}& =&\,nH_n\\
\dfrac{n^2}{2}<\dfrac{n(n+1)}{2}=\sum_{i=1}^n i \le &\sum_{i=1}^n i\left\lfloor\dfrac{n}{i}\right\rfloor& \le &\sum_{i=1}^n i\dfrac{n}{i}& =&\,n^2
\end{alignat*}
となる $(\text{ただし}\,\,H_n=\displaystyle\sum_{k=1}^n\dfrac{1}{k})$。そして,次のように評価が出来る。
\begin{align*}
\dfrac{\log{n^2}-\log{2}}{\log{n}+\log{H_n}} < \dfrac{\log{\sigma(1) +\sigma (2)+\cdots \sigma (n)}}{\log{d(1)+d(2)+\cdots d(n)}} < \dfrac{\log{n^2}}{\log{n}} = 2 \quad \marunum{1}
\end{align*}
次に十分大きい $n$ で $H_n<2\log{n}$ が成り立つこと, 及び次を示す。
\[ \lim_{n\to\infty}\dfrac{\log{n^2}-\log{2}}{\log{n}+\log{H_n}}=\lim_{n\to\infty}\dfrac{2-\log_n{2}}{1+\log_n{H_n}}=2 \]
$2\le k\le n$ なる整数 $k$ で, (曲線 $y=\dfrac{1}{x}$ の $0<x$ における単調減少性と面積評価から) $\dfrac{1}{k}<\displaystyle\int_{k-1}^{k} \dfrac{1}{x} dx$ が成り立つから, 総和を取り
\[ H_n-1<\displaystyle\int_1^{n}\dfrac{1}{x} dx=\log{n} \]
よって $H_n<\log{n}+1<2\log{n}$ である。これを用いて
\begin{align*}
\lim_{n\to\infty}\dfrac{2-\log_n{2}}{1+\log_n{2}+\log_n{\log{n}}}&\le\lim_{n\to\infty}\dfrac{2-\log_n{2}}{1+\log_n{H_n}}\\
&\le\lim_{n\to\infty}\dfrac{2-0}{1+0}=2 \quad \marunum{2}
\end{align*}
となる。$\log_n{\log{n}}=\dfrac{\log{\log{n}}}{\log{n}}=\dfrac{t}{e^t}$ で(ただし $n= e^{(e^t)}$) $n\rightarrow \infty$ のとき $t\rightarrow \infty$ なので, $\log_n{\log{n}}=\dfrac{t}{e^t}\rightarrow 0$ であるから
\[ \lim_{n\to\infty}\dfrac{2-\log_n{2}}{1+\log_n{2}+\log_n{\log{n}}}=\dfrac{2-0}{1+0+0}=2 \]

しかるに\marunum{2}とはさみうちの原理から $\displaystyle\lim_{n\to\infty}\dfrac{2-\log_n{2}}{1+\log_n{H_n}}=2$, そして\marunum{1}とはさみうちの原理から
\[\displaystyle\lim_{n\to\infty }\dfrac{\log{\sigma(1) +\sigma (2)+\cdots \sigma (n)}}{\log{d(1)+d(2)+\cdots d(n)}}=2\]
