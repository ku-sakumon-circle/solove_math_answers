\begin{thm}{042}{\hosi 6}{東工大 (1969)}
 実数$a,b,c,x,y,z,p$が次の4条件を満たしている: $a^2-b^2-c^2>0$, $ax+by+cz=p$, $ap<0$, $x>0$。このとき、$x^2-y^2-z^2$ の符号を調べよ。
\end{thm}

$p-ax=by+cz$であって、この両辺2乗する。左辺について$a^2>b^2+c^2$から、
\[ (p-ax)^2=p^2-2apx+a^2x^2>p^2-2apx+(b^2+c^2)x^2 \]
一方右辺は、コーシー・シュワルツの不等式を用いて
\[ (by+cz)^2\le (b^2+c^2)(y^2+z^2) \]
となる。$ap<0$, $x>0$より$-2apx>0$であるから、
\begin{align*}
 &p^2-2apx+(b^2+c^2)x^2<(b^2+c^2)(y^2+z^2) \\
 \dou\quad 0<&p^2-2apx<-(b^2+c^2)(x^2-y^2-z^2)
\end{align*}
を得る。明らかに$b^2+c^2>0$だから、$x^2-y^2-z^2<0$である。