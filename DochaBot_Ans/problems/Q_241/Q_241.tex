\begin{thm}{241}{\hosi ?}{}
 任意の自然数$N$に対して、座標空間内の球であって、その内部に格子点を$N$個含むものが存在することを証明せよ。
\end{thm}

$\mr{P}=(\sqrt{2}, \sqrt{3}, \sqrt{5})$とする\footnote{これが球の中心となるが、これ以外にも様々な取り方がある。後述の補題が成り立つような座標を取るとよい。}。$\mr{P}$と格子点$(l,m,n)$との距離は、
\[ \sqrt{(l-\sqrt{2})^2+(m-\sqrt{3})^2+(n-\sqrt{5})^2} \]
である。この距離の2乗を$f(l,m,n)$とおく。2つの格子点$(l_1,m_1,n_1)$, $(l_2,m_2,n_2)$に対して$f(l_1,m_1,n_1)=f(l_2,m_2,n_2)$が成り立っていたとする。このとき、
\begin{align*}
 f(l_1,m_1,n_1)-f(l_2,m_2,n_2)&=0 \\
 \dou\quad (l_1^2+m_1^2+n_1^2-l_2^2-m_2^2-n_2^2) \qquad\qquad\qquad\qquad & \\
  + 2(l_2-l_1)\sqrt{2}+2(m_2-m_1)\sqrt{3}+2(n_2-n_1)\sqrt{5}&=0
\end{align*}
となる。ここで次の補題を用意する\footnote{ここでは証明を略すが、さほど難しくない。}。
\begin{subthm}{241.1}
 $\sqrt{2}, \sqrt{3}, \sqrt{5}$は無理数である。$a, b, c, d$を有理数とし、
 \[ a+b\sqrt{2}+c\sqrt{3}+d\sqrt{5}=0 \]
 が成り立つとする。このとき、$a=b=c=d=0$である。
\end{subthm}
この補題によって、
\begin{align*}
 2(l_2-l_1)=0 \,,\,\, 2(m_2-m_1)=0 \,,\,\, 2(n_2-n_1)=0 \\
 l_1^2+m_1^2+n_1^2-l_2^2-m_2^2-n_2^2=0
\end{align*}
が従うので、$(l_1, m_1, n_1)=(l_2, m_2, n_2)$である。すなわち
\[ f(l_1, m_1, n_1)=f(l_2, m_2, n_2) \,\Rightarrow\, (l_1, m_1, n_1)=(l_2, m_2, n_2) \]
だから、対偶を取れば
\[ (l_1, m_1, n_1)\neq (l_2, m_2, n_2) \,\Rightarrow\, f(l_1, m_1, n_1)\neq f(l_2, m_2, n_2) \]
つまり格子点全体の集合で定義された実数値関数$f(l,m,n)$は、異なる二つの格子点に対して異なる値を返すような関数である。よって、座標空間内の格子点全体を$f$の値が小さい順に$(l_1, m_1, n_1)$, $(l_2, m_2, n_2), \ldots$ と並べることができる。

このように並べたとき、$f(l_i, m_i, n_i) < f(l_{i+1}, m_{i+1}, n_{i+1})$が成り立っている。そこで正の実数$R_N$を、
\[ f(l_N, m_N, n_N) < R_N^2 < f(l_{N+1}, m_{N+1}, n_{N+1}) \]
が成り立つようにとる。このとき、球
\[ (x-\sqrt{2})^2+(y-\sqrt{3})^2+(z-\sqrt{5})^2 < R_N^2 \]
は、その内部に$(l_i, m_i, n_i)$ $(i=1,2,\dots,N)$という$N$個の格子点だけを含む球となっている。