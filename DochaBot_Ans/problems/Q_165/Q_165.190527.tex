\begin{thm}{165}{\hosi 7\maru}{東京大}
 $xy$平面上の各格子点を中心として半径$r$の円が描かれており、傾き$\dfrac{2}{5}$の任意の直線はこれらのどれかと共有点を持つという。このような性質を持つ実数$r$の最小値を求めよ。
\end{thm}

傾き$\dfrac{2}{5}$の直線の方程式は$5x+2y-k=0$ (ただし$k$は実数) の形で表される。いかなる$k$に対しても、ある格子点$(m, n)$が少なくとも1つ存在して、
\[ \frac{|5m+2n-k|}{\sqrt{5^2+2^2}}=\frac{|5m+2n-k|}{\sqrt{29}} \]
が$r$以下となることが必要十分である。

適当な整数$N$に対して、$m=N$, $n=-2N$とおけば$5m+2n=N$とできる。またいかなる格子点$(m, n)$に対しても明らかに$5m+2n$は整数になる。すなわち、$(m, n)$が格子点全体を動くとき、$5m+2n$は整数全体を動く。したがって、$N$を整数全体で動く変数とみて、$|5m+2n-k|=|N-k|$が最小となるような$N$のときに$\dfrac{|N-k|}{\sqrt{29}}\le r$が成り立つ。

実数$k$に対して、$M_k+\dfrac{1}{2}\ge k$を満たす最小の整数$M_k$が存在し、これは$M_k-\dfrac{1}{2}<k\le M_k+\dfrac{1}{2}$を満たす。$k-M_k=j$とおくと、$-\dfrac{1}{2}<j\le\dfrac{1}{2}$であって、$|N-k|=|N+M_k-j|$。

$|N+M_k|\ge 1$であると、$|N+M_k-j|\ge\dfrac{1}{2}$となる。すなわち、$\dfrac{|N-k|}{\sqrt{29}}\ge\dfrac{1}{2\sqrt{29}}$となるような$N$が必ず存在するから、$r\ge\dfrac{1}{2\sqrt{29}}$が$r$の十分条件である。特に$k=\dfrac{1}{2}$の場合を考えれば、$\left|N-\dfrac{1}{2}\right|$の最小値は$\dfrac{1}{2}$であるから、そのとき直線と格子点の距離は$\sqrt{1}{2\sqrt{29}}$となり得るため、$r<\dfrac{1}{2\sqrt{29}}$は題意を満たさない。

以上により、$r\ge\dfrac{1}{2\sqrt{29}}$が必要十分で、$r$の最小値は$\dfrac{1}{2\sqrt{29}}$である。