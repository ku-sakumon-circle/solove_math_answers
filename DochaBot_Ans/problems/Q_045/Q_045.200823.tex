\begin{thm}{045}{\hosi 7}{横浜国立大 2009}
 \begin{enumerate}
  \item $x^2+y^2=41$ を満たす自然数$x,y$の組を全て求めよ。
  \item $(ac-bd)^2+(ad+bc)^2$ を因数分解せよ。
  \item 任意の自然数$n$に対して、$x^2+y^2=2009^n$を満たす自然数の組$(x,y)$が存在することを証明せよ。
 \end{enumerate}
\end{thm}

\syoumon{1}
$x^2$, $y^2$は正である。$41-y^2\le 40$ より $1\le x^2\le 40$ を満たすことが必要。よって$x=1,2,3,4,5,6$ の中から調べれば十分。これらの中から探せば、
\[ (x,y)=(4,5)\,,\,\, (5,4) \]

\syoumon{2}
計算によって、
\begin{align*}
 &(ac-bd)^2+(ad+bc)^2 \\
 =& (ac)2-2abcd+(bd)^2+(ad)^2+2abcd+(bc)^2 \\
 =& a^2c^2+a^2d^2+b^2c^2+b^2d^2 \\
 =& (a^2+b^2)(c^2+d^2)
\end{align*}
である\footnote{これは、整数$x,y$を用いて$x^2+y^2$の形で表されるような整数全体の集合は積について閉じていることを主張している。}。

\syoumon{3}
$2009=41\times49=(4^2+5^2)(0^2+7^2)$ に注意すると、(2)で$a=4$, $b=5$, $c=0$, $d=7$として
\[ (-35)^2+28^2=35^2+28^2=2009 \]
を得る。よって$n=1$の場合はよい。$n=k$ ($k\ge 1$) において$a_k^2+b_k^2=2009^k$ となるような自然数$a_k$, $b_k$ が存在したと仮定する。このとき、
\begin{align*}
 2009^{k+1}&=(35^2+28^2)(a_k^2+b_k^2) \\
 &=(35a_k-28b_k)^2+(35b_k+28a_k)^2 \\
 &=(28a_k-35b_k)^2+(28b_k+35a_k)^2
\end{align*}
である。よって$35a_k-28b_k\neq 0$であるなら、
\[ a_{k+1}=|35a_k-28b_k| \,,\,\, b_{k+1}=35b_k+28a_k \]
とし、$35a_k-28b_k=0$であるなら$28a_k-35b_k$は0とはならないことがわかるので
\[ a_k+1=|28a_k-35b_k| \,,\,\,\, b_{k+1}=28b_k+35a_k \]
とおくことによって$2009^{k+1}=a_{k*1}^2+b_{k+1}^2$ とすることができる。よって数学的帰納法により、全ての$n$で$a_n^2+b_n^2=2009^n$となるように自然数$a_n$, $b_n$を構成することができる。