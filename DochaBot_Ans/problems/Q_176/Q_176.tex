\begin{thm}{176}{\hosi 9}{suiso\_728660 様}
 「$A, B, C, D$ は相異なる」ということを、コンマで区切らずに一文字ずつ不等号`$\neq$'で繋ぎ、一つの数式で表現する際に「$A\neq B\neq C\neq D$」とするのは誤りである。なぜならば、この数式は$A=C=1$, $B=D=0$ でも成立しているといえるからである。正しくは「$A\neq B\neq C\neq D\neq A\neq C\neq B\neq D$」であり、最低でも$\neq$が7個必要である。この場合、$B\neq C$を2回参照しているためにまだ多いのではないか、と思われるかもしれないが、しかし6本以下で表すことは不可能である。\\
 では、「2018個の定数$a_k$ $(k=1, 2, \cdots, 2018)$ は相異なる」ということを同じような方法で表現するために、不等号`$\neq$'は最低でも何個必要だろうか。
\end{thm}

\verb|https://twitter.com/solove_math/status/1618203564731666473?s=20&t=35UjLXcB48qYh3KxW2BPYQ| 
を参照してください(ツイートに飛びます).  