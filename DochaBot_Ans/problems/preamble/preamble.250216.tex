% packages
\usepackage{amsmath,amssymb,amsfonts,amsthm,ascmac}
\usepackage{bm,braket,bigints}
\usepackage{fancyhdr,framed,multicol}
\usepackage{color}
\usepackage{makeidx}
\usepackage{enumitem}
\usepackage[dvipdfmx]{graphicx}
\usepackage{wrapfig}
\usepackage{itembkbx}
\usepackage{here}
\usepackage{ulem}
\usepackage{tcolorbox,cleveref}
\usepackage[%
top    = 25truemm,%
bottom = 25truemm,%
left   = 7truemm,%
right  = 7truemm]{geometry}

%% 索引を出力するための環境を調整
\makeatletter
\renewenvironment{theindex}{
    % \if@twocolumn
    %   \onecolumn\@restonecolfalse
    % \else
    %   \clearpage\@restonecoltrue
    % \fi
     \columnseprule.4pt \columnsep 2zw
    % \ifx\multicols\@undefined
    %   \def\presectionname{}\def\postsectionname{}%
    %   \twocolumn[\section*{\indexname}]%
    % \else
    %   \ifdim\textwidth<\fullwidth
    %     \setlength{\evensidemargin}{\oddsidemargin}
    %     \setlength{\textwidth}{\fullwidth}
    %     \setlength{\linewidth}{\fullwidth}
    %     \def\presectionname{}\def\postsectionname{}%
    %     \begin{multicols}{3}[\section*{\indexname}]%
    %   \else
    %     \def\presectionname{}\def\postsectionname{}%
    %     \begin{multicols}{2}[\section*{\indexname}]%
    %   \fi
    % \fi
    \@mkboth{\indexname}{\indexname}%
    \plainifnotempty % \thispagestyle{plain}
    \parindent\z@
    \parskip\z@ \@plus .3\jsc@mpt\relax
    \let\item\@idxitem
    \raggedright
    \footnotesize\narrowbaselines
   }{
  %   \ifx\multicols\@undefined
  %     \if@restonecol\onecolumn\fi
  %   \else
  %     \end{multicols}
  %   \fi
     \clearpage
   }
\makeatother

\renewcommand{\plainifnotempty}{\relax}

% Enumerate 環境
%\def\theenumi{\arabic{enumi}}
%\def\labelenumi{(\theenumi)}
\SetLabelAlign{CenterWithParen}{\hfil(\makebox[0.7em]{#1})\hfil}

% 図・数式周囲の余白調整/改ページ許可
\def\lengthparam{
\setlength{\textfloatsep}{0pt}
\setlength{\intextsep}{0pt}
\setlength{\abovedisplayskip}{0.5pt}
\setlength{\belowdisplayskip}{0.5pt}
}
\allowdisplaybreaks[4]

% 問題を記述するための環境定義
\def\probs{}
\def\none{none}
\def\breakboxparindent{0pt}
\newenvironment{thm}[3]%
{\bigskip \global\expandafter\def\expandafter\probs\expandafter{\probs \quad Q.#1} %
 \begin{breakitembox}[l]{Q.#1\hspace{0.4eM}#2\hspace{0.4eM}#3}}
{\end{breakitembox}\smallskip}

% 問題の総数---------------------------------------------------
\newcounter{qnum}
\setcounter{qnum}{300} 

% 補題を記述するための環境定義
\definecolor{gray}{rgb}{0.75,0.75,0.75}
\newenvironment{subthm}[1]%
{\def\FrameCommand{\textcolor{gray}{\vrule width 5pt} \hspace{3pt}}
 \MakeFramed {\advance\hsize-\width \FrameRestore}
 \hspace{6pt} \textbf{補題#1} \\}
{\endMakeFramed}

% ヘッダの設定
\pagestyle{fancy}
\lhead{\probs \global\def\probs{}}
\rhead{}

% 問題索引の作成を宣言
\makeindex

%ローマ数字を出力
\newcounter{rom}
\newcommand{\Rnum}[1]{\setcounter{rom}{#1} \hspace{-0.1ex}\Roman{rom}\hspace{-0.1ex}}
\newcommand{\rnum}[1]{\setcounter{rom}{#1} \hspace{-0.1ex}\roman{rom}\hspace{-0.1ex}}

% (Re)New commands
%\newcommand{\thm}{\begin{itembox}[l]}
\newcommand{\syoumon}[1]{\vspace{1ex}\hspace{1eM}{\large ({\bf #1})} \par \vspace{1ex}}
\newcommand*{\vvv}{\overrightarrow}
\newcommand{\combi}[2]{{}_{#1}\mathrm{C}_{#2}}
\newcommand{\permu}[2]{{}_{#1}\mathrm{P}_{#2}}
\newcommand{\disp}{\displaystyle}
\newcommand{\dou}{\Leftrightarrow}  %%同値
\newcommand{\naraba}{\Rightarrow}
\newcommand{\irr}{\mbox{Irr}}
\newcommand{\jap}{\mbox}
\renewcommand{\leq}{\leqq}
\renewcommand{\geq}{\geqq}
\newcommand{\nea}{\nearrow}
\newcommand{\hosi}{$\bigstar$}
\newcommand{\maru}{$\circledcirc$}
\newcommand{\marunum}[1]{\raise0.2ex\hbox{\textcircled{\footnotesize{#1}}}}
\newcommand{\sea}{\searrow}
\newcommand{\mb}{\mathbb}
\newcommand{\mr}{\mathrm}
\newcommand{\parena}[1]{\left(#1\right)}

\newcommand{\ovl}{\overline}
\newcommand{\bbox}{\quad\rule{2.2mm}{3mm}}
\newcommand{\aran}{\leftarrow}
\newcommand{\step}[2]{\textbf{Step #1.} (#2)\quad} 
\newcommand{\dothom}[1]{ \underset{#1}{\overset{\cdot}{\simeq}}  }
\newcommand{\prf}{\noindent{\large (\textbf{Proof.})\quad}} 
\newcommand{\ten}{$\bullet\quad$} %黒丸
\newcommand{\vin}{\rotatebox{90}{$\in$}}
\newcommand{\ben}{\begin{enumerate}}
\newcommand{\een}{\end{enumerate}}
\newcommand{\bcas}{\begin{cases}}
\newcommand{\ecas}{\end{cases}}
\newcommand{\spec}[1]{\mr{Spec}\parena{#1}}
%%%ふぉんと
\newcommand{\maf}{\mathfrak} %%%フラクトゥール(素イデアルなど)
\newcommand{\hana}{\mathscr} %%%花文字(圏のやつ)
\newcommand{\mac}{\mathcal} %%%関手のやつ
\newcommand{\besha}{\begin{shadebox}}
\newcommand{\ensha}{\end{shadebox}}
\newcommand{\ken}[1]{\text{\textbf{#1}}}
\newcommand{\bealn}[1]{\begin{align*}#1\end{align*}}

\newcommand{\bolm}[1]{{\mbox{\boldmath $#1$}}}
\newcommand{\tf}{\textbf}

\newcommand{\bracket}[1]{%
\[
 \left(
 \begin{tabular}{p{0.9\hsize}}
  #1
 \end{tabular}
 \right)
\]}

\renewcommand{\leq}{\leqq}
\renewcommand{\geq}{\geqq}
\renewcommand\thefootnote{*\arabic{footnote}}
\newcommand{\kuro}[1]{\mr{\textbf{#1}}}  %%複体の文字とか

%%%代数学
\newcommand{\tens}{\mathbin{\otimes}} %%%テンソル
\newcommand{\ilim}[1][]{\mathop{\varinjlim}\limits_{#1}}%%逆極限・射影極限
\newcommand{\plim}[1][]{\mathop{\varprojlim}\limits_{#1}}%順極限・帰納極限
\DeclareMathOperator{\Tr}{Tr}
\newcommand{\im}[1]{\mr{Im}\,#1}
\newcommand{\coker}[1]{\mr{Coker}\,#1}
\newcommand{\rank}[1]{\mr{rank}\, #1}
\newcommand{\tor}[2]{\mr{Tor}_{#1}^{#2}}  %%Tor{i番目}{環}
\newcommand{\ext}[2]{\mr{Ext}^{#1}_{#2} }
\renewcommand{\hom}[1]{\mr{Hom}\parena{#1}} %Hom関手
\newcommand{\norgr}{\rotatebox{90}{$\triangle$}}  %%正規部分群
\newcommand{\bep}{\begin{pmatrix}}
\newcommand{\enp}{\end{pmatrix}} %行列
\newcommand{\tr}[1]{\mr{Tr}\parena{#1}} %Trace
%\newcommand{\gal}[1]{\mr{Gal}\parena{#1}} %Galois群
\newcommand{\ord}[2]{\mr{ord}_{#1}{\parena{#2}}}
\DeclareMathOperator{\ch}{ch}
\newcommand{\trans}[1]{{}^{\mr{T}}\parena{#1}} %transpose

%%%幾何学
\newcommand{\supp}[1]{\mr{Supp}(#1)}
\newcommand{\opensub}{\overset{\mr{open}}{\subset}}
\newcommand{\sub}[1]{\overset{#1}{\subset}}




%\renewcommand{\labelenumi}{\fbox{\arabic{enumi}}}
%\renewcommand{\labelenumii}{(\arabic{enumii})}
%\renewcommand{\labelenumiii}{(\roman{enumiii})}
\renewcommand{\labelenumi}{(\arabic{enumi})}
\renewcommand{\labelenumii}{(\roman{enumii})}

\parindent=0pt






%newtheorem 
\newtheorem{supple}{補足}
\newtheorem{claim}{Claim}
\newtheorem{rem}{注意}