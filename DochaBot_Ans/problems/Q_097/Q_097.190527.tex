\begin{thm}{097}{\hosi 5}{ドイツTST (1977) 誘導追加}
 $N=4444^{4444}$の各桁の数の和を$A$、$A$の各桁の数の和を$B$、$B$の各桁の数の和を$C$とする。なお、桁は十進法で考える。
 \begin{enumerate}
  \item $A<200000$ を示せ。
  \item $C\le 12$ を示せ。
  \item $N-C$ は9の倍数であることを示せ。
  \item $C$を求めよ。
 \end{enumerate}
\end{thm}

\syoumon{1}
\[ 4444^{4444}<10000^{4444}=10^{17776} \]
より、$N$の桁の個数は17776個以下である。桁の個数が17776個以下で各桁の和が最も大きくなるのは、$\underbrace{99\dots 99}_{17776\text{個}}$なので、
\[ A\le 9\times 17776 = 159984 < 200000 \]

\syoumon{2}
200000未満の数であって、各桁の和が最も大きくなるのは199999のときで、このときの各桁の数の和は46であるから、$B\le 46$。さらに、46以下の数であって、各桁の数の和が最も大きくなるのは39のときで、このときの各桁の数の和は12であるから、$C\le 12$が示された。

\syoumon{3}
一般にある自然数$a$について、桁の個数を$k$、$10^i$の位の数字を$a_i$とおくと、
\begin{align*}
 a&=10^0a_0+10^1a_1+10^2a_2+\dots+10^{k-1}a_{k-1} \\
 &=9\times(1a_1+11a_2+\dots+\underbrace{11\dots 11}_{k-1\text{個}}a_{k-1}) \\
 &\qquad + (a_0+a_1+a_2+\dots+a_{k-1})
\end{align*}
となるから、ある自然数と、その各桁の数の和を9で割った余りはそれぞれ等しい。

したがって、$N$, $A$, $B$, $C$は、すべて9で割った余りが等しい。よって$N-C$が9の倍数であることが示された。

\syoumon{4}
$N=4444^{4444}$を9で割った余りを考える。
\[ 4444^{4444} \equiv (-2)^{3\times 1481+1} \equiv (-2)\times 1^{1482} \equiv 7 \pmod{9} \]
これによって、$C$は1以上12以下でかつ9で割った余りが7となるような数だから、$C=7$。

\footnote{編者註:~はてなブログを参照しつつ、追加された誘導に合うように一部修正・追加しました}