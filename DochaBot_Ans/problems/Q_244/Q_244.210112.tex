\begin{thm}{244}{}{}
 次の命題を証明または反証せよ。
 \begin{enumerate}
  \item $P(x)\in\mathbb{R}[x]$ が全ての整数$n$で$P(n)\in\mathbb{Z}$ ならば、$P(x)\in\mathbb{Z}[x]$
  \item $\mathbb{R}-\{0\}$ で定義された関数$f(x)$ が $f'(x)=\dfrac{1}{x}$ を満たすならば、ある定数$C$が存在して$f(x)=\log |x|+C$
  \item 有理数上0である連続関数 $f:\mathbb{R}\rightarrow\mathbb{R}$ は定数関数0である。
 \end{enumerate}
\end{thm}

\syoumon{1}
偽である。たとえば$P(x) = \dfrac{x(x-1)}{2}$とせよ。$x$が整数のとき, $x(x+1)$は必ず偶数であるが, 明らかに整数係数多項式ではない。よってこれが反例である。ちなみに, このような整数の上で常に整数値となる多項式$P(x)$は\textbf{整数値多項式}と呼ばれており, 一般に次のような形で表示できることと同値であることが知られている。
\[ P(x) = \sum_{k=0}^{N} a_kp_{k}(x) \]
ただし, $a_k\in\mb{Z}, N\geq 0,  p_{k}(x) = \dfrac{x(x-1)\cdots (x-k+1)}{k!}$.  先の解答で挙げた$P(x)$は$p_2(x)$にあたる。

\syoumon{2} \textbf{偽である!}(本問は中々の引っかけ問題である.\footnote{アンケート機能使ったら偽と答えた方が半分未満だった。} 間違えても気にしなくてよい)

たとえば次のような関数はもれなく$f'(x)=\dfrac{1}{x}$を満たすけれども, すべての実数で$f(x) = \log{|x|}+C$となるわけではない:
\begin{align*}
 f(x)=\left\{
 \begin{aligned}
  \log{x}+A \quad(x>0) \\
  \log{(-x)}+B \quad (x<0)
 \end{aligned}
 \right.
\end{align*}
つまり, 積分定数にあたるものを2カ所に与えても問題ないということである。

(\textbf{余談}) この現象は高度な話でde Rham Cohomology(ド・ラームコホモロジー)という概念に関連があり, $f(x)$が$x\neq 0$でしか定義されてないことに起因する問題である。さて, 上の$f$を天下りに与えはしたが, 次のように考えれば自然と現れるものであることが理解できる。$g(x) := f(x) - \log{|x|}$は$\mb{R}-\{ 0 \}$上の無限回微分可能な関数で, $g'(x) = 0$を満たしている。結果論的には, 
\begin{align*}
 g(x)=\left\{
 \begin{aligned}
  A \quad (x>0) \\
  B \quad (x<0)
 \end{aligned}
 \right.
\end{align*}
であるから, このようなものに限ることを示せばいいわけだが, $g'(x) = 0$ということは$g(x)$は$\mb{R}-\{ 0 \}$の各点$x$の十分近くでは定数関数である。しかし$g(x)$は$\mb{R} - \{ 0 \}$全体の定数関数にはならない。なぜなら, この$\mb{R} - \{ 0 \}$という領域が, 原点で断絶を起こしているから。より数学的には「$\mb{R} - \{ 0 \}$は二つの連結成分$\mb{R}_{>0}$と$\mb{R}_{<0}$に分割される」から。一方で原点を埋めた$\mb{R}$全体で$F'(x) = 0$ならば$F(x)$は定数であることは紛れもなく真であって, 局所定数関数$g$が二つの半直線$\mb{R}_{>0}$と$\mb{R}_{<0}$の上では定数であることも平均値の定理から容易に分かることである。だから, $g(x)$は「二つの連結成分(半直線)に実数$A,B$を割り当てるしかない」のだから, このように決まってしまうのである。

ようは「微分方程式は, ``何処''で解くかで様子が変わることがある」と言えるのだ。

一般に$C^{\infty}$多様体$M$(局所的には$\mb{R}^{n}$のようななめらかな座標が取れる空間概念)に関する``不変量''としてde Rham Cohomology $H^{\ast}(M)$というものが定義される。物としては$\mb{R}$ベクトル空間なので$0,\mb{R},\mb{R}^2,\mb{R}^3,\cdots$といった``値''をとるものであり\footnote{少しややこしいが, この``値''としての$\mb{R}$はどちらかというと「代数構造の入った$\mb{R}$」であり, 多様体としての$\mb{R}, \mb{R}-\{ 0 \}$とは少し性格が違うものではある。もちろん, $\mb{R}$は$\mb{R}$でしかないが, 群(対称的な構造を持った集合)とも思えたり多様体とも思えたりするということだ。}, 不変量なので, これを用いて多様体という図形を分類できたりできなかったりするという代物だ。$H^{\ast}(M)$は$H^{0}(M)$, $H^{1}(M),H^{2}(M),\cdots$というものたちに分解され, それらが「wedge積」と呼ばれるかけ算でひとつの環\footnote{集合であって, 積や和と``呼ばれる''諸々の条件を満たした演算が組み込まれたもの。たとえば$\mb{C},\mb{R},\mb{Z}$など。}をなすようなものである。とくにこの中の$H^{0}(M)$は「$M$上の無限回微分可能な関数であって, 微分して$0$であるようなもの全体の集合」と同じである。だから, 先と同様に「$M$上の局所定数関数全体」の集合である。$M=\mb{R}$なら, 局所定数関数は本当の定数関数しかないから$H^{0}(\mb{R}) = \mb{R}$.  一方で本問のように$H^{0}(\mb{R}-\{ 0 \}) = \mb{R}^2$である。右辺の$\mb{R}$の指数が, \textbf{$M$の連結成分の個数に等しい}のだろうと想像できると思う(厳密に示すにはやはり少し言葉が必要なのだが)。

さて, このことをふまえた上で再び次の問題に挑戦してみたいという人はいるかな?

(\textbf{問題}): $M=\mb{R}-\{ 0,1,2,\cdots, 100 \}$で定義された関数$f(x)$であって
\[ f'(x) = \dfrac{1}{x^2} + \dfrac{1}{(x-1)^2} + \dfrac{1}{(x-2)^2 }+ \cdots + \dfrac{1}{(x-100)^2} \]
を満たすものをすべて求めよ。(\textbf{Hint:} $H^{0}(M) = \mb{R}^{102}$)

de Rham Cohomologyは代数的位相幾何学という分野で登場する。京大数学科では3回生で習う程度のものである。参考書としてはRaoul Bott , Loring W.Tu のDifferential Forms in Algebraic Topologyという本が有名である(前提知識としては多様体論, 加群論の初歩, 線形代数程度, であろうか)。

\syoumon{3}
\textbf{真である。} 任意の無理数$p$を取る。このとき$p$を近似する有理数列$q_1,q_2,\cdots,$が存在する。具体的には, $p$の10進展開を小数第$n$位で切り捨てるなどとすればよい。連続性から
\[ f(p) =  f(\lim_{n\to \infty}q_n) = \lim_{n\to \infty} f(q_n) = \lim_{n\to \infty} 0 = 0 \]
なので示された。
