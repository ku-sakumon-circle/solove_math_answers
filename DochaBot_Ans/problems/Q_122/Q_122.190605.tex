\begin{thm}{122}{\hosi 7}{AwesomeMath}
 $d(n)$を$n$の正の約数の個数とする。$x_0$は2以上の整数であり、漸化式 $x_{n+1}=d(x_n)^2$~$(n=0, 1, 2, \ldots)$ により数列$\{x_n\}$を定めたとき、$\disp \lim_{n\to\infty} x_n = 9$ を示せ。
\end{thm}

$x_1=d(x_0)^2$は1より大きい平方数なので、$x_0$が1より大きい平方数の場合を考えるだけでよい。さらに、平方数の約数の個数は奇数であるから、このときに$d(x_0)^2$は1より大きい奇数の平方数となるので、$x_0$が1より大きい奇数の平方数の場合を考えるだけでよい。このとき$x_0\ge 9$である。次の補題を示す。

\begin{subthm}{122.1}
 $n$は5以上の奇数とする。このとき次が成り立つ。
 \[ d(n^2)^2<n^2 \]
 また、$n=3$では上の不等号を統合にしたものが成立する。
\end{subthm}

$n=3$で等号が成立することは、実際に計算すれば明らか。

まず、$n=p^r$の場合の主張を示す(ただし$p$は奇素数、$r$は自然数で、$(p,r)\neq (3, 1)$)。
\[ d(n^2)^2=d(p^{2r})^2=(2r+1)^2 \]
であるから、次の不等式を示せばよいことになる。
\[ p^r-2r-1>0 \]
$p=3$の場合、$r\ge 2$であるから、二項展開により、
\[ 3^r-2r-1=(2+1)^r-2r-1>\binom{r}{2}\cdot 2^r > 0 \]
となるのでよい。$r=1$も含めれば$3^r-2r-1\ge 0$といえる。$p\ge 5$の場合、$p=3$の場合に帰着して、
\[ p^r-2r^1>3^r-2r-1\ge 0 \quad\Rightarrow p^r-2r-1>0 \]
となるのでよい。よって$n=p^r$の場合が示された。

一般の5以上の奇数$n$について示す。素因数分解により
\[ n=p_1^{e_1}p_2^{e_2}\dots p_N^{e_N} \]
と表示したとする (各$p_i$は奇素数、$e_i$は自然数、$p_1<p_2<\dots <p_N$)。$N\ge 2$としてよく、$2\le i\le N$の場合には$p_i>3$であるから
\[ d(p_i^{2e_i})^2<p_i^{2e_i} \]
となる。$p_1$については$p_1=3$, $e_i=1$の場合も考慮して
\[ d(p_1^{2e_1})^2\le p_1^{2e_1} \]
となる。$N\ge 2$としたので、
\[ d(n^2)=d\left(\prod_{i=1}^N p_i^{2e_i}\right) = \prod_{i=1}^N d(p_i^{2e_i})^2<\prod_{i=1}^Np_i^{2e_i}=n^2 \]
が成り立ち、補題が示された。

漸化式から帰納的に、$x_0\ge 9$が奇数の平方数のとき、$x_1, x_2, \dots$も1より大きい奇数の平方数だから$x_n\ge 9$である。よって補題122.1から$\{x_n\}$は非増加数列であって、
\[ x_0\ge x_1\ge x_2\ge \dots \ge x_n \ge x_{n+1}\ge \dots \]
を満たす。もし$x_n>9$を満たす$n$が無限個存在するならば、この非増加性より``すべての''$n$で$x_n>9$となる。ところがこの場合は補題122.1より$x_{n+1}=d(x_n)^2<x_n$となるから、
\[ x_0>x_1>x_2>\dots > x_n>x_{n+1}>\dots \]
が成り立つ。各$x_n$は自然数だから、いくらでも小さい自然数が存在することになり矛盾する。

したがって、$x_n>9$を満たす$n$は高々有限個しか存在しない。すなわち、十分大きい自然数$M$が存在して、$n\ge M$ならば$x_n=9$であるから、題意は示された。