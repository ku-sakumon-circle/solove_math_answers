\begin{thm}{125}{\hosi 10}{BotBot07080546 様}
 十進法で末尾が0でない正の整数$n$の末尾を首位へと移し、新しい整数$f(n)$を作る。例えば、$f(334)=433$, $f(893)=389$ である。$f(5^p)=2^q$ を満たすような3以上の整数の組$(p,q)$を全て求めよ。
\end{thm}

3以上の整数$p$について、$5^p$の下3桁は$p$が奇数のときの125と偶数のときの625の2通り。よって$f(5^p)$の下2桁は12か62のいずれかである。一方で、ある整数$a, b$を用いて$2^q=100a+b$と表すことを考えると、$q\ge 3$だから、$b=2^q-100a$は4の倍数である。このことから$2^q$の下2桁は62にはなり得ない。

よって$p=2k+1$~(ただし$k$は1以上の整数) とし、$5^p=1000x+125$とおく。すると
\[ 5^{2k+1}=5\times 25^k=125(8x+1) \quad\dou\quad 25^{k-1}=8x+1 \]
を得る。$5^p$の桁数は3以上だから、$x$の桁数を$n$~(ただし$n$は0以上の整数) とおく。すると$5^p$と$f(5^p)$はともに$n+3$桁であって、
\[ f(5^p)=5\times 10^{n+2}+100x+12=2^q \]
である。これについて整理して以下を得る。
\begin{align*}
 2^q&=5\times 10^{n+2}+100\cdot\frac{25^{k-1}-1}{8}+12 \\
 \dou\quad 2^{q+1}&=10^{n+3}+25^k-1 \quad\cdots\text{(*)}
\end{align*}

一方で、$2^q$は首位が5で$n+3$桁の数であるから、
\[ 5\times 10^{n+2} \le 2^q < 6\times 10^{n+2} \]
が成り立つ。常用対数をとる。$a=\log_{10}2$, $b=\log_{10}3$とおいて、
\[ n+3-a \le aq < n+2+a+b \]
を得る。これを整理して、
\[ aq-a-b+1 < n+3 \le aq+a=a(q+1) < q+1 \]
を得た。すなわち、$n+3 < q+1$が成り立っている。

$v_2(m)$で$m$が2で割り切れる回数を表すことにすると、$v_2(2^{q+1})=q+1$, $v_2(10^{n+3})=n+3$である。さらに補題125.1\footnote{LTEの補題の特別な場合}によって
\[ v_2(25^k-1)=v_2(25-1)+v_2(k)=3+v_2(k) \]
とわかる。$n+3 < q+1$であったから、(*)が成り立つためには、$v_2(k)=n$でなければならない。よって$k=c\cdot2^n$と書ける~(ただし$c$は奇数)。

$5^p=5^{2k+1}$は$n+3$桁の数であるから、
\[ 10^{n+2} \le 5^{2k+1} < 10^{n+3} \]
が成り立つ。常用対数をとる。$a=\log_{10}2$, $b=\log_{10}3$とおいて、
\[ n+2 \le (2k+1)(1-a) < n+3 \]
を得る。これを整理して、
\[ \frac{n+a+1}{2(1-a)} \le k=c\cdot 2^n < \frac{n+a+2}{2(1-a)} \]
を得た。これを満たすような0以上の整数$n$と奇数$c$は$(n, c)=(0, 1), (1, 1)$のみであり\footnote{詳細は帰納的にするなどすれば証明できる。}、これが(*)が成り立つための必要条件である。

$(n, c)=(0, 1)$のとき$k=1$で、$10^3+25^1-1=1024=2^{10}$となるから$q=9$が(*)を満たせる。$(n, c)=(1, 1)$のとき$k=2$で、$10^4+25^2-1=10624$となるから(*)を満たす$q$は存在しない。

$k=1$のとき$p=3$であり、
\[ f(5^3)=f(125)=512=2^9 \]
となるので実際に題意を満たせる。よって求めるものは、$(p, q)=(3, 9)$。

\begin{subthm}{125.1}
 $v_2(a)$で$a$が2で割り切れる回数を表すものとする。\\
 整数$n$、奇数$x, y$であって、$x-y$が4で割り切れるとき、次が成り立つ。
 \[ v_2(x^n-y^n)=v_2(x-y)+v_2(n) \]
\end{subthm}
はじめに、$n$が奇数のときを考える。
\[ x^n-y^n=(x-y)(x^{n-1}+x^{n-2}y+\cdots+xy^{n-2}+y^{n-1}) \]
であり、$x-y$が4で割り切れるから$x\equiv y\pmod{2}$であって、
\begin{align*}
 &x^{n-1}+x^{n-2}y+\cdots+xy^{n-2}+y^{n-1} \\
 \equiv &x^{n-1}+x^{n-2}x+\cdots+xx^{n-2}+x^{n-1} \\
 \equiv & nx^{n-1} \equiv 1 \pmod{2}
\end{align*}
となるから、$v_2(n)=0$に注意して、$v_2(x^n-y^n)=v_2(x-y)+v_2(n)$が成り立つ。

一般の$n$については、奇数$k$と整数$m$を用いて$n=k\cdot 2^m$と書ける。このとき、
\[ v_2(x^n-y^n)=v_2\left((x^{2^m})^k-(y^{2^m})^k\right)=v_2(x^{2^m}-y^{2^m}) \]
とできる。ここで、$x, y$は奇数だから、$x^{2^c}+y^{2^c}\equiv 2 \pmod{4}$となるので2で一回しか割り切れない。また、
\begin{align*}
 &x^{2^m}-y^{2^m} \\
 =&(x^{2^{m-1}}-y^{2^{m-1}})(x^{2^{m-2}}-y^{2^{m-2}})\dots(x^2+y^2)(x+y)(x-y)
\end{align*}
と因数分解できるから、
\[ v_2(x^{2^m}-y^{2^m})=m+v_2(x-y) \]
となっていることがわかる。最後に、$v_2(n)=v_2(k\cdot 2^m)=m$だから、一般の$n$に対して
\[ v_2(x^n-y^n)=v_2(x-y)+v_2(n) \]
となることが示された。