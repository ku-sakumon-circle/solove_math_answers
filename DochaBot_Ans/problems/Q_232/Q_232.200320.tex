\begin{thm}{232}{}{滋賀医科大 (2016)}
 分母が奇数、分子が整数の分数で表される有理数を「控えめな有理数」と呼ぶことにする。1個以上の控えめな有理数$a_1,\dots,a_n$に対して、集合$S\langle a_1,\dots, a_n \rangle$を
\[ S\langle a_1,\dots,a_n\rangle=\left\{\sum_{i=1}^n x_ia_i \middle| \text{各$x_i$は控えめな有理数}\right\} \]
 と定める。例えば、$1\cdot\left(-\dfrac{1}{3}\right)+\dfrac{2}{3}\cdot 2=1$であるから、$1\in S\langle -\dfrac{1}{3},2\rangle$ である。
 \begin{enumerate}
  \item 控えめな有理数$a_1,\dots, a_n$が定める集合$S\langle a_1,\dots,a_n\rangle$の要素は控えめな有理数であることを示せ。
  \item 0でない控えめな有理数$a$が与えられたとき、$S\langle a\rangle=S\langle 2^t \rangle$となる0以上の整数$t$が存在することを示せ。
  \item 控えめな有理数 $a_1,\dots,a_n$ が与えられたとき、 $S\langle a_1,\dots,a_n\rangle=S\langle b\rangle$ となる控えめな有理数$b$が存在することを示せ。
  \item 2016が属する集合$S\langle a_1,\dots,a_n\rangle$はいくつあるか。ただし、$a_1,\dots,a_n$は控えめな有理数であるとし、$a_1,\dots,a_n$と$b_1,\dots,b_n$が異なっていても、$S\langle a_1,\dots,a_n\rangle =S\langle b_1,\dots,b_n\rangle$であれば、$S\langle a_1,\dots,a_n\rangle$と$S\langle b_1,\dots,b_n\rangle$は1つの集合として数える。
 \end{enumerate}
\end{thm}

ここに解答を記述。