\begin{thm}{256}{\hosi 6}{$\cot$の部分分数分解}
 $\tau\in\mathbb{C}-\mathbb{Z}$として、次を示したい。
 \[ \frac{1}{\tau}+\sum_{d=1}^\infty\left(\frac{1}{\tau-d}+\frac{1}{\tau+d}\right)=\pi\cos\pi\tau \]
 \begin{enumerate}
  \item 左辺は$\mathbb{C}-\mathbb{Z}$で絶対収束することを示せ。
  \item 十分大きい$N\in\mathbb{N}$として$R=N+\dfrac{1}{2}$とする。$\pm R\pm Ri$を4頂点とする正方形経路$C$を用いて次を示せ。
	\[ \text{(右辺)}-\text{(左辺)}=\lim_{N\to\infty}\frac{1}{2\pi i}\oint_C\!\frac{\pi\cos\pi\zeta}{\zeta-\tau} \,d\zeta \]
  \item $\disp \frac{1}{2\pi i}\oint_C \frac{\pi\cos\pi\zeta}{\zeta} \,d\zeta=0$を利用して与式を示せ。
 \end{enumerate}
\end{thm}
ここに解答を記述。