\begin{thm}{168}{\hosi ?}{自作}
 $d(k)$で$k$の正約数の個数、$\phi(k)$でオイラーの$\phi$関数、$\sigma(k)$で$k$の正約数の総和、$\pi(k)$で$k$以下の素数の約数とする。 \\
 $n$が2以上の整数のとき、
 \[ \sum_{k=2}^n \left\lfloor \frac{d(k)+\phi(k)}{\sigma(k)} \right\rfloor = \pi(n) \]
 が成立することを示せ。
\end{thm}

ここに解答を記述。