\begin{thm}{147}{\hosi 7}{BotBot07080546 様}
 点$(0,19)$を通り、傾きが整数値の直線が、放物線$y=x^2$によって切り取られる線分の長さもまた、整数値をとるという。その長さを求めよ。
\end{thm}

直線を$y=nx+19$~$(n\in\mathbb{Z})$とし、
\[ x^2=nx+19 \quad\dou\quad x=\frac{n\pm\sqrt{n^2+76}}{2} \]
の小さい方を$\alpha$、大きい方を$\beta$とすれば、切り取られる線分の長さは$(\alpha, \alpha^2)$と$(\beta, \beta^2)$の距離なので、
\[ \sqrt{(\beta-\alpha)^2+(\beta^2-\alpha^2)^2} = \sqrt{(n^2+1)(n^2+76)} \]
となり、$(n^2+1)(n^2+76)$は平方数である。

$n$が奇数のとき、$n^2\equiv 1 \pmod{4}$で、$(n^2+1)(n^2+76)\equiv 2\cdot 77 \equiv 2 \pmod{4}$となり、平方剰余でないから、$n$は偶数。

ユークリッドの互除法より、$n^2+76$と$n^2+1$の最大公約数は、75と$n^2+1$の最大公約数と等しい。この最大公約数を$g$とする。$g=1$のとき、$n^1+1$も$n^2+76$も同時に平方数でなければならない。しかし、$n^2$が平方数である一方で$n^2+1$が平方数になるような整数$n$は存在しない。よって$g\neq 1$。

$75=3\times 5^2$より$g$は、$g=3, 5, 15, 25, 75$が考えられる。$g$が3の倍数のときは、
\[ n^2+1\equiv 0 \quad\dou\quad n^2\equiv -1 \pmod{3} \]
となって不適。よって$g$は5または25である。

(i)~$g=5$のとき、5と互いに素な自然数$k$を用いて$n^2+1=5k$とし、
\[ (n^2+1)(n^2+76)=5k(5k+75)=25(k^2+15k) \]
より、$k^2+15k$が平方数ならよい。$k\ge 50$だと、$(k+7)^2<k^2+15k<(k+8)^2$が成り立つから、$k<50$でなければならない。よって$n^2<249$から、$|n|\ge 15$。$n^2+1\equiv 0 \pmod{5}$より$n\equiv 2, 3 \pmod{5}$でかつ、$n$は偶数より、$n=\pm2, \pm8, \pm12$。このうち適するのは$n=\pm 2$のときで、長さは20。

(ii)~$g=25$のとき、25と互いに素な自然数$k$を用いて$n^2+1=25k$とし、
\[ (n^2+1)(n^2+76)=25k(25k+75)=25^2(k^2+3k) \]
より、$k^2+3k$が平方数ならよい。しかし、$k\ge 1$のもとで$(k+1)^2<k^2+3k<(k+2)^2$が成り立つから、平方数にはなり得ない。

以上より、線分の長さは20。