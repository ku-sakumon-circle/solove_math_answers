\begin{thm}{055}{\hosi 6}{学コン 2009/11/02}
 $x,y$座標がともに有理数の点を`有理点'とする。
 \begin{enumerate}
  \item 点$(a,b)$ を直線$y=mx$ に関して対称移動した点の座標を求めよ。
  \item 原点を中心とする円C上に有理点が1つでもあれば、C上に有理点が無数にあることを示せ。
 \end{enumerate}
\end{thm}

\syoumon{1}
移動した先の点の座標を$(s, t)$とおく。移動前後の点を結んだ線分は直線$y=mx$に直交し、かつこれら2点の中点は直線$y=mx$上にあるから、
\begin{align*}
 \left\{
 \begin{aligned}
  \frac{t-b}{s-a}&=-\frac{1}{m} \\
  \frac{t+b}{2}&=m\frac{s+a}{2}
 \end{aligned}\right.
\end{align*}
これらを$s, t$について解けば求める点の座標を得る。
\[ (s, t) = \left( \frac{1-m^2}{1+m^2}a+\frac{2m}{1+m^2}b \,,\,\, \frac{2m}{1+m^2}a-\frac{1-m^2}{1+m^2}b \right) \]

\syoumon{2}
原点$\mr{O}(0,0)$と、有理点$\mr{A}(a, b)$をとる。原点を中心とし、点$\mr{A}$を通る円をCとする。有理数$m$によって直線$y=mx$をおき、有理点$\mr{A}$をこの直線に関して対称移動した点を$\mr{S}(s, t)$とおく。(1)の結果によって、
\[ (s, t) = \left( \frac{1-m^2}{1+m^2}a+\frac{2m}{1+m^2}b \,,\,\, \frac{2m}{1+m^2}a-\frac{1-m^2}{1+m^2}b \right) \]
であるが、$a, b, m$がすべて有理数であるから、$s, t$はともに有理数となる。したがって点$\mr{S}$は有理点である。ここで、
\[ s^2+t^2=\left[\left(\frac{1-m^2}{1+m^2}\right)^2+\left(\frac{2m}{1+m^2}\right)^2\right](a^2+b^2)=a^2+b^2 \]
であるから、$\mr{OA}=\mr{OS}$が成り立つ。つまり、点$\mr{S}$は円C上にある。有理数$m$は任意にとることができるから、円C上に点$\mr{A}$と異なる有理点を無数に作ることができる。よって題意は示された。