\begin{thm}{191}{\hosi 7}{自作}
 三辺の長さが整数値である直角三角形$T$の内接円の半径$r$が素数であるとする。以下のの問に答えよ。
 \begin{enumerate}
  \item $T$の面積は偶数であることを示せ。
  \item $T$の斜辺の長さが偶数のとき、$r$を求めよ。また、$T$の各辺の長さを求めよ。
 \end{enumerate}
\end{thm}

\syoumon{1}
3辺を$a, b, c$とし、$c$が斜辺とする。三平方の定理から$c^2=a^2+b^2$で、面積は$\dfrac{ab}{2}$。$ab$が4で割り切れることを示せばよい。$a, b$がともに偶数ならば明らか。ともに奇数の場合、$a^2+b^2\equiv 2 \pmod{8}$であり、一方明らかに$c$は偶数で、このとき$c^2\equiv 0, 4 \pmod{8}$となるから矛盾となる。$a, b$の偶奇が異なる場合、偶数のものが4の倍数でないとすると、$a^2+b^2\equiv 5 \pmod{8}$であり、一方$c$は明らかに奇数で、このとき$c^2\equiv 1 \pmod{8}$となるから矛盾である。

以上のことから、$a, b$の一方は4の倍数であり、よって$ab$も4の倍数である。よって題意は示された。

\syoumon{2}
$c=a+b-2r$である。面積について、
\[ \frac{ab}{2}=\frac{r}{2}(a+b+c)=r(c+r) \]
である。$r$が奇数とすると、いま$c$が偶数であるから、$r$と$r+c$がともに奇数である。一方$\dfrac{ab}{2}$は偶数であるから不適。よって$r$は偶数である。さらに$r$は素数であったから$r=2$。このとき、$c=a+b-4$で、
\[ \frac{ab}{2}=a+b+c \quad\dou\quad (a-4)(b-4)=8 \quad \text{(*)}\]
を得る。$c$が偶数なので$a, b$の偶奇が一致するが、(1)での議論からこれらはともに偶数であることがわかる。よって(*)を満たすのは$(a, b)=(6, 8), (8, 6)$のみである。このとき三平方の定理から$c=10$。

以上より、$r=2$、各辺の長さは6, 8, 10。