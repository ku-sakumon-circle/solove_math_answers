\begin{thm}{229}{\hosi 7}{学コン 2019-6-5}
 $n$を正の整数、$e$を自然対数の底とする。
 \begin{enumerate}
  \item 関数$\disp f_n(x)=1+x+\frac{x^2}{2!}+\frac{x^3}{3!}+\dots +\frac{x^n}{n!}$ とするとき、$x>0$において、$0<e^x-f_n(x)<\dfrac{x^{n+1}e^x}{(n+1)!}$であることを示せ。
  \item $\disp \lim_{n\to\infty} \cos\frac{2\pi e n!}{3}=-\frac{1}{2}$ を示せ。
 \end{enumerate}
\end{thm}

\syoumon{1}

$g_n(x)=1-e^{-x}f_n(x)$とする。
\[ g'_n(x)=-e^{-x}\bigl(f'_n(x)-f_n(x)\bigr)=\frac{e^{-x}x^n}{n!}>0 \]
より$g_n(x)$は増加するから、$g_n(x)>g_n(0)=0$となる。$e^x>0$をかけることで$e^x-f_n(x)>0$を得る。

$h_n(x)=\dfrac{x^{n+1}}{(n+1)!}-g_n(x)$とする。
\[ h'_n(x)=\frac{x^n}{n!}-\frac{e^{-x}x^n}{n!}=\frac{x^n}{n!}(1-e^{-x})>0 \]
より$h_n(x)$は増加するから、$h_n(x)>h_n(0)=0$となる。$e^x$をかけることで
\[ 0<e^xh_n(x) \quad\dou\quad e^x-f_n(x)<\frac{x^{n+1}e^x}{(n+1)!} \]
を得る。以上より題意は示された。

\syoumon{2}

(1)のから$x=1$とし$n!$をかけると、
\[ 0<en!-\left(n!+\frac{n!}{1!}+\frac{n!}{2!}+\dots +\frac{n!}{(n-1)!}+1\right) < \frac{e}{n+1} \]
となる。ここで、
\[ m_n=n!+\frac{n!}{1!}+\frac{n!}{2!}+\dots +\frac{n!}{(n-1)!}+1 \]
とおく。$m_n$は整数である。さらに$n\ge 3$のとき ($n\to\infty$を考えるので3以上としてよい)、
\begin{align*}
 m_n&= 1+\permu{n}{1}+\permu{n}{2}+\sum_{k=3}^n\permu{n}{k} \\
 &= (n^2+1)+\sum_{k=3}^nk!\combi{n}{k} \\
 \dou\qquad \frac{m_n}{3}&= \frac{n^2+1}{3}+\sum_{k=3}^n\frac{k!}{3}\combi{n}{k}
\end{align*}
であり、$\disp \sum_{k=3}^n\frac{k!}{3}\combi{n}{k}$は整数である。これを$S_n$とおく。したがって以下を得る。
\[ \frac{n^2+1}{3}\cdot 2\pi < \frac{2\pi e n!}{3}-2S_n\pi < \frac{n^2+1}{3}\cdot 2\pi+\frac{2e\pi}{3(n+1)} \]

次に$\cos\left(\dfrac{n^2+1}{3}\cdot 2\pi\right)$の値について考える。$n\equiv 1, 2 \pmod{3}$のとき、$n^2+1\equiv 2$であるため、$n^2+1=3N+2$となる整数$N$が存在し、
\[ \cos\left(\frac{n^2+1}{3}\cdot 2\pi\right)=\cos\left(2N\pi+\frac{4}{3}\pi\right)=\cos\frac{4}{3}\pi=-\frac{1}{2} \]
である。また$n\equiv 0 \pmod{3}$のとき、$n^2+1=3N+1$となる整数$N$が存在し、
\[ \cos\left(\frac{n^2+1}{3}\cdot 2\pi\right)=\cos\left(2N\pi+\frac{2}{3}\pi\right)=\cos\frac{2}{3}\pi=-\frac{1}{2} \]
である。つまり任意の整数$n$に対して$\cos\left(\dfrac{n^2+1}{3}\cdot 2\pi\right)=-\dfrac{1}{2}$である ($\cdots$ *)。

さてこれらのことは、
\[ \frac{n^2+1}{3}\cdot 2\pi \equiv \frac{2}{3}\pi \,,\,\, \frac{4}{3}\pi \pmod{2\pi} \]
を意味しているから、$n$を$0<\dfrac{2e\pi}{3(n+1)}<\dfrac{\pi}{3}$を満たすように大きくとれば、$\dfrac{n^2+1}{3}\cdot 2\pi<x<\dfrac{n^2+1}{3}+\dfrac{2e\pi}{3(n+1)}$の範囲で$\cos x$は単調に増加または減少する。したがって、
\[ \cos\left(\frac{n^2+1}{3}\cdot 2\pi\right) \,,\,\, \cos\left(\frac{n^2+1}{3}\cdot 2\pi+\frac{2e\pi}{3(n+1)}\right) \]
のうち、小さいほうを$L_n$、大きいほうを$H_n$とおけば、
\[ L_n \le \cos\left(\frac{2\pi e n!}{3}-2S_n\pi\right)=\cos\frac{2\pi e n!}{3} \le H_n \quad\cdots\text{(#)} \]
である。ただし中央の等号において$S_n$が整数であることを用いた。

$L_n$, $H_n$が$-\dfrac{1}{2}$に近づくことを論じる。

$n\equiv 1, 2 \pmod{3}$のとき、
\begin{align*}
 \cos\left(\frac{n^2+1}{3}\cdot 2\pi+\frac{2e\pi}{3(n+1)}\right)&=\cos\left(\frac{4}{3}\pi+\frac{2e\pi}{3(n+1)}\right) \\
 &=\cos\left(\frac{2}{3}\pi-\frac{2e\pi}{3(n+1)}\right)
\end{align*}
$n\equiv 0 \pmod{3}$のとき、
\begin{align*}
 \cos\left(\frac{n^2+1}{3}\cdot 2\pi+\frac{2e\pi}{3(n+1)}\right)=\cos\left(\frac{2}{3}\pi+\frac{2e\pi}{3(n+1)}\right)
\end{align*}
であり、$0<\dfrac{2e\pi}{3(n+1)}=\dfrac{\pi}{3}$を満たすようにとっているので、
\[ \dfrac{2}{3}\pi-\dfrac{2e\pi}{3(n+1)}\le x\le \dfrac{2}{3}\pi+\dfrac{2e\pi}{3(n+1)} \]
の範囲で$\cos x$は単調減少する。したがって
\begin{align*}
 \cos\left(\frac{2}{3}\pi+\frac{2e\pi}{3(n+1)}\right)&\le \cos\left(\frac{n^2+1}{3}\cdot 2\pi+\frac{2e\pi}{3(n+1)}\right) \\
 &\le \cos\left(\frac{2}{3}\pi-\frac{2e\pi}{3(n+1)}\right)
\end{align*}
となり、この式から挟み撃ちの原理により
\[ \lim_{n\to\infty}\cos\left(\frac{n^2+1}{3}\cdot 2\pi+\frac{2e\pi}{3(n+1)}\right)=\cos\frac{2}{3}\pi=-\frac{1}{2} \]
である。よって(*)とあわせれば$\disp \lim_{n\to\infty}L_n=\lim_{n\to\infty}H_n=-\frac{1}{2}$となるから、式(#)においてはさみうちの原理により
\[ \lim_{n\to\infty} \cos\frac{2\pi e n!}{3}=-\frac{1}{2} \]