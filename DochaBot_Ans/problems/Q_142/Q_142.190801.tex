\begin{thm}{142}{}{}
 \begin{enumerate}
  \item 19で割って14余る平方数は存在するか。 \hosi 4
  \item $\dfrac{2a^2-1}{b^2+2}$ が整数になる整数の組$(a,b)$は存在するか。 \hosi 9
  \item $\dfrac{3^n-1}{2^n-1}$ が整数になる$n\in\mathbb{N}$ を全て求めよ。 \hosi 9
  \item $p!+p$ が平方数になる素数$p$を全て求めよ。 \hosi 11
 \end{enumerate}
\end{thm}

本問では、平方剰余のルジャンドル記号を利用する。

\syoumon{1}
次の計算により存在しない。
\begin{align*}
 \left(\frac{14}{19}\right) &= \left(\frac{-5}{19}\right) = \left(\frac{-1}{19}\right)\left(\frac{5}{19}\right) \\
 &= (-1)^{\frac{19-1}{2}}\left(\frac{5}{19}\right) \quad\quad\text{(第一補充則より)} \\
 &= (-1)\left[(-1)^{\frac{19-1}{2}\cdot\frac{5-1}{2}}\left(\frac{19}{5}\right)\right] \quad\quad\text{(相互法則より)} \\
 &=(-1)\left(\frac{19}{5}\right) = (-1)\left(\frac{4}{5}\right) = -1
\end{align*}

\syoumon{2}
分子は奇数なので、分母は奇数である必要があり、$b$は奇数となる。このとき$b^2+2\equiv\pmod{8}$なので、$b^2+2$の素因数$p$であって、$o\equiv\pm 3 \pmod{8}$なるものが存在する。

$b^2+2$は$p$で割れるから、$2a^-1$もこの$p$で割り切れなければならない。すなわち$2a^2-1\equiv 0\pmod{p}$の必要がある。
\[ 2a^2\equiv 1\equiv 1+p \quad\dou\quad a^2\equiv \frac{p+1}{2} \pmod{p} \]
となるような$a$を考えなければならない。

平方剰余の第二補充法則より、
\[ \left(\frac{2}{p}\right)=\left(-1\right)^{\frac{p^2-1}{8}} \]
さらに平方剰余に乗法性があることから、
\[ \left(\frac{2}{p}\right)\left(\frac{\frac{p+1}{2}}{p}\right)=\left(\frac{p+1}{p}\right)=\left(\frac{1}{p}\right)=1 \]
したがって、
\[ \left(\frac{\frac{p+1}{2}}{p}\right)=\left(\frac{2}{p}\right)=\left(-1\right)^{\frac{p^2-1}{8}} \]

$p=8k+3$~($k\ge 0$) のとき、$\dfrac{p^2-1}{8}=(2k+1)(4k+1)$となり奇数である。一方$p=8k-3$~($k\ge 1$) のとき、$\dfrac{p^2-1}{8}=(2k-1)(4k-1)$となり奇数である。よっていずれの場合も$\left(-1\right)^\frac{p^2-1}{8}=-1$で、$\left(\dfrac{\frac{p+1}{2}}{p}\right)=-1$となる。これは$a^2\equiv \dfrac{p+1}{2} \pmod{p}$となる整数$a$が存在しないことを示すので、分子は$p$で割り切れず、$\dfrac{2a^2-1}{b^2+2}$は整数とならない。

\syoumon{3}
$n=1$は明らかに解。$n\geq 2$とするとき, $2^n-1$には奇素因数が存在するが,それを任意に一つ選んで$p$とする。$p|2^n-1$かつ, $p|3^n-1$でなければならない。$K=\dfrac{3^n-1}{2^n-1}$が整数のとき, $2^n-1$は3で割れない。よって$n$は奇数である。そのとき, 
\[3^n-1\equiv 0 (\mbox{mod} p)\dou 3^{n+1}\equiv 3 (\mbox{mod} p)\]
となり, $n+1$が偶数であるため$3$は法$p$における平方剰余でなければならない。つまり
\[\left(\dfrac{3}{p}\right)=1\]
である。一方, 相互法則 から
\[\left(\dfrac{3}{p}\right)=(-1)^{\frac{p-1}{2}}\left(\dfrac{p}{3}\right)\]
である。つまり,\\
・$p|2^n-1$ が $p\equiv 1$ (mod 4)ならば, $p\equiv 1$ (mod 3)\\
・$p|2^n-1$ が $p\equiv 3$ (mod 4)ならば, $p\equiv 2$ (mod 3)\\
である。そこで, $2^n-1$を素因数分解したときに現れる$4k+1$型素因数のを$S$, $4k+3$型素因数の個数を$T$とすると,
\[2^n-1\equiv 1^S\cdot 3^{T} (\mbox{mod} 4)\]
である。$n\geq 2$なので $2^n-1\equiv -1$ (mod 4)であり, $T$は奇数となる。\\
前に述べたことより, $2^n-1$の$3k+1$型素因数の個数も$S$で, $3k+2$型素因数の個数は$T$になるから,
\[2^n-1\equiv 1^S\cdot 2^T \equiv 2 (\mbox{mod} 3) (\because T\mbox{は奇数})\]
すると$2^n\equiv 0$ (mod 3)になり矛盾する。以上より~$n=1$.

\syoumon{4}
$p=2$および$p=3$の場合、
\[ 2!+2=4=2^2 \,,\,\, 3!+3=9=3^2 \]
となるのでよい。

$p>4$の場合を考える。ある整数$m$を用いて$p!+p=m^2$と書けたとする。このとき、
\[ m^2\equiv p!+p\equiv p \pmod{4} \]
が成り立つ。$p$は明らかに奇数でありかつ4を法とした平方剰余であることから、$p\equiv 1 \pmod{4}$でなければならない。$p=5$の場合、$5!+5=125$より不適。$p=7$は$7\equiv 3\pmod{4}$より不適。

$p>8$の場合を考える。$p\equiv 1\pmod{4}$であったから、$p\equiv 1, 5\pmod{8}$のいずれかである。
\[ m^2\equiv p!+p\equiv p \pmod{8} \]
より、$p$は8を法とした平方剰余でなければならないので、$p\equiv 1\pmod{8}$のみが適する。

さて、$q<p$を満たす任意の奇素数$q$について、
\[ m^2\equiv p!+p\equiv p \pmod{q} \]
が成り立つ。これによって、
\begin{align*}
 \left(\frac{p}{q}\right)&=1 \\
 \Rightarrow\quad \left(\frac{p}{q}\right)\left(\frac{q}{p}\right)&=\left(\frac{q}{p}\right) \\
 \Rightarrow\quad (-1)^{\frac{(p-1)(q-1)}{2}}&=\left(\frac{q}{p}\right)
\end{align*}
が得られる。$p\equiv 1\pmod{4}$であったから、$(-1)^{\frac{(p-1)(q-1)}{2}}=1$。すなわち$\left(\dfrac{q}{p}\right)=1$である。また、$p\equiv 1\pmod{8}$かつ$p+1$は偶数だから
\[ \left(\frac{2}{p}\right)=(-1)^{\frac{p^2-1}{8}}=(-1)^{\frac{(p-1)(p+1)}{8}}=1 \]
である。したがって、$p$より小さい全ての素数は$p$を法とした平方剰余となる。さらに、平方剰余には乗法性があることから、$p$より小さい任意の整数は$p$より小さい素数の積で表せるので、$p$より小さい任意の整数は$p$を法とした平方剰余であるとわかる。しかしこのことは、平方剰余は$\dfrac{p-1}{2}$個しか存在しないことに矛盾する。

以上のことから、$p!+p$が平方数となるのは、$p=2, 3$。