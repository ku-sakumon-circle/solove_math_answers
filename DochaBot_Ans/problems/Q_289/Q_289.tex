\begin{thm}{289}{\hosi 10-高}{自作, 2023-3月号大数宿題}
$f(x)$は次数$1$以上の有理数係数多項式とし, 最高次係数は正, ある整数$m$で$f(m)$が整数であるとする. このとき, 正の整数$n$であって, $f(n)$が正の整数でかつ$10$進法で表したときの桁として$0$が現れるようなものが無限個存在することを証明せよ. 
\end{thm}

 \begin{enumerate}
        \item $f(x) = \dfrac{1}{a} g(x)$と表す(ただし$a$は正の整数, $g(x)$は整数係数多項式). $f(m)\in \Z$なので$g(m) \in a\Z$である. よって$g(ax + m)$の係数は\textbf{$a$の倍数係数}なので\footnote{$g(x) = \sum_{i=1}^{d} n_i (x-m)^{i} + g(m)$と表せば($n_i\in \Z )$), $g(ax + m) = \sum_{i=1}^{d}a^i n_i x^i+ g(m)$であるため, }, $f(ax + m)$は$x$に関する整数係数多項式である. 以降$F(x):= f(ax+m)$とかく. 
        \item $F$の最高次係数は正なので, 十分大きい$N\in\mb{N}$に対して, $x\geq 0$で常に$F(x+N) = f(ax + aN + m) \geq 0$かつ狭義単調増加としてよい(ここで$\deg{F} \geq 1$を用いた).
        \item $F(N) = f(aN + m) = a_0$とし, $F(x+N) - F(N) = xh(x)$ とおくと$h$は整数係数である. (2)より$k>0$が整数のとき$h(k) > 0$である. 

    \end{enumerate}
      (1),(2)より$a_0 = f(aN + m)$は非負整数である. これが10進法で$k$桁\textbf{以下} ($a_0 < 10^{k}$)であるとする. このとき$F(10^{k+1} + N)$を考えれば, $F(10^{k+1} + N) = 10^{k+1}h(10^{k+1}) + a_0 \geq 10^{k+1}\cdot 1 + a_0$だからこれは$k+2$桁以上で, 

 \[
 F(10^{k+1} + N) =  10^{k+1} h(10^{k+1}) + a_0 \equiv a_0 \pmod{10^{k+1}}
 \]
 である. これは$k+2$桁以上の正の整数$f(a\cdot 10^{k+1} + aN + m)$の下$k+1$桁が$a_0$であることを意味し, $a_0$は$k$桁以下だから$k+1$桁目は0になる. $k$の取り方は無数にあるので$n = a\cdot 10^{k+1} + aN + m$ ($k$は$a_0 < 10^k$となる任意の正の整数) とすれば無限個の$n$で$f(n)$は桁に0が現れる. 