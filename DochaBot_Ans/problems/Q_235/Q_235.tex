\begin{thm}{235}{\hosi 7}{京都府立医科大 (2020)}
 $n$は自然数とする。変量$x$についての$2n$個のデータの値を$x_i$ ($1\le i\le 2n$) とし、変量$y$についての$2n$個のデータの値を$y_i$ ($1\le i\le 2n$) とする。$k$は $1\le k\le 2n-1$を満たす整数とする。変量$x$と$y$の$2n$個の組を、
 \begin{align*}
  (x_i,y_i)=\left\{
  \begin{aligned}
   &(i,i+2n-k) & &(1\le i\le k) \\
   &(i,i-k) & &(k+1\le i\le 2n)
  \end{aligned}
  \right.
 \end{align*}
 で与える。$x$と$y$の共分散を$s_{xy}$とし、相関係数を$r$とする。
 \begin{enumerate}
  \item $s_{xy}$を$k$と$n$を用いて表せ。
  \item $r=0$となる$k$は存在しないことを証明せよ。
  \item 自然数$n$に対して$r$を最小にする$k$を取り、そのときの$r$を$r_n$と表す。$\disp \lim_{n\to\infty} r_n$ を求めよ。
 \end{enumerate}
\end{thm}

\syoumon{1}
$\{x_i\}$の平均$\overline{x}$は、
\[ \overline{x}=\frac{1}{2n}\sum_{i=1}^{2n}x_i=\frac{1}{2n}\sum_{i=1}^{2n}i=\frac{1}{2n}\cdot\frac{2n(2n+1)}{2}=n+\frac{1}{2} \]
である。$\{y_i\}_{i=1}^{2n}$は$\{1,\dots,2n\}$の置換であることに注意すると、この平均について$\overline{y}=n+\dfrac{1}{2}$である。よって$s_{xy}$は、
\footnotesize
\begin{align*}
 s_{xy}&=\frac{1}{2n}\sum_{i=1}^{2n}(x_i-\overline{x})(y_i-\overline{y}) \\
 &=\frac{1}{2n}\sum_{i=1}^{2n}(x_iy_i)-\frac{1}{2n}\left(n+\frac{1}{2}\right)\sum_{i=1}^{2n}(x_i+y_i)+\frac{1}{2n}\sum_{i=1}^{2n}\left(n+\frac{1}{2}\right)^2 \\
 &=\frac{1}{2n}\sum_{i=1}^{2n}(x_iy_i)-\left(n+\frac{1}{2}\right)(\overline{x}+\overline{y})+\left(n+\frac{1}{2}\right)^2 \\
 &=\frac{1}{2n}\left[\sum_{i=1}^ki(i+2n-k)+\sum_{i=k+1}^{2n}i(i-k)\right]-\left(n+\frac{1}{2}\right)^2 \\
 &=\frac{1}{2n}\left[\sum_{i=1}^{2n}i(i-k)+2n\sum_{i=1}^ki\right]-\left(n+\frac{1}{2}\right)^2 \\
 &=\frac{1}{2n}\left[\frac{2n(2n+1)(4n+1)}{6}-k\cdot\frac{2n(2n+1)}{2}\right]+\frac{k(k+1)}{2}-\left(n+\frac{1}{2}\right)^2 \\
 &=\frac{(2n+1)(4n+1)}{6}+k\left(\frac{k+1}{2}-\frac{2n+1}{2}\right)-\frac{(2n+1)(3n+\frac{3}{2})}{6} \\
 &=\frac{4n^2-1}{12}+\frac{1}{2}k(k-2n)
\end{align*}
\normalsize

\syoumon{2}
$x,y$の分散をそれぞれ$s_x, s_y$として、$r=\dfrac{s_{xy}}{\sqrt{s_xs_y}}$である\footnote{編者註: 相関係数の分母は標準偏差を用いる旨改訂}。よって$1\le k\le 2n$に対して$s_{xy}\neq 0$を言えばよい。(1)の結果から、このような$k$が存在したとすると
\begin{align*}
 & s_{xy}=\frac{4n^2-1}{12}+\frac{1}{2}k(k-2n)=0 \\
 \dou\quad & 3k^2-6nk+\frac{1}{2}(4n^2-1)=0 \\
 \therefore\quad & k=n\pm \frac{1}{6}\sqrt{12n^2+6}
\end{align*}
であるが、
\begin{align*}
 6k-6n&=\pm\sqrt{12n^2+6} \,(\in\mathbb{Z}) \\
 \Rightarrow \quad (6k-6n)^2&=12n^2+6 \\
 \Rightarrow \quad 0&\equiv 2 \pmod{4}
\end{align*}
となって矛盾。よって$r\neq 0$。

\syoumon{3}
まず、$k$が動いても$\{x_i\}$, $\{y_i\}$は常に$\{1,\dots,2n\}$の置換であるから$s_x$, $s_y$は不変であり、その値は$z_i=i$ ($i=1,\dots,2n$)という変量$z$の分散に等しい。よって
\begin{align*}
 s_x=s_y&=\frac{1}{2n}\sum_{i=1}^{2n}i^2-\left(\frac{1}{2n}\sum_{i=1}^{2n}i\right)^2 \\
 &=\frac{1}{2n}\cdot\frac{2n(2n+1)(4n+1)}{6}-\left(\frac{1}{2n}\cdot\frac{2n(2n+1)}{2}\right)^2 \\
 &=\frac{1}{3}n^2-\frac{1}{12}
\end{align*}
一方、$s_{xy}=\dfrac{4n^2-1}{12}+\dfrac{(k-n)^2}{2}-\dfrac{n^2}{2}$ より、これは$k=n$で最小となる。このとき$s_{xy}=-\frac{2n^2+1}{12}$となる。以上より、
\begin{align*}
 \lim_{n\to\infty}r_n=\frac{-\frac{2n^2+1}{12}}{\frac{1}{3}n^2-\frac{1}{12}}=-\frac{1}{2} 
\end{align*}