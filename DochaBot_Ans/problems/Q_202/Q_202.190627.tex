\begin{thm}{202}{\hosi 1}{関西医大 (2017)}
 次の5つの命題がある。

 \begin{tabular}{cc}
  ~~~命題1: & 「命題1は真」\\
  ~~~命題2: & 「命題1は偽」\\
  ~~~命題3: & 「命題2は真」\\
  ~~~命題4: & 「命題3は偽」\\
  ~~~命題5: & 「命題4は偽」\\
 \end{tabular}

 この5つの命題の中で、2つの命題が真で、3つの命題が偽であるとき、真の命題はどれか答えよ。
\end{thm}

命題1が真であることを仮定すると、命題2は偽、命題3は偽、命題4は真、命題5は偽、と順に真偽が定まる。

一方、命題1が偽であることを仮定すると、命題2は真、命題3は真、命題4は偽、命題5は真、と順に真偽が定まる。

このうち、5つの命題の中で2つの命題が真となるのは、命題1が真の場合であるから、真の命題は、命題1と命題4である。\footnote{編者の感想: 命の字がゲシュタルト崩壊しました。}