\begin{thm}{182}{\hosi 5}{弘前大 理工 後期 (2017) 改題}
 漸化式 $a_1=2$, $a_{n+1}=1+a_1a_2\cdots a_n$ によって定められる数列$\{a_n\}$について、以下の問に答えよ。
 \begin{enumerate}
  \item $a_1$, $a_2$, $a_3$, $a_4$, $a_5$ を求めよ。また、$a_5$は素数でないことを示せ。
  \item $n$が2以上の整数のとき、$a_n$は$a_k$ $(k=1, 2, \dots, n-1)$ と互いに素であることを示せ。
  \item (2)を用いて、素数が無限に存在することを示せ。
  \item $a_{n+1}-1=a_n(a_n-1)$ $(n\ge 2)$ を示せ。
  \item $\disp \sum_{k=1}^\infty \frac{1}{a_k}$ を求めよ。
 \end{enumerate}
\end{thm}

\syoumon{1}
次の計算によって求まる。
\begin{align*}
 a_1&=2 \\
 a_2&=1+a_1=3 \\
 a_3&=1+a_1a_2=7 \\
 a_4&=1+a_1a_2a_3=43 \\
 a_5&=1+a_1a_2a_3a_4=1807=13\times 139
\end{align*}

\syoumon{2}
\[ a_1a_2\dots a_{n-1}=a_kb_k \quad (k=1, 2, \dots n-1) \]
と書くと、$a_n\times 1-a_kb_k=1$となる。一般に、整数$a, b$に対して$ax+by=1$となる整数$x, y$が存在することと、$a, b$が互いに素であることは同値であるから、$a_n, a_k$は互いに素である。

\syoumon{3}
(2)より、$k=1, 2, \dots, n-1$として、$a_n$は$a_k$と互いに素であって、かつ明らかに$a_n>1$であるから、$a_k$のいずれの約数でもないが$a_n$の約数ではあるような素数$p_n$がひとつ取れる。数列$\{p_n\}$はどの2つの項も一致しない素数の列であるから、素数は無限個あることが示される。

\syoumon{4}
$n\ge 2$では、$a_n-1=a_1a_2\dots a_{n-1}$となるから、
\[ a_{n+1}-1=(a_1a_2\dots a_{n-1})a_n=(a_n-1)a_n \]
よりよい\footnote{$n=1$の場合でも、$a_2-1=2=2\times 1=a_1(a_1-1)$より成り立っている。}。

\syoumon{5}
まず、$a_n\ge n+1$を示す。$n=1$は明らかによい。$m\ge 1$として、$n=1, 2, \dots, m$で$a_m\ge m+1$ならば、
\[ a_{m+1}=1+a_1a_2\dots a_m \ge 1+a_m \ge 1+(m+1) \]
を得る。よって帰納的に$a_n\ge n+1$である。

(4)の結果より、
\[ \frac{1}{a_{k+1}}=\frac{1}{a_k(a_k-1)}=\frac{1}{a_k-1}-\frac{1}{a_k} \quad\dou\quad \frac{1}{a_k}=\frac{1}{a_k-1}-\frac{1}{a_{k+1}-1} \]
である。これについて総和をとれば、
\[ \sum_{k=1}^n\frac{1}{a_k}=\frac{1}{a_1-1}-\frac{1}{a_{n+1}-1}=1-\frac{1}{a_{n+1}-1} \]
となる。$n\to\infty$の極限を考えると、$a_n\ge n+1$から$a_n\to\infty$だから、
\[ \sum_{k=1}^\infty\frac{1}{a_k}=\lim_{n\to\infty}\left(1-\frac{1}{a_{n+1}-1}\right)=1 \]