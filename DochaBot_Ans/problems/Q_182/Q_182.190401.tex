\begin{thm}{182}{\hosi 5}{弘前大 理工 後期 (2017) 改題}
 漸化式 $a_1=2$, $a_{n+1}=1+a_1a_2\cdots a_n$ によって定められる数列$\{a_n\}$について、以下の問に答えよ。
 \begin{enumerate}
  \item $a_1$, $a_2$, $a_3$, $a_4$, $a_5$ を求めよ。また、$a_5$は素数でないことを示せ。
  \item $n$が2以上の整数のとき、$a_n$は$a_k$ $(k=1, 2, \cdots, n-1)$ と互いに素であることを示せ。
  \item (2)を用いて、素数が無限に存在することを示せ。
  \item $a_{n+1}-1=a_n(a_n-1)$ $(n\ge 2)$ を示せ。
  \item $\disp \sum_{k=1}^\infty \frac{1}{a_k}$ を求めよ。
 \end{enumerate}
\end{thm}

\syoumon{1}
ここに解答を記述。

\syoumon{2}
ここに解答を記述。

\syoumon{3}
ここに解答を記述。

\syoumon{4}
ここに解答を記述。

\syoumon{5}
ここに解答を記述。