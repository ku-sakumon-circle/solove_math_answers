\begin{thm}{124}{\hosi 3}{組み合わせ論の精選}
$n$を1より大きな奇数とする。${}_{n}\mbox{C}_1,{}_{n}\mbox{C}_2,\cdots, {}_{n}\mbox{C}_{\frac{n-1}{2}}$の中には奇数が奇数個あることを示せ。 
\end{thm}

\begin{proof}
${}_{n}\mbox{C}_1+{}_{n}\mbox{C}_2+\cdots +{}_{n}\mbox{C}_{\frac{n-1}{2}}=K$とすると, ${}_n\mbox{C}_i={}_n\mbox{C}_{n-i}$から $2K=\displaystyle\sum_{i=1}^{n-1}={}_n\mbox{C}_i =2^n-2$ となる。ゆえに$K=2^{n-1}-1$で, $n>1$のため$K$は奇数である。仮に奇数が偶数個あるなら, その中で偶数個ある奇数のみの総和をとると偶数である。偶数のみの総和は明らかに偶数なので, $K$はそれらをあわせて偶数になる。これは矛盾。よって,奇数が奇数個ある。
\end{proof}

