\begin{thm}{057}{\hosi 4}{京大 (2006)}
 $Q(x)$を2次式とする。整式$P(x)$は$Q(x)$では割り切れないが、$\{P(x)\}^2$は$Q(x)$で割り切れるという。このとき2次方程式$Q(x)=0$は重解を持つことを示せ。
\end{thm}

$P(x)$を$Q(x)$で割った商を$A(x)$、余りを$R(x)$とおくと、$R(x)$は高々1次式であって、
\[ P(x)=A(x)Q(x)+R(x) \]
と書ける。これを用いて、
\[ \{P(x)\}^2=\{A(x)\}^2\{Q(x)\}^2+2A(x)Q(x)R(x)+\{R(x)\}^2 \]
となるから、$\{P(x)\}^2$が$Q(x)$で割り切れることは、$\{R(x)\}^2$が$Q(x)$で割り切れることに等しい。

$R(x)$が0次式とする\footnote{0次式は定義上$R=0$を含まない。一方で$P(x)$は$Q(x)$で割り切れないから$R=0$は除かれるべきである。}と、$\{R(x)\}^2$も0次式だから$Q(x)$で割り切れず不適。よって$R(x)$は1次式。これはすなわち$Q(x)$は1次式$R(x)$の二乗に因数分解されることを意味するから、2次方程式$Q(x)=0$は重解を持つ。