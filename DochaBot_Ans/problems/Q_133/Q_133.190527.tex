\begin{thm}{133}{\hosi 7\maru}{京大実戦理系 (2016)}
 $\angle\mr{C}$が$90^\circ$の直角三角形ABCに対して、$\angle\mr{A}$の二等分線とBCの交点をPとする。ABとACの長さが素数であるとき、BPおよびCPの長さを求めよ。
\end{thm}

CP$=p$, BP$=q$, AB$=m$, AC$=n$ とする。角の二等分線の性質から、
\[ n:m=p:q \,\dou\, mp=nq \]
であり、$m$は斜辺なので$m>n$。よって$p<q$となる。これによって$p, q$は異なる素数だから互いに素である。よって$n$は$p$の倍数、$m$は$q$の倍数であるから、自然数$k$を用いて$n=kp$, $m=kq$とすると、三平方の定理により
\[ (p+q)^2=m^2-n^2=k^2(q^2-p^2) \]
$p+q>0$で両辺割って整理すると、$k^2-1=\dfrac{2p}{q-p}$を得る。左辺は正だから$k\ge 2$。右辺は整数だから、$q-p$は$2p$の正の約数である。$p$と$q$が互いに素であることから、$q-p=1, 2$のいずれか。

$q-p=1$を満たすのは$p=2$、$q=3$の場合だが、$4=k^2-1$となり$k$が自然数とならず不適。

$q-p=2$の場合、$p=k^2-1=(k-1)(k+1)$になる。$k\ge 3$であると$p$が2つの2以上の整数の積となり素数でなく不適。よって$k=2$で、このとき$p=3$, $q=5$である。

以上より、BP$=5$, CP$=3$