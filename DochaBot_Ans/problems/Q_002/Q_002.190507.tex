\begin{thm}{002}{}{}
 正の整数$n$の正の約数の積を$P(n)$とする。
 \begin{enumerate}
  \item $P(n)=24^{240}$ となる $n$ は1つだけある。 $n$ を求めよ。 \hosi 3 (JMO予選 2012)
  \item 実は$P(n)=P(m)$ならば $n=m$ である。このことを示せ。 \hosi ? (自作)
 \end{enumerate}
\end{thm}

\syoumon{1}
$P(n)=24^{240}=2^{720} 3^{240}$であることから、$n$は2と3のみを素因数に持つ。そこで$n=2^a 3^b$とおく。$n$の約数のうち、2で$k$回 ($1\le k\le a$) 割り切れるものは、
\[ 2^k,\, 2^k 3,\, 2^k 3^2,\, \cdots,\, 2^k 3^{b-1},\, 2^k 3^b \]
の$b+1$個ある。よって、$P(n)$の中には、$2^k$が$b+1$個掛け算されているから、$P(n)$の2の指数について、
\[ \sum_{k=1}^a k(b+1) = \frac{a(a+1)(b+1)}{2} = 720 \]
が成り立つことがわかる。$P(n)$の3の指数についても同様に考えて、
\[ \sum_{k=1}^b k(a+1) = \frac{b(b+1)(a+1)}{2} = 240 \]
を得る。第1式を第2式で除すことで、$a=3b$を得る。これを用いれば$b$について
\[ b(b+1)(3b+1)=480 \]
が得られる。480の約数のうち3で割って1余るものは4, 16, 40, 160であることを踏まえれば、$b=5$を解の一つとして得る。題意を満たす$n$はただ一つであることがわかっているから、これが$n$における3の指数である。さらに$a=15$とわかるから、
\[ n=2^{15} 3^5=7962624 \]

\syoumon{2}
$a,b\in\mathbb{N}$に対して, $b$ が $a$ で割り切れるということ,すなわち $a$ が $b$ の正の約数であることを "$a\mid b$" と表す。\\
これを用いて,$n$のすべての正の約数の総積$P(n)$を $\displaystyle\prod_{d\mid n}d$ と表す。 また, $d\mid n$ならば $\dfrac{n}{d}\mid n$であるから, $P(n)=\displaystyle\prod_{d\mid n}\dfrac{n}{d}$ とも表せる。これより, $D_n$で$n$の正の約数の個数を表すとすると,
\begin{eqnarray*}
P(n)^2 = \left(\displaystyle\prod_{d\mid n}d\right)\left(\displaystyle\prod_{d\mid n}\dfrac{n}{d}\right) = \displaystyle\prod_{d\mid n}d\cdot \dfrac{n}{d} = \displaystyle\prod_{d\mid n}n = n^{D_n}
\end{eqnarray*}
となる。したがって,$P(n)>0, n^{\frac{D_n}{2}}>0$ であることから $P(n)=n^{\frac{D_n}{2}}$ と表される。\\
\\
よって, $n^{\frac{D_n}{2}}=m^{\frac{D_m}{2}}$ ならば $n=m$ であることを示せばよい。ただし, 負の値になることがないので, $n^{D_n}=m^{D_m}$ で考えればよい。\\
ある素数$p$に対して $p\mid n\Leftrightarrow p\mid m$ なので, $n$と$m$の素因数の種類は一致する。次に, $n$が素数$p$で割ることができる最大の回数を$M_p(n)$ と置く。このとき, $n^{D_n}=m^{D_m}$ であることは, $n$ (または $m$)を割り切る任意の素数$p$ に対して (左辺と右辺をそれぞれ何回割ることができるかを考えて) $D_nM_p(n)=D_mM_p(m)$ が成り立つことと同値である。\\
\\
対称性より, $D_m\le D_n$ とする。このとき,
\begin{eqnarray*} 
D_nM_p(n)=D_mM_p(m)\le D_nM_p(m) \Rightarrow M_p(n)\le M_p(m)
\end{eqnarray*}
が成立する。 よって, 任意の$n$を割り切る素数$p$ に関して, $m$ のほうが, $n$ と同じ回数, あるいは$n$ より多い回数だけ$p$ で割ることができる。よって, $\dfrac{m}{n}$ は整数であり, すなわち$n\mid m$ である。\\
$m$ が $n$ の倍数ならば,  $D_n\le D_m$ であり, $D_m\le D_n\le D_m$ だから, $D_n=D_m$ となる。よって, $n^{D_n}=m^{D_m}=m^{D_n}$ であり, $n,m\in\mathbb{N}$ だから $n=m$ が示された。  $\Box$
