\begin{thm}{064}{\hosi 8}{Z会京大理系実戦演習}
 $a$を実数の定数とする。$x$の方程式 $x^5-x^4+ax^3+ax^2-x+1=0$ の5つの解のうち、少なくとも2つの解が一致するとき、$a$の値と、一致する解を求めよ。
\end{thm}

\syoumon{解法1}

与式は$x=-1$を解に持つから\footnote{一般に奇数次の相反方程式は$x=-1$を解に持つ。}、
\[ x^5-x^4+ax^3+ax^2-x+1=(x+1)\bigl[x^4-2x^3+(a+2)x^2-2x+1\bigr] \]
である。$f(x)=x^4-2x^3+(a+2)x^2-2x+1$とおく。

(i)~$f(-1)=8+a$だから、$a=-8$とすれば$f(x)$が$(x+1)$で割り切れ、
\[ (x+1)f(x)=(x+1)^3(x^2-4x+1) \]
だから$x=-1$が重解である。

(ii)~$f(-1)\neq 0$かつ$f(1)=0$の場合を考える。$f(1)=a$より、これは$a=0$で、このとき$f(x)=(x-1)^2(x^2+1)$となるから、与方程式は$x=1$を重解に持つ。

(iii)~$f(-1)\neq 0$かつ$f(1)=0$の場合を考える。これは$a\neq 0, 8$。また$f(0)=1$であるから、$x=0, \pm 1$は方程式$f(x)=0$の解でない。この方程式の重解を$p$、それ以外の2解を$q, r$とおくと、
\[ f(x)=(x-p)^2(x-q)(x-r) \quad \text{ただし$p, q, r\neq 0, \pm 1$} \]
と書ける。一方で、$x^4f\left(\dfrac{1}{x}\right)=f(x)$であるから、
\[ f(t)=0 \quad\dou\quad f\left(\frac{1}{t}\right)=0 \]
である。$p\neq p^{-1}$なので\footnote{$p$は$0$でも$\pm 1$でもないから}、$q, r$のいずれかが$p$に等しいが、$p^{-1}=q$としてよい。よって
\[ f(x)=(x-p)^2(x-p^{-1})(x-r) \]
となるが、解と係数の関係により$ppp^{-1}r=1$なので、$r=p^{-1}$。つまり
\[ f(x)=(x-p)^2(x-p^{-1})^2=x^4-2x^3+(a+2)x^2-2x+1 \]
である。$x^3$の係数を比較して、$2(p+p^{-1})=2$である。続いて$x^2$の係数を比較すると$p^2+4+p^{-2}=a+2$であるから、
\[ a=p^2+2+p^{-2}=(p+p^{-1})^2=1 \]
となる。このとき、
\[ f(x)=\left(x-\frac{1+\sqrt{-3}}{2}\right)^2\left(x-\frac{1-\sqrt{-3}}{2}\right)^2 \]

以上より、
\begin{align*}
 \left\{
 \begin{aligned}
  a&=0 & \text{一致する解は}&\, 1 \\
  a&=1 & \text{一致する解は}&\, \frac{1+\sqrt{-3}}{2}\,,\,\, \frac{1-\sqrt{-3}}{2} \\
  a&=-8 & \text{一致する解は}&\, -1
 \end{aligned}
 \right.
\end{align*}

\syoumon{解法2}

与方程式について、
\[ (x+1)\bigl[(x-1)^2(x^2+1)+ax^2\bigr]=0 \]
と整理できるから、明らかに$x=-1$を解の1つに持ち、残りの4解については、
\[ f(x)=(x-1)^2(x^2+1)+ax^2 \]
の振る舞いを調べればよい。

$a=0$のとき、与方程式は
\[ (x-1)^2(x+1)(x^2+1)=0 \,\dou\, x=1 \,,\,\, -1 \,,\,\, \pm i \]
と解ける。このことから一致する解は$1$である。

$a>0$のとき、常に$f(x)>0$であるから、与方程式は$x=-1$以外に虚数解を4つもつ。ここで$a$は実数であるから、この4つの虚数解は、2組の共役な虚数である。互いに共役な虚数同士は等しくなりえないから、少なくとも2つの解が一致するためには、この2組が一致しなければならない。このことから、
\[ f(x)=(x-1)^2(x^2+1)+ax^2 = (x^2+sx+t)^2 \]
と因数分解することを考える。これの係数を比較することで$(s, t, a)=(-1, 1, 1)$を得る。実際に$a=1$のとき、与方程式は、
\[ (x+1)(x^2-x+1)^2=0 \,\dou\, x=-1\,,\,\, \frac{1\pm\sqrt{3}i}{2} \]
と解けて、一致する解は$\dfrac{1\pm\sqrt{3}i}{2}$と求まる。

$a<0$のとき、$f(0)=1$, $f(1)=a<0$から、中間値の定理によって、$0<\alpha<1$なる実数$\alpha$によって$f(\alpha)=0$となることがわかる。また、
\[\lim_{x\to\infty} f(x) = \lim_{x\to\infty} x^4\left[\left(1-\frac{1}{x}\right)^2\left(1+\frac{1}{x^2}\right)^2+\frac{a}{x^2}\right] =+\infty\]
から、十分大きい実数$M>0$に対して、$f(M)>0$がいえる。これより、$f(1)=a<0$と$f(M)>0$から、中間値の定理によって、$\beta>1$なる実数$\beta$によって$f(\beta)=0$となることがわかる。これらより、$a<0$の場合には、方程式$f(x)=0$の虚数解は高々2つであるから、重解は存在するなら実数である。

一般に、方程式$f(x)=0$の実数解$x=t$が重解\footnote{少なくとも2次の}であることは、$f(t)=0$かつ$f'(t)=0$と同値である。
\begin{align*}
 f(t)&=(t-1)^2(t^2+1)+at^2=0 \\
 f'(t)&=2(t-1)(2t^2-t+1)+2at=0
\end{align*}
これらから$a$を消去し整理して、
\[ (t-1)(t+1)(t^2-t+1)=0 \quad\cdots\text{(*)}\]
が、$x=t$が方程式$f(x)=0$の重解となるための条件である。ここで、先述の$\alpha$, $\beta$は、明らかに(*)を満たさないから、$x=\alpha, \beta$は重解になり得ない。

ここまでのことから、$a<0$の場合に、与方程式が重解を持つためには、方程式$f(x)=0$が$x=-1$を解に持たなければならない。
\[ f(-1)=8+a=0 \quad\dou\quad a=-8 \]
よって、$a=-8$のとき、一致する解として$-1$が得られる。

以上によって、求めるものは
\begin{align*}
 \left\{
 \begin{aligned}
  a&=0\,\,\text{のとき、一致する解は}\,\, x=1 \\
  a&=1\,\,\text{のとき、一致する解は}\,\, x=\frac{1\pm\sqrt{3}i}{2} \\
  a&=-8\,\,\text{のとき、一致する解は}\,\, x=-1
 \end{aligned}
 \right.
\end{align*}