\begin{thm}{036}{\hosi 6}{九州大 理系 2018}
 1から4までの数字を1つずつ書いた4枚のカードが箱に入っている。箱の中から1枚カードを取り出してもとに戻す試行を$n$回続けて行い、$n$回目までに取り出したカードの数字のすべての積を$X$とする。$X$を4で割った余りが0, 1, 2, 3である確率をそれぞれ求めよ。
\end{thm}

$n$回目時点で$X$を$4$で割った余りが0,2である確率を$a_n,b_n$とおき, $X$が奇数である確率を$s_n$とする. $a_1 = b_1 = \dfrac{1}{4}$, $s_1 = \dfrac{1}{2}$, $a_n+b_n+s_n = 1$が成り立つ. $X$が奇数であるのは, 常に奇数を出し続けるときなので$s_n = \dfrac{1}{2^n}$である. \par 
$X\equiv 2\pmod{4}$なのは$n$回目までに2を一回出しそれ以外奇数を取り出す確率なので$b_n = \dfrac{2^{n-1}\cdot 1\cdot {}_n\mr{C}_1}{4^n} = \dfrac{n}{2^{n+1}} $である. よって, 
\[P(X\equiv 0) = a_n = 1-b_n - s_n = 1 - \dfrac{n+2}{2^{n+1}}\]
\[P(X\equiv 2) = b_n = \dfrac{n}{2^{n+1}}\]
$c_n = P(X\equiv 1)$, $d_n = P(X\equiv 3)$とおく. $c_1=d_1 = dfrac{1}{4}$, $c_n + d_n = s_n$である. $X\equiv 1$であるような取り出し方$W$は, 最後に取り出した奇数を別の奇数に入れ替えた取り出し方$W'$ ($W$で最後が1なら$W'$で最後を3にする)とペアにすれば, $X\equiv 3$であるような取り出し方と同じ個数だけあるといえるので$c_n = d_n = \dfrac{1}{2}s_n = \dfrac{1}{2^{n+1}}$\\
\\
\fbox{$c_n,d_n$を求める別方針}
 $n+1$回目で$X\equiv 1$であるのは, $n$回目で$X\equiv \pm 1$から$\pm 1$を取り出すときである(複合同順). よって
\[c_{n+1} = \dfrac{c_n+d_n}{4}\]
様に考えると$d_{n+1} = \dfrac{c_n+d_n}{4}$である. よって帰納的に$c_n  = d_n$なので, $c_n = d_n = \dfrac{1}{2}s_n = \dfrac{1}{2^{n+1}}$. 
