\begin{thm}{277}{\hosi 8}{2022東工大}
    $a,b,c$は自然数で最大公約数が1であるとする. 
    \begin{enumerate}
    \item $a+b+c$, $ab+bc+ca$, $abc$の最大公約数は1であることを示せ. 
    \item $a+b+c$, $a^2+b^2+c^2$, $a^3+b^3+c^3$の最大公約数となりうる正の整数を全て求めよ.
    \end{enumerate}
\end{thm}
$s=a+b+c$, $t=ab + bc + ca$, $u = abc$とおく. $n\mb{Z}$は$n$の倍数全体の集合とする. 
\syoumon{1}
$g=\mr{gcd}(s,t,u)$とおく. $a,b,c$は$X^3 - sX^2 + tX - u = 0$の3つの解なので, $X=a,b,c$とおいた式が成り立つ. 
それらの $s,t,u\in g\mb{Z}$なので$a^3,b^3,c^3\in g\mb{Z}$. 
よって$g$がある素数で割れれば, $a^3,b^3,c^3$はその素数で割れるので互いに素であることに反する. よって$g=1$.\\  
\syoumon{2}
$a^i+b^i+c^i$($i=1,2,3$)の最大公約数を$G(a,b,c)$とおく. $G(1,2,2) = 1, G(1,1,2) = 2, G(1,1,1) = 3, G(1,1,4) = 6$である. 
$a^2 + b^2 + c^2 = s^2 - 2t$ , $a^3 + b^3 + c^3 = s(a^2 + b^2 + c^2) - t(a+b+c) + 3u = s^3 - 3st  +  3u$である. 
よって$G(a,b,c)$を割り切る素数$p$を取ると, $p|s, p|2t, p|3u$である. $p|s$なので, $p\not | t$または$p\not | u$である.
 よって$p|2$または$p|3$である. よって$p=2,3$なので, $G(a,b,c)$は2,3以外の素因数を持たない. \par
$G(a,b,c)$は4の倍数にも9の倍数にもならないことを証明する. $G(a,b,c)$が4の倍数だとすると, $a,b,c$は互いに素なので, $a,b,c$のうち奇数が二つである.
 このとき$a^2 + b^2 + c^2\equiv 2\pmod{4}$なので矛盾. 
 $a,b,c$が9の倍数だとすると, $x^3 \equiv 0,\pm 1\pmod{9}$ ($x\in \mb{Z}$)なので,
  $a^3 + b^3 + c^3$が9で割れるためには「$a,b,c$のうち3の倍数が1つだけである」ことが分かる. 
  すると$a^2 + b^2 + c^2 \equiv 2\pmod{3}$なので矛盾. よって$G(a,b,c) \in \set{1,2,3,6}$が分かり,
   $G(a,b,c) = \bolm{1,2,3,6}$ですべてであることが分かる. 
