\begin{thm}{144}{\hosi 6}{自作 DMP4.5th}
 $P(x)$は$n$次の整式である ($n\ge 1$)。方程式$P(x)=0$が異なる$n$個の実数解を持つとき、方程式$P(x)=P'(x)$は異なる$n$個の実数解をもつことを証明せよ。
\end{thm}

$n=1$のとき、$P(x)=ax+b$とおけば、
\[ P(x)=P'(x) \,\dou\, ax+b=a \]
となるので明らかに1つの実数解を持ち、題意を満たしている。以降$n\ge 2$について考える。

方程式$P(x)=0$の解を、$a_1, a_2, \dots , a_n$とし、$P(x)$の$x^n$の係数を$k$とすれば、
\[ P(x)=k(x-a_1)(x-a_2)\dots(x-a_n) \]
と表される。このとき、
\begin{align*}
 P'(x)&=k(x-a_1)' (x-a_2)\dots(x-a_n) \\
 &\qquad +k(x-a_1)(x-a_2)' \dots (x-a_n)+\dots \\
 &\qquad +k(x-a_1)(x-a_2)\dots(x-a_n)'
\end{align*}
となる。$i=1,2,\dots, n$に対して、$x=a_i$を代入すると、
\[ P(a_i)=k(a_i-a_1)(a_i-a_2)\dots(a_i-a_{i-1})(a_i-a_{i+1})\dots(a_i-a_n) \neq 0 \]
となるから、$a_i$は方程式$P(x)=P'(x)$の解ではない。よって$x\neq a_i$とすると$P(x)\neq 0$であって、
\[ P(x)=P'(x) \,\dou\, \frac{P'(x)}{P(x)}=\frac{1}{x-a_1}+\frac{1}{x-a_2}+\dots+\frac{1}{x-a_n}=1 \]
となる。以降、$a_1<a_2<\dots<a_n$として考える。また、$\disp Q(x)=\sum_{i=1}^n \frac{1}{x-a_i}$とおく。すると、
\[ Q'(x)=-\sum_{i=1}^n \frac{1}{(x-a_i)^2} < 0 \]
より$Q(x)$は常に単調減少する。$x<a_1$では明らかに$Q(x)<0$なので、区間$(-\infty, a_1)$に$Q(x)=1$の解は無い。続いて、区間$(a_i, a_{i+1})$ を考える (ただし$i=1, 2, \dots, n-1$)。$Q(x)$は単調減少であって、
\[ \lim_{x\to a_i+0} Q(x)=\infty, \qquad \lim_{x\to a_{i+1}-0} Q(x)=-infty \]
より、中間値の定理から、この区間において$Q(x)=1$の解がただ一つ存在する。最後に、区間$(a_n, infty)$では、$Q(x)$は単調減少であって、
\[ \lim_{x\to a_n+0} Q(x)=\infty, \qquad \lim_{x\to\infty} Q(x)=0 \]
より、同様にの区間において$Q(x)=1$の解がただ一つ存在する。

以上より、区間$(a_i, a_{i+1})$ ($i=1,2,\dots, n-1$) と区間$(a_n, \infty)$の中に解が1つずつ存在し、これらは明らかに相異なるので、題意は示された。