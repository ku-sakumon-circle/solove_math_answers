\begin{thm}{247}{\hosi ?}{元ネタ: 京大院試 (英語)}
 0でない実数係数多項式 $a_nx^n+\cdots +a_1x+a_0$ の次数は$a_n\neq 0$ならば$n$とされる。では0の次数はどのように定めるのが自然か。
\end{thm}

あえて答えがいくつかに分かれそうな書き方をした。自由に論じてもらえば良いが, もっともらしいことを言わなければならない。\par
$f\in \mb{R}[x]\setminus{\{ 0 \}}$に対して, その次数を$\deg{f}$と書くことにしよう。このとき, 次が成り立つことは明らかである。
\[\deg{(fg)} = \deg{f} + \deg{g}\quad (f,g\in \mb{R}[x]\setminus{\{ 0 \}})\]
仮に$\deg{0}$を定義しようものなら, この公式を$f=0, g=0$の場合にも拡張したいと考えたくなるものである。では形式的に$g=0$としてみると
\[\deg{0} = \deg{f} + \deg{0}\]
となり, $\deg{0}$は普通の実数のような数では定義しづらい。そこで, 極限計算的には正しい$\infty + x = \infty$, $-\infty + x = -\infty$のような式を思い出し, $\deg{0} = \infty$または$\deg{0} = -\infty$とするとこの公式は「ある意味では崩れない」感じがする。\par 
さて, 現段階では$\pm \infty$のどちらも妥当である。そこで, 他に$\deg{f}$にまつわる公式がないかを探し, その公式を$\deg{0}$が入る場合に拡張出来ないかということを再び考えてみよう。たとえば次のようなものがある。
\[\deg{(f+g)} \leq  \max{\{ \deg{f}, \deg{g}\} } \quad (f,g,f+g\neq 0)\]
形式的に$f+g = 0$であるとすると
\[\deg{0} \leq \max{\{ \deg{f}, \deg{(-f)}\} } = \deg{f}\]
となる。この式を見るに, $\deg{0} = \infty$とはしづらい。$\infty \leq \deg{f}$は奇妙だからである。一方で$\deg{0} = -\infty$はこの点に関しては問題がない!\par 
その他, $\deg{0} = -\infty$とするのが有用そうであると思える点はある。たとえば, 
\begin{enumerate}
 \item 剰余の定理を思いだそう。$g(x)\neq 0$とする。$f$の$g$による割り算を実行した$f(x) = g(x)Q(x) + R(x)$ ($Q(x)$は整式, $R(x) = 0$または$R\neq 0$かつ$\deg{R} < \deg{g}$) における$R$の条件は, $\deg{0} = -\infty$, $-\infty < r$ ($r\in \mb{R}$)と取り決めるならば「$\deg{R} < \deg{g}$」とだけ書いてもよいだろう。  
\end{enumerate}

$\deg{0} = \infty$としたい理由もいくつか考えられる。
\begin{enumerate}
 \item $f\neq 0$のとき, $\deg{f}$は方程式$f(x) = 0$の複素数解の個数である。だから$\deg{0} = \infty$が相応しい。
 \item etc ... 
\end{enumerate}
個人的には, これを理由にするのはいささか微妙なのではと思う。確かに(1)のような事実はある。しかしながらそれは$\deg{f}$の「生の情報」と言えるかどうかは少し怪しい。つまり, 多項式の零点の個数は$\deg{f}$と同じだという事実が先走り, それが$\deg{f}$の本質だ, と主張できるかどうかは微妙ではないかと思うのである。まるで「1ってなんですか?」という哲学的な問いに対して「乗法単位元です」と答えているかのようだ\footnote{実際に見たことがある。}。多項式は関数とも見れるが, 関数と思わない視点もある。$\deg{f}$は多項式を関数と思わずとも定まる概念である。だから, 零点の個数(関数の視点による情報)はやや$\deg{f}$の概念から離れているのではないかと思う。\par
結局, 「$\deg{}$というのは$\mb{R}[x]\setminus{\{ 0 \}}$に対して整数値を与える写像にすぎない」という立場からすれば, 零点の個数と話を結びつけるのは自然とは言いがたい。一方で$\deg{0} = -\infty$を採用するに至った方法では, $\deg{}$という写像に関する話の範囲内である。だから, より生の$\deg{}$を使って話が展開できているのではないだろうか。
