\begin{thm}{011}{\hosi 8}{}
 有理数$a,b,c$について、$a^2\pm (a+b+c), b^2\pm (a+b+c), c^2\pm (a+b+c)$の全てが平方数であるとする。このとき、$a,b,c$はいずれも整数であり、かつ$a+b+c=0$を満たすことを証明せよ。
\end{thm}

$a^2\pm (a+b+c)=s_\pm^2$とおく。ここで$s_\pm$は整数である。辺々引くと
\[ 2(a+b+c)=s_+^2-s_-^2 \]
を得る。右辺は明らかに整数であるから、$2(a+b+c)$は整数である。
\[ 2a^2\pm 2(a+b+c)=2s_\pm^2 \quad\dou\quad 2a^2=2s_\pm^2\mp 2(a+b+c) \]
から、$2a^2$が整数であることがわかる。これを$n$とおく。有理数$a$を、正整数$j$と、これと互いに素な整数$i$を用いて$a=\dfrac{i}{j}$と表すと、
\[ 2a^2=n \quad\dou\quad 2i^2=nj^2 \]
である。$n$が奇数であることを仮定する。このとき$j$が偶数で無ければならず、$j=2j'$とおく。すると$i^2=2nj'^2$となって$i$も偶数で無ければならない。これは$i, j$が互いに素であるとしたことに矛盾する。したがって$n$は偶数でなければならない。このとき$n=2n'$とおけば、$i^2=n'j^2$である。これは$i^2$が$j^2$の倍数であることを示しているが、$i, j$は互いに素であるから、$j=1$のみが適する。すなわち、$a=\dfrac{i}{j}$は整数である。同様の議論によって$b, c$も整数であることが示される。

$a+b+c\neq 0$を仮定する。ここで、対称性から$a^2\le b^2\le c^2$を考えれば十分である。このとき少なくとも$c^2>0$である。加えて、$a+b+c>0$として一般性を失わない。なぜなら、$a+b+c<0$の場合には、$a, b, c$を$-a, -b, -c$と取り換えればよいから。
\[ c^2 < c^2+(a+b+c) \le c^2+3|c| < c^2+4|c|+4=(|c|+2)^2 \]
であるから、$c^2+(a+b+c)=(|c|+1)^2$である。よって、$a+b+c=2|c|+1$。このとき、
\[ c^2-(a+b+c)=c^2-2|c|-1=(|c|-1)^2-2=k^2 \]
について、$(|c|+k-1)(|c|-k-1)=2$を得るが、これを満たす整数$|c|$, $k$は存在しない。したがって$c^2-(a+b+c)$は平方数になり得ず、$a+b+c>0$は不適であるから$a+b+c=0$であることが示された。
