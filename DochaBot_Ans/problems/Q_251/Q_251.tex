\begin{thm}{251}{\hosi 2}{}
 $xy$平面に$y^2=x(x+1)^2$によって表されるグラフを図示せよ。(注意せよ!)\\
\end{thm}
(Hint) グラフは, 曲線と1点になる. \\
\\
(\textbf{注意}) 図形は, 「1点」と「連結な部分」の和になっている。この図形と似たものとして楕円曲線というものがある(注:本問のグラフは楕円曲線とは言わない). 楕円曲線の 実平面上での一般的な式は, 
\[y^2 = x^3+ ax + b\quad (a,b\in \mb{R}, 4a^3 - 27b^2 \neq 0)\]
という方程式である。最後の条件が成り立てばグラフ全体は一つにつながっているか, 「閉じた曲線」と一つの曲線の和集合になっているのだが, 逆にこの条件を外すと楕円曲線が変な形になるのである. この問では, この「閉じた曲線」が1点に縮んでいった瞬間の$a,b$を本問で選んできたというわけである. そういうわけで, この1点を非常に見落としやすいという, そういう問題である.