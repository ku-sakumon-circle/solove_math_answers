\begin{thm}{273}{}{${}_{p}\mr{C}_{k}$}
次を示せ. 
\ben 
\item 素数$p$, $1\leq k\leq p-1$なる整数$k$に対し${}_{p}\mr{C}_{k}$は$p$の倍数 (\hosi 1) 
\item $n\in\mb{N}$に対し$n^{p} - n$は$p$の倍数 (\hosi 2)
\item すべての整数$1\leq k\leq n-1$に対して${}_{n}\mr{C}_{k}$が$n$で割れるような2以上の整数$n$は素数. (\hosi ?)
\een 

\end{thm} 
\syoumon{1} 
$p! = k! (p-k)! {}_{p}\mr{C}_{k}$であり, 左辺は$p$の倍数である. $p$は素数だから, $1$から$p-1$までの整数はいずれも$p$の倍数ではない. よって, $p$が素数であることから$k! (p-k)!$も$p$の倍数ではない. よって${}_{p}\mr{C}_{k}$が$p$の倍数でなければならない. 

\syoumon{2} (いわゆるEulerの定理) \\
数学的帰納法で示す. $n=1$では$1^p - 1 = 0$より明らか. $n=m\geq 1$で主張が正しいとするとき, $n=m+1$では
\[
(m+1)^{p} - (m+1) = (m^{p} - m) + \sum_{i=1}^{p-1} {}_{p}\mr{C}_{k} m^{k}
\]
となる. 第一項は帰納法の仮定から$p$の倍数であり, 第二項は(1)より$p$の倍数である. よって$n=m+1$の場合も正しく, 数学的帰納法により示された. 

\syoumon{3}  $n$の素因数$p$をとる. $n$が素数でないと仮定する. このとき$p<n$であることに注意. そこで, ${}_{n}\mr{C}_{p}$を考える. これは$n$の倍数なので
\[
\dfrac{{}_{n}\mr{C}_{p}}{n} = \dfrac{(n-1)!}{p!(n-p)!} = \dfrac{(n-p+1)(n-p+2) \dots (n-1)}{p!}
\]
は整数である. しかし, $n-p+1$から$n-1$までの間の整数には$p$の倍数が無いので, 最右辺の分子は$p$の倍数ではない. 分母は$p$の倍数であるからこの値が整数であることに反する. よって$n$は素数である. 