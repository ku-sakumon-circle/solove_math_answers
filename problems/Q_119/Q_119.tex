\begin{thm}{119}{\hosi 8}{首都大 (2015)}
 座標平面において、楕円$C$: $\dfrac{x^2}{16}+\dfrac{y^2}{9}=1$ とする。全ての辺が$C$に接する長方形のうち、面積が最大となるものの面積を求めよ。
\end{thm}

\syoumon{解1}
まず、長方形の一辺が$y$軸に平行のとき、4直線$x=\pm 4$, $y=\pm 3$が長方形を構成するので、この面積は$6\times 8=48$。

続いて$y=mx+k$ ($m\neq 0$, $k\ge 0$)\footnote{$y=mx-k$も長方形の一辺を作るから、$k\ge 0$としてよい。}が一辺を構成するとき、$y=mx+k$は$C$に接するので、
\begin{align*}
 \frac{x^2}{16}+\frac{(mx+k)^2}{9}&=1 \\
 \left(\frac{1}{16}+\frac{m^2}{9}\right)x^2+\frac{2mk}{9}x+\left(\frac{k^2}{9}-1\right)&=0
\end{align*}
を$x$の方程式と見たとき、解は重解である。この判別式$D$を見て、
\begin{align*}
 D&=\left(\frac{2mk}{9}\right)^2-\left(\frac{k^2}{9}-1\right)\left(\frac{1}{4}\frac{4}{9}m^2\right) \\
 &=-\frac{1}{36}k^2+\frac{1}{4}\frac{4}{9}m^2=0 \\
 \therefore\quad k&=\sqrt{9+16m^2} \qquad (\because\, k\ge 0)
\end{align*}
を得る。すなわち、傾き$m$で$C$に接する直線は、
\[ y=mx\pm\sqrt{9+16m^2} \]
の2つである。この2つが長方形の2辺を作るとき、残りの2辺は傾き$-\dfrac{1}{m}$で$C$に接する直線なので、次のようになる。
\[ y=\left(-\frac{1}{m}\right)x\pm\sqrt{9+16\left(-\frac{1}{m}\right)^2} \]
原点と、直線$y=mx+\sqrt{9+16m^2}$との距離$d(m)$は、$d(m)=\dfrac{\sqrt{9+16m^2}}{m^2+1}$である。同様に、$d\left(-\dfrac{1}{m}\right)$は、
\[ d\left(-\frac{1}{m}\right)=\frac{\sqrt{9+16(-1/m)^2}}{\sqrt{(-1/m)^2+1}}=\frac{\sqrt{16+9m^2}}{\sqrt{1+m^2}} \]
これを用いて長方形の面積は、
\begin{align*}
 &2d(m)\times2d\left(-\frac{1}{m}\right) \\
 =& 4 \frac{\sqrt{(9+16m^2)(16+9m^2)}}{1+m^2}=4\frac{\sqrt{\left(\dfrac{9}{m}+16m\right)\left(\dfrac{16}{m}+9m\right)}}{m+\dfrac{1}{m}} \\
 =& 4 \frac{\sqrt{16^2+9^2+144\left(m^2+\dfrac{1}{m^2}\right)}}{m+\dfrac{1}{m}}=4 \frac{\sqrt{337+144(a^2-2)}}{a} \\
 =& 4 \frac{\sqrt{49+144a^2}}{a}=4\sqrt{\frac{49}{a^2}+144}
\end{align*}
と計算できる。ここで、$a=m+\dfrac{1}{m}$とした。$m\neq 0$のとき、$a$は$|a|\ge 2$の範囲を動き、$m=\pm 1$のときに$|a|=2$となる。このときに面積は最大化されることが分かるから、
\[ 4\sqrt{\frac{49}{a^2}+144}\le 4\sqrt{\frac{49}{2^2}+144}=50 \]
となる。よって求める最大値は50。\footnote{辺が$x,y$軸と平行になる場合($m=0$に相当)よりも面積は大きくできる。}

\syoumon{解2}
楕円$C$の外部の点$(p, q)$から、楕円$C$に2本の接線を引くことを考える。$p\neq\pm 4$ならば、接線の式は傾きを$a$として $y=a(x-p)+q$とかける。これが楕円$C$と共有点を1つだけもつので、2次方程式
\begin{align*}
 &\quad\frac{x^2}{16}+\frac{1}{9}[a(x-p)+q]^2=1 \\
 \dou\quad &\left(\frac{1}{16}+\frac{a^2}{9}\right)x^2-\frac{2}{9}a(ap-q)x+\frac{(ap-q)^2}{9}-1=0
\end{align*}
の判別式$D$は0である。すなわち、
\begin{align*}
 \frac{D}{4}&=\frac{1}{9^2}a^2(ap-q)^2-\left(\frac{1}{16}+\frac{a^2}{9}\right)\left(\frac{(ap-q)^2}{9}-1\right) \\
 &=\frac{1}{16\cdot9}\Bigl[(p^2-16)a^2-2pqa+q^2-9\Bigr]=0
\end{align*}
点$(p, q)$~(ただし$p\neq\pm 4$) に対して、この$a$についての方程式の解が2接線の傾きを与える。

2接線が直交するような$(p, q)$の条件を求める。$a$についての方程式、
\[ (p^2-16)a^2-2pqa+q^2-9=0 \]
の2実数解の積が$-1$であることが必要条件であるから、解と係数の関係によって、
\[ \frac{q^2-9}{p^2-16}=-1 \quad\dou\quad p^2+q^2=25 \]
を得る。逆に、$p^2+q^2=25$かつ$p\neq\pm 4$ならば、もとの方程式は積が$-1$となる2実数解をもつ\footnote{きちんと調べるには判別式をみたらよい}。一方で$p=\pm 4$のときには、$q=\pm 3$とすれば (複合任意)、$x=p$, $y=q$の2直線が楕円$C$に対する直交する2接線となり、さらに$p^2+q^2=25$を満たす。したがって、楕円$C$に直交する2接線の交点は、円$x^2+y^2=25$の全体を動く\footnote{楕円の準円}。この円を$D$とする。このことから、楕円$C$に外接する長方形の各頂点は円$D$上にあるといえる。

円$D$に内接する長方形の2辺の長さを$s, t$とおく (ただし$s, t>0$)。対角線の長さは円の直径に等しく$10$であるから、$s^2+t^2=10^2$が成り立つ。$s, t>0$であるから、相加相乗平均の不等式により
\[ 100=s^2+t^2\ge 2\sqrt{s^2t^2}=2st \quad\dou\quad st\le 50 \]
が成り立つ。等号の成立は$s=t$のとき。$st$は長方形の面積に等しいから、円$D$に内接する長方形のうち面積が最大となるのは、正方形となるときで、その面積は$50$となる。

最後に、円$D$に内接する正方形であって、楕円$C$に外接するものが存在することを示す。4直線$y=\pm x\pm5$は、4交点$(\pm5, 0)$, $(0, \pm5)$を持つ(複合任意)が、これらは円$D$上にあり正方形をなす。さらにこれら4直線と楕円$C$は、
\begin{align*}
 \frac{x^2}{16}+\frac{(\pm x\pm 5)^2}{9}&=1 \\
 \dou\quad \frac{25}{16\cdot 9}x^2\pm \frac{10}{9}x+\frac{16}{9}&=0 \\
 \dou\quad \left(\frac{5}{12}x \pm \frac{4}{3}\right)^2&=0 \\
\end{align*}
より接する。したがって、この正方形が求める長方形であって、その面積は50。