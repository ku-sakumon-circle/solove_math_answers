\begin{thm}{066}{\hosi ?}{(1) 東京大 文科 (2001), (2) 京都大 文系 (2006)}
 \begin{enumerate}
  \item 白石180個と黒石181個の合わせて361個の碁石が横に一列に並んでいる。碁石がどのように並んでいても、次の条件を満たす黒の碁石が少なくとも一つあることを示せ。
	(条件)~その黒の碁石とそれより右にある碁石を全て除くと、残りは白石と黒石が同数となる。ただし、碁石が一つも残らない場合も同数とみなす。
  \item $n, k$は正の整数であり、$k\le n$とする。穴の開いた$2k$個の白玉と$2n-2k$個の黒玉にひもを通して輪を作る。このとき、適当な2箇所でひもを切って$n$個ずつの部分に分け、どちらの組も白玉$k$個、黒玉$n-k$個からなるようにできることを示せ。
 \end{enumerate}
\end{thm}

\syoumon{1}
ここに解答を記述。

\syoumon{2}
ここに解答を記述。