\begin{thm}{151}{\hosi ?}{}
 超越数とは有理数係数多項式の根とならない数である。$\sqrt{2}$は$x^2-2=0$の解なので超越数ではないが、円周率 $\pi$やネイピア数 $e$ は超越数である。ところで, $e+\pi$と$e\pi$が無理数であるかは現在も未解決の問題ではあるが、この2つの数のうち少なくとも一方は無理数であることは分かる。なぜか。
\end{thm}

\begin{proof}
$e+\pi=a, e\pi =b$がともに有理数であるとする。 解と係数の関係より
\[x^2-ax+b=0\]
の解が$\pi ,e$である。これらはどちらも超越数であるから, 有理数係数方程式の根とならないことから矛盾する。従って $a,b$のうち少なくとも一方は無理数である。
\end{proof}
