\begin{thm}{197}{\hosi 7\maru}{自作 DMO4th 理6}
 三角形$T$の1つの辺の長さは平方数で、残りの辺の長さは素数である。また、$T$の面積は整数で、外接円の半径は素数である。$T$の各辺の長さを求めよ。
\end{thm}

$T$ の三辺の長さを$p,q,n^2$とし, $T$の面積を $S$, $T$の外接円の直径を$d$とする。ただし,  $n. S$は自然数, $d,p,q$は素数である。\\
長さ$n^2$の辺の対角を$\theta$とすると, 
\[S=\dfrac{1}{2}pq\sin{\theta}\]
であり, 正弦定理より
\[\dfrac{n^2}{\sin{\theta}}=d\]
なので
\[S=\dfrac{pqn^2}{2d}\]
となる。$S$が整数なので, 右辺が整数となるためには $pqn^2$が$d$で割り切れ, $d$は素数なので $p,q,n$ のうちの少なくともひとつが$d$で割り切れる。また, \uwave{三角形の各辺は$T$の外接円における弦であって, その長さは直径を超えてはならない}から
\[p\leq d, q\leq d, n^2\leq d\]
がすべて成り立つ。もし$n$が$d$で割り切れれば, 
\[d^2\leq n^2\leq d\]
となり, $d^2\leq d$を満たす素数$d$は存在しないので不適である。よって $p$または$q$が$d$で割り切れることになる。一般性を失わず$q$が$d$で割り切れるとしてよい。$q\leq d$であったから $q=d$ である。($q$は素数なのでよい.)\\
このとき, $T$のある一つの辺が直径に等しくなるから$T$は斜辺の長さが$q$の直角三角形である。そこで, 三平方の定理により
\[q^2=(n^2)^2+p^2\]
が成り立つから, 整理すると $p^2=(q-n^2)(q+n^2)$ と因数分解ができ, $q\pm n^2$は自然数, $q-n^2<q+n^2$だから
\[q-n^2=1 かつ q+n^2=p^2\]
でなければならない。この和を取れば \underbar{$2q=p^2+1$となる。}さらに 2つ目の式から $q=(p-n)(p+n)$ となり, 同様に考えると
\[p-n=1 かつ p+n=q\]
である。 この和を取れば \underbar{$2p=q+1$ となる。}下線部の2式から
\[p^2+1=2q=2(2p-1) \dou p^2-4p+3=0\]
となり, ここから$p=1,3$と出るので $p=3$であることが必要である。その他の値を求めると
\[q=2p-1=5, n=p-1=2\]
となる。$n^2=4$だから $T$は各辺の長さが$3,4,5$の三角形であることが必要である。\footnote{$S=\dfrac{pqn^2}{2d}$において分子が$d$で割り切れる条件しか見ておらず, 十分条件となっているかが自明な議論ではないため, 十分性を最後に確認せねばならない。}そして, このような$T$は$S=6, d=5$となっているので実際に条件を全て満たしている。\\
\\
以上より, $T$は各辺の長さが$3,4,5$の直角三角形に限る。\footnote{本問は, 2018年大学への数学3月号「読者と作るページ」に掲載された。}