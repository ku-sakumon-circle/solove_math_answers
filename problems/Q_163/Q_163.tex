\begin{thm}{163}{\hosi 8\maru}{学コン}
 $N$を4以上の整数とする。$n$を $1\le n \le N$を満たす整数とし、$\disp S_n=n\left(\sum_{k=n}^N \frac{1}{k}\right)$ とおく。
 \begin{enumerate}
  \item $S_n$が最大となるような$n$は1つまたは2つ存在することを示せ。
  \item (1)で定めた$n$の1つを$k$とする。$\disp \lim_{N\to\infty} \frac{S_k}{N^\alpha}$ が0でない値に収束するような実数の定数$\alpha$と、そのときの極限値を求めよ。
 \end{enumerate}
\end{thm}

\syoumon{1}
$2\le n\le N-1$として、
\begin{align*}
 S_{n+1}-S_n&=(n+1)\sum_{k=n+1}^N\frac{1}{k}-n\sum_{k=n}^N\frac{1}{k} \\
 &=(n+1)\sum_{k=n+1}^N\frac{1}{k}-n\left(\sum_{k=n+1}^N\frac{1}{k}+\frac{1}{n}\right) \\
 &=\sum_{k=n+1}^N\frac{1}{k}-1 < \sum_{k=n}^N\frac{1}{k}-1 = S_n-S_{n-1}
\end{align*}
が成り立つ。
\begin{align*}
 S_2-S_1&=\sum_{k=2}^N\frac{1}{k} -1 \ge \frac{1}{2}+\frac{1}{3}+\frac{1}{4}-1=\frac{1}{12}>0 \\
 S_N-S_{N-1}&=\frac{1}{N}-1 < 0
\end{align*}
より、ある整数$k$が$2\le k\le N-1$に存在し、
\begin{align*}
 S_N-S_{N-1}<S_{N-1}-S_{N-2}<\dots <S_{k+1}-S_{k}\le 0 \\
 \quad \le S_{k}-S{k-1}<\dots <S_3-S_2<S_2-S_1
\end{align*}
となるので、これを整理して
\begin{align*}
 S_N<S_{N-1}<S_{N-2}<\dots<S_{k+1}\le S_k \\
 S_1<S_2<S_3<\dots<S_{k-1}\le S_k
\end{align*}
が得られ、$S_{k-1}$, $S_k$, $S_{k+1}$が最大となる候補である。

これらが3つが等しいとすると、$S_k-S_{k-1}=0=S_{k+1}-S_k$であって、これは$S_k-S_{k-1}>S_{k+1}-S_k$に矛盾する。よって最大値をとり得るのは、$S_k$と$S_{k+1}$、$S_k$と$S_{k-1}$、あるいは$S_k$のみ、であるから、題意は示された。

\syoumon{2}
(1)の結果から、
\begin{align*}
 S_{k+1}-S_k&\le 0\le S_k-S_{k-1} \\
 \dou\quad \sum_{i=k+1}^N\frac{1}{i}-1 &\le -\le \sum_{i=k}^N\frac{1}{i}-1 \\
 \dou\quad -1-\frac{1}{k}&\le -\sum_{i=k}^N\frac{1}{i}\le -1 \\
 \dou\quad k+1&\ge k\sum_{i=k}^N\frac{1}{i}=S_k\ge k \quad\cdots\marunum{1} \\
 \dou\quad \frac{k+1}{N^\alpha}&\ge \frac{S_k}{N^\alpha}\ge \frac{k}{N^\alpha} \quad\cdots\marunum{2}
\end{align*}
を得る。$2\le i\le N$で$\disp \int_i^{i+1}\!\frac{dx}{x}< \frac{1}{i} < \int_{i-1}^i\!\frac{dx}{x}$が成り立つから、$k\le i\le N$で総和をとることで
\begin{align*}
 \int_{k}^{N+1}\!\frac{dx}{x} &< \sum_{i=k}^N\frac{1}{i}<\int_{k-1}^N\!\frac{dx}{x} \\
 \dou\quad k\log\left(\frac{N+1}{k}\right) &< S_k < k\log\left(\frac{N}{k-1}\right)
\end{align*}
となる。\marunum{1}から、
\[ k\log\left(\frac{N+1}{k}\right) < k+1 \,,\,\, k\log\left(\frac{N}{k-1}\right) > k \]
となるから、
\begin{align*}
 k\log\left(\frac{N+1}{k}\right)-1 &< k < k\log\left(\frac{N}{k-1}\right) \\
 \dou\quad \log\left(\frac{N+1}{k}\cdot e^{-\frac{1}{k}}\right) &< 1 < \log\left(\frac{N}{k-1}\right) \\
 \dou\quad \frac{N+1}{k}\cdot e^{-\frac{1}{k}} &< e < \frac{N}{k-1} \\
 \dou\quad \frac{N+1}{N}\cdot e^{-\frac{1}{k}-1} &< \frac{k}{N} < \frac{k}{k-1}\cdot \frac{1}{e}
\end{align*}
明らかに$\disp \lim_{N\to\infty} k=\infty$なので\footnote{これが収束すると仮定すると、$\disp \sum_{i=k}^N\frac{1}{i}$は発散するから\marunum{1}が満たされなくなる。}、
\begin{align*}
 \lim_{N\to\infty}\frac{k}{k-1}\cdot \frac{1}{e}&=\lim_{N\to\infty}\frac{1}{1-\frac{1}{k}}\cdot\frac{1}{e}=\frac{1}{e} \\
 \lim_{N\to\infty} \frac{N+1}{N}\cdot e^{-\frac{1}{k}-1}=1\cdot e^{-1}=\frac{1}{e}
\end{align*}
と計算される。よってはさみうちの原理から$\disp \lim_{N\to\infty} \frac{k}{N}=\frac{1}{e}$と求まった。\marunum{2}において、
\[ \frac{k+1}{N}\cdot \frac{1}{N^{\alpha-1}}\ge \frac{S_k}{N^\alpha}\ge \frac{k}{N}\cdot \frac{1}{N^{\alpha-1}} \]
であるから、$\dfrac{1}{N^{\alpha-1}}$が$N\to\infty$の極限で0でない値に収束すればよい。これを満たすのは$\alpha=1$で、このときはさみうちの原理から
\[ \lim_{N\to\infty} \frac{S_k}{N^\alpha}=\frac{1}{e} \]
