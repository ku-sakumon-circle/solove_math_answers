\begin{thm}{237}{\hosi 5}{京都府立医科大 (2020)}
 実数全体で定義された関数$f(x)$は微分可能で $f(0)=0$を満たし、その導関数$f'(x)$は連続かつ単調に減少しているとする。
 \begin{enumerate}
  \item $n$を自然数とし、$k$は$1\le k\le n$を満たす整数とする。$\dfrac{k-1}{n}\le x\le\dfrac{k}{n}$ のとき、以下の不等式(a), (b)が成り立つことを証明せよ。
  \begin{description}
   \item[(a)] $f\left(\dfrac{k}{n}\right)+f'\left(\dfrac{k-1}{n}\right)\left(x-\dfrac{k}{n}\right)\le f(x)$
   \item[(b)] $f(x)\le f\left(\dfrac{k}{n}\right)+f'\left(\dfrac{k}{n}\right)\left(x-\dfrac{k}{n}\right)$
  \end{description}
  \item $\disp a_n=\int_0^1\!f(x) \,dx-\frac{1}{n}\sum_{k=1}^nf\left(\frac{k}{n}\right)$ ($n=1,2,\dots$) とおく。このとき、$\disp \lim_{n\to\infty} na_n=-\frac{1}{2}f(1)$ であることを証明せよ。
 \end{enumerate}
\end{thm}

\syoumon{1}
(a): $x<\dfrac{k}{n}$なら、平均値の定理よりある$x<c<\dfrac{k}{n}$が存在して、
\[ f'(c)\left(x-\frac{k}{n}\right)=f(x)-f(\frac{k}{n}) \quad\cdots\text{(*)} \]
が成り立つ。ここで$f'(x)$は単調減少するから、$f'(c)\le f'(\frac{n-1}{k})$であり、$x-\dfrac{k}{n}<0$であることに注意すると、
\[ f(x)-f(\frac{k}{n})=f'(c)\left(x-\frac{k}{n}\right)\ge f'\left(\frac{k-1}{n}\right)\left(x-\frac{k}{n}\right) \]
より与不等式が示された。また、$x=\dfrac{k}{n}$なら与不等式は$f\left(\dfrac{k}{n}\right)+0\le f\left(\dfrac{k}{n}\right)$となり明らかに成り立つ。

(b): (*)において、$f'(c)\ge f'(\frac{k}{n})$から(a)と同様にして得られる。

\syoumon{2}
(1a)の左辺を$A_1^{(n,k)}(x)$、(1b)の右辺を$A_2^{(n,k)}(x)$とおくと、$\dfrac{k-1}{n}\le x\le\dfrac{k}{n}$において$A_1^{(n,k)}(x)\le f(x)\le A_2^{(n,k)}(x)$である ($k=1,2,\dots,n$)。よって、
\[ \sum_{k=1}^n \int_{\frac{k-1}{n}}^{\frac{k}{n}}\!A_1^{(n,k)}(x) \,dx \le \int_0^1\! f(x) \,dx \le \sum_{k=1}^n \int_{\frac{k-1}{n}}^{\frac{k}{n}}\!A_2^{(n,k)}(x) \,dx \quad\cdots\text{(\hosi)} \]
を得る。ここで、
\begin{align*}
 \int_{\frac{k-1}{n}}^{\frac{k}{n}}\!A_1^{(n,k)}(x)\,dx &=\Bigl[\frac{1}{2}f'\left(\frac{k-1}{n}\right)\left(x-\frac{k}{n}\right)^2+f\left(\frac{k}{n}\right)\cdots\Bigr]^{\frac{k}{n}}_{\frac{k-1}{n}} \\
 &=\frac{1}{n}f\left(\frac{k}{n}\right)-\frac{1}{2n^2}f'\left(\frac{k-1}{n}\right) \\
 \int_{\frac{k-1}{n}}^{\frac{k}{n}}\!A_2^{(n,k)}(x)\,dx &=\frac{1}{n}f\left(\frac{k}{n}\right)-\frac{1}{2n^2}f'\left(\frac{k}{n}\right)
\end{align*}
よって(\hosi)より、
\begin{align*}
 &-\frac{1}{2n^2}\sum_{k=1}^nf'\left(\frac{k-1}{n}\right)\le a_n \le -\frac{1}{2n^2}\sum_{k=1}^nf'\left(\frac{k}{n}\right) \\
 \dou\quad & -\frac{1}{2n}\sum_{k=1}^nf'\left(\frac{k-1}{n}\right)\le na_n \le -\frac{1}{2n}\sum_{k=1}^nf'\left(\frac{k}{n}\right)
\end{align*}
これについて$n\to\infty$の極限を考えると、左辺と右辺はともに区分求積法によって $\disp -\frac{1}{2}\int_0^1\! f'(x) \,dx$ となる。$\disp \int_0^1\!f'(x)\,dx=f(1)-f(0)$であり$f(0)=0$だから、
\[ \lim_{n\to\infty} na_n=-\frac{1}{2}f(1) \]