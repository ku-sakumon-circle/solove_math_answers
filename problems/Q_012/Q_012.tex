\begin{thm}{012}{\hosi 4}{中京大 (1997)}
 $a$を定数とし、$0\le x \le \dfrac{\pi}{2}$を満たす$x$に対して、
 \begin{align*}
  f(x)=\lim_{n\rightarrow\infty}\frac{(e^x+a)\cos^{n+1}x+\sin^{n+1}x}{\cos^nx+\sin^nx}
 \end{align*}
 とする。$f(x)$が$0\le x \le \dfrac{\pi}{2}$で連続であるとき、$\displaystyle \int^{\frac{\pi}{2}}_0\!\! f(x) \,dx$の値を求めよ。
\end{thm}

まず、与式における極限をとるべき式について、
\begin{align*}
 \frac{(e^x+a)\cos^{n+1}x+\sin^{n+1}x}{\cos^nx+\sin^nx}=\frac{(e^x+a)\cos x}{1+\tan^n x}+\frac{\sin x}{\frac{1}{\tan^n x}+1}
\end{align*}
と整理できる。このとき、$0\le x<\dfrac{\pi}{4}$のとき、$0\le\tan x<1$であり、$\dfrac{\pi}{4}<x\le\dfrac{\pi}{2}$のとき$\tan x>1$であり、さらに$x=\frac{\pi}{4}$のとき$\tan x=1$であることを踏まえて、
\begin{align*}
 f(x)&=\lim_{n\to\infty} \frac{(e^x+a)\cos x}{1+\tan^n x}+ \lim_{n\to\infty} \frac{\sin x}{\frac{1}{\tan^n x}+1} \\
 &=\left\{\,
 \begin{alignedat}{2}
  &(e^x+a)\cos x & &\left(0\le x<\frac{\pi}{4}\right) \\
  &\frac{e^\frac{\pi}{4}+a+1}{2\sqrt{2}} & \quad &\left(x=\frac{\pi}{4}\right) \\
  &\sin x & &\left(\frac{\pi}{4}<x\le\frac{\pi}{2}\right)
 \end{alignedat} \right.
\end{align*}
と求められる。したがって$x=\dfrac{\pi}{4}$で連続となればよい。
\[ \lim_{x\to\frac{\pi}{4}-0} (e^x+a)\cos x = \lim_{x\to\frac{\pi}{4}+0} \sin x = f\left(\frac{\pi}{4}\right) \]
を考えて、$a=1-e^\frac{\pi}{4}$である。これを用いて求める積分は、
\begin{align*}
 &\int_0^{\frac{\pi}{2}}\! f(x) \,dx = \int_0^{\frac{\pi}{4}}\! (e^x+a)\cos x \,dx + \int_{\frac{\pi}{4}}^{\frac{\pi}{2}}\! \sin x \,dx \\
 =& \Bigl[\frac{1}{2}e^x(\sin x+\cos x)\Bigr]_0^{\frac{\pi}{4}}+a\Bigl[\sin x \Bigr]_0^{\frac{\pi}{4}}+\Bigl[-\cos x\Bigr]_{\frac{\pi}{4}}^{\frac{\pi}{2}} = \sqrt{2}-\frac{1}{2}
\end{align*}



