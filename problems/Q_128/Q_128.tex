\begin{thm}{128}{\hosi 4}{JBMO SLP 改}
 $p^2(p^3-1)=q(q+1)$ を満たす素数$p, q$について調べよう。
 \begin{enumerate}
  \item $p<q$ を示し、$q$を$p$で割った余りを求めよ。
  \item $p^2+p+1$ は$q$の倍数であることを示せ。
  \item $p^2\le q+1$ を示せ。
  \item 組$(p,q)$は存在するか。
 \end{enumerate}
\end{thm}

\syoumon{1}
$p\ge q$と仮定すると、
\[ p^2(p^3-1)=q(q+1)\le p(p+1) \]
であるから、$p\ge 2$であることに注意して、
\begin{align*}
 0&\ge p^2(p^3-1)-p(p+1) \\
  &= p^5-2p^2-p > p^5 > 0
\end{align*}
となるから矛盾。よって$p<q$である。

与式から、$q(q+1)$は$p$で割り切れるが、$p$と$q$は互いに素であるから、$q+1$が$p$で割り切れる。したがって、$q$を$p$で割った余りは$p-1$。

\syoumon{2}
与式は、$p^2(p-1)(p^2+p+1)=q(q+1)$と整理できる。ここで$p^2$と$q$は互いに素で、$p-1<q$より明らかに$p-1$と$q$も互いに素である。したがって、$p^2+p+1$は$q$の倍数である。

\syoumon{3}
やはり$p^2(p-1)(p^2+p+1)=q(q+1)$からはじめて、$p^2$と$q$は互いに素であるから、$q+1$は$p^2$の倍数である。よって、$p^2\le q+1$といえる。

\syoumon{4}
$p=2$とすると、$2^2(2^3-1)=q(q+1)$を満たす素数$q$は存在しない。よって$p\ge 3$。
(2)の結果から、$q\le p^2+p+1$といえる。(3)の結果とあわせて、
\[ p^2\le q+1\le p^2+p+2 < p^2 \quad (\because\, p\ge 3)\]
となる。$p^2\le q+1<2p^2$でありかつ$q+1$は$p^2$の倍数なので、$q+1=p^2$である。しかしこのとき
\[ q=p^2-1=(p+1)(p-1) \]
で、$p\pm 1$はともに2以上だから$q$が素数であることに矛盾する。

よって、$p^2(p^3-1)=q(q+1)$を満たす素数$p, q$は存在しない。
\footnote{編者註: 小問に沿うように微修正しています}