% packages
\usepackage{amsmath,amssymb,amsfonts,amsthm,ascmac}
\usepackage{bm,braket,bigints}
\usepackage{fancyhdr,framed,multicol}
\usepackage{color}
\usepackage{makeidx}
\usepackage{enumitem}
\usepackage[dvipdfmx]{graphicx}
\usepackage{wrapfig}
\usepackage{itembkbx}
\usepackage{here}
\usepackage{ulem}
\usepackage[%
top    = 25truemm,%
bottom = 25truemm,%
left   = 7truemm,%
right  = 7truemm]{geometry}

% Enumerate 環境
%\def\theenumi{\arabic{enumi}}
%\def\labelenumi{(\theenumi)}
\SetLabelAlign{CenterWithParen}{\hfil(\makebox[0.7em]{#1})\hfil}

% 図・数式周囲の余白調整/改ページ許可
\def\lengthparam{
\setlength{\textfloatsep}{0pt}
\setlength{\intextsep}{0pt}
\setlength{\abovedisplayskip}{0.5pt}
\setlength{\belowdisplayskip}{0.5pt}
}
\allowdisplaybreaks[4]

% 問題を記述するための環境定義
\def\probs{}
\def\none{none}
\def\breakboxparindent{0pt}
\newenvironment{thm}[3]%
{\bigskip \global\expandafter\def\expandafter\probs\expandafter{\probs \quad Q.#1} %
 \begin{breakitembox}[l]{Q.#1\hspace{0.4eM}#2\hspace{0.4eM}#3}}
{\end{breakitembox}\smallskip}

% 問題の総数
\newcounter{qnum}
\setcounter{qnum}{228}

% 補題を記述するための環境定義
\definecolor{gray}{rgb}{0.75,0.75,0.75}
\newenvironment{subthm}[1]%
{\def\FrameCommand{\textcolor{gray}{\vrule width 5pt} \hspace{3pt}}
 \MakeFramed {\advance\hsize-\width \FrameRestore}
 \hspace{6pt} \textbf{補題#1} \\}
{\endMakeFramed}

% ヘッダの設定
\pagestyle{fancy}
\lhead{\probs \global\def\probs{}}
\rhead{}

% 問題索引の作成を宣言
\makeindex

%ローマ数字を出力
\newcounter{rom}
\newcommand{\Rnum}[1]{\setcounter{rom}{#1} \hspace{-0.1ex}\Roman{rom}\hspace{-0.1ex}}
\newcommand{\rnum}[1]{\setcounter{rom}{#1} \hspace{-0.1ex}\roman{rom}\hspace{-0.1ex}}

% (Re)New commands
%\newcommand{\thm}{\begin{itembox}[l]}
\newcommand{\syoumon}[1]{\vspace{1ex}\hspace{1eM}{\large ({\bf #1})} \par \vspace{1ex}}
\newcommand*{\vvv}{\overrightarrow}
\newcommand{\combi}[2]{{}_{#1}\mathrm{C}_{#2}}
\newcommand{\permu}[2]{{}_{#1}\mathrm{P}_{#2}}
\newcommand{\disp}{\displaystyle}
%\newcommand{\enthm}{\end{itembox}\\}
\newcommand{\dou}{\Leftrightarrow}  %%同値
\newcommand{\naraba}{\Rightarrow}
\newcommand{\irr}{\mbox{Irr}}
\newcommand{\jap}{\mbox}
\renewcommand{\leq}{\leqq}
\renewcommand{\geq}{\geqq}
\newcommand{\nea}{\nearrow}
\newcommand{\hosi}{$\bigstar$}
\newcommand{\maru}{$\circledcirc$}
\newcommand{\marunum}[1]{\raise0.2ex\hbox{\textcircled{\footnotesize{#1}}}}
\newcommand{\sea}{\searrow}
\newcommand{\mb}{\mathbb}
\newcommand{\mr}{\mathrm}
\newcommand{\bracket}[1]{%
\[
 \left(
 \begin{tabular}{p{0.9\hsize}}
  #1
 \end{tabular}
 \right)
\]}
%\renewcommand{\labelenumi}{\fbox{\arabic{enumi}}}
%\renewcommand{\labelenumii}{(\arabic{enumii})}
%\renewcommand{\labelenumiii}{(\roman{enumiii})}
\renewcommand{\labelenumi}{(\arabic{enumi})}
\renewcommand{\labelenumii}{(\roman{enumii})}

\parindent=0pt
