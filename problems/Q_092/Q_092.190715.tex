\begin{thm}{092}{\hosi 5}{JMO予選 2008-7}
 6桁の平方数の上三桁として考えられるものは全部でいくつあるか。 \hfil
\end{thm}

自然数$n$の2乗が6桁になるとき、
\[ 10^5\le n^2 < 10^6 \quad \dou\quad 316<10^{\frac{5}{2}}\le n < 10^3=1000 \]
であって、$316^2=99856$, $317^2=100489$, $999^2=998001$なので、$n$は$317\le n\le 999$の範囲を動く。

$n^2$の上3桁を$A_n$とおく。ここで
\[ (n+1)^2-n^2=2n+1 \]
$317\le n\le 499$のときには$2n+1<1000$なので、$A_{n+1}$は$A_n$か$A_n+1$のいずれかである。したがって、$A_{317}=100\le i\le A_{500}=250$を満たす任意の$i$について、$A_k=i$を満たす$k$が317から500までの間に少なくとも一つ存在する。一方で$500\le n\le 998$のときには、$2n+1>1000$であるから、$A_{n+1}$は$A_n+1$か$A_n+2$のいずれかである。したがって、$501\le n \le 999$における$A_n$は全て異なる値をとる。

以上のことから$A_n$は、100から250までの151通りに加え、$501\le n\le 999$の場合の499通りが考えられるから、あわせて650通り。