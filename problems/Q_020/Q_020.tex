\begin{thm}{020}{\hosi 11}{IMO マスターデーモン}
 2以上の整数$n$で、$\dfrac{2^n+1}{n^2}$が整数値をとる$n$の値を全て求めよ。 \\
\end{thm}

$n=1$は明らか。$n\ge 2$のとき、分子は奇数だから$n$も奇数である必要がある。$n$を割り切る最小の素因数 (2ではない) が存在するので、それを$p$とおく。このとき、
\[ 2^n+1\equiv 0 \quad\Rightarrow 2^{2n}\equiv 1 \pmod{p} \quad \cdots \marunum{1} \]
また$p$は2と互いに素だからフェルマーの小定理により$2^{p-1}\equiv 1 \pmod{p}$。これと\marunum{1}より、2の$\pmod{p}$の位数は$\gcd(2n, p-1)$の約数である。

$p-1$は$p$より小さい素数で素因数分解されるため、$p$の最小性から$n$とは互いに素であるが、$p-1$は明らかに偶数であるから、$\gcd(2n, p-1)=2$である。よって位数は1または2となる。

位数が1と仮定すると、
\[ 2^1\equiv 1 \quad\dou\quad 1\equiv 0 \pmod{p} \]
だが、$p\ge 3$なので不適。位数は2で$3\equiv 0$となり$p=3$である。よって$n$は3の倍数である。$n$が3で割り切れる回数を$k$とおく (ただし$k\ge 1$)。

LTEの補題を用いる。$v[3](a)$で、$a$が3で割り切れる回数を表すとして、
\[ v[3](2^n+1^n)=v[3](2+1)+v[3](n)=1+v[3](n)=1+k \]
となる。一方分母は$v[3](n^2)=2k$である。分母の方が3で割り切れる回数が少なくないといけないので、$1+k\ge 2k$より$k=1$。よって$n$は3で一回だけ割り切れる。

次に$n$の2番目に小さい素因数の存在を仮定する。これを$q$とし、同様の議論から2の$\pmod{q}$の位数は$\gcd(2n, q-1)$の約数である。$q-1$は$q$より小さい素因数($p=3$も含まれる)で素因数分解されるから、$q-1$が3と互いに素ならば$\gcd$は2で、$q-1$が3の倍数ならば$gcd$は6である。

よって位数は2または6の約数として、1, 2, 3, 6の場合を調べればよい。以下$\pmod{q}$で、1なら$1\equiv 0$で不適。2なら$3\equiv 0$で$q\ge 5$より不適。3なら$7\equiv 0$で$q=7$があり得る。6なら$63\equiv 0$で$q=7$があり得る。

したがって$q$が存在するならば$q=7$。このとき$n$は7の倍数であり、$2^n+1$も7で割り切れる必要があるが、$2^n+1\equiv 3, 5, 2 \pmod{7}$であるから不適。よって$q=7$であってはならないから、$n$には3より大きい素因数が存在せず、3で一回しか割り切れないから$n=3$が必要であり、実際に$n=3$は$\dfrac{2^3+1}{3^2}=1$となるからよい。

以上より、求める$n$は$n=1, 3$。