\begin{thm}{198}{\hosi 4}{駿台EXSテスト 改}
 $P(x)$は定数でない整数係数多項式とする。$P(1)>0$のとき, $P(n+P(1))-P(1)$は$P(1)$の美数であることを示せ。また, 任意の正の整数$n$に対して $P(n)$が素数であるような $P(x)$ は存在しないことを示せ。
\end{thm}

$d>0$,  $a_0, a_1, \cdots, a_d$は整数, $a_d\neq 0$として, 
\[P(x)=\disp\sum_{k=0}^{d}a_kx^k\]
とおいたとき,
\[P(n+P(1))=\disp\sum_{k=0}^{d} a_k(n+P(1))^k \equiv \disp\sum_{k=0}^{d} a_k n^k =P(n) (\jap{mod } P(1))\]
から前半の主張は明らか。題意を満たす$P(x)$が取れたとしよう。$P(n)$が素数のとき, $P(1)=q$とおくと $q>0$ であって, すべての自然数$n$に対して $P(n+q)-P(n)$ は $q$ の倍数である。特に$n=kq+1$ ($k\geq 0$) とおいたとき,
\[P(kq+1)\equiv P((k-1)q+1)\equiv \cdots \equiv P(1)=q\equiv 0 (\jap{mod} q)\]
だから $P(kq+1)$は$q$で割り切れる素数になる。よって, すべての自然数に対して
\[P(kq+1)=q\]
であるが, $0\leq k\leq d$ で考えたときに $P(x)$ は$ x- kq-1$ で割り切れ, すると$d+1$次以上の多項式になってしまい仮定に反する。よって存在しない。\qed
