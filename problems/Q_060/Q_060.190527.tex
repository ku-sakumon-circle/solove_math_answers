\begin{thm}{060}{}{整数 素数編}
 次の各問の条件を満たす素数$p, q, r, s$を求めよ。
 \begin{enumerate}
  \item $p^q+q^p=r$ (京大)
  \item $p+q=r$, $pr=q+s$ (学コン)
  \item $pq-1$, $qr-1$が平方数でかつ$pr-1$が素数の6乗 (学コン)
  \item $2^{p^2-q^2}-1=pqrs$ (AwesomeMath)
 \end{enumerate}
\end{thm}

\syoumon{1}
$p, q$がともに奇素数だと、$p^q+q^p$が明らかに2より大きい偶数となるので不適。よって$p, q$のどちらかは2である。対称性から、$q=2$のときを考えればよく、すなわち$p^2+2^p$が素数となる奇素数$p$を求めればよい。

$p\not\equiv 0 \pmod{3}$のとき、$p^2\equiv 1 \pmod{3}$である。一方で、$p$は奇数であるから、
\[ 2^p \equiv (-1)^p \equiv -1 \pmod{3} \]
である。したがって、$p$が3で割り切れないとき、$p^2+2^p\equiv 0 \pmod{3}$が成り立ち、不適である。よって$p$は3のみが適する。このとき$r=2^3+3^2=17$。

以上より、$(p, q, r)=(3, 2, 7), (2, 3, 7)$。

\syoumon{2}
$p,q$が奇数だと$r>2$かつ$r$が偶数なので不適。$p=2$または$q=2$である。

$p=2$のとき $r=q+2$より, $q+s=2r=2(q+2)$だから整理して $s=q+4$. よって, $q,q+2,q+4$が素数である。$q$が3の倍数でないとき, $q+2, q+4$のどちらかひとつが3の倍数だが, 明らかに$q+4>q+2>3$なので, 素数かつ3の倍数である3にはなりえないため不適。$q=3$なら $r=5, s=7$でよい。

$q=2$のとき $r=p+2$より $p(p+2)-2=s$. $p=3$のとき$r=5, s=13$ なのでよい。$p\equiv 1  (\mbox{mod} 3)$だと$r>3$かつ$r$が3で割れて不適。$p\equiv 2$のとき$s>3$かつ$s$が3で割れて不適。以上より $(p,q,r,s)=(2,3,5,7), (3,2,5,13)$

\syoumon{3}
$rp-1=s^6$ としたとき($s$は素数)
\[rp=(s^2+1)(s^4-s^2+1)\]
であって, 各因数は1より大きい。$p,r$は対称性があることに注意すると, $p=s^2+1. s^4-s^2+1=r$としてよい。($r=s^2+1$の場合も同様)\\
すると, $s$が奇素数では$p>2$かつ偶数になるので不適。$s=2$であり, $p=5, r=13$である。$5q-1=a^2, 13q-1=b^2$ ($a,b$は自然数) と置けるとする。
\[8q=b^2-a^2=(b-a)(b+a)\]
となる。$q=2$のとき, $a=3, b=5$なのでよい。$q\geq 3$のとき, $2<4<2q<4q$であり, $a\pm b$は偶数でかつ $b-a<b+a$から $(b-a,b+a)=(2, 4q), (4,2q)$の場合に限られる。つまり, $(a,b)=(2q-1, 2q+1), (q-2, q+2)$の場合に限られる。前者のとき,
\[5q-1=(2q-1)^2 \dou 4q^2-9q+2=0\]
で, $q\geq 3$なる解はない。後者のとき
\[5q-1=(q-2)^2 \dou q^2-9q+5=0\]
で整数解もなく不適。以上より $(p,q,r)=(5,2,13), (13,2,5)$

\syoumon{4}
右辺は正より $p^2-q^2>0$で, そのとき左辺は奇数なので $p>q>2$である。$p^2\equiv q^2$ (mod $8$)なので$p^2-q^2=8k$とすると
\[2^{8k}-1=(2^k-1)(2^k+1)(4^k+1)(16^k+1)\]
$k=1$のとき, $(p+q)(p-q)=8$となる$p,q$は存在しないので不適。$k\geq 2$が奇数のとき, 
\[2^k+1=(2+1)(2^{k-1}-2^{k-2}+\cdots +1)\]
となり, $2^{8k}-1$は2以上の整数5個以上の積になることから$pqrs$と表されることはない。$k\geq 2$が偶数とする。$k=2m$ として, $m=1$ならば 
\[(p+q)(p-q)=16\]
より $p+q=8, p-q=2$ の場合しかない。$p=5, q=3$である。\\
$m\geq 2$のときは
\[2^{k}-1=(2^m-1)(2^m+1)\]
で, $2^{m}-1\geq 3$だから $2^{8k}-1$は2以上の整数5個以上の積になり不適。\\
$2^{16}-1=65535=5\times 3\times 17\times 257$ より,
\[(p,q,r,s)=(5,3,17,257), (5,3,257,17)\]
