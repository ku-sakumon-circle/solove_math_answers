\begin{thm}{184}{}{}
 \begin{enumerate}
  \item 方程式 $x^3+3x^2-(k-7)x+k-11=0$ はちょうど2つの実数解を持つ。実数$k$の値を求めよ。 \hosi 2 (東京電機大 2017)
  \item $(a,-9)$ を通る曲線 $y=x^4-6x^2$ の接線が2本あるとき、$a$の値を求めよ。 \hosi 8 (学コン 2017-5-4)
 \end{enumerate}
\end{thm}

\syoumon{1}
\[ x^3+3x^2-(k-7)x+k-11=(x-1)(x^2+4x+11-k) \]
より、$k$によらず$x=1$は与えられた方程式の実数解である。与方程式がちょうど2つの実数解をもつとき、方程式$x^2+4x+11-k=0$が、(i) 1でない重解を持つ、(ii) 1と、1以外の実数解をもつ、の場合がある。

(i)~重解を持つ場合は、判別式を調べることで$k=7$とわかり、このときの重解は-2であるから、これは適する。

(ii)~1を実数解に持つ場合、$1^2+4\cdot 1+11-k=0$より$k=16$となり、1以外の実数解は-5となるから、これは適する。

以上より、$k=7, 16$。

\syoumon{2}
曲線$y=x^4-6x^2$について、$y'=4x^3-12x$より、点$(t, t^4-6t^2)$における接線は、
\[ y=(4t^3-12t)(x-t)+t^4-6t^2 \,\dou\, y=(4t^3-12t)x-3t^4+6t^2 \]
となる。この接線を$L_t$と呼ぶことにする。

ここで、異なる実数$p, q$ (ただし$p>q$とする)であって$L_p=L_q$となる\footnote{これを複接線という}ことがあるかを調べる。これは、
\[ 4p^3-12p=4q^3-12q \quad\text{かつ}\quad -3p^4+6p^2=-3q^4+6q^2 \]
が成り立つことと同値である。$p-q\neq 0$に注意して、
\begin{align*}
 4p^3-12p=4q^3-12q \,&\dou\, p^2+pq+q^2=3 \quad \marunum{1} \\
 -3p^4+6p^2=-3q^4+6q^2 \,&\dou\, (p+q)(p^2+q^2)=2(p+q) \quad \marunum{2}
\end{align*}
を得る。まず、$p+q\neq 0$のとき、$\marunum{2}$により$p^2+q^2=2$となり、\marunum{1}に代入すれば$pq=1$となる。$p^2+q^2=(p+q)^2-2pq=2$から$p+q=\pm 2$を得る。解と係数の関係から、$p, q$は2次方程式$x^2-2x+1=0$の2解であるが、これは明らかに重解を持つ。しかし、$p\neq q$に矛盾するので不適。すなわち、$p+q\neq 0$でかつ$L_p=L_q$となるような$p, q$は存在しない。

続いて$p+q=0$の場合、\marunum{2}は成り立つ。また$p\neq q$であるから、$p=-q\neq 0$。これを\marunum{1}に代入して、$p=\sqrt{3}$, $q=-\sqrt{3}$となる。

以上のことから、$L_p=L_q$となるのは、$p=\sqrt{3}$, $q=-\sqrt{3}$のときのみ。

さて、$L_t$が点$(a, -9)$を通ることから、
\[ -9=(4t^3-12t)a-3t^4+6t^2 \,\dou\, (t^2-3)(3t^2+4at+3)=0 \]
を満たすような実数$t$に対して、異なる$L_t$が2つだけであるような実数$a$を定めればよい。まず$t=\pm\sqrt{3}$という解が得られているが、$L_{\sqrt{3}}$と$L_{-\sqrt{3}}$は等しい直線を示すのであったから、これが1本目の接線である。接線が2本だけ得られるならば、方程式$3t^2+4at+3=0$は、(i)~$\pm\sqrt{3}$ではない重解をもっている、(ii)~相異なる2つの実数解を持ち、一方は$\pm\sqrt{3}$に等しく、もう一方は$\pm\sqrt{3}$でない、のいずれかである。

(i)~この場合には方程式の判別式を見て、$a=\dfrac{3}{2}$となり、その重解は$t=\mp 1$となるから適する (複合同順)。

(ii)~この場合には$3(\pm\sqrt{3})^2+4a(\pm\sqrt{3})+3=0$が満たされるので、これより$a=\mp\sqrt{3}$を得る。このときもう一方の解は$t=\pm\dfrac{1}{\sqrt{3}}$であるから適する。

以上より、
\begin{align*}
 a=\frac{3}{2}\quad&\text{}\quad t=\pm\sqrt{3}, -1 & a=-\frac{3}{2}\quad&\text{}\quad t=\pm\sqrt{3}, 1 \\
 a=\sqrt{3}\quad&\text{}\quad t=\pm\sqrt{3}, -\frac{1}{\sqrt{3}} & a=-\sqrt{3}\quad&\text{}\quad t=\pm\sqrt{3}, \frac{1}{\sqrt{3}}
\end{align*}
が方程式$(t^2-3)(3t^2+4at+3)=0$の解になっており、対応する$L_t$の種類が2つになる。また、これ以外の$a$の場合には題意を満たさない。

以上より、求める$a$は$a=\pm\dfrac{3}{2}, \pm\sqrt{3}$