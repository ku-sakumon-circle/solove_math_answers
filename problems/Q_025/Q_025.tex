\begin{thm}{025}{\hosi 4}{学コン 2015-6-2}
 $a,b,c$が$a \ge b \ge c > 0$ を満たして動くとき、$\dfrac{2a+b}{c}+\dfrac{2b+c}{a}$の最小値を求めよ。
\end{thm}

$F(a, b, c)=\dfrac{2a+b}{c}+\dfrac{2b+c}{a}$ とおく。$a, c$を固定すると、$F$は$b$に関して増加する関数である。よって、$a, c$を固定した上では$b$は$a\le b\le c$を動くから、$b=c$とすると小さくできる。すなわち、
\[ F(a, b, c)\ge F(a, c, c)=1+\frac{2a}{c}+\frac{3c}{a} \]
である。$a, c>0$だから、相加相乗平均の不等式より、
\[ F(a, c, c)\ge 1+2\sqrt{\frac{2a}{c}\cdot\frac{3c}{a}}=1+2\sqrt{6} \]
が成立する。等号の成立は$2a^2=3c^2$のときなので、たとえば$b=c=2$, $a=\sqrt{6}$とすると、
\[ F(\sqrt{6}, 2, 2)=1+2\sqrt{6} \]
が実現する。よって求める最小値は$1+2\sqrt{6}$。