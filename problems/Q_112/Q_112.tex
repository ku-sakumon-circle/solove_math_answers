\begin{thm}{112}{\hosi ?}{センター\Rnum{1}A (2016)}
 $N$市では温度の単位として摂氏(${^circ}$C)のほかに華氏(${}^\circ$F)も使われている。華氏(${}^\circ$F)での温度は、摂氏(${}^\circ$C)での温度を$\dfrac{9}{5}$倍し、32を加えると得られる。例えば、摂氏$10^\circ$Cは、$\dfrac{9}{5}$倍し32を加えることで華氏$50^\circ$Fとなる。

したがって、$N$市の最高気温について、摂氏での分散を$X$、華氏での分散を$Y$とすると、$\dfrac{Y}{X}$は \textgt{\fbox{\hspace{1ex}ツ\hspace{1ex}}} になる。

東京(摂氏)と$N$市(摂氏)の共分散を$Z$、東京(摂氏)と$N$市(華氏)の共分散を$W$とすると、$\dfrac{W}{Z}$は \textgt{\fbox{\hspace{1ex}テ\hspace{1ex}}} になる (ただし、共分散は2つの変量のそれぞれの偏差の積の平均値)。

東京(摂氏)と$N$市(摂氏)の相関係数を$U$、東京(摂氏)と$N$市(華氏)の相関係数を$V$とすると、$\dfrac{V}{U}$は \textgt{\fbox{\hspace{1ex}ト\hspace{1ex}}} になる。
\end{thm}

ここに解答を記述。