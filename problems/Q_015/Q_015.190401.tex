\begin{thm}{015}{\hosi 7}{数検1級二次対策}
 二進法を変形して$-2$を基底に取ると、$0$と$1$からなる列が符号をつけずに十進法の全ての整数を表すことができる。これを負二進法という。例えば、$101_{(-2)}=(-2)^2+1=5$ である。では負二進法で$m$桁の整数は十進法でどのような整数の範囲を表すか。
\end{thm}

ここに解答を記述。
