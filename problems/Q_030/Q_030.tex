\begin{thm}{030}{\hosi 7}{(1) 千葉大、 (2) ガロア祭2012}
 \begin{enumerate}
  \item $n$は3以上の自然数とする。整数$x,y,z$が$x^n+2y^n=4z^n$ を満たすならば、$(x,y,z)=(0,0,0)$であることを示せ。
  \item 有理数$x,y,z,w$が、$x^2-2y^2+5(z^2-3w^2)=0$ を満たすならば、$(x,y,z,w)=(0,0,0,0)$であることを示せ。
 \end{enumerate}
\end{thm}

\syoumon{1}
まず、$(x, y, z)\neq (0,0,0)$ なる解が得られたとする。このとき$x,y,z$の最大公約数$g$が定義できて、
\[ x^n+2y^n=4z^n \quad\Rightarrow\quad \left(\frac{x}{g}\right)^n+2\left(\frac{y}{g}\right)^n=4\left(\frac{z}{g}\right)^n \]
であるから、$(x,y,z)$が解ならば$\disp\left(\frac{x}{g},\frac{y}{g},\frac{z}{g}\right)$も解であることが従う。よって初めから$x,y,z$の最大公約数は1であることを仮定してよい。
\[ x^n=4z^n-2y^n \]
より、$x$は偶数でなければならない。$n\ge 3$であるから$x^n$は8の倍数である。$\pmod{4}$を考えると、
\[ 0\equiv 2y^n \pmod{4} \]
となるから、$y$は偶数でなければならない。続いて$\pmod{8}$を考えると、$x^n, y^n$はともに8の倍数だから、
\[ 0\equiv 4z^n \pmod{8} \]
となる。よって$z$も偶数でなければならない。$x,y,z$がすべて偶数なので、このことは最大公約数を1としたことに反する。よって$(x,y,z)=(0,0,0)$であることが必要で、これは明らかに解である。\footnote{無限降下法的な議論でもよい。無限降下法はどの項の次数も同じであるようなときに使えることが多い。(個人的にはこのように最大公約数を取る方が好きである。)}

\syoumon{2}
$(x,y,z,w)\neq(0,0,0,0)$ なる解があったとする。このとき十分大きい自然数$N$によって$(N!x,N!y,N!z,N!w)$ を考えることで$(0,0,0,0)$でない整数解を得ることができる。そこで初めから$x,y,z,w$が整数であるとしてよい。また今回の場合も前問と同様にして$x,y,z,w$の最大公約数が1であるとしてよい。$\pmod{5}$を考えることにより、
\[ x^2\equiv 2y^2 \pmod{5} \]
を得る。$y\not\equiv 0 \pmod{5}$であるなら、$yk\equiv 1 \pmod{5}$となる整数$k$が取れるので
\[ (kx)^2\equiv 2 \pmod{5} \]
となる。しかし2は$\pmod{5}$における平方剰余ではないので不適である\footnote{もちろん、$2y^2\equiv 0,2,3 \pmod{5}$であることを見て、0以外は非平方剰余なので、と言ってもよい。}。よって$y\equiv 0\pmod{5}$であり、$x^2\equiv 0\pmod{5}$なので$x$も5の倍数である。

$x^2,y^2$は25の倍数である。$\pmod{25}$を考えると、
\[ 5(z^2-3w^2) \equiv 0 \pmod{25} \]
より、$z^2\equiv 3w^2 \pmod{5}$でなければならない。$w\not\equiv 0\pmod{5}$であるなら、$wl\equiv 1 \pmod{5}$となる整数$l$が取れるので
\[ (lz)^2\equiv 3 \pmod{5} \]
となる。しかし3は$\pmod{5}$における平方剰余ではないので不適である。よって$w\equiv 0\pmod{5}$であり、$z\equiv 0\pmod{5}$も従う。

以上より、$x,y,z,w$は全て5の倍数となり、最大公約数を1としたことに反する。よって$(0,0,0,0)$でない有理数解は存在しない。また$(0,0,0,0)$は明らかに解である。