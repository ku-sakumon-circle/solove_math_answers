\begin{thm}{264}{\hosi 3}{位数}
$n$を自然数, $a$は$n$と互いに素な整数とする. 次を示せ. 
\begin{enumerate} 
\item $a^k\equiv 1\pmod{n}$を満たす自然数$k$が存在する. 
\item (1)の$k$のうち最初のものを$d$とおく. これを$a$の$\bmod{n}$における位数という. このとき$k$は必ず位数$d$の倍数である.
\end{enumerate}
\end{thm}

\syoumon{1} 
$a^{m}$を$n$で割った余りとしてありうるのは有限個であるから, 鳩ノ巣原理より$a^{m_1}\equiv a^{m_2}\pmod{n}$となるような相異なる自然数$m_1,m_2$が存在する.つまり, $a^{m_1} - a^{m_2}$は$n$の倍数である. $m_1>m_2$であるとしてよい.  このとき, $k=m_1-m_2$とおくと$k$は自然数で, $a^{m_1}(a^{k} - 1)$は$n$の倍数である. $a$と$n$は互いに素なので$a^{k} - 1$が$n$の倍数であることが分かり, 題意が従う. \qed  \\

\syoumon{2}
$k$を$d$で割って$k=qd + r$とおく($0\leq r<d$).  このとき, $1\equiv a^{k} = a^{dq + r} \equiv (a^{d})^{q} a^r\equiv 1^{q} \cdot a^{r} \equiv a^{r} \pmod{n}$なので, $a^{r} \equiv 1\pmod{n}$である. もし$r\neq 0$なら, $0< r<d$と$d$を定義する最小性に矛盾する. よって$r=0$だから$k=dq$となり, $k$は$d$の倍数であることが示された. 