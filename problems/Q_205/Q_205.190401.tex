\begin{thm}{205}{\hosi 6}{奈良県医大 (2009)}
 $n$を2以上の整数とし、1から$n$までの相異なる$n$個の整数を横一列に並べて得られる各順列$\sigma$に対して、左から$i$番目の数字を$\sigma(i)$と記す。このとき、$1\le i < j \le n$ かつ $\sigma(i)>\sigma(j)$ を満たす整数の対$(i,j)$の個数を$l(\sigma)$とおく。さらに、1から$n$までの順列$\sigma$全体のなす集合を$S$とする。$\sigma$が$S$全体を動くとき、$l(\sigma)$の総和$\disp \sum_{\sigma\in S} l(\sigma)$ を求めよ。
\end{thm}

ここに解答を記述。