\begin{thm}{028}{\hosi 9}{(東大実戦)}
 $n$を0以上の整数とする。$0\leq k\leq n$ を満たすすべての整数 $k$ のうち, $2^k$ の最高位の数字が1のものの個数を $a_n$, 最高位の数字が4のものの個数を $b_n$ とする。$\disp\lim_{n\to \infty} \dfrac{a_n}{b_n}$の値を求めよ。 (最高位の数字は10進法で考えるものとする。)
\end{thm}

$n\geq 4$ とする。\\
$2^n$ が $k_n$桁であるとする。 (この時, $k_n\geq 2$) また,  $2^i$が$k_n$桁であるような最大の整数を$M(n)$として,
\[\disp\lim_{n\to \infty} \dfrac{a_{M(n)}}{b_{M(n)}}\]
を代わりに考える。$k_n=k_{M(n)}$に注意する。
\\
\\
$2^{j}$と$2^{j+1}$で桁数が異なるとき, ある自然数$s$が存在して$5\cdot 10^{s-1}\leq 2^j< 10^{s}$ となるが, この不等式から
\[10^{s}\leq 2^{j+1}< 2\cdot 10^{s}\] 
が言えるため, $2^{j+1}$の最高位の数は1になる。また, このとき $2^{j+2}$は明らかに最高位の数が2,3になる。\\
\\
このことから, $s$桁($s=1,\cdots, k_n$)の2の非負整数乗であって, 最高位の数が1であるようなものがただひとつ存在するので,$a_n=k_n$,  $a_{M(n)}=k_{n}$である。\\
\\
集合$S_d=\{ 2^{k} | 0\leq k\leq M(n), 2^k\mbox{の最高位の数が$d$}\}$  ($d=1,2,\cdots ,9$) を考える。このとき, $S_1$の各元を2倍すると$M(n)$の取り方から\footnote{$M(n)$で取れば, $S_1$の元を2倍したときに必ず$S_2\cup S_3$に属するようになる。逆に$n=7$等で考えると, $128\in S_1$を2倍しても $256$は大小の問題により $S_2\cup S_3$には属さなくなる。}
必ず $S_2\cup S_3$に属し, 逆に $S_2\cup S_3$の各元を2で割ると $S_1$に属する。よって, 集合の元の間に1対1の対応(全単射)が存在し, 集合の元の個数に関して
\[|S_1|=k_n=|S_2\cup S_3|=|S_2|+|S_3|\]
が成立する。同様な議論により, $S_2\cup S_3$と $S_4\cup S_5\cup S_6 \cup S_7$ の間に全単射が存在するので
\[|S_2|+|S_3|=k_n=|S_4|+|S_5|+|S_6|+|S_7|\]
また同様に $S_4$と $S_8\cup S_9$の間に全単射が存在するので
\[|S_4|=b_{M(n)}=|S_8|+|S_9|\]
また, $S_1\cup S_2\cup \cdots \cup S_9$は$\{2^{k}| k=0,1,\cdots , M(n)\}$なので
\[M(n)+1=\disp\sum_{d=1}^9 |S_d|\]
であり, 右辺は
\[|S_1| +(|S_2|+|S_3|)+(|S_4|+\cdots +|S_7|) +(|S_8|+|S_9|)=3k_n+b_{M(n)}\]
となるから 
\[b_{M(n)}=M(n)+1-3k_{n}\]
を得る。これらにより
\[\dfrac{a_{M(n)}}{b_{M(n)}}=\dfrac{k_n}{M(n)+1-3k_{n}}=\dfrac{1}{\frac{M(n)}{k_n} +\frac{1}{k_n}-3}\]
となる。\\
$2^{M(n)}$は$k_n$桁であるから
\[10^{k_n-1}\leq 2^{M(n)}<10^{k_n}\]
が成り立つ。底を2とする対数をとり,整理すると
\[\left(1- \dfrac{1}{k_n}\right)\log_2{10}\leq \dfrac{M(n)}{k_n} < \log_2{10} \]
となる。明らかに $k_n\to \infty$ なので はさみうちの原理により$\disp\lim_{n\to \infty} \dfrac{M(n)}{k_n}=\log_2{10}$である。よって
\[\disp\lim_{n\to \infty}\dfrac{a_{M(n)}}{b_{M(n)}}=\dfrac{1}{\log_2{10}-3}=\dfrac{1}{\log_2{5}-2}\]
を得る。\\
$b_{M(n)}-b_n$ は, $ n< k\leq M(n)$ なる整数$k$のうち$2^k$の最高位が4であるものの個数であるが, $2^n, 2~{M(n)}$は桁が同じなのでそのような数は高々ひとつである。よって
\[0\leq b_{M(n)}-b_n\leq 1\]
から
\[1\leq \dfrac{b_{M(n)}}{b_{n}}\leq 1+\dfrac{1}{b_n}\]
$n\to \infty$のとき, $a_{M(n)}=k_n\to \infty$で, $\dfrac{a_{M(n)}}{b_{M(n)}}$は収束するので $b_{M(n)}\to \infty$ が必要である。$b_{M(n)}+1\leq b_n$ だから $b_n\to \infty$ である。よって
\[\disp\lim_{n\to \infty} \dfrac{b_{M(n)}}{b_n}=1\]
である。このことと, $a_n=a_{M(n)}$から
\[\disp\lim_{n\to \infty}\dfrac{a_n}{b_n}=\disp\lim_{n\to \infty} \left(\dfrac{a_{M(n)}}{b_{M(n)}}\cdot \dfrac{b_{M(n)}}{b_n}\right) = \mbox{{\boldmath $\dfrac{1}{\log_2{5}-2}$}}\]
