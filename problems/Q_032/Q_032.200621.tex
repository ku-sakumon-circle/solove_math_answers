\begin{thm}{032}{\hosi 4}{阪大 医(保健)学部 (2007)}
 $n$を2以上の自然数とする。1つのさいころを$n$回投げ、第1回目から第$n$回目までに出た目の最大公約数を$G$とする。$G$の期待値を$n$の式で表せ。
\end{thm}

最大公約数が$G$となる確率を$P(G)$と書く。$G=4,5,6$となるのは、それぞれ出た目が全て4,5,6の場合のみであるから、
\[ P(4)=P(5)=P(6)=\frac{1}{6^n} \]
$G=3$となる場合は、出た目の全てが3,6のどちらかのみである場合から、$n$回全て6であった場合を除けばよい。よって、
\[ P(3)=\frac{1}{3^n}-\frac{1}{6^n} \]
$G=2$となる場合は、出た目の全てが2,4,6のいずれかである場合から、$n$回全て4であった場合と6であった場合を除けばよい。よって、
\[ P(2)=\frac{1}{2^n}-\frac{2}{6^n} \]
上掲の場合に該当しない場合はすべて$G=1$となる。以上のことを用いて、$G$の期待値は、
\begin{align*}
 &\sum_{G=1}^6 G\cdot P(G) \\
 =& 1\times \left(1-\frac{1}{2^n}-\frac{1}{3^n}\right)+2\times\left(\frac{1}{2^n}-\frac{2}{6^n}\right)+3\times\left(\frac{1}{3^n}-\frac{1}{6^n}\right) \\
 &\quad +(4+5+6)\times\frac{1}{6^n} \\
 =& 1+\frac{1}{2^n}+\frac{2}{3^n}+\frac{8}{6^n}
\end{align*}