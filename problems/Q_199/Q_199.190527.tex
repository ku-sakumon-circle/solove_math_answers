\begin{thm}{199}{\hosi 6}{開成模試 文系}
 座標平面上で半径1、原点Oを中心とする球面をS、点$(0,0,1)$をAとする。平面pはベクトル$(1,1,1)$に垂直で、pとSの共有点の集合は円Cをなし、点Aを頂点としてCを底面とする円錐Tが存在する。Tの体積の最大値を求めよ。
\end{thm}

平面pは、$k$を実数として$x+y+z=k$で表される。pとSが共有点を持ち、それが1点でないような$k$の条件は、pと原点Oの距離が1未満であることが必要十分。よって、
\[ \frac{|0+0+0-k|}{\sqrt{1^2+1^2+1^2}}=\frac{|k|}{\sqrt{3}} < 1 \]
より、$-\sqrt{3}<k<\sqrt{3}$である。円錐Tが存在するためには、平面pは点Aを通ってはいけないので、$0+0+1\neq k$より$k\neq 1$。よって、$k$のとり得る範囲は$-\sqrt{3}<k<\sqrt{3}, k\neq 1$。

点Oを頂点とし円Cと底面とする円錐を考えれば、三平方の定理によって円Cの半径は
\[ \sqrt{1^2-\left(\frac{|k|}{\sqrt{3}}\right)^2} = \sqrt{1-\frac{k^2}{3}} \]
である。

円錐Tの高さは、平面pと点Aの距離に等しいから、
\[ \frac{|0+0+1-k|}{\sqrt{1^2+1^2+1^2}}=\frac{|1-k|}{\sqrt{3}} \]
したがってTの体積は
\[ \frac{1}{3}\pi \left(1-\frac{k^2}{3}\right) \frac{|1-k|}{\sqrt{3}} = \frac{\sqrt{3}\pi}{9}\left(1-\frac{k^2}{3}\right)|1-k| \]
と求められる。これを$f(k)$とおく。

$0\le k < 1$, $1<k<\sqrt{3}$のとき、$|1-k|\le 1$である。一方、$-\sqrt{3}<k<0$のとき、$|1-k|>1$である。したがって、$0\le k<1$, $1<k<\sqrt{3}$のとき、$f(-k)\ge f(k)$が成り立つ。このことから、$-\sqrt{3}<k\le 0$の範囲で$f(k)$の最大値を求めればよい。またこのとき$|1-k|=1-k$である。

$f(k)=\dfrac{\sqrt{3}\pi}{9}\left(1-\dfrac{k^2}{3}\right)(1-k)$について$k$で微分して$f'(k)=\dfrac{\sqrt{3}\pi}{9}\left(x^2-\dfrac{2}{3}x-1\right)$であるから、$-\sqrt{3}<k\le 0$において、$k=\dfrac
{1-\sqrt{10}}{3}$で極大となるから、求める値は、
\[ f\left(\frac{1-\sqrt{10}}{3}\right) = \frac{4\sqrt{3}}{729}(13+5\sqrt{10})\pi \]