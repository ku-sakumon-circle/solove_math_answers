\begin{thm}{183}{\hosi 5}{一橋 (2014)}
 $(2\times3\times5\times7\times11\times13)^{10}$ は十進法において何桁であるか求めよ。 \\
 (注: 対数の値は本問では与えられていない。)
\end{thm}

$7\times11\times13=1001$より、
\[ (2\times3\times5\times7\times11\times13)^{10}=10^{10}\times3^{10}\times1001^{10} \]
である。計算によって、$3^{10}=59049$と求まる。続いて、
\begin{align*}
 1001^{10}=(1000+1)^{10}=\sum_{k=0}^{10} \combi{10}{k}1000^k\cdot1^{10-k}
\end{align*}
である。$1000^k$は$k$が1増えるごとに3桁増えるのに対し、
\[ \combi{10}{k}\le \combi{10}{5}=252 \quad(k=0,1,\dots 9,10) \]
と高々2桁であることから、$\combi{10}{k}1000^k$ (ただし$k=0,1,\dots, 9$)の桁数は$1000^{10}$の桁数よりも小さいことがわかる。したがって、$1001^{10}$の桁数は$k=10$の項のみで決まる。これは
\[ \combi{10}{10}1000^{10}\cdot1^0=10^{30} \]
である。これを用いて求める桁数は、
\[ 10^{10}\times3^{10}\times10^{30}=5.9049\times10^{44} \]
と等しいから、45桁。