\begin{thm}{064}{\hosi 8}{Z会京大理系実戦演習}
 $a$を実数の定数とする。$x$の方程式 $x^5-x^4+ax^3+ax^2-x+1=0$ の5つの解のうち、少なくとも2つの解が一致するとき、$a$の値と、一致する解を求めよ。
\end{thm}

与式は$x=-1$を解に持つから\footnote{一般に奇数次の相反方程式は$x=-1$を解に持つ。}、
\[ x^5-x^4+ax^3+ax^2-x+1=(x+1)\bigl[x^4-2x^3+(a+2)x^2-2x+1\bigr] \]
である。$f(x)=x^4-2x^3+(a+2)x^2-2x+1$とおく。

(i)~$f(-1)=8+a$だから、$a=-8$とすれば$f(x)$が$(x+1)$で割り切れ、
\[ (x+1)f(x)=(x+1)^3(x^2-4x+1) \]
だから$x=-1$が重解である。

(ii)~$f(-1)\neq 0$かつ$f(1)=0$の場合を考える。$f(1)=a$より、これは$a=0$で、このとき$f(x)=(x-1)^2(x^2+1)$となるから、与方程式は$x=1$を重解に持つ。

(iii)~$f(-1)\neq 0$かつ$f(1)=0$の場合を考える。これは$a\neq 0, 8$。また$f(0)=1$であるから、$x=0, \pm 1$は方程式$f(x)=0$の解でない。この方程式の重解を$p$、それ以外の2解を$q, r$とおくと、
\[ f(x)=(x-p)^2(x-q)(x-r) \quad \text{ただし$p, q, r\neq 0, \pm 1$} \]
と書ける。一方で、$x^4f\left(\dfrac{1}{x}\right)=f(x)$であるから、
\[ f(t)=0 \quad\dou\quad f\left(\frac{1}{t}\right)=0 \]
である。$p\neq p^{-1}$なので\footnote{$p$は$0$でも$\pm 1$でもないから}、$q, r$のいずれかが$p$に等しいが、$p^{-1}=q$としてよい。よって
\[ f(x)=(x-p)^2(x-p^{-1})(x-r) \]
となるが、解と係数の関係により$ppp^{-1}r=1$なので、$r=p^{-1}$。つまり
\[ f(x)=(x-p)^2(x-p^{-1})^2=x^4-2x^3+(a+2)x^2-2x+1 \]
である。$x^3$の係数を比較して、$2(p+p^{-1})=2$である。続いて$x^2$の係数を比較すると$p^2+4+p^{-2}=a+2$であるから、
\[ a=p^2+2+p^{-2}=(p+p^{-1})^2=1 \]
となる。このとき、
\[ f(x)=\left(x-\frac{1+\sqrt{-3}}{2}\right)^2\left(x-\frac{1-\sqrt{-3}}{2}\right)^2 \]

以上より、
\begin{align*}
 \left\{
 \begin{aligned}
  a&=0 & \text{一致する解は}&\, 1 \\
  a&=1 & \text{一致する解は}&\, \frac{1+\sqrt{-3}}{2}\,,\,\, \frac{1-\sqrt{-3}}{2} \\
  a&=-8 & \text{一致する解は}&\, -1
 \end{aligned}
 \right.
\end{align*}