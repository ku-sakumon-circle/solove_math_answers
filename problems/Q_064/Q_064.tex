\begin{thm}{064}{\hosi 8}{Z会京大理系実戦演習}
 $a$を実数の定数とする。$x$の方程式 $x^5-x^4+ax^3+ax^2-x+1=0$ の5つの解のうち、少なくとも2つの解が一致するとき、$a$の値と、一致する解を求めよ。
\end{thm}

\syoumon{解法1}

与式は$x=-1$を解に持つから\footnote{一般に奇数次の相反方程式は$x=-1$を解に持つ。}、
\[ x^5-x^4+ax^3+ax^2-x+1=(x+1)\bigl[x^4-2x^3+(a+2)x^2-2x+1\bigr] \]
である。$f(x)=x^4-2x^3+(a+2)x^2-2x+1$とおく。$x=0$は解ではないので、以下では$x\neq 0$のもとで考える。 $x^{-2}f(x) = (x^2 + x^{-2}) - 2(x+x^{-1}) + (a+2)$であり、$t=x+x^{-1}$とすると$x^{-2}f(x) = t^2 - 2t + a$である。これは
\[(t - \alpha)(t-\beta), \alpha = 1+\sqrt{1-a},\quad \beta = 1-\sqrt{1-a}\]
と因数分解される。よって
\[f(x) = x^2(t-\alpha)(t-\beta) = (x^2 - \alpha x + 1)(x^2-\beta x + 1)\]
である。$A(x) = x^2-\alpha x  +1 $、$B(x) = x^2-\beta x + 1$とする。\par
さて、方程式$(x+1)A(x)B(x) = 0$が重解を持つとすれば、次の4パターンの可能性がある。
\begin{enumerate} 
\item $f(-1) = A(-1)B(-1)  = 0$である場合。
\item $A(-1)B(-1)\neq 0$であるが、$A(x)$に重根がある場合。
\item $A(-1)B(-1)\neq 0$であるが、$B(x)$に重根がある場合。
\item $A(-1)B(-1)\neq 0$で$A(x)$にも$B(x)$にも重根はないが、$A(x)$と$B(x)$が共通根を持つ場合。
\end{enumerate}
(1): $f(-1) = a+8$より$a=-8$の場合である。このとき$\alpha = 4, \beta = -2$なので$A(x) = x^2-4x + 1$、$B(x) = x^2 + 2x + 1$だから
\[(x+1)f(x) = (x+1)^3 (x^2-4x + 1)\]
なので一致する解は$-1$である。\par 
以降は$a\neq 8$とする。このとき$f(1)\neq 0$である。\\
(2): $x^2 - \alpha x + 1 = (x-\alpha /2)^2 + 1 - \alpha^2/4$なので$1-\alpha^2/4 = 0$、つまり$\alpha = \pm 2$である。$\alpha\neq -2$であるような$a$はなく、$\alpha=2$となるのは$a=0$のときである。$a=0$のとき、$A(x) = (x-1)^2$、$B(x) =x^2+1$だから
\[(x+1)f(x) = (x+1)(x-1)^2(x^2+1)\]
であり、一致する解は1である。\\
(3): 同様に、$\beta = \pm 2$となる。$\beta = 2$となる$a$はなく、$a\neq -8$なので$\beta = 2$も考えなくてよい(つまりこのパターンは(1)と兼ねている)。\\
(4): $A(z) = B(z) = 0$となる$z$があるとすると、$B(z) - A(z) = (\alpha - \beta)z = (2\sqrt{1-a} )z = 0$である。$A(0),B(0)\neq 0$なので$z\neq 0$である。よって$\sqrt{1-a}=0$だから$a=1$を得る。このとき$\alpha = \beta = 1$だから
\[(x+1)f(x) = (x+1)(x^2-x+1)^2\]
だから、一致する解は($x^2-x+1$の根である)$\dfrac{1 \pm \sqrt{-3}}{2}$である。\par 
以上より
\begin{align*}
 \left\{
 \begin{aligned}
  a&=0 & \text{一致する解は}&\, 1 \\
  a&=1 & \text{一致する解は}&\, \frac{1+\sqrt{-3}}{2}\,,\,\, \frac{1-\sqrt{-3}}{2} \\
  a&=-8 & \text{一致する解は}&\, -1
 \end{aligned}
 \right.
\end{align*}

\syoumon{解法2}

与方程式について、
\[ (x+1)\bigl[(x-1)^2(x^2+1)+ax^2\bigr]=0 \]
と整理できるから、明らかに$x=-1$を解の1つに持ち、残りの4解については、
\[ f(x)=(x-1)^2(x^2+1)+ax^2 \]
の振る舞いを調べればよい。

$a=0$のとき、与方程式は
\[ (x-1)^2(x+1)(x^2+1)=0 \,\dou\, x=1 \,,\,\, -1 \,,\,\, \pm i \]
と解ける。このことから一致する解は$1$である。

$a>0$のとき、常に$f(x)>0$であるから、与方程式は$x=-1$以外に虚数解を4つもつ。ここで$a$は実数であるから、この4つの虚数解は、2組の共役な虚数である。互いに共役な虚数同士は等しくなりえないから、少なくとも2つの解が一致するためには、この2組が一致しなければならない。このことから、
\[ f(x)=(x-1)^2(x^2+1)+ax^2 = (x^2+sx+t)^2 \]
と因数分解することを考える。これの係数を比較することで$(s, t, a)=(-1, 1, 1)$を得る。実際に$a=1$のとき、与方程式は、
\[ (x+1)(x^2-x+1)^2=0 \,\dou\, x=-1\,,\,\, \frac{1\pm\sqrt{3}i}{2} \]
と解けて、一致する解は$\dfrac{1\pm\sqrt{3}i}{2}$と求まる。

$a<0$のとき、$f(0)=1$, $f(1)=a<0$から、中間値の定理によって、$0<\alpha<1$なる実数$\alpha$によって$f(\alpha)=0$となることがわかる。また、
\[\lim_{x\to\infty} f(x) = \lim_{x\to\infty} x^4\left[\left(1-\frac{1}{x}\right)^2\left(1+\frac{1}{x^2}\right)^2+\frac{a}{x^2}\right] =+\infty\]
から、十分大きい実数$M>0$に対して、$f(M)>0$がいえる。これより、$f(1)=a<0$と$f(M)>0$から、中間値の定理によって、$\beta>1$なる実数$\beta$によって$f(\beta)=0$となることがわかる。これらより、$a<0$の場合には、方程式$f(x)=0$の虚数解は高々2つであるから、重解は存在するなら実数である。

一般に、方程式$f(x)=0$の実数解$x=t$が重解\footnote{少なくとも2次の}であることは、$f(t)=0$かつ$f'(t)=0$と同値である。
\begin{align*}
 f(t)&=(t-1)^2(t^2+1)+at^2=0 \\
 f'(t)&=2(t-1)(2t^2-t+1)+2at=0
\end{align*}
これらから$a$を消去し整理して、
\[ (t-1)(t+1)(t^2-t+1)=0 \quad\cdots\text{(*)}\]
が、$x=t$が方程式$f(x)=0$の重解となるための条件である。ここで、先述の$\alpha$, $\beta$は、明らかに(*)を満たさないから、$x=\alpha, \beta$は重解になり得ない。

ここまでのことから、$a<0$の場合に、与方程式が重解を持つためには、方程式$f(x)=0$が$x=-1$を解に持たなければならない。
\[ f(-1)=8+a=0 \quad\dou\quad a=-8 \]
よって、$a=-8$のとき、一致する解として$-1$が得られる。

以上によって、求めるものは
\begin{align*}
 \left\{
 \begin{aligned}
  a&=0\,\,\text{のとき、一致する解は}\,\, x=1 \\
  a&=1\,\,\text{のとき、一致する解は}\,\, x=\frac{1\pm\sqrt{3}i}{2} \\
  a&=-8\,\,\text{のとき、一致する解は}\,\, x=-1
 \end{aligned}
 \right.
\end{align*}