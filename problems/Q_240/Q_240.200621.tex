\begin{thm}{240}{\hosi ?}{}
 定規とコンパスによって後に示す操作(a), (b)を有限回行うことだけが許されている。座標平面内の2点$(0,0)$, $(1,0)$を始めに与え、有限回の操作の組み合わせから得られる平面上の点全体を$A$とする。このとき、$(\cos\dfrac{2}{17}\pi,0)\in A$を示せ。また、正17角形が有限回の操作(a), (b)だけで得られることを示せ。

 \begin{itemize}
  \setlength{\leftskip}{3eM}
  \item[操作(a)] 与えられた2点を結ぶ直線を描く
  \item[操作(b)] 与えられた点を中心とし、与えられた長さを半径とする円を描く
 \end{itemize}

 なお、$\cos\dfrac{2}{17}\pi$が次に示す値であることは認めてよい。
 \[ \!\!\frac{1}{16}\!\!\left(\sqrt{17}-1+\sqrt{34-2\sqrt{17}}+2\sqrt{17+3\sqrt{17}-\sqrt{170+38\sqrt{17}}}\right) \]
\end{thm}

ここに解答を記述。