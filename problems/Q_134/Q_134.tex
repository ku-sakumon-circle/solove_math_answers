\begin{thm}{134}{}{}
 $k, n$を正の整数とし、$S_k(n)=1+2^k+\cdots+(n-1)^k+n^k$ とする。
 \begin{enumerate}
  \item $S_2(n)$ が素数になる$n$を求めよ。 \hosi 1
  \item $S_4(n)$ が素数になる$n$を求めよ。 \hosi 5
  \item $m$を奇数とするとき、$S_m(n)$が素数になる組$(m,n)$ を求めよ。 \hosi 9
 \end{enumerate}
\end{thm}
(解答Ver.1: 2021/11/28)


\syoumon{1} 
教科書より$S_2(n) = \dfrac{n(n+1)(2n+1)}{6}$. この値が素数$p$であるとせよ. すなわち$n(n+1)(2n+1) = 6p$である. まず$n=1,2$で$p=1,5$なので$n=2$が適する. 以降$n\geq 3$とする. さて, $6p$に掛けられてある素数の個数は3個である. 一方で$n(n+1)(2n+1)$は3つの1でない整数の積であるから, それぞれの因数は素因数を一つは持つ. そのため, $6p$と等しくなるためには, $n$も$n+1$も$2n+1$もすべて素数でなければならない. $n\geq 3$としたので$n$が素数なら$n$は奇数だが, このとき$n+1$は4以上の偶数であるため不適. よって$n=2$のみ. ($S_2(2)=5$)

\syoumon{2}
$S_4(n)$を閉じた式で表したい. これはどうやるのかというと, 
\[\sum_{k=1}^{n}((k+1)^5 - k^5)\]
を二通りの方法で計算するのである. \\
\fbox{一つ目: 望遠鏡和で計算} \\
和がパタパタと消えるやつ(望遠鏡和)により$(n+1)^5 - 1^5 = n^5 + 5n^4 + 10n^3 + 10n^2 + 5n$に等しい.\\
\fbox{二つ目: 中身を展開する} $(k+1)^5 - k^5 = 5k^4 + 10k^3 + 10k^2 + 5k + 1$である. よってこの和は
\[5S_4(n) + 10(S_3(n) + S_2(n)) + 5S_1(n) + S_0(n)\]
である. 教科書より$S_3(n) = \dfrac{n^2(n+1)^2}{4}$, $S_1(n) = \dfrac{n(n+1)}{2}$, $S_0(n) = n$なので
{\small 
\begin{align*} 
&5S_4(n) +  \dfrac{10n^2(n+1)^2}{4} + \dfrac{10n(n+1)(2n+1)}{6} + \dfrac{5n(n+1)}{2} + n\\
&= 5S_4(n) + \dfrac{15n^2(n+1)^2 + 10n(n+1)(2n+1) + 15n(n+1) + 6n}{6} \\
&= 5S_4(n) + \dfrac{15n^4 + 50n^3 + 60n^2 + 31n)}{6}
\end{align*}
} 
以上二つの計算方法を比較して
\begin{align*} 
5S_4(n) &= (n+1)^5 - 1 - \dfrac{15n^4+50n^3 + 60n^2 +31n)}{6}\\
&= \dfrac{6n^5 + 15n^4 + 10n^3 - n}{6}\\
&= \dfrac{n(6n^4 + 15n^3 + 10n^2 - 1)}{6}\\
&= \dfrac{n(n+1)(6n^3 +9n^2 + n -1)}{6}\\
&= \dfrac{n(n+1)(2n+1)(3n^2 + 3n -1)}{6}
\end{align*}
を得る\footnote{$6x^3 + 9x^2 + x-1$を有理数の範囲で因数分解しようと期待する場合, $x=\dfrac{1}{\pm (6の約数)}$ の形の有理数解を探すとよい. $x=-\dfrac{1}{2}$で0になることが分かるので, $2x+1$で割れるというわけだ. }. 
さて, $S_4(n) = p$が素数であるとせよ. このとき
\[30p = n(n+1)(2n+1)(3n^2 + 3n - 1)\]
左辺は4つの素数の積であり, 右辺は4つの整数の積である. まず$n=1$の場合は$p=1$となり不適. $n\geq 2$の場合, (1)と同様にして右辺の4つの因数はすべて素数でなければならず, $n,n+1$の偶奇に注意すれば $n=2$でなければならない. $S_4(2) = 1+16 = 17$だから実際に素数になる. よって$n=2$が適する. ($S_4(2)=17$) 

\syoumon{3}
$S_m(n)$を閉じた式で表すことは難しいが別の方法がある. ヒントとしては, $S_1(n),S_2(n),S_3(n),S_4(n)$はいずれも因数分解したときの形に$n,n+1$が現れることだ. これと$M$が奇数であることを活用するとうまくいく. \\
さて, $S_m(n) = p$が素数であるとせよ. このとき
\[2p = \sum_{k=1}^{n} k^m + \sum_{k=1}^{n} (n+1-k)^{m} = \sum_{k=1}^{n} \left\{ k^m + (n+1 -k)^m \right\}\]
である. ここで$m$が奇数なので$k^m + (n+1-k)^m$は二項展開からも分かるように, $(-k)^m = -k^m$が$k^m$と相殺するので$n+1$で割り切れる(この時点で$n+1=2,p,2p$に絞りにかかっても良いと思う). さらに, 
\[2p = 2n^m + \sum_{k=1}^{n-1} (k^m + (n - k)^m) \]
であり, $2n^m$も二項目の和も$n$で割り切れると分かる. よって$n$も$n+1$も$2p$の約数である. $2p$の正の約数は$1,2,p,2p$であり, $p$と$2p$は差が1以上であることに注意すれば$(n,n+1) = (1,2),(2,p)$の可能性がありうる. 前者は不適である(
 $S_m(1) = 1$は素数でないから). 後者では$p=3$であり, $S_m(2) = 1 + 2^m =  3$となる$m$は$m=1$に限る. よって 求める組は
\[(m,n) = (1,2)\]


