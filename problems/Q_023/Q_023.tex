\begin{thm}{023}{\hosi 3}{}
 $2^n\times2^n$のチェス盤から、$1\times1$の正方形を1つ取り除いたものを`欠損チェス盤'と呼ぶことにする。この欠損チェス盤は3つの$1\times1$の正方形からなるL字形のブロックを用いて充填できることを示せ。
\end{thm}

数学的帰納法により証明する。$n=1$であるとき、欠損チェス盤はL字型ブロックそのものであるから、明らかに充填可能である。

\begin{wrapfigure}[12]{r}[0pt]{100pt}
 \centering
 \includegraphics[width=\linewidth]{../problems/Q_023/A_023.png}
\end{wrapfigure}

$n=k$~($k\ge 1$) において、$2^k\times 2^k$の任意の欠損チェス盤が充填可能であったとする。$2^{k+1}\times 2^{k+1}$の欠損していないチェス盤を、4つの$2^k\times 2^k$のチェス盤$A, B, C, D$の集まりと考える。ここから1つの$1\times 1$の小正方形を任意に取り除く。すると$A, B, C, D$のうちの一つから小正方形が取り除かれるが、それを$A$としても一般性は失われない。小正方形を取り除いた$A$を$A^*$と表すとき、$A^*$は$2^k\times 2^k$の欠損チェス盤とみなせるから、帰納法の仮定によりL字型ブロックで充填できる。続いて、$B$の左下隅、$C$の右上隅、$D$の左上隅の小正方形(図の赤部)を取り除いたものをそれぞれ$B^*, C^*, D^*$とおくと、これらはすべて$2^k\times 2^k$の欠損チェス盤とみなせるから帰納法の仮定によりL字型ブロックで充填できる。最後に、$B, C, D$から取り除いた小正方形は1つのL字型ブロックで埋められる。以上により、$2^{k+1}\times 2^{k+1}$の欠損チェス盤全体がL字型ブロックで充填できる。

したがって、数学的帰納法により、全ての自然数$n$で$2^n\times 2^n$の欠損チェス盤をL字型ブロックで充填できる。