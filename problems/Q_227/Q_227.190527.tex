\begin{thm}{227}{\hosi 8}{自作, 学コン 2018-9-3}
 $xy$平面上の放物線G:~$y=mx^2+px+q^n$ を考える。Gと$x$軸は2つの交点を持つものとする。各交点におけるGの2本の接線と$x$軸で囲まれる部分の面積を$S$とするとき、
 \begin{enumerate}
  \item $S$を$m, n, p, q$で表せ。
  \item $m$を整数、$n$を自然数、$p, q$を素数とする。$S$が自然数となるとき、$p$および$S$を求めよ。
 \end{enumerate}
\end{thm}

$m=0$ではGは直線であるので、以下では$m\neq 0$とする。

\syoumon{1}
Gは$x$軸と2つの交点を持つことから、2次方程式$mx^2+px+q^n=0$の判別式を考えて$p^2-4mq^n>0$であることが必要。この2つの実数解を$\alpha, \beta$~$(\alpha>\beta)$とおく。解と係数の関係から、
\[ m(\alpha+\beta)=-p,\quad m\alpha\beta=q^n \]
が導かれる。

Gにおいて、$y'=2mx+p$より、$x=t$における接線の方程式は、
\[ y=(2mt+p)x-mt^2+q^n \]
である。$t=\alpha, \beta$での式を連立することで、
\[ (2m\alpha+p)x-m\alpha^2+q^n=(2m\beta+p)x-m\beta^2+q^n \,\dou\, x=\frac{\alpha+\beta}{2} \]
より、2接線の交点の$x$座標は$\dfrac{\alpha+\beta}{2}$とわかる。交点の$y$座標については、
\[ (2m\alpha+p)\frac{\alpha+\beta}{2}-m\alpha^2+q^n=-\frac{m}{2}(\alpha-\beta)^2 \]
と求まる。$S$は、底辺の長さが$\alpha-\beta$、高さが$\dfrac{|m|}{2}(\alpha-\beta)^2$の三角形の面積であるから、
\begin{align*}
 S&=\frac{1}{2}(\alpha-\beta)\cdot\frac{|m|}{2}(\alpha-\beta)^2 = \frac{|m|}{4}(\alpha-\beta)^3 \\
 &=\frac{|m|}{4}\left(\frac{\sqrt{p^2-4mq^n}}{|m|}\right)^3 = \frac{(p^2-4mq^n)^{\frac{3}{2}}}{4m^2}
\end{align*}
と求められた。

\syoumon{2}
$S$が自然数となるので、$S^2=\dfrac{(p^2-4mq^n)^3}{16m^4}$も自然数である。この分母は偶数であるから、分子も偶数で無ければならず、$p$は偶数である。かつ$p$は素数であるから、$p=2$。

$p^2-4mq^n=4(1-mq^n)>0$であることが必要であったから、$mq^n\le 0$より$m<0$である。$-m=M$とおけば$M$は自然数であり、
\[ S^2=\frac{(4+4Mq^n)^3}{16M^4}=\frac{4(1+Mq^n)^3}{M^4} \]
も自然数である。ここで$M$が偶数であるとすると、$1+Mq^n$は奇数のため、分子$4(1+Mq^n)^3$は8の倍数でない。一方で分母$M^4$は16の倍数であり、$S^2$が自然数でなくなるため不適。よって$M$は奇数である。さらに、
\[ 4(1+Mq^n)\,\,\text{が$M$の倍数} \quad\dou\quad \text{4が$M$の倍数} \]
である。つまり$M$は4の正の約数のうち奇数のものであるから、$M=1$。

$M=1$を代入して$S^2=4(1+q^n)^3=(2+2q^n)^2(1+q^n)$となる。$S^2$と$(2+2a^n)^2$がともに平方数なので、$1+q^n$も平方数である。よっては2以上の自然数$k$を用いて
\[ 1+q^n=k^2 \,\dou\, q^n=(k-1)(k+1) \]
とかける。$q$は素数であるから、$q^n$の約数は$1, q, q^2, \dots, q^n-1, q^n$である。

$k\ge 3$のとき、$k\pm 1$は1より大きい$q^n$の約数であるから、$k\pm 1$の双方がqの倍数である。したがって$(k+1)-(k-1)=2$も$q$の倍数なので、$q=2$である。$2^n=(k-1)(k+1)$において、$k$は3以上の奇数であり、$k-1$と$k+1$は隣り合う偶数ゆえ、どちらか一方は4の倍数ではない。かつ1より大きい$2^n$の約数であるから、$k-1=2$, $k+1=4$のみが適するから、このとき$k=3$, $n=3$である。

$k=2$のとき、$q^n=3$より、$q=3$, $n=1$である。

以上より、$(p, q, m, n)=(2, 2, -1, 3), (2, 3, -1, 1)$が題意を満たす組で、このとき$S$は
\[ S=\frac{(p^2-4mq^n)^\frac{3}{2}}{4m^2}=54, 16 \]