\begin{thm}{051}{}{}
 $a, b, c$を非負整数とする。
 \begin{enumerate}
  \item 任意の自然数$n$に対して、$an^2+bn+c$が素数であるならば、$a=b=0$かつ$c$は素数であることを示せ。 \hosi 3
  \item 任意の自然数$n$に対して、$a^n+b^{n-1}$が素数ならば、$a=b=1$であることを示せ。 \hosi ? (suiso\_728660 様)
 \end{enumerate}
\end{thm}

\syoumon{1}
$c=0$とすると、$an^2+bn$は$a, b$が0でなければ常に$n$の倍数となるから不適であるし、$a=b=0$ならば$an^2+bn+c$は常に0となるのでやはり不適である。よって$c\neq 0$。

$n=c$として、$ac^2+bc+c=c(ac+b+1)$が素数だから、(i)~$c$が素数かつ$ac+b+1=1$、または、(ii)~$c=1$かつ$ac+b+1$が素数、であることが必要。

(i)~この場合には、$a=b=0$が必要であり、実際に$a=b=0$かつ$c$が素数の場合には、任意の自然数$n$に対して$an^2+bn+c$が素数となるからよい。

(ii)~この場合には$a+b+1$が素数であることが必要である。これを$p$とおく。$n=p+1$として、
\[ a(p+1)^2+b(p+1)+1=ap^2+2ap+bp+(a+b+1)=p\left[ap+2a+b+1\right] \]
が素数であることが必要である。よって$ap+2a+b=0$でなければならない。これを満たすのは$a=b=0$のみであるが、$a+b+1$が素数であることと矛盾する。よってこの場合は不適である。

以上より、$a=b=0$かつ$c$が素数であることが示された。

\syoumon{2}
$n=1$として、$a+1$が素数だから、$a\ge 1$であることが必要。

$n=2$として、$a^2+b$が素数である。これを$p$とおく。

$n=p+1$として、$a^{p+1}+b^p$は素数である。これを$q$とおく。$a, b$は$p$と互いに素であるから、Fermatの小定理により
\[ a^{p+1}+b^p\equiv a^2+b\equiv 0 \pmod{p} \]
となるので、$q$は$p$で割れる素数だから$q=p$でなければならない。すなわち、
\[ a^{p+1}+b^p=a^2+b \]
を満たさなければならない。$a\ge 1$, $b\ge 0$により
\[ 0\le a^{p+1}-a^2=b-b^p\le 0 \]
だから$a^{p+1}=a^2$かつ$b=b^p$である。よって$a, b$ともに0か1に限られる。

$a\ge 1$であったから、$a=1$である。$b=0$とすると、$n=2$のとき$p=1$となって素数で無いから不適。よって$b=1$に限られる。$a=b=1$のとき、$a^n+b^{n-1}=2$は実際に全ての$n$で素数となるからよい。よって$a=b=1$であることが示された。
