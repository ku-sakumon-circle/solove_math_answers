\begin{thm}{228}{}{}
 $f(x)=x^2+x+1$とする。また、$i$を虚数単位として $\omega=\dfrac{-1+\sqrt{3}i}{2}$とする。
 \begin{enumerate}
  \item $\left|f\left(\sqrt[3]{2}\omega\right)\right|<1$でることを示せ。
  \item $f\left(\sqrt[3]{2}\right)^{2019}$に最も近い整数を8で割った余りを求めよ。
 \end{enumerate}
\end{thm}

\syoumon{1}
$f(x)$が実数係数であるから, $f(\sqrt[3]{2}\omega )$の共役複素数は
\[f(\sqrt[3]{2}\omega^2)=\sqrt[3]{4}\omega^4 +\sqrt[3]{2}\omega +1\]
なので, $f(x)=\dfrac{x^3-1}{x-1}$ と$\omega^3=1$により
\begin{align*}
\left|f\left(\sqrt[3]{2}\omega\right)\right|^2&=\dfrac{(\sqrt[3]{2}\omega)^3-1}{\sqrt[3]{2}\omega-1}\cdot \dfrac{(\sqrt[3]{2}\omega^2)^3-1}{\sqrt[3]{2}\omega^2-1}\\
&=\dfrac{1\cdot 1}{\sqrt[3]{4}-(\omega +\omega^2)\sqrt[3]{2} +1}\\
&= \dfrac{1}{\sqrt[3]{4}+\sqrt[3]{2}+1}\\
&< \dfrac{1}{1+1+1}=\dfrac{1}{3} <1
\end{align*}
により $|f(\sqrt[3]{2})|<\dfrac{1}{\sqrt{3}}<1$で成立する。\qed

\syoumon{2}
$f(x)^{2019}=P(x)$とおく。$P(\sqrt[3]{2})$は, 実は「ほぼ整数」である。そして,この数に最も近い整数は
\[P(\sqrt[3]{2})+P(\sqrt[3]{2}\omega)+P(\sqrt[3]{2}\omega^2)\]
で与えられる。そのことを今から示そう。\\
$P(x)$を展開すると, 正の整数$a_0,a_1,\cdots a_{4038}$を用いて
\[P(x)=\disp\sum_{k=0}^{4038}a_kx^k\]
と表される。このとき,
\[P(x)+P(\omega x)+P(\omega^2 x)=\disp\sum_{k=0}^{4038}a_kx^k(1+\omega^k+\omega^{2k})\]
となる。$1+\omega^k +\omega^{2k}$は $k$が$3$で割り切れるときは$3$に, それ以外の場合には$0$となることがわかる。つまり上の多項式は$x^{3k}$の項しか登場せず
\[P(x)+P(\omega x)+P(\omega^2 x)=\disp\sum_{m=0}^{1346}3a_{3m}x^{3m}\]
となる。$x=\sqrt[3]{2}$を代入すると
\[P(\sqrt[3]{2})+P(\sqrt[3]{2}\omega )+P(\sqrt[3]{2}\omega^2)=\disp\sum_{m=0}^{1346}3a_{3m}\cdot 2^m\]
となり, 右辺で根号が消滅し, これが整数になることがわかる。この整数を$N$とおくと
\[P(\sqrt[3]{2})=N-(P(\sqrt[3]{2}\omega)+P(\sqrt[3]{2}\omega^2))\]
とかけるが, $P(x)$もまた実数係数多項式なので$P(\sqrt[3]{2}\omega )$の共役複素数が$P(\sqrt[3]{2}\omega^2)$である。よって
\[P(\sqrt[3]{2})=N-2\mbox{Re}(P(\sqrt[3]{2}\omega))\]
である。$|Re|\leq \sqrt{(Re)^2+(Im)^2}$と, (1)より
\[|\mbox{Re}(P(\sqrt[3]{2}\omega))|\leq |P(\sqrt[3]{2}\omega)|<\left(\dfrac{1}{\sqrt{3}}\right)^{2019}<0.25\]
だから, 
\[N-0.5<N-2\mbox{Re}(P(\sqrt[3]{2}))<N+0.5\]
となり, $P(\sqrt[3]{2})$に最も近い整数が$N$であることを意味する。\\
つまりこの$N$を8で割ったあまりを考察すればよい。
\[N=3a_0+6a_3+12a_6+24a_9+\cdots\]
であったから, $\disp\sum_{m=0}^{1346}3a_{3m}\cdot 2^m$の$m\geq 3$以降の項は8を法として0になることが分かる。つまり
\[N\equiv 3a_0+6a_3+12a_6 (mod 8)\]
なので $a_0,a_3,a_6$が求まればよい。

$3a_0=3P(0)=3$である。\\
$a_3$は, $(1+x+x^2)^{2019}$の$3$次の係数だから, $1^{2017}(x)^1(x^2)^1$ $1^{2016}(x)^3(x^2)^0$の組み合わせで多項定理により係数を求めると
\[a_3=\dfrac{2019!}{2017!1!1!}+\dfrac{2019!}{2016!3!0!}\]
となるから,
\[6a_3\equiv 6\cdot 2018\cdot 2019 + 2017\cdot 2018\cdot 2019 \equiv 4+6\equiv 2 \]
続いて$a_6$を計算する。
\[ 1^{2013}(x)^6(x^2)^0, 1^{2014}(x)^4(x^2)^1, 1^{2015}(x)^2(x^2)^2, 1^{2016}(x)^0(x^2)^3 \]
のパターンを調べれば
\[a_6=\dfrac{2019!}{2013!6!0!}+\dfrac{2019!}{2014!4!1!}+\dfrac{2019!}{2015!2!2!}+\dfrac{2019!}{2016!0!3!}\]
となる。先頭3つについては, 分母にある$6!0!, 4!1!, 2!2!$が高々2で4回しか割れず, $2016$が2で6回割れることを考慮すれば偶数になることが分かる。つまり, 先頭3つは12倍して8で割れる数になるから
\[12a_6\equiv 0+0+0+12\cdot\dfrac{2017\cdot 2018\cdot 2019}{6}\equiv 4\cdot (奇数)\equiv 4 (mod 8)\]
と分かる。以上より
\[N\equiv 3a_0+6a_3+12a_6\equiv 3+2+4\equiv 1\]
だから答えは1である。
