\begin{thm}{175}{\hosi 10}{自作}
 $n$を2以上の整数とする。任意の素数$p$に対して $\dfrac{p^n+1}{p+1}$ は$n^2$で割り切れないことを証明せよ。
\end{thm}

分子が分母で割りきれなければならないので, $p^n+1\equiv 0 \, (\mbox{mod} p+1)$ である。$(-1)^n+1\equiv 0$ (mod  $p+1$)だから $n$は奇数でなければならない。$n$は3以上の奇数なので, 素因数が存在しており, かつそれらはすべて奇素因数である。\\
そのような奇素因数のうち, 最小のものを$q$とおく。以下で登場する合同式はすべて mod $q$で考えるものとする。$\dfrac{p^n+1}{p+1}$は$n^2$の倍数なら $q^2$の倍数なので $p^n\equiv -1$ . 二乗して $p^{2n}\equiv 1$  である. $p=q$だとこの式は成り立たないので $p\neq q$であり, $p,q$は互いに素である。ゆえにフェルマーの小定理より $p^{q-1}\equiv 1$ である。\\
\\
$p$のmod $q$における位数を$d$とする。つまり, $d$は $p^d\equiv 1$ を満たす最小の正の整数である。このとき, $d$は$p^m\equiv 1$  を満たす整数$m$を常に割り切るので, $d$は$2n$と$q-1$ を割り切る整数になっている。($d$はこの2数の公約数である) ここで, $q-1$は$q$未満の素数で素因数分解され, $q$は$n$の最小素因数をとったので $n$ と $q-1$には共通した素因数が存在しない。したがって$d$は$2$と$q-1$の公約数でもあり, $1,2$ があり得る。\\
\\
$d=1$ の とき, $p\equiv 1$ と $p^n\equiv -1$ から $1\equiv -1$ となり, $q\geq 3$に矛盾する。\\
$d=2$ の とき, $p^2\equiv 1$ であり,  $p\equiv -1$ となる。($p\equiv 1$では位数の定義に矛盾)\\
このとき$p+1\equiv 0$だから $p+1$は$q$で割り切れる。よって, $v_{q}(p+1)\geq 1$ で, $p^n+1=p^n-(-1)^n$ に対して LTE lemma を適用することができるので
\[v_q(p^n+1)=v_q(p+1)+v_q(n)\]
となる。$\dfrac{p^n+1}{p+1}$が$q$で割り切れる回数は $v_q{(p^n+1)}-v_q(p+1)=v_q(n)$ である。 ここでもし自然数$k$が存在して $\dfrac{p^n+1}{p+1}=kn^2$ となるなら, $v_q(n)>0$に注意して
\[v_q(kn^2)=2v_q(n)+v_q(k)\geq 2v_q(n)>v_q(n)=v_q\left(\dfrac{p^n+1}{p+1}\right)\]
となって矛盾するから, $n\geq 3$の奇数において, いかなる素数$p$を取っても $\dfrac{p^n+1}{p+1}$は$n^2$の倍数とならない。よって, 題意は示された。

