\begin{thm}{254}{\hosi 8}{京府医3 (2021)}
 $a$は$a>1$を満たす実数とする。1辺の長さ$a$の正方形である面を1つ、3辺の長さが$a,1,1$の二等辺三角形である面を1つ、4辺の長さが$a,1,1,1$の台形である面を2つ用意し、これらを組み合わせて5つの面で囲まれた立体$F$ができたとする。
 \begin{enumerate}
  \item 立体$F$において、正方形の面に平行な長さ$1$の辺がある。その辺上の点から正方形の面に引いた垂線の長さ$h$を$a$で表せ。
  \item 立体$F$において、正方形の面と台形の面のなす角を$\theta_1$とし、正方形の面と二等辺三角形の面のなす角を$\theta_2$とするとき、$\theta_1+\theta_2=\dfrac{\pi}{2}$となる$a$の値を求めよ。
  \item (2)で求めた$a$の場合を考える。1辺の長さが$a$の立方体にいくつかの$F$を正方形の面でうまくはり合わせると正十二面体ができる。この事実を利用して1辺の長さが$1$の正十二面体の体積を求めよ。
 \end{enumerate}
\end{thm}

ここに解答を記述。