\begin{thm}{252}{\hosi 8\maru}{自作 学コン2020-11-5 原題}
 \begin{enumerate}
  \item $x^2+y^2=3z^2$ の整数解は $(x,y,z)=(0,0,0)$に限ることを証明せよ。
  \item $xy$平面において、$x$座標と$y$座標がともに有理数であるような点を有理点と呼ぶこととする。$\theta$を$0<\theta\le\dfrac{\pi}{2}$を満たす定数とし、次の条件を満たす$xy$平面上の三角形$\mr{ABC}$を考える。
  \begin{itemize}
   \item[(条件)] $\mr{A}$は有理点であり、$\mr{AB}$の長さは有理数である。さらに$\angle\mr{ACB}=\theta$である。
  \end{itemize}
  三角形$\mr{ABC}$の外心を$\mr{X}$とする。次の問に答えよ。
  \begin{itemize}
   \item[(a)] $\theta=\dfrac{\pi}{3}$のとき、$\mr{X}$は有理点でないことを証明せよ。
   \item[(b)] $\sin\theta=\dfrac{1}{\sqrt{2020}}$のとき、$\mr{X}$が有理点となることはあるか。ないならば証明し、あるならば$\mr{A,B,C}$の座標を挙げそれらが条件を満たすことを示せ。 
  \end{itemize}
 \end{enumerate}
\end{thm}

\syoumon{1}
$(x,y,z) \neq (0,0,0)$ なる解が存在したとする。$x^2 + y^2\equiv 0\pmod{3}$であるが, 一般に自然数$n$に対して$n^2\equiv 0,1\pmod{3}$であるため, $x,y$はともに$3$の倍数でなければならない。このとき, $3z^2$は$9$の倍数となるから, $z$は$3$の倍数である。すると, この方程式の解に関して
\[(x,y,z) \,\text{が整数解ならば} \,\left(\frac{x}{3}, \frac{y}{3}, \frac{z}{3}\right) \,\text{が整数解} \]
が言えるから, この議論を繰り返して$N=0,1,2\cdots$に対して$(\frac{x}{3^{N}}, \frac{y}{3^{N}}, \frac{z}{3^{N}})$ が整数解となる。しかし, $x,y,z$のどれかが$0$ではないから, その$0$でないものに関して, 十分大きい$N$に対してそれを$3^{N}$で割ったものが整数でなくなるから矛盾する。よってこのような解は存在せず, $(x,y,z) = (0,0,0)$は自明な解であるから示された。

\syoumon{2a}
背理法によって証明する。$\mr{X}$が有理点であるような三角形$\mr{ABC}$が存在したとする。ここで, そのようなものが存在するならば, $\mr{A}$の座標は$(0,0)$であるとしてよい。なぜなら, $\mr{A}$と原点はともに有理点であり, $A$を原点に動かす平行移動によってすべての有理点は有理点へと移動し, $X$の移動先も有理点であるからである。さらに, $\mr{AB}$は有理数であるから, 有理数倍の原点を中心とした拡大縮小によって有理点が有理点に移ることを利用し, $\dfrac{1}{\mr{AB}}$倍によって$\mr{AB}=1$であるとしてもよい。

$X$の座標を$(p,q)$とする($p,q$は有理数)。三角形$\mr{ABC}$の外接円の半径を$R$とすれば, その外接円の方程式は
\[(x-p)^2 + (y-q)^2 = R^2\]
で与えられ これは原点($=\mr{A}$) を通るので$p^2 + q^2 = R^2$が従う。正弦定理によって
\[R = \dfrac{1}{2\sin{\theta}} = \dfrac{1}{\sqrt{3}}\]
であるから, $p^2 + q^2 = \dfrac{1}{3}$である。$p,q$は有理数であったから, $t=p,q$に関して 自然数$x_t$, 整数$y_t$を取って
\[t = \dfrac{y_t}{x_t}\]
と表示することができる。すると, 
\begin{align*}
 & p^2+q^2 = \dfrac{1}{3} \\
 \dou\quad & \left(\frac{y_p}{x_p}\right)^2 + \left(\frac{y_q}{x_q}\right)^2 = \dfrac{1}{3}\\
 \dou\quad & (3y_px_q)^2 + (3y_qx_p)^2 = 3(x_px_q)^2
\end{align*}
となり, $x_px_q \neq 0$であるから(1)に反する。よって矛盾であり, $\mr{X}$は有理点でない。

\syoumon{2b}
\bf{存在する。}まず, 不定方程式$x^2 + y^2 = 2020z^2$は非自明な有理数解$(x,y,z) = (12,19,\frac{1}{2})$を持つ\footnote{$(21,8,\frac{1}{2})$でもOK。}ことに注意する。$x_0=12, y_0=19$とおく。

$A$を原点とする。$\mr{AB} = 1$となるように作ろう。このとき, 正弦定理より外接円半径は$\sqrt{505}$である。有理点$\mr{X}(p,q)$が取れたとするなら, 外接円の方程式は
\[(x-p)^2 + (y-q)^2 = 505\]
を満たし, それが原点を通るから$p^2 + q^2 = 505$を満たすべきである。そこで, $p,q$として$p=x_0$, $q=y_0$を取ろう。

この$\mr{A}(0,0)$と$\mr{X}(x_0,y_0)$から条件を満たすように$\mr{B},\mr{C}$を取ろう。$\mr{B}$の座標を$(a,b)$とする。$\mr{AB}$の垂直二等分線上に外心$\mr{X}$があるように点$\mr{B}$を取るべきであるが, その垂直二等分線の方程式は
\[ax+by = \dfrac{1}{2}\]
である。よって, $a,b$は$a^2 + b^2 = 1$かつ$12a + 19b=\dfrac{1}{2}$を満たすように取ればよく, これを解いて
\[a= \frac{6}{505} \pm \frac{19\sqrt{2019}}{1010},\quad b=\frac{19}{1010}\mp \frac{6\sqrt{2019}}{505}\quad (複合同順)\]
を得るので, そのうちの一つとして
\[B = \left(\frac{6}{505} - \frac{19\sqrt{2019}}{1010},\frac{19}{1010}+ \frac{6\sqrt{2019}}{505}\right) \]
を選ぶ。$\mr{XA} = \mr{XB}$を満たすように取ったから, $(x-x_0)^2 + (y-y_0)^2 = 505$上に$\mr{A},\mr{B}$が存在する。$\mr{C}$は, この円の弧$\mr{AB}$のうち, 長いほうから任意にとればよい。なぜなら, その弧に対する円周角($\angle{\mr{ACB}}$)は鋭角であり, 正弦定理より$\sin{\angle{\mr{ACB}}} = \dfrac{1}{2\sqrt{505}} =\dfrac{1}{\sqrt{2020}}$であるから, $\angle{\mr{ACB}} = \theta$が満たされる。たとえば, $\mr{C}$として$(0,2y_0) = (0,38)$を取る。この点は, $\mr{X}$を通る直線$y=y_0$より上にある。一方で, $\mr{A},\mr{B}$は下にあるので, この$\mr{C}$は弧$\mr{AB}$のうち長いほうに属することも明らかである。以上より, $\mr{A},\mr{B},\mr{C}$の一例として
\[ \mr{A}(0,0), \mr{B}\left(\dfrac{6}{505} - \dfrac{19\sqrt{2019}}{1010},\dfrac{19}{1010}+ \dfrac{6\sqrt{2019}}{505}\right), \mr{C}(0,38)\]
を与えることが出来る。

\syoumon{余談}
一般化すると次が成り立つ。
\begin{subthm}{252.1}
$0< \theta  \leq \frac{\pi}{2}$を満たす定数$\theta$に対して, 次は同値である。
 \begin{enumerate}
  \item (\bf{条件})を満たす三角形$\mr{ABC}$が存在する。
  \item $\sin{\theta}$は有理数である。または, $4k+3$型素数の約数と平方因子を持たない自然数$n$と, (正の)有理数$r$を用いて$\sin{\theta} = r\sqrt{n}$と表すことが出来る。
 \end{enumerate}
\end{subthm}
この証明のためには, (2)の条件にあるような$n$に関して$x^2 + y^2 = nz^2$が非自明な整数解を持ち, 逆に平方因子を含まず$4k+3$型素数の約数を持てば非自明な整数解を持たないことを示す必要があり, それは「二平方和定理」より従う。このことを事実として認めれば, (2) (a)と同じ方法によって, (1)$\naraba$ (2)が従う。(2)$\naraba$(1)は(2) (b)と同じような構成で分かる。また, $\theta$を鈍角にしてもよい($\mr{C}$を取るときに短い方の弧$\mr{AB}$から取ればよい)。
