\begin{thm}{242}{\hosi 1}{}
 $\sin 1$, $\sin 2$, $\sin 3$, $\sin 4$, $\pi-3$, $\cos 1$, $\tan 1$ を小さい順に並べよ。\\
 なお、$\pi=3.14\cdots$ である。
\end{thm}

はじめに、$\sin 1$, $\sin 2$, $\sin 3$, $\cos 1$ の大小を決定する。そのためにこれらを全て$\sin x$ ($0\le x\le \dfrac{\pi}{2}$)の形で表したい。$\sin 1$についてはすでによい。その他は
\[ \sin 2=\sin (\pi-2) \,,\,\, \sin 3=\sin (\pi-3) \,,\,\, \cos 1=\sin \left(\frac{\pi}{2}-1\right) \]
である。$\sin x$が$0\le x\le \dfrac{\pi}{2}$で単調増加であることを踏まえれば、これらの大小を決定するには$1, \pi-2, \pi-3, \dfrac{\pi}{2}-1$の大小を見ればよい。$3.14<\pi=3.14\dots <3.15$より、
\[ \pi-3 < \frac{\pi}{2}-1 < 1 < \pi-2 \]
であるから、
\[ \sin 3 < \cos 1 < \sin 1 < \sin 2 \]

$\pi < 4 < 2\pi$より、$\sin 4 < 0$である。これ以外の数は全て正である。

$\tan 1 > \tan\dfrac{\pi}{4}=1$ である。これ以外の数は全て1未満である。

よって、$\pi-3$以外の数については次のように大小が決まる。
\[ \sin 4 < \sin 3 < \cos 1 < \sin 1 < \sin 2 < \tan 1 \]

$x > 0$のとき、$\sin x < x$であった\footnote{この証明は非常に基本的なので略}\footnote{編集者註: 原文では「$0\le x$のとき$\sin x\le x$」となっていましたが、等号を含まない方が議論に都合がよいので変更しました。}。これに$x=\pi-3$を代入すれば$\sin (\pi-3)=\sin 3 < \pi -3$となる。

$3.14<\pi=3.14\dots <3.15$と、$0\le x\le \dfrac{\pi}{2}$で$\cos x$が単調減少であることから、
\[ \cos 1 > \cos \frac{\pi}{3}=\frac{1}{2} > 0.15 > \pi -3 \]
なので、$\pi -3 < \cos 1$である。

以上より、
\[ \sin 4 < \sin 3 < \pi -3 < \cos 1 < \sin 1 < \sin 2 < \tan 1 \]