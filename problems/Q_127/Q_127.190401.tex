\begin{thm}{127}{\hosi 7}{自作}
 正の約数の個数が$\sqrt{N}$個であるような正の整数$N$を求めよう。
 \begin{enumerate}
  \item $N$は奇数であることを示せ。
  \item $\sqrt{N}$以下の$N$の正の約数はいくつあるか。
  \item $N$を全て決定せよ。
 \end{enumerate}
\end{thm}

\syoumon{1}
$\sqrt{N}$が個数を表すためには自然数でなければならないので,  自然数$n$を用いて $N=n^2$ とおける。
\[n=p_1^{e_1}p_2^{e_2}\cdots p_r^{e_r}\]
と素因数分解できたとする。($p_1, \cdots, p_r$は相異なる素数, $e_i\geq 1$ ($\forall i$)) $n^2$の約数の個数は
\[(2e_1+1)(2e_2+1)\cdots (2e_r +1)\]
となり, 奇数の積になるから奇数である。これが$\sqrt{N}=n$に等しいので, $n$は奇数。

\syoumon{2}
$N=n^2$の約数$d_1,d_2,\cdots, d_{2k-1}$が
\[1=d_1<d_2<\cdots d_{2k-2}<d_{2k-1}=N\]
となるとする。このとき, $i=1,2,\cdots ,k$に対して$d_id_{2k-i}=N$となり, とくに$d_k^2=N$となるから$d_k=n$である。よって, $\sqrt{N}$以下の約数は$d_1,\cdots, d_k$の$k$個であり, $2k-1=\sqrt{N}=n$ だから $\dfrac{n+1}{2}$ 個。

\syoumon{3}
$N=1$はよいので $N\geq 9$のときを考える。$1$から$n$までの奇数は$\dfrac{n+1}{2}$個あり,(2)のものと一致するので, $1,3,\cdots, n$がすべて$n$の約数にならなければならない。特に, $n-2$が$n$の約数になるので
\[\dfrac{n}{n-2}=1+\dfrac{2}{n-2}\]
が整数になる。そのような$n$は$n=3$のみで$N=9$となる。よって$N=1,9$

