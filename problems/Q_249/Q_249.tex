\begin{thm}{249}{\hosi 7}{東北大数学科院試 H31共同}
 $\mathbb{R}$上で定義された実数値関数$\Phi$が凸関数であるとき、すなわち任意の$x,y\in\mathbb{R}$と$t\in (0,1)$ に対して
 \[ \Phi\bigl(tx+(1-t)y\bigr)\le t\Phi(x)+(1-t)\Phi(y) \]
 を満たすとき、次の問に答えよ。
 \begin{enumerate}
  \item $\Phi$は$\mathbb{R}$上連続であることを示せ。
  \item $\disp \sum_{j=1}^n t_j=1$ を満たす$t_j>0$と$x_j\in\mathbb{R}$ $(j=1,2,\dots, n)$ に対して $\disp \Phi\left(\sum_{j=1}^nt_jx_j\right)\le \sum_{j=1}^nt_j\Phi(x_j)$ が成り立つことを示せ。
  \item $a>0$とする。区間$[0,a]$上の実数値連続関数$f$に対して $\disp \Phi\left(\frac{1}{a}\int_0^a\!f(x)\,dx\right)\le\frac{1}{a}\int_0^a\!\Phi\bigl(f(x)\bigr)\,dx$ が成り立つことを示せ。
 \end{enumerate}
\end{thm}

ここに解答を記述。