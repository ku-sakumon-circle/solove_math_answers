\begin{thm}{270}{\hosi 7}{自作問題・学コン2021-6-6}
$\dfrac{n!}{m^{n-1}}$が整数であるような$2$以上の整数$m,n$の組をすべて求めよ. 
\end{thm}

分母と分子の2で割り切れる回数について考える. $m=2^{e}\cdot a$と表す($e\geq 0, a\geq 1$, $a$は奇数). 
このとき$m^{n-1}$は$2$で$e(n-1)$回割り切れる. 一方で$n!$は
\[
\sum_{k=1}^{\infty} \left[ \dfrac{n}{2^k} \right]    
\] 
回割り切れる. この値を$V(n)$と呼ぼう. $\dfrac{n!}{m^{n-1}}$が整数であるためには, $V(n)\geq e(n-1)$以上でなければならないが, 
\[
    V(n) <  \sum_{k=1}^{\infty} \dfrac{n}{2^k} = n    
\]
なので, $e(n-1) < n$の必要がある. これは$e< \frac{n}{n-1} = 1 + \dfrac{1}{n-1} \leq 2$を導くので, $e=1$を得る. 
もし$\dfrac{n!}{m^{n-1}}$が整数であるなら, $\dfrac{n!}{a^{n-1}}$も整数である. もし$a>1$であるなら, $a$はある素数$p\geq 3$で割り切れ, $\dfrac{n!}{p^{n-1}}$は整数である. 
よって$n!$は$p$で$n-1$回は割り切れなければならないが, 実際に割り切れる回数は
\[
\sum_{k=1}^{\infty} \left[ \dfrac{n}{p^k}\right] < \sum_{k=1}^{\infty} \dfrac{n}{3^k} = \dfrac{n}{2} \leq n-1     
\]
と評価されるため, 矛盾である. したがって$a=1$であり, $m = 2$である. \par 
次に$V \geq n-1$となるような$n$の条件について調べる. $n=2^{f}\cdot b$と表す($f\geq 0, b\geq 1$, $b$は奇数). 
\begin{align*}
    V(n) &= \sum_{k=1}^{f} \left[ \dfrac{2^{f}b}{2^{k}} \right] + \sum_{k>t} \left[\dfrac{b}{2^{k-t}}\right] \\ 
      &= \sum_{k=1}^{f} 2^{f-k} \cdot b + V(a) \\ 
      &= b(2^{f} - 1) + V(b) \\ 
      &= n - b + V(b) 
\end{align*}
よって, $V(n) \geq n-1$であることと$V(b) \geq b-1$であることは同値である. $b$は奇数であるので, $b!$と$(b-1)!$の2で割り切れる回数は同じである. よって同様の評価で
\[
V(b) = \sum_{k=1}^{\infty} \left[ \dfrac{b-1}{2^{k}} \right] \leq \sum_{k=1}^{\infty} \left[\dfrac{b-1}{2^k} \right] = b-1   
\]
が得られる. よって$V(b) \geq b-1$と$V(b) = b-1$は同値であり, このとき上の不等式評価で等号が成立していることを考えれば, すべての$k\geq 1$で
\[
\left[\dfrac{b-1}{2^k}\right] = \dfrac{b-1}{2^k}    
\]
が成り立つ. これが成り立つのは$b-1=0$のときに限る. よって$b=1$より$n=2^{f}$を得る. $n\geq 2$なので$f\geq 1$である. 逆に$n=2^{f}$ ($f\geq 1$), $m=2$であるとき
\[V(n) = \sum_{k=1}^{f} [2^{f-k}] = \sum_{k=1}^{f} 2^{f-k} =  2^{n} - 1  = n-1\]
なので$\dfrac{(2^{f})!}{2^{n-1}}$は整数であることがわかる. 以上より, 求める$(m,n)$は
\[
\bolm{(m,n) = (2,2^{f})\qquad (f\geq 1)}
\]