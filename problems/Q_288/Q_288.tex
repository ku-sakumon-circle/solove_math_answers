\begin{thm}{287}{\hosi 7-大}{math\_akumon様 作}
$a>1$を実数定数とする. 連続関数$f:\mathbb{R}\to \mathbb{R}$が関数方程式$x = f(f(ax))$を満たし, かつ$f'(0)$が存在するとき, $f(x) = \pm \frac{x}{\sqrt{a}}$であることを示せ. 
\end{thm}

$ax = t$とおくことで, $f(f(t)) = a^{-1} t$となる. 続けて$f$に$2$回入れると
$$
f(f(f(f(t)))) = f(f(a^{-1}t)) = a^{-2}t
$$
となる. 帰納的に, 非負整数$n$に対し
$$
f^{2n}(t) = a^{-n} t
$$
が従う. さらに, 
$$
f^{2n}(t) = f(f^{2n-2}(f(t))) = f(a^{-(n-1)}f(t) ) = a^{-n}t
$$
であるから, $n\to \infty$とすることで
$$
\lim_{n\to \infty} f(a^{-(n-1)} f(t)) = f(0) = \lim_{n\to \infty} a^{-n}t = 0
$$
が従う. ただし, 最初の等号は$f$の連続性と$a>1$を用いた. 

よって $f'(0)$ は
\[
f'(0) = \lim_{h\to 0} \frac{f(h) - f(0)}{h} = \lim_{h\to 0} \frac{f(h)}{h}
\]
として与えられる. 

ここで, 関数方程式より $f$の全単射性($\mathbb{R}\to \mathbb{R}$を1対1に移す関数であること)が従うので, $x\neq 0$に対して$f(x) \neq 0$ であることに注意すると, 任意の$x\neq 0$にたいして
\[
\frac{x}{f(ax)} = \frac{f(f(ax))}{f(ax)}
\]
が成り立つ. この式から$x\to 0$とすれば, 連続性より$f(ax)\to f(0) = 0$であるので, 
\[
\lim_{x\to 0}\frac{x}{f(ax)} = \frac{1}{a}f'(0)^{-1} = \lim_{x\to 0} \frac{f(f(ax))}{f(ax)} = f'(0)
\]
が従い, $f'(0) = \pm a^{-1/2}$でなければならないことが分かる. 

ここで, 零でない実数$x_0$を任意に一つ取る. いま, 
\[
f'(0) = \lim_{n\to \infty} \frac{f(a^{-n}x_0)}{a^{-n}x_0}
\]
であるが, $f(a^{-n}x_0) = f^{2n+1}(x_0) = a^{-n}f(x_0)$であるので, 
\[
f'(0) = \lim_{n\to \infty} \frac{a^{-n}f(x_0)}{a^{-n}x_0} = \frac{f(x_0)}{x_0}
\]
となるので, すべての零でない実数$x_0$に対して$f(x_0) = f'(0)x_0$であることが示された. よって$x_0=0$の場合と合わせることで, $f(x) = \pm\frac{1}{\sqrt{a}}x$であることが必要となる. また, 明らかにこの関数は条件を満たしている. 

以上より題意は示された. 

