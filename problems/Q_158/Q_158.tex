\begin{thm}{158}{\hosi 2}{河合マーク\Rnum{1}A}
 $a\in\mathbb{N}$, $b\in\mathbb{Z}$, $A=\{3, 4, 7, 9, a^2-a\}$, $B=\{0, 8, a+b+1, 2a+b\}$ とする。
 \begin{enumerate}
  \item $A\cap B = \{3, 4\}$ のとき、$a$と$b$を求めよ。
  \item $A-B = \{2, 3, 7, 9\}$ のとき、$a$と$b$を求めよ。
 \end{enumerate}
\end{thm}

\syoumon{1}
$a$は自然数なので、$(2a+b)-(a+b+1)=a-1\ge 0$であるから、
\[ a+b+1=3 \quad\text{かつ}\quad 2a+b=4 \]
であればよい。これを解いて $(a, b)=(2, 0)$ を得る ($a$は自然数でかつ$b$は整数なのでよい)。このとき、
\[ A=\{3, 4, 7, 9, 2\} \,,\,\, B=\{0, 8, 3, 4\} \]
なので、実際に$A\cap B=\{3, 4\}$である。よって、$(a, b)=(2, 0)$。

\syoumon{2}
$2\in A\setminus B$より、$a^2-a=2$である。$a$は自然数なので、$a=2$である。これによって$B=\{0, 8, b+3, b+4\}$。$4\in A$かつ$4\notin A\setminus B$なので、$b+3$か$b+4$のいずれかが4である。よって$b=0$または$b=1$。

$b=0$とすると、$3\in A\cap B$となるので$3\in A\setminus B$に反する。$b=1$とすると
\[ A=\{3, 4, 7, 9, 2\} \,,\,\, B=\{0, 8, 4, 5\} \]
より$A\setminus B=\{3, 4, 7, 9\}$である。よって$(a, b)=(2, 1)$。