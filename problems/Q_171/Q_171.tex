\begin{thm}{171}{\hosi 12}{自作}
 (前略) \\
 4つの素数$a, b, n, x$があり、これら4つの中から上手く3つを選ぶと、それらはある順番で等差数列をなす。さらに、$\dfrac{a+b^n}{n}=x$ を満たす。このような組$(a, b, n, x)$を全て求めよ。 \\
 (後略)
\end{thm}

$a,b,n,x$のうち3つからなる等差数列を$S$と呼ぶことにし、階差は非負となるように並べるものとする。たとえば$a$が$S$に選ばれていないという状況を$a\notin S$で表すことにする。方程式
\[ a+b^n=nx \qquad\cdots(\text{*}) \]
について考える。

{\bf \large Step. 1} まず、
\[ n\le (1+1)^{n-1}=2^{n-1}\le b^{n-1} \]
より、$bn\le b^n$が従うので、$bn<a+b^n=nx$ より$b<x$ である。さらに$n^2\le 1+2^n$ が成り立つことに注意すると、
\[ n^2\le 1+2^n\le 1+b^n < a+b^n =nx \]
となるので、$n<x$ である。また、このことから$x\neq 2$が分かる。

{\bf \large Step. 2} $S$の階差が0であるとき、$S$は$(p,p,p)$ ($p$は素数) となる。$b,n$のいずれかは$S$に選ばれているので、Step. 1 より$x\notin S$が従う。よって$a=b=n=p$ とすると $p+p^p=px$ より$1+p^{p-1}=x$ である。$p>2$ であるとすると左辺は2より大きい偶数となって$x$は素数でないから不適。よって$p=2$となり$x=3$となる。よって$(a,b,n,x)=(2,2,2,3)$は解の一つである。

\vspace{1zw}

以降$S$の階差$d$は正であるとする。まず、$a,b,n,x$がすべて奇数であるとすると、(*)の左辺は偶数、右辺は奇数なので不適。よって$a,b,n$のいずれかが2である。このとき、$2\in S$であるとするとこれは$S$のうち最小の元であって、最大の元は$2+2d$と書ける。これが$a,b,n$のいずれかに等しいから$2+2d$は素数でなければならないが、$d>0$よりこれは2より大きい偶数となり素数でない。よって$2\notin S$である。同じ理由によって$a,b,n$の中に2は一つしか存在しない。

$2\notin S$とStep. 1 を踏まえると、$a,b,n,x$と$S$としてあり得る状況は次のいずれかである。
\begin{align*}
 a&=2 \,\text{で、}& S&=(b,n,x) & &\text{or} & &(n,b,x) & & & &\\
 b&=2 \,\text{で、}& S&=(a,n,x) & &\text{or} & &(n,a,x) & &\text{or} & &(n,x,a) \\
 n&=2 \,\text{で、}& S&=(a,b,x) & &\text{or} & &(b,a,x) & &\text{or} & &(b,x,a)
\end{align*}

{\bf \large Step. 3} $a=2$ の場合、考えるべき方程式(*)は
\[ 2+b^n=nx \]
である。$S=(s_1,s_2,s_3)$とするとき、等差数列だから$s_1+s_3=2s_2$が成り立つ。このとき$s_3<2s_2$が成り立っていることに注意する。

{\bf \large Step. 3.1} $S=(b,n,x)$ の場合、$3\le b < n$ より$n\ge 5$ でなければならず、$x<2n$ だから
\[ 2+2^n < 2+b^n=nx < 2n^2 \]
となるが、$n\ge 7$では$2+2^n>2n^2$となるので不適。よって$n=5$で、$b=3$でなければならない。このとき階差は2だから$x=7$である。しかしこれらは(*)の解にならない。

{\bf \large Step. 3.2} $S=(n,b,x)$ の場合、$n\ge 3$かつ$b\ge 5$であり、$x<2b$が成り立つ。よって
\[ b^n<2+b^n<nx<2bn \]
より、$b^{n^1}<2n$だから$5^{n-1}<2n$となる。これは$n\ge 3$で成り立たないので不適。

{\bf \large Step. 4} $b=2$ の場合、考えるべき方程式(*)は
\[ a+2^n=nx \]
である。$n$はこの場合奇素数だから、$n\equiv \pm 1\pmod{6}$ である。すると$n=6k\pm 1$と書ける。$n=6k+1$なら $2^n\equiv 64^k\times 2\equiv 2 \pmod{9}$ で、$n=6k-1$なら $2^n\equiv 5 \pmod{9}$ となることがわかる。

また、$S=(s_1,s_2,s_3)$の階差$d$は、$s_1>3$である限りは3の倍数でなければならない。なぜなら、$d\equiv\pm 1\pmod{3}$, $s_1\equiv\pm 1\pmod{3}$であり、いずれの場合においても$s_1+d$, $s_1+2d$のいずれかが$\pmod{3}$ で0となるが、3より大きい素数$s_1+d$, $s_1+2d$に不適であるから。またこのとき、$s_1\equiv s_2\equiv s_3 \pmod{3}$ を満たすことに注意する。

{\bf \large Step. 4.1} $S=(a,n,x)$ の場合、$3\le a$, $x<2n$ だから、
\[ 3+2^n\le a+2^n=nx<2n^2 \]
より、$3+2^n<2n^2$ である。これは$n\ge 7$で成り立たない。一方$3\le a$から$5\le n$なので、$n=5$に決まる。よって$a=3$, $x=7$で、
\[ 3+2^5=35=5\times 7 \]
は(*)を満たす。よって、$(a,b,n,x)=(3,2,5,7)$は解の一つである。

{\bf \large Step. 4.2} $S=(n,a,x)$ の場合を考える。$n=3$とすると$a=\dfrac{x+3}{2}$ から、
\[ a+2^3=3x \quad\dou\quad 19+x=6x \]
となるが、このような$x$は存在しない。よって$n>3$であり、階差$d$は3の倍数であり、$n=6k\pm 1$となる。

$n=6k+1$のとき、$2^n\equiv 2\pmod{9}$, $a=n+d\equiv 1\pmod{3}$。さらに$nx=(a-d)(a+d)$を用いて(*)の$\pmod{9}$ を考えると、
\[ a+2\equiv a^2-d^2\equiv a^2 \pmod{9} \]
より、$a^2-a-2\equiv 0 \pmod{9}$ である。
\[ a^2-a-2\equiv (a+4)^2 \equiv 0\pmod{9} \]
なので、この解は$a+4$が3の倍数のとき、すなわち$a\equiv 2 \pmod{3}$に限るが、これは$a\equiv 1\pmod{3}$に反する。

$n=6k-1$のとき、$2^n\equiv 5\pmod{9}$, $a=n+d\equiv 2\pmod{3}$。同様に考えて、
\[ a+5\equiv a^2 \pmod{9} \]
となる。$a\equiv 2\pmod{3}$より、$a\equiv 2,5,8 \pmod{9}$ であるが、この中に上式を満足するものはない。

{\bf \large Step. 4.3} $S=(n,x,a)$ の場合、$a=2x-n$ より、
\[ 2x-n+2^n=nx \]
となる。$n=3$とすると、$2x+5=3x$ より$x=5$で、階差は2だから$a=7$となる。これらは全て素数で、
\[ 7+2^3=15=3\times 5 \]
となり(*)は確かに満たされる。よって$(a,b,n,x)=(7,2,3,5)$は解の一つとなる。

$n>3$とする。階差$d$は3の倍数だから$n\equiv x\equiv a \pmod{3}$でなければならない。

$n=6k+1$ のとき\footnote{実はこの場合は$\pmod{3}$でもわかる。一方$n=6k-1$では$\pmod{9}$を見ないとわからない。}、
\begin{align*}
 2x-n+2&\equiv nx \pmod{9} \\
 (n-2)(x+1)&\equiv 0\pmod{9}
\end{align*}
である。明らかに$n-2$は3の倍数ではないから、$x+1$は3で割り切れる。しかし$x\equiv n\equiv 1\pmod{3}$であったことに矛盾する。

$n=6k-1$ のとき、
\begin{align*}
 2x^n+5&\equiv nx\pmod{9} \\
 (n-2)(x+1)&\equiv 3 \pmod{9}
\end{align*}
である。この場合では$n\equiv x\equiv 2 \pmod{3}$だから、$n-2$, $x+1$はともに3の倍数となり、上式の左辺は9の倍数となるから矛盾となる。

{\bf \large Step. 5} $n=2$の場合、考えるべき方程式は次のようになる。
\[ a+b^2=2x \]

{\bf \large Step. 5.1} $S=(a,b,x)$ の場合、$b\ge 5$である。また$x=2b-a$であるから、
\[ a+b^2=4b-2a \quad\dou\quad 3a=b(4-b) < 0 \]
より、解なし。

{\bf \large Step. 5.2} $S=(b,a,x)$ の場合、$x=2a-b$から、
\[ a+b^2=4a-2b \quad\dou\quad 3a=b(b+2) \]
である。$a>b$がともに素数であることから、$b=3$, $a=b+2$ のみが適する。よって$a=5$, 階差が2なので$x=7$となる。これは
\[ 5+3^2=14=2\times 7 \]
となって(*)を満たす。よって$(a,b,n,x)=(5,3,2,7)$は解の一つである。

{\bf \large Step. 5.3} $S=(b,x,a)$ の場合、$2x=a+b$ だから
\[ a+b^2=a+b \quad\dou\quad b^2=b \]
だがこれは解にならない。

\vspace{1zw}

以上のことから、求める$(a,b,n,x)$は、
\[ (a,b,n,x)=(2,2,2,3) \,,\,\, (3,2,5,7) \,,\,\, (7,2,3,5) \,,\,\, (5,3,2,7) \]