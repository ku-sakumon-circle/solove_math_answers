\begin{thm}{210}{\hosi 7\maru}{京大理系 (2017)}
 $w$ を0でない複素数, $x,y$ を $w+\dfrac{1}{w}=x+yi$ を満たす実数とする.\footnote{これはジェーコフスキー変換という名がついており,航空学などに応用がされている。}
 \begin{enumerate}
  \item 実数 $R$ は $R>1$ を満たす定数とする. $w$ が絶対値 $R$ の複素数全体を動くとき, $xy$ 平面上の点 $(x,y)$ の軌跡を求めよ。
  \item 実数 $\alpha$ は $0<\alpha<\dfrac{\pi}{2}$ を満たす定数とする. $w$ が偏角 $\alpha$ の複素数全体を動くとき, $xy$ 平面上の点 $(x,y)$ の軌跡を求めよ。
 \end{enumerate}
\end{thm}

\syoumon{1}
ここに解答を記述。

\syoumon{2}
ここに解答を記述。