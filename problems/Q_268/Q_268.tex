\begin{thm}{268}{\hosi ?}{京都大院試H2専門}
$\mb{Q}$係数多項式$f(X)$が与えられたとする. 任意の代数的整数(すなわち, 最高次係数が1の$\mb{Z}$係数多項式の根) $a$に対して$f(a)$がまた代数的整数になるならば, $f(X)$は$\mb{Z}$係数となることを示せ. \textcolor{red}{($f(X)$は定数でないとする. )}
\end{thm}
$f(X)$が$\mb{Z}$係数でないと仮定する. このとき, 2以上の整数$n$, 整数係数多項式$g(x)$を上手く選んで, $f(x) = \dfrac{1}{n} g(x)$と表せる. ここで$n$は最小であるように取る. 仮定より$n\geq 2$であり, $n$には素因数$p$がある. 具体的に$g(X) = \sum_{k=0}^{d} g_k X^{k}$と表しておく. ($d = \deg{g} \geq 1$, $g_k \in \mb{Z}$)
%$g(x)$は$\mb{Z}$係数である. $n$を割り切る素数$p$が存在するので, それを一つ固定する. $n=pm$と表すとき, $m$は整数なので$mf(x) = \dfrac{1}{p}g(x)$もまた問題と同じ条件を満たすことに注意せよ. $n$は最小に取っているので, $g(x)$のすべての係数の最大公約数は$n$と互いに素である. よって$\dfrac{1}{p}g(x)$は整数係数多項式ではないが, 問題と同じ条件を満たす. よって$n=p$の場合に帰着できる. 
%さらに$g(x)$を係数が$p$で割れるものとそうでないものに分離して$g(x) = pQ(x) + r(x)$と表せば, $\dfrac{r(x)}{p}$もまた条件を満たす. $\dfrac{1}{p}r(x)$という多項式(ただし$r(x)$の)が条件を満たさないことを見ればよい.
 $\deg{g} < p^{k}$であるような整数$k$を取るとき, ある$\alpha\in \mb{F}_{p^{k}}$が存在して$g(\alpha)\neq 0$である. $\alpha$の$\mb{F}_{p}$上の最小多項式を考え, それを整数係数かつモニックにリフトした多項式$A(T)$を考える. $A(T)$は$\mb{Z}$上既約である(もし可約なら$\mb{F}_{p}$でも可約になるから). 環$R=\mb{Z}[T]/(A(T))$を取る. $A(T)$はモニックだから, その任意の根$a$は代数的整数である.  $a\in \mb{C}$と考えて, 代数的整数環$\ovl{\mb{Z}}$の部分環$\mb{Z}[a]$と$R$は同型で,  構成(および有限体$\mb{F}_{p^{k}}$の一意性) より $R/pR \cong \mb{F}_{p}[T]/(A(T)) \cong \mb{F}_{p^{k}}$ である. このように同型写像$\phi:\mb{Z}[a]/p\mb{Z}[a] \to \mb{F}_{p^{k}}$を得るが, これによって$\phi(s\bmod{p}) = \alpha$であるような$s\in \mb{Z}[a]$を取る.   \par 
さて, 問題の条件から$f(s) = \dfrac{1}{n} g(s) \in \mb{Z}[a]$だから$g(s)\in n\mb{Z}[a]$である. とくに$g(s) \in p\mb{Z}[a]$である. すなわち
\[
\sum_{k=0}^{d} g_k (s\bmod{p})^{k} = 0 \quad (\text{in}  \mb{Z}[a]/p\mb{Z}[a])
\]
この式を$\phi$で送ると, $\phi(s\bmod{p}) = \alpha$により
\[
\sum_{k=0}^{d} g_k \alpha^{k} = 0 \quad (\text{in} \mb{F}_{p^{k}})
\]
つまり$g(\alpha) = 0$なので$\alpha$の取り方に反する. よって$n=1$であり$f(X)$は整数係数であることが示された.  