\begin{thm}{168}{\hosi ?}{自作}
 $d(k)$で$k$の正約数の個数、$\phi(k)$でオイラーの$\phi$関数、$\sigma(k)$で$k$の正約数の総和、$\pi(k)$で$k$以下の素数の約数とする。 \\
 $n$が2以上の整数のとき、
 \[ \sum_{k=2}^n \left\lfloor \frac{d(k)+\phi(k)}{\sigma(k)} \right\rfloor = \pi(n) \]
 が成立することを示せ。
\end{thm}

まず、次の補題を示す。
\begin{subthm}{168.1}
 $c(n) = \dfrac{d(n) + \phi(n)}{\sigma(n)}$とおく。このとき, 次が成り立つ。
 \begin{enumerate}
  \item $n$が素数ならば、$c(n)=1$
  \item $n\geq 2$が素数でないならば、$0<c(n)<1$
 \end{enumerate}
\end{subthm}

(証明 1)~$n$を素数とする。このとき,$n$の正の約数は$1,n$の{\bf $2$個} で, その和は$n+1$であり, $n$以下の$n$と互いに素な自然数の個数は$1$から$n-1$までの全ての整数であり, {\bf $n-1$個}である。したがって, $d(n) = 2, \sigma(n)=n+1, \phi(n) = n-1$なので, 
\[c(n) = \dfrac{2+ (n-1)}{n+1} = 1\]
よりよい。\qed\\

(証明 2)~$n\geq 2$を素数でないとする。$0<c(n)$であることは明らか。$n$は$1$と$n$以外にも約数を持ち, $1\neq n$であるから, $n+1< \sigma(n) $が分かる。次の2つの集合
\begin{align*}
 D &= \{ k\in \mb{Z}\cap [1,n]\mid \text{$k$は$n$の約数である} \} \\
 \Phi &= \{ k\in \mb{Z}\cap [1,n] \mid \text{$k$は$n$と互いに素である} \}
\end{align*}
を定めると、$d(n),\phi(n)$は$D$, $\Phi$の元の個数である。さて, $1$は$n$の約数であり, $n$と互いに素であるから, $1\in D, 1\in \Phi$であることが分かる。続いて, $2\le k\le n$なる自然数について, $k\in D$を満たすとすると, $k$と$n$の最大公約数は$k\neq 1$であるからこの$k$は$\Phi$に属さない。以上から, $D\cap \Phi = \{ 1 \}$であると分かる。有限集合$X$に対してその元の個数を$|X|$で書くことにする。定義より明らかに$D\cup \Phi \subset \{ 1,2,\dots, n\}$であるから, 
\begin{align*}
 d(n)+\phi(n) &= |D|+|\Phi| \\
 &= |D\cup\Phi|+|D\cap\Phi| = |D\cup\Phi|+|\{1\}| \\
 &\le |\{1, 2, \dots, n\}|+|\{1\}|=n+1<\sigma{n}
\end{align*}
となる。ゆえに, $c(n) <1$が得られた。\qed\\
\\
補題より, $k\geq 2$に対して
\begin{align*}
 \bigl\lfloor c(k) \bigr\rfloor=\left\{
 \begin{aligned}
  1 &\quad \text{($k$が素数)} \\
  0 &\quad \text{($k$が素数でない)}
 \end{aligned}
 \right.
\end{align*}
となるので, $\disp\sum_{k=2}^{n} \lfloor c(k) \rfloor $は$2$から$n$までの素数の個数を計上する。したがって$\pi(n)$に一致する。\qed



