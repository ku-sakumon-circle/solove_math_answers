\begin{thm}{010}{\hosi 3}{新潟大 (2016)}
 $P(x)=x^4+x^3+x-1$ とする。3次以下の整式$Q(x)$であって、$P(1)=Q(1)$, $P(-1)=Q(-1)$, $P(2)=Q(2)$, $P(-2)=Q(-2)$ を全て満たすようなものを求めよ。
\end{thm}

$R(x)=P(x)-Q(x)$とおこう。条件より、
\[ R(1)=R(-1)=R(2)=R(-2)=0 \]
を満たす。因数定理より$R(x)$は$(x-1)(x+1)(x-2)(x+2)$で割り切れる。しかも、$P(x)$が4次、$Q(x)$が3次以下だから、$R(x)$は4次である。さらに$P(x)$の$x^4$の係数が1だから$R(x)$の$x^4$の係数も1であって、
\[ R(x)=(x-1)(x+1)(x-2)(x+2)=x^4-5x^2+4 \]
である。よって、
\begin{align*}
 Q(x)&=P(x)-R(x) \\
&=(x^4+x^3+x-1)-(x^4-5x^2+4) \\
&= x^3+5x^2+x^5
\end{align*}