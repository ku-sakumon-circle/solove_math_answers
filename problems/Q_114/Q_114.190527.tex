\begin{thm}{114}{\hosi 4}{広島大}
 2次関数$f(x)=x^2+ax+b$は、
 \[ \int_0^1\! xf(x) \,dx = \int_0^1\! x^2f(x) \,dx \]
 を満たす。$f(x)=0$は相異なる実数解を持つことを示せ。また、その実数解のうち少なくとも一つは0と1の間にあることを示せ。
\end{thm}

与式を変形して、$\disp \int_0^1\!(x^2-x)f(x) \,dx$を得る。ここで$0<x<1$において$x^2-x<0$である。$f(x)=0$が$0<x<1$に解を持たないとすると、$0<x<1$において、常に$f(x)>0$であるか常に$f(x)<0$であるかのいずれかである。このとき、$(x^2-x)f(x)$は常に負であるか正であるかのいずれかとなるが、積分した値は0になり得ず不適。したがって、$f(x)=0$は$0<x<1$の範囲に解を持つ。

この解が重解となると、$f(x)$の2次の係数が正であることから、$0<x<1$の範囲で$f(x)\ge 0$である。すると$(x^2-x)f(x)\le 0$となるからやはり積分値は0にならず不適。よって$f(x)=0$は異なる2つの実数解を持つ。

以上より、$f(x)=0$は異なる2つの実数解をもち、少なくとも1つは0と1の間にあることが示された\footnote{編者註: はてなブログを参考にしつつ、重解についての議論を追加。}。