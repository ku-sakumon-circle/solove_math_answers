\begin{thm}{245}{}{}
 次の命題を証明または反証せよ。
 \begin{enumerate}
  \item $x,y,z\in\mathbb{R}$ が$x+y+z,xy+yz+zx,xyz\in\mathbb{Q}$ ならば$x,y,z\in\mathbb{Q}$
  \item $1+\sqrt{-1}$ は16の8乗根である
  \item 単調増加かつ微分可能な $f:\mathbb{R}\rightarrow\mathbb{R}$ がある定数$M$によって $f(x)<M$ となるならば、$\disp \lim_{x\to\infty} f'(x)=0$
  \item $x>0$が無理数ならば$\sqrt{x}$は無理数である。
 \end{enumerate}
\end{thm}

\syoumon{1}
\textbf{偽である。} $x=0, y=\sqrt{2}, z=-\sqrt{2}$とすればよい。実際
\[x+y+z = 0,  xy+yz+zx = -2, xyz = 0.\]
別のアプローチも述べておこう。$x+y+z\in \mb{Q},\cdots$などという条件から, 解と係数の関係をふまえれば
\[(T-x)(T-y)(T-z)\]
という$T$の多項式は実解を3つ持つ有理数多項式にになるわけである。だから, そのような多項式の根は果たして有理数なのか?という視点で解く事もできる。センスがあるのかないのかわからない反例だが,  
\[T(T-\cos{\dfrac{2\pi}{3}})(T-\cos{\dfrac{4\pi}{3}}) = T^3 + \dfrac{3}{4}T +\dfrac{1}{4}\]
などもよいだろう。これは
\[\cos{3\theta} = -4\cos^{3}{\theta} + 3\cos{\theta}\]
の$\cos{3\theta} = 1$となる場合, すなわち$\theta =0, \dfrac{2\pi}{3}, \dfrac{4\pi}{3}$から現れる三次方程式である。 

\syoumon{2}
\textbf{真である。} $n$乗根は何も実数に限る話ではない。単に$n$乗して$x$になったら$x$の$n$乗根というのである。実際に8乗して$16$になることは容易である:
\[(1+\sqrt{-1})^8 = (2\sqrt{-1})^{4} = (-4)^2  = 16\]

\syoumon{3}
\textbf{偽である。} 本問の中ではこれが最もトリッキーである。数学界の反例探しというのを甘く見てはいけない。ただ, 正直なところ具体的な``$f(x)$の数式''を与えるのは面倒(かつ意義はない)だしその式だけ見ても分かりづらいので, どういうグラフであるかの説明をするだけで想像をしてほしい。

まず$-\pi/2 \leq x\leq 0$では$f(x) = \cos{x} - 1$とする。$x=-\frac{\pi}{2}$での傾きが$1$だから, そこから微分可能になるように別の単調増加グラフをつなげればいい。たとえば$y=e^{x}$の$x\leq 0$の部分を, $(0,1)\to (-\frac{\pi}{2}, -1)$という平行移動で繋げれば良い。これで$f(x)$の$x\leq 0$の部分は完成した。

次に$0\leq x\leq \pi$では$\sin{x}$の$-\pi/2\leq x\leq \pi/2$のグラフ(これは単調増加)を平行移動して$f(0)=0$のところで繋げる(この繋げ方は微分可能である)。次に$\pi \leq x\leq 2\pi$においては, $\dfrac{1}{2^{2}}\sin{x}$の$-\pi/2\leq x\leq \pi/2$の部分を繋げればよい。以下このように繋げていく。すなわち, この次には$y=\dfrac{1}{3^{2}}\sin{3^2x}$ ($-\pi/18\leq x\leq\pi/18$)を繋げてから$y=\dfrac{1}{4^{4}}\sin{x}$ ($-\pi/2\leq \pi/2$)を繋げていき, 以下$n=3,4,5,6,\cdots$に対しても
\[y=\dfrac{1}{(2n-1)^2}\sin{(2n-1)^2x}\]
の$(-\pi/\{2(2n-1)^2\}\leq x\leq \pi/\{2(2n-1)^2\}$の部分を繋げてから
\[y=\dfrac{1}{(2n)^2}\sin{x}\]
の$-\pi/2\leq\pi/2$の部分を繋げる。

よくみると$-\pi/2\leq x\leq \pi/2$の部分にあるサインカーブを(縦に縮めて)何度も繋げているから, 横方向にいくらでも$f(x)$が伸びていくことが分かる。これで$f(x)$が帰納的な構成を通して実数全体で定義されたことになる。繋げていった物としてはサインカーブの増加部分しか使っていないわけだから, (狭義)単調増加性は明らか。微分可能性もつながりの部分で傾きが$0$になっていることから分かるであろう。

さて, この$f(x)$はいわば階段状であるが, 天にまで登っていくわけではない。なぜなら, 用いたサインカーブは$\dfrac{1}{k^2}\sin{Ax}$という形であって, 一つ上昇するときに上昇値は$\dfrac{2}{k^2}$しかないので, $\sum_{k=1}^{\infty}\dfrac{2}{k^2}<\infty$というよく知られた話によって$f(x)<M$が常に満たされるように定数$M$を取ることが出来る。しかしながら$\lim_{x\to \infty} f'(x) = 0$は満たされない。なぜなら, 繋げたサインカーブのうち$\dfrac{1}{(2k-1)^2}\sin{(2k-1)^2x}$というものを微分すると$\cos{(2k-1)^2x}$であって, この導関数の値が1になるような点をこのサインカーブは含んでしまっていて, $f'(x)$が無限回$1$という値を取ることがわかるからである。

よってこのような$f(x)$を与えると良い。

\syoumon{4}
\textbf{真である。}有名問題ではあるが, 対偶,あるいは背理法を使わないと難しいというのはなぜか不思議な感じがする。さて, $\sqrt{x}$が有理数だと仮定して$x$が有理数であることを示せばよいが, 有理数の積は有理数なのであたりまえである。