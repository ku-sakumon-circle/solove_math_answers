\begin{thm}{141}{\hosi 3}{神戸大理 前期 (2019)}
 $n$を2以上の整数とする。2個のさいころを同時に投げるとき、出た目の数の積を$n$で割った余りが1となるような確率を$P(n)$とする。以下の問に答えよ。
 \begin{enumerate}
  \item $P(2)$, $P(3)$, $P(4)$ を求めよ。
  \item $n\ge 36$ のとき、$P(n)$を求めよ。
  \item $P(n)=\dfrac{1}{18}$となる$n$を全て求めよ。
 \end{enumerate}
\end{thm}

2つの出目に対して、その積から1を引いた表を作ると、次のようになる。
\begin{align*}
 \begin{array}{c||c|c|c|c|c|c|}
    & 1 & 2 & 3 & 4 & 5 & 6 \\ \hline\hline
  1 & 0 & 1 & 2 & 3 & 4 & 5 \\ \hline
  2 & 1 & 3 & 5 & 7 & 9 & 11\\ \hline
  3 & 2 & 5 & 8 & 11& 14& 17\\ \hline
  4 & 3 & 7 & 11& 15& 19& 23\\ \hline
  5 & 4 & 9 & 14& 19& 24& 29\\ \hline
  6 & 5 & 11& 17& 23& 29& 35\\ \hline
 \end{array}
\end{align*}
$P(n)$は、この表の36個の値のうち、$n$で割り切れるものの個数を$N(n)$として、$\dfrac{N(n)}{36}$で求められる。

\syoumon{1}
$N(2)=9$より、$P(2)=\dfrac{1}{4}$ \\
$N(3)=8$より、$P(3)=\dfrac{2}{9}$ \\
$N(4)=5$より、$P(4)=\dfrac{5}{36}$

\syoumon{2}
$n\ge 36$においては、この表に現れる数字は35以下であるから、$n$で割り切れるものは0のみなので$N(n)=1$。よって$P(n)=\dfrac{1}{36}$

\syoumon{3}
$N(n)=2$となる$n$を求めればよい。1つは0なので、これ以外に1つだけ$n$の倍数が存在する。ここで表は対角線に沿って対称なので、対角線上にあるものから見ればよい。3, 8, 15, 24, 35の約数のうち、$N(n)=2$となるものを探せば、
\[ n=6, 12, 15, 24, 35 \]
である。