\begin{thm}{096}{\hosi 7}{京大実戦}
 座標平面上の2つの曲線(または直線) $C_1$: $y=ax^2+bx$、$C_2$: $y=e^{-x}$が第一象限の点Pにおいて共通の接線を持つという。$C_1$と$x$軸で囲まれた領域の面積が最大となるときの実数$a, b$を求めよ。
\end{thm}

まず、$a=0$のときは、$C_1$は直線$y=bx$となる。$b\le 0$のとき、明らかに$C_1$と$C_2$は第一象限に共有点を持たないので不適。$b>0$のとき、$C_1$において$y'=b>0$だが、$C_2$において$y'=-e^{-x}<0$となって、明らかに共通の接線を持てないのでやはり不適。

$a\neq 0$のとき、$C_1$は点$(0,0)$, $\left(-\dfrac{b}{a}, 0\right)$を通る放物線である。$a>0$のとき、これは下に凸であり、グラフを考えれば、第一象限内では常に単調増加である。よって同様の議論から、これは$C_2$と共通の接線を持ち得ない。したがって、$a<0$のみが適する。

以降$a<0$で考える。$b\le 0$のとき、$C_1$は第一象限を通らないので不適。よって$b>0$。これによって、$C_1$と$x$軸が囲む領域の面積は、
\[ \int_0^{-\frac{b}{a}}\! (ax^2+bx) \,dx = \int_0^{-\frac{b}{a}}\! ax\left(x+\frac{b}{a}\right) \,dx = -\frac{a}{6}\left(-\frac{b}{a}-0\right)^3=\frac{b^3}{6a^2} \]
これの最大値を考える。

点Pの$x$座標を$p$~(ただし$p>0$) とおく。この点において、$C_1$, $C_2$が共通の接線を持つことは、
\[ ap^2+bp=e^{-p} \quad\text{かつ}\quad 2ap+b=-e^{-p} \]
である。第2式の両辺に$p$を乗じ、$bp$, $ap^2$をそれぞれ消去すると、
\[ a=-\frac{1+p}{p^2}e^{-p} \,,\,\, b=\frac{2+p}{p}e^{-p} \quad\cdots\text{(*)} \]
を得る。これを用いて、
\[ \frac{b^3}{6a^2}=\frac{p(2+p)^3}{6(1+p)^2}e^{-p} \]
となった。右辺を$p$の関数とみて$f(p)$とおく。ここで任意の正の実数$p$について、(*)によって得る$a, b$が$C_2$と共通の接線を持つ$C_1$を作るから、$f(p)$の定義域は正の実数全体としてよい。$f(p)$を微分して、
\begin{align*}
 f'(p)&=\left\{\left[\frac{p(2+p)^3}{6(1+p)^2}\right]'-\frac{p(2+p)^3}{6(1+p)^2}\right\}e^{-p} \\
 &=-\frac{(p-1)(p+2)^2(p^2+2p+2)}{6(p+1)^3}e^{-p}
\end{align*}
より、$0<p<1$で$f'(p)>0$、$f'(1)=0$、$p>1$で$f'(p)<0$だから、$f(p)$は$p=1$で最大となる。

したがって求める$a, b$の値は、(*)に$p=1$を代入して、
\[ a=-\frac{2}{e} \,,\,\, b=\frac{3}{e} \]