\begin{thm}{049}{\hosi 7\maru}{京都大 理系 (2012)}
 さいころを$n$回投げて出た目を順に $X_1, X_2, \ldots, X_n$ とする。さらに、
 \begin{align*}
  Y_1=X_1,\quad Y_k=X_k+\dfrac{1}{Y_{k-1}} \quad (k=2,3,\ldots, n)
 \end{align*}
 によって $Y_1, Y_2, \ldots, Y_n$を定める。$\dfrac{1+\sqrt{3}}{2}\le Y_n \le 1+\sqrt{3}$となる確率$p_n$を求めよ。
\end{thm}
(解答: 2021/11/25)
$n\geq 1$とする.  $Y_{n+1} = X_{n+1} + \dfrac{1}{Y_n}$がその不等式を満たすために「$X_{n+1}$がどう出るべきか」 ということについて考えたい. それを考えるためには$Y_n$の大きさに関する場合分けが必要である. \par 
まず$Y_n$は帰納的に有理数であるため, 不等式の等号は実現することがない. そこで, $\dfrac{1+\sqrt{3}}{2} < Y_n < 1+\sqrt{3}$を満たす確率を$p_n$, 事象を$P_n$ ($n\geq 1$) とおく. $n+1$回目でこの不等式が成り立っていたとしよう. すると, $\dfrac{1}{\sqrt{3} \pm 1} = \dfrac{\sqrt{3}\mp 1}{2}$により
\begin{align*}
&\dfrac{1+\sqrt{3}}{2} < Y_{n+1} < 1 + \sqrt{3}\\
\iff& \dfrac{\sqrt{3}-1}{2} < \dfrac{1}{Y_{n+1}} < \sqrt{3} - 1 \\
\iff& \dfrac{\sqrt{3}-1}{2} - \dfrac{1}{Y_n} < X_{n+1} < \sqrt{3} - 1 - \dfrac{1}{Y_n}
\end{align*}
よって, 最後の不等号の両側の数がどうであるかによって$X_n$がどうあるべきかが以下のように分かる: 
\begin{enumerate}[label=(\roman*)]
\item $2\leq \sqrt{3} + 1 - \dfrac{1}{Y_n}$かつ$\dfrac{1+\sqrt{3}}{2} - \dfrac{1}{Y_n} > 1$; すなわち$Y_n > \sqrt{3}+1$であるとき, 事象$P_{n+1}$が起こるのは $X_n = 2$ のときである. 
\item $2\leq \sqrt{3} + 1 - \dfrac{1}{Y_n}$かつ$\dfrac{1+\sqrt{3}}{2} - \dfrac{1}{Y_n} \leq 1$; すなわち$Y_n > \sqrt{3}+1$であるとき, 事象$P_{n+1}$が起こるのは $X_{n+1} = 1,2$ のときである. 
\item $2> \sqrt{3} + 1 - \dfrac{1}{Y_n}$かつ$\dfrac{1+\sqrt{3}}{2} - \dfrac{1}{Y_n} < 1$; すなわち$Y_n < \dfrac{\sqrt{3}+1}{2}$であるとき, 事象$P_{n+1}$が起こるのは $X_{n+1} = 1$ のときである. 
\end{enumerate}
そこで, $Y_n > \sqrt{3}+1$を満たす確率を$q_n$, 事象を$Q_n$とおき, $Y_n < \dfrac{\sqrt{3}+1}{2}$を満たす確率を$r_n$, 事象を$R_n$とおく. このとき$P_n \cup Q_n \cup R_n$は全事象であるから$p_n + q_n + r_n = 1$. そして, 上の考察により$P_{n+1}$であるのは「$P_n$と$X_{n+1} = 1,2$が起こるとき」と「$Q_{n}$と$X_{n+1}=2$が起こるとき」と「$R_n$と$X_{n+1} = 1$が起こるとき」である. これらは排反であるから, 
\[p_{n+1} = \dfrac{2}{6}p_n + \dfrac{1}{6}q_n + \dfrac{1}{6}r_n = \dfrac{2}{6}p_n + \dfrac{1}{6
}(1-p_n) = \dfrac{1}{6}p_n+ \dfrac{1}{6}\]
となる. $Y_1 = X_1$が不等式を満たすのは$X_1 = 2$のときであるから $p_1 = \dfrac{1}{6}$. よって漸化式を解く事で$p_{n+1}$ ($n\geq 1$)が求まる: 
\begin{align*}
p_{n+1} - \dfrac{1}{5} &= \dfrac{1}{6} \left( p_n - \dfrac{1}{5}\right) \\
&= \dfrac{1}{6^n} \left(p_1 - \dfrac{1}{5} \right) \\
&= -\dfrac{1}{5\cdot 6^{n+1}} \\
\therefore \quad p_{n+1} &= \dfrac{1}{5} \left( 1 - \dfrac{1}{6^{n+1} } \right) \quad (n\geq 1)
\end{align*}
これは$n=0$でも正しいので
\[{\boldmath p_n = \dfrac{1}{5} \left( 1 - \dfrac{1}{6^n} \right) }\]

