\begin{thm}{250}{\hosi 10}{東北大数学科院試 H28選択 改}
 $p$を奇素数とする。
 \begin{enumerate}
  \item 正の整数$d$に対して次を示せ。
  \begin{align*}
   \sum_{k=0}^{p-1} k^d \equiv \left\{\,
   \begin{aligned}
    -1 & \quad(p-1\text{が}d\text{を割り切るとき}) \\
    0 & \quad(\text{otherwise})
   \end{aligned}
   \right.\!\!\! \pmod{p}
  \end{align*}
  \item $p\equiv 1\pmod{4}$のときは$B_p=\combi{\frac{p-1}{2}}{\frac{p-1}{4}}$とし、$p\equiv 3\pmod{4}$のときは$B_p=0$とする。$p$未満の非負整数の組$x,y$であって$y^2\equiv x(x^2+1) \pmod{p}$を満たすようなものの個数を$N_p$とするとき、$N_p\equiv -B_p \pmod{p}$ であることを証明せよ。
 \end{enumerate}
\end{thm}

ここに解答を記述。