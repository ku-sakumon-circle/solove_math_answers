\begin{thm}{121}{\hosi 7}{東北大AO (2017)}
 $x \geq 0$における関数
 \[ \int_0^x\! \frac{(1+x^2+t)^2}{(1+x^2-t)^3} \,dt \]
 の最大値を求めよ。
\end{thm}

$s:=1+x^2-t$と置換し, $1+x^2=f$とおいて
\[\disp\int_{1+x^2}^{1+x^2-x} \dfrac{(2+2x^2-s)^2}{s^3} \,(-1)ds=\disp\int_{f-x}^{f} \dfrac{(2f-s)^2}{s^3} \,ds \]
展開して積分すると
\[=\disp\int_{f-x}^{f}\left(\dfrac{1}{s} -\dfrac{4f}{s^2}+\dfrac{4f^2}{s^3}\right)ds=\log{\dfrac{f}{f-x}}-4\dfrac{f}{f-x} +2\dfrac{f^2}{(f-x)^2}+2\]
これは$\dfrac{x}{f}=g$とおくと
\[\log{\dfrac{1}{1-g}}-4\dfrac{1}{1-g}+2\dfrac{1}{(1-g)^2}+2\]
となり,さらに$\dfrac{1}{1-g}=h$とおくと,
\[\log{h}-4h+2h^2+2\]
である。\\
$\disp\lim_{x\to\infty}g=0, g>0, x\leq 0$において$1+x^2\geq 2x$から $0<g\leq\dfrac{1}{2}$の範囲を動くので, $h$は$1<h\leq 2$の範囲を動く。$h$で$\log{h}-4h+2h^2+2$を微分すると
\[\dfrac{1}{h}+4(h-1)\]
となり, $h$の範囲からこれは正値をとる。よって, $h=2$つまり$x=1$で最大値を取り,その値は
\[2+\log{2}\]
である。
