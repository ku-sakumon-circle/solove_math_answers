\begin{thm}{279}{\hosi 8}{2022新歓ビラ問題}
$x,y\in \mb{C}$とする. 正の整数$n$に対し
\[
    m_n = x^n - y^n,\quad p_n = x^n + y^n 
\]
とおく. 
\begin{enumerate}
\item $x\neq y$とする. すべての$n$で$m_n$が整数であるならば, $x,y$はともに制すであることを証明せよ. 
\item すべての$n$で$p_n$が整数であるような組$(x,y)$はどのように表せるか. 
\end{enumerate}

\end{thm}
\syoumon{1} 条件より$x-y$が0でない整数なので, $x+y = m_2/m_1$は有理数である. よって$x,y$は有理数である. $x = \dfrac{a}{c}$と既約分数で表すとき(ただし$c>0$), $y = \dfrac{a-m_1c}{c}$と書けるが, これも既約分数であることに注意する. $a-m_1c = b$とおくと$x=a/c, y=b/c$でどちらも既約分数である. 題意を示すためには, $c=1$を示せばよい. 背理法で$c>1$であるとする. このとき, $c$の素因数$p$が少なくとも一つ存在する. 条件より
$m_n = \dfrac{a^n - b^n}{c^n}$は整数なので, 特に$a^n - b^n$は$p^n$の倍数である. \par 
$v_n$を$a^n - b^n$が$p$で割り切れる回数として定義する. いま, $v_n \geq n$が成り立つことが分かっている. とくに$n=p^e$の場合に見てみよう. 因数分解
\[X^p - Y^p = (X-Y)\sum_{k=0}^{p-1} X^{k} Y^{p-1-k}\]
を考えると, 
\[v_{p^{e+1}} = v_{p^e} + \parena{\sum_{k=0}^{p-1} (a^{p^e})^{k} (b^{p^{e}})^{p-1-k}がpで割り切れる回数}\]
が分かる. ここで, 次を示す\footnote{以下の発想はLTEの補題から来るものである. 知っている人はLTEの補題を適用すればもうそろそろオチが分かるであろう. }.  . 
\begin{claim}
$X,Y$が$p$で割れない整数で$X-Y$が$p^2$の倍数であるとき, $\sum_{k=0}^{p-1} X^{k}Y^{p-1-k}$が$p$でちょうど1回割り切れる. 
\end{claim}
\prf 
条件より$Y = X + p^2m$とおける. $\bmod{p^2}$で見ると
\[\sum_{k=0}^{p-1} X^k(X + p^2m)^{p-1-k} \equiv \sum_{k=0}^{p-1} X^{k}\cdot X^{p-1-k} \equiv \sum_{k=0}^{p-1} X^{p-1} = pX^{p-1} \]
であるから, $X$が$p$で割れないことより, これは$\bmod{p^2}$で0ではないから主張は示された. 
\qed \\
いま, $v_{p} \geq p$より$a-b$は$p^2$で割れる. $a,b$は$c$と互いに素であるから$p$では割れない. したがって主張より
\[v_{p^{e+1}} = v_{p^e} + 1\]
であるが, $v_{p^e} = v_{p} + (e-1)$となる. ところが, $v_{p} +(e-1) \geq p^{e}$は十分大きい$e$では成り立たないので矛盾する. よって$c=1$が得られ, $x,y$は整数であることが証明された. \qed 

\syoumon{2} $x+y = p_1$, $x^2 + y^2 = p_2$は整数である. $xy = \dfrac{1}{2}(p_1^2 - p_2)$であるから, $x,y$は二次方程式
\[T^2 = p_1T - \dfrac{1}{2}(p_1^2 - p_2)\]
の解である. 詳しく言うと, 
\[x = \dfrac{p_1 \pm \sqrt{2p_2-p_1^2}}{2},\quad y=\dfrac{p_1 \mp \sqrt{2p_2-p_1^2}}{2}\]
と表せる. よって$T^{n+2} = p_1T^{n+1} - \dfrac{1}{2}(p_1^2-p_2) T^n$の解であるから, $p_{n+2} = p_1 p_{n+1} - \dfrac{1}{2}(p_1^2 - p_2)p_n$が成立する. すなわち$k=p_1^2 - p_2$(これは整数)とおくとき, 次が成立する.
\[kp_n = 2p_1p_{n+1} - 2p_{n+2}\]
いまここで$k$が奇数であると仮定すると, $kp_n$が偶数なので$p_n$が常に偶数となる. とくに$p_1,p_2$も偶数なので$k=p_1^2 - p_2$も偶数となり, 矛盾する. したがって$k$は偶数である. $x,y$は
\[x = \dfrac{p_1 \pm \sqrt{p_1^2 - 2k}}{2},\quad y=\dfrac{p_1 \mp \sqrt{p_1^2 -2k}}{2}\]
とも書ける. $k=2m$($m$は整数)と表すと$p_1^2 - 2k = p_1^2- 4m $なので, $p_1$の偶奇で分けると$x,y$は次の二つのパターンで表せることが必要となる. 
\bealn{ 
(x,y) &= \parena{\dfrac{a\pm \sqrt{b}}{2}, \dfrac{a\mp \sqrt{b}}{2}}\quad (aが奇数, b\equiv 1\pmod{4})\\
&, \parena{c \pm \sqrt{d}, c\mp \sqrt{d}}\quad (c,dは整数) 
}
逆にこれらが条件を満たすことを示す. 上のパターンの場合, $p_{1} = a, p_2 = (a^2+b)/2$で, $a,b$が奇数なので$p_2$は整数である. また, $p_n$は漸化式$p_{n+2} = ap_{n+1} - \dfrac{a^2 - b}{4}p_{n}$を満たし, 条件より$\dfrac{a^2-b}{4}$は整数だから, 帰納的にすべての$p_n$が整数となる. よってこの場合は条件を満たす. 下のパターンが条件を満たすことも同様にできる(あるいは二項定理から明らかである). 
