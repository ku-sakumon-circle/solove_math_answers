\begin{thm}{283}{\hosi 3}{有名問題}
$f(x,y)$ を $x,y$ に関する実数係数の多項式とする. 円 $x^2 + y^2 = 1$ 上の関数としていたるところで$f(x,y) = 0$ であるならば, $f(x,y)$は$x^2 + y^2 - 1$で割り切れることを証明せよ. 
\end{thm}

$f(x,y)$を$x^2 + y^2 - 1$で割ることで
$$
f(x,y) = (x^2 + y^2 - 1)Q(x,y) + xg_1(y) +  g_2(y)
$$
となるような $y$ の多項式 $g_1(y), g_2(y)$が存在する($y$をいったん定数とみて考えるとよい).

円$x^2 + y^2 = 1$のパラメータ$t$を取って$x=\cos{t}, y=\sin{t}$とおくことで
$$
f(\cos{t}, \sin{t}) = 0 = \cos{t}g_1(\sin{t}) + g_2(\sin{t})
$$ 
がすべての実数$t$で成立する. 

よって, 
$$
g_2(\sin{t})^2 = \cos^{2}{t}g_1(\sin{t})^2 = (1-\sin^{2}{t})g_1(\sin{t})^2 
$$
となるので, $\sin{t} = z$とおくことで
$$
g_2(z)^2 = (1-z^2)g_1(z)^2
$$
が$-1\leq z\leq 1$で成り立つので, 多項式としても一致する(多項式一致の定理). ここで, $g_1, g_2$が零多項式でないと仮定したとき, 両辺が $(z-1)$で何回割り切れるかを考えると, 左辺は偶数回で右辺は奇数回であるため, 矛盾である. よって$g_1g_2 = 0$であり, $g_1 = g_2 = 0$ も従う. これは$f(x,y)$が$x^2 + y^2 - 1$で割れることを示しているから, 題意は示された. 