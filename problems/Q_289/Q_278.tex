\begin{thm}{278}{\hosi 3}{}
    $x$は実数とする. 実数値関数$a(x)$を
    \[
    a(x) = 0 \quad (x\leq 0),\quad e^{-\frac{1}{x}} \quad (x>0)
    \]
    で定める. 
    \begin{enumerate}
        \item $k\in \mb{N}$とする. このときある多項式$P_k(T)$によって, $x>0$では$a(x)$の$k$階微分が$P_k(\frac{1}{x}) e^{-\frac{1}{x}}$と表せることを示せ. 
        \item $a(x)$が無限回微分可能であることを証明せよ.
    \end{enumerate}
\end{thm}
\syoumon{1} $x>0$とする. 
帰納法で示す. $k=1$では$a'(x) = \dfrac{1}{x^2} e^{-\frac{1}{x}}$なので$P_1(T) = T^2$とおけばよい. $k=n$で主張が正しいと仮定する. このとき
\[
\dfrac{d^{k+1}}{dx^{k+1}} a(x) = \dfrac{d}{dx} (P_k(\frac{1}{x})e^{-\frac{1}{x}}) = \parena{ \dfrac{1}{x^2} P_k(\frac{1}{x}) - \dfrac{1}{x^2} P_k'(\frac{1}{x})  }e^{-\frac{1}{x}}     
\]
なので, $P_{k+1}(T) = T^2 P_k(T) - T^2 P_k'(T)$と置けばよい. よって$k=n+1$でも正しく, 数学的帰納法よりすべての$k$で主張は正しい. \\
\syoumon{2}
$x\neq 0$における無限回微分可能性は明らかであるので, $x=0$におけるそれを示せばよい.\par 
$a$の$k$階微分を$a_k(x)$で表し, まずすべての$k$で$a_k(0) = 0$であることを示そう. $k=0$, すなわち$a_0(x) = a(x)$に対しては, 定義より明らかである. 
続いてある$n\geq 0$で$a_n(0) = 0$であるとする. 定義より
\[a_{n+1}(0) = \lim_{t\to 0} \dfrac{a_{n}(t) - a_{n}(0)}{t} = \lim_{t\to 0} \dfrac{a_n(t)}{t}\]
であるが, 明らかに$a_n(t)$は$t<0$で0なので, 左極限$t\to -0$では右辺も$0$である. 右極限$t\to +0$では, $s=\frac{1}{t}$とおくと
\[
   \dfrac{a_n(t)}{t} = \dfrac{sP_n(s)}{e^{s}}
\]
である. $t\to +0$のとき $s\to +\infty$であり, 指数関数が多項式関数より速く増大するので, この極限は0となる. よって$a_{n+1}(0) = 0$も示された. 
よってすべての$k\geq 0$で$a_{k+1}(0) = a_k'(0) = 0$であると言える. これは $a_{k}(x)$が$x=0$で微分可能であることを意味するので, 題意は示された. 