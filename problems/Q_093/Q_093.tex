\begin{thm}{093}{\hosi 7}{suiso\_728600 様}
 $n$を自然数、$p$を素数とする。
 \begin{align*}
  n^{28}+2016 \equiv 0 \pmod p
 \end{align*}
 が解をもつような$p$は無限に存在することを示せ。
\end{thm}

$p$が有限個しか存在しないとして, そのような$p$の集合を$K$とする。$p=2,3,7$のとき, $n=2016$が解になっているので $2,3,7\in K$である。そこで, $L=K-\{2,3,7\}$とおく。($2017$は素数であり, $p=2017$のとき$n=1$で解を持っているから $2017\in L$, つまり$L\neq \varnothing$.)仮定より$L$は有限集合としているので, $L=\{p_1,\cdots, p_m\}$とおき, $P=(42p_1p_2\cdots p_m)^{28} +2016$とおく。\footnote{たとえ$L=\varnothing$だとしても,$L$の元のすべての積を1とおいて, $P=42^{28}+1$とすれば問題ない} $P$は$L$の中の任意の素数を約数として持たない。しかし, 2,3,7の冪で表される数でもないから, $P$を素因数分解したときに現れる,2,3,7でない素数$q$であって, $q\notin K$であるものがとれる.このとき,  
\[n^{28}+2016\equiv 0 \pmod{q} \]
は $n=42p_1p_2\cdots p_m$を解にもち, $q\notin K$に矛盾する。よって$K$は無限集合。\qed
